\PassOptionsToPackage{unicode=true}{hyperref} % options for packages loaded elsewhere
\PassOptionsToPackage{hyphens}{url}
%
\documentclass[oneside,11pt,french,]{extbook} % cjns1989 - 27112019 - added the oneside option: so that the text jumps left & right when reading on a tablet/ereader
\usepackage{lmodern}
\usepackage{amssymb,amsmath}
\usepackage{ifxetex,ifluatex}
\usepackage{fixltx2e} % provides \textsubscript
\ifnum 0\ifxetex 1\fi\ifluatex 1\fi=0 % if pdftex
  \usepackage[T1]{fontenc}
  \usepackage[utf8]{inputenc}
  \usepackage{textcomp} % provides euro and other symbols
\else % if luatex or xelatex
  \usepackage{unicode-math}
  \defaultfontfeatures{Ligatures=TeX,Scale=MatchLowercase}
%   \setmainfont[]{EBGaramond-Regular}
    \setmainfont[Numbers={OldStyle,Proportional}]{EBGaramond-Regular}      % cjns1989 - 20191129 - old style numbers 
\fi
% use upquote if available, for straight quotes in verbatim environments
\IfFileExists{upquote.sty}{\usepackage{upquote}}{}
% use microtype if available
\IfFileExists{microtype.sty}{%
\usepackage[]{microtype}
\UseMicrotypeSet[protrusion]{basicmath} % disable protrusion for tt fonts
}{}
\usepackage{hyperref}
\hypersetup{
            pdftitle={SAINT-SIMON},
            pdfauthor={Mémoires XVIII},
            pdfborder={0 0 0},
            breaklinks=true}
\urlstyle{same}  % don't use monospace font for urls
\usepackage[papersize={4.80 in, 6.40  in},left=.5 in,right=.5 in]{geometry}
\setlength{\emergencystretch}{3em}  % prevent overfull lines
\providecommand{\tightlist}{%
  \setlength{\itemsep}{0pt}\setlength{\parskip}{0pt}}
\setcounter{secnumdepth}{0}

% set default figure placement to htbp
\makeatletter
\def\fps@figure{htbp}
\makeatother

\usepackage{ragged2e}
\usepackage{epigraph}
\renewcommand{\textflush}{flushepinormal}

\usepackage{indentfirst}
\usepackage{relsize}

\usepackage{fancyhdr}
\pagestyle{fancy}
\fancyhf{}
\fancyhead[R]{\thepage}
\renewcommand{\headrulewidth}{0pt}
\usepackage{quoting}
\usepackage{ragged2e}

\newlength\mylen
\settowidth\mylen{...................}

\usepackage{stackengine}
\usepackage{graphicx}
\def\asterism{\par\vspace{1em}{\centering\scalebox{.9}{%
  \stackon[-0.6pt]{\bfseries*~*}{\bfseries*}}\par}\vspace{.8em}\par}

\usepackage{titlesec}
\titleformat{\chapter}[display]
  {\normalfont\bfseries\filcenter}{}{0pt}{\Large}
\titleformat{\section}[display]
  {\normalfont\bfseries\filcenter}{}{0pt}{\Large}
\titleformat{\subsection}[display]
  {\normalfont\bfseries\filcenter}{}{0pt}{\Large}

\setcounter{secnumdepth}{1}
\ifnum 0\ifxetex 1\fi\ifluatex 1\fi=0 % if pdftex
  \usepackage[shorthands=off,main=french]{babel}
\else
  % load polyglossia as late as possible as it *could* call bidi if RTL lang (e.g. Hebrew or Arabic)
%   \usepackage{polyglossia}
%   \setmainlanguage[]{french}
%   \usepackage[french]{babel} % cjns1989 - 1.43 version of polyglossia on this system does not allow disabling the autospacing feature
\fi

\title{SAINT-SIMON}
\author{Mémoires XVIII}
\date{}

\begin{document}
\maketitle

\hypertarget{chapitre-premier.}{%
\chapter{CHAPITRE PREMIER.}\label{chapitre-premier.}}

1720

~

{\textsc{Le roi commence à monter à cheval et à tirer.}} {\textsc{-
L'Espagne remet la Sicile à l'empereur, et le roi de Sicile devient roi
de Sardaigne.}} {\textsc{- Mariage du duc d'Albret avec
M\textsuperscript{lle} de Gordes.}} {\textsc{- Suite de ses mariages.}}
{\textsc{- Fortune prodigieuse de M. et de M\textsuperscript{me} de
Beauvau par le duc de Lorraine.}} {\textsc{- Pension de dix mille livres
à la nouvelle duchesse d'Albret.}} {\textsc{- Survivance du gouvernement
de Franche-Comté au duc de Tallard, et de sous-gouverneur du roi au fils
aîné de Saumery.}} {\textsc{- Mariage de M. de Mailloc avec une fille de
la maréchale d'Harcourt.}} {\textsc{- Duc de Noailles s'accommode avec
Bloin, pour son second fils, de la survivance d'intendant des ville,
châteaux et parcs de Versailles et de Marly.}} {\textsc{- M. le comte de
Charolais et le maréchal de Montesquiou entrent au conseil de régence.}}
{\textsc{- Mort de M\textsuperscript{me} de Coetquen, et curiosités sur
elle.}} {\textsc{- Chabot.}} {\textsc{- Mort et caractère de l'abbé de
Chaulieu.}} {\textsc{- Mort de Sousternon.}} {\textsc{- Arrêt du conseil
du 22 mai 1720, qui manifeste le désordre des actions de la banque, et
qui a de tristes suites.}} {\textsc{- Malice noire d'Argenson.}}
{\textsc{- Mouvements du parlement.}} {\textsc{- L'arrêt est révoqué,
dont l'effet entraîne à la fin la perte de Law.}} {\textsc{- Conduite de
l'abbé Dubois à l'égard de Law.}} {\textsc{- M. le duc d'Orléans me
confie, et à deux autres avec moi, l'arrêt avant de le donner.}}
{\textsc{- Je tâche en vain de l'en détourner.}} {\textsc{- Conduite du
parlement et de M. le duc d'Orléans.}} {\textsc{- Arrêt qui révoque au
bout de six jours celui du 22 mai.}} {\textsc{- Law est ôté du contrôle
général des finances.}} {\textsc{- Beuzwaldt, avec seize Suisses, en
garde chez lui.}} {\textsc{- Il voit le régent après un refus simulé\,;
travaille avec lui et en est traité avec la bonté ordinaire.}}
{\textsc{- La garde se retire de chez lui.}} {\textsc{- L'agio est
transféré de la rue Quincampoix en la place de Vendôme.}} {\textsc{- M.
le duc d'Orléans me veut donner les sceaux, et m'en presse deux jours
durant.}} {\textsc{- Je tiens ferme à les refuser.}} {\textsc{- Law et
le chevalier de Conflans envoyés sonder et persuader le chancelier.}}
{\textsc{- Ils réussissent et le ramènent de Fresnes.}} {\textsc{- Les
sceaux redemandés à Argenson et rendus au chancelier.}} {\textsc{-
Retraite d'Argenson en très bon ordre et fort singulière.}}

~

Le roi commença a monter a cheval au pas, et galopa un peu quelque temps
après, puis commença à tirer.

Les Espagnols évacuèrent la Sicile, dont l'empereur prit possession, et
de tous les droits du tribunal fameux, dit de la monarchie, dont Rome
n'osa lui disputer la moindre partie, après tout ce qui en était arrivé
entre cette cour et le duc de Savoie, qu'on a vu ici en son temps. Ce
prince, qui avec toute son adresse n'avait pu parer ce fâcheux coup,
renonça malgré lui à la Sicile, en eut la faible compensation de la
Sardaigne, dont {[}il{]} prit le titre de roi, au lieu de celui de roi
de Sicile.

Le duc d'Albret épousa M\textsuperscript{lle} de Gordes, de la maison de
Simiane, fille unique du premier mariage de M\textsuperscript{me} de
Rhodes, qui était Simiane aussi, et veuve en secondes noces de M. de
Rhodes, dernier de la maison de Pot, qui avait été autrefois grand
maître des cérémonies, et fort de la cour et du grand monde, avec
beaucoup d'esprit et de galanterie, depuis perdu de goutte et fort
retiré, mort depuis longtemps. M. d'Albret perdit cette troisième femme
au bout de deux ans. Il avait deux fils de sa première femme, et un de
la seconde, mais il était infatigable en mariages. Il épousa en
quatrièmes noces, en 1725, une fille du comte d'Harcourt-Lorraine, qui
prit le nom postiche de Guise, si odieux aux vrais François, mais si
cher à cette maison. Il avait obtenu en don une terre en Lorraine du duc
de Lorraine, à laquelle il fit donner le nom de Guise, d'où il prit le
nom de comte, puis de prince de Guise. Il n'y eut point d'enfants de ces
deux derniers mariages du duc d'Albret, qu'une fille fort contrefaite,
qui a depuis épousé le fils aîné de M. de Beauvau, qui, lui et sa femme,
ont fait une si prodigieuse fortune par la faveur du dernier duc Léopold
de Lorraine, et qui s'est fait grand d'Espagne, prince de l'empire,
chevalier de la Toison d'Or, gouverneur de la Toscane, avec d'immenses
biens.

M. le duc d'Orléans donna à la nouvelle duchesse d'Albret une pension de
dix mille livres, la survivance du gouvernement de Franche-Comté au duc
de Tallard, et celle de sous-gouverneur du roi au fils aîné de Saumery,
qui valait beaucoup mieux que le père, car il était sage, instruit,
honnête homme, et dans les bornes de ce qu'il était\,; mais pour ce
genre de survivance, et d'un père plein de santé, qui n'avait pas besoin
de secours, mais qui en voulait perpétuer les appointements dans sa
famille, c'est une invention qui n'avait point d'exemple pour de pareils
emplois, et que le père qui l'obtint était bien loin de mériter par le
peu qu'il valait, dont il avait fait force preuves et des plus étranges,
comme on l'a vu ici en son lieu, et moins encore de la grâce de M. le
duc d'Orléans que de qui que ce pût être. Le maréchal de Tallard ni les
siens n'en avaient pas mieux mérité.

Le vieux marquis de Mailloc, riche, mais fort extraordinaire, épousa peu
après une fille de la maréchale d'Harcourt, à qui elle n'avait pas
grand'chose à donner. Il n'y en eut point d'enfants.

Le duc de Noailles, toujours à l'affût de tout, trouva que Versailles et
Saint-Germain, dont il avait le gouvernement et la capitainerie, étaient
faits l'un pour l'autre. Il tourna donc Bloin, dont il acheta pour son
second fils la survivance d'intendant des ville, châteaux et parcs de
Versailles et de Marly. Il prévoyait que dans quelques années ce morceau
serait bon à s'en être nanti, et il ne se trompa pas.

M. le comte de Charolais fut admis au conseil de régence, dont il ne fit
pas grand usage\,; il vit d'abord ce que c'était. Le maréchal de
Montesquiou y entra aussi en même temps\,; il y fit le trentième.

M\textsuperscript{me} de Coetquen mourut en Bretagne, où elle s'était
retirée depuis assez longtemps dans ses terres. Elle était Chabot, fille
de l'héritière de Rohan, et soeur du duc de Rohan, de la belle et habile
M\textsuperscript{me} de Soubise, et de M\textsuperscript{me} d'Espinoy,
cadette de l'une, aînée de l'autre. La beauté de M\textsuperscript{me}
de Soubise avait fait son mari prince\,; et que ne fit-elle pas\,?
M\textsuperscript{me} d'Espinoy jouissait du tabouret de grâce que le
crédit du vieux Charost avait obtenu, lorsque le prince d'Espinoy épousa
sa fille en premières noces. Cela faisait dire à M\textsuperscript{me}
de Coetquen assez plaisamment qu'elle était par terre entre deux
tabourets. C'était une femme d'esprit, de fort grande mine, avec de la
beauté, qui avait fait du bruit, haute et impérieuse, fort unie a ses
sueurs. Elle est célèbre par la passion que M. de Turenne eut pour elle,
qui lui arracha le secret du siège de Gand, que le roi n'avait confié
qu'à lui et à Louvois. M\textsuperscript{me} de Coetquen le laissa
échapper à dessein de se parer de son empire sur M. de Turenne, mais à
quelqu'un d'assez discret, et qui en sentit assez la conséquence pour
qu'il n'allât pas plus loin. Le roi ne laissa pas d'être averti qu'il
avait transpiré. Il le dit à Louvois, qui lui protesta qu'il n'en était
pas coupable. Le roi envoya quérir M. de Turenne, qui était alors aux
couteaux tirés avec Louvois. Il eut alors plus de probité que de
haine\,: il rougit et avoua sa faiblesse, et lui en demanda pardon. Le
roi, qui n'ignorait pas quel est l'empire de l'amour, se contenta d'en
rire un peu et de s'amuser aux dépens de M. de Turenne et avec lui, de
le trouver encore si sensible à son âge. Il le chargea de faire en sorte
que M\textsuperscript{me} de Coetquen fut plus secrète et tâchât de
fermer la bouche à qui elle avait eu l'indiscrétion de parler, car le
roi n'apprit que par M. de Turenne que c'était par M\textsuperscript{me}
de Coetquen, à qui il avait confié ce secret, qu'il s'était su. Mais
heureusement il n'avait pas été plus loin, et cette aventure ne porta
aucun préjudice à cette grande exécution. Le feu roi considérait
M\textsuperscript{me} de Coetquen\,; elle était dans la confidence de sa
soeur et fut assez avant en beaucoup de choses\,; elle était faite pour
la cour et pour le grand monde, où elle figura longtemps.

L'abbé de Chaulieu mourut quelques jours après\,: c'était un agréable
débauché de fort bonne compagnie, qui faisait aisément de jolis vers,
beaucoup du grand monde, et qui ne se piquait pas de religion. Il montra
malgré lui qu'il n'était guère plus attaché à l'honneur. Il l'était
depuis bien des années à MM\hspace{0pt}. de Vendôme, et fut très
longtemps le maître de leur maison et de leurs affaires. Le duc de
Vendôme s'en reposait entièrement sur le grand prieur son frère et sur
l'abbé de Chaulieu sous lui. On a vu ici en son temps que M. de Vendôme
se trouva ruiné, que son frère et l'abbé de Chaulieu s'entendaient et le
volaient\,; qu'il chassa Chaulieu de chez lui, se brouilla avec le grand
prieur, lui ôta tout maniement de ses affaires et de la dépense de sa
maison, et eut recours au roi, qui chargea Crozat l'aîné, beau-père
depuis du comte d'Évreux, de l'administration des affaires et de la
maison de M. de Vendôme. Chaulieu n'en rabattit rien de son ton dans le
monde, demeura de plus en plus étroitement lié avec le grand prieur, et
se moqua de tout ce qu'on en pouvait dire avec l'impudence qui lui était
naturelle. Mais cependant il n'osait plus paraître à la cour, quoiqu'on
n'en eût pas fait assez de cas pour le lui défendre. Il n'était que
tonsuré, se prétendait gentilhomme, et avait fourré un neveu dans la
gendarmerie, qui ne s'est point poussé. Cette noblesse était pour le
moins obscure, et le bien de la famille fort court. Cette friponnerie
lui fit perdre beaucoup de sociétés.

Sousternon mourut subitement chez M. de Biron qu'il était allé voir. Il
était fils d'un frère du feu P. de La Chaise, ancien lieutenant général
fort borné, en sorte qu'il lui était arrivé des malheurs à la guerre. Il
était aussi capitaine des gardes du comte de Toulouse, comme gouverneur
de Bretagne.

Le 22 mai de cette année devint célèbre par la publication d'un arrêt du
conseil d'État concernant les actions de la compagnie des Indes, qui est
ce qu'on connaissait sous le nom de Mississipi, et sur les billets de
banque. Cet arrêt diminuait par degrés les actions et les billets de
mois en mois, en sorte qu'à la fin de l'année ils se trouveraient
diminués chacun de la moitié de leur valeur. Cela fit ce qu'on appelle
en matière de finance et de banqueroute montrer le cul, et cet arrêt le
montra tellement à découvert qu'on crut tout perdu beaucoup plus à fond
qu'il ne se trouva, et parce que ce n'était pas même un remède au
dernier des malheurs. Argenson, qui par l'occasion de Law était arrivé
aux finances, et parvenu aux sceaux, qui, dans sa gestion, l'avait
finement barré en tout ce qu'il avait pu, et qui enfin s'était vu
nécessité de lui quitter les finances, fut très accusé d'avoir suggéré
cet arrêt par malice et en prévoyant bien tous les maux. Le vacarme fut
général et fut épouvantable. Personne de riche qui ne se crut ruiné sans
ressource ou en droiture, ou par un nécessaire contre-coup\,; personne
de pauvre qui ne se vit réduit à la mendicité. Le parlement, si ennemi
du système par son système, n'eut garde de manquer une si belle
occasion. Il se rendit protecteur du public par le refus de
l'enregistrement et par les remontrances les plus promptes et les plus
fortes, et le public crut lui devoir en partie la subite révocation de
l'arrêt, tandis qu'elle ne fut donnée qu'aux gémissements universels et
à la tardive découverte de la faute qu'on avait commise en le donnant.
Ce remède ne fit que montrer un vain repentir d'avoir manifesté l'état
intérieur des opérations de Law, sans en apporter de véritable. Le peu
de confiance qui restait fut radicalement éteint, jamais aucun débris ne
put être remis à flot.

Dans cet état forcé, il fallut faire de Law un bouc émissaire. C'était
aussi ce que le garde des sceaux avait prétendu\,; mais, content de sa
ruse et de sa vengeance, il se garda bien de se déceler en reprenant ce
qu'il avait été obligé de quitter. Il était trop habile pour vouloir des
finances en chef, en l'état ou elles se trouvaient. En peu de temps de
gestion, on eut oublié Law, et on s'en serait pris à lui\,; il en savait
trop aussi pour souffrir un nouveau contrôleur général, qui, pour le
temps qu'il aurait duré, eut été le maître\,; et c'est ce qui en fit
partager l'emploi en cinq départements. Véritablement, il choisit celui
qu'il voulut, et ayant ainsi remis un pied dans la finance, ses quatre
collègues le furent moins que ses dépendants. Ce fut une autre comédie,
que celle que donna le régent, en refusant de voir Law, amené par le,
duc de La Force par la portée ordinaire, et peut-être par une suggestion
du garde des sceaux, qui les haïssait tous deux, pour leur en donner la
mortification\,; puis de voir le même Law amené des le lendemain par
Sassenage, par les derrières, et reçu. M. le Duc, M\textsuperscript{me}
sa mère, et tout leur entour, étaient trop avant intéressés dans les
affaires de Law, et en tiraient trop gros pour l'abandonner. Ils
accoururent de Chantilly, et ce fut un autre genre de vacarme que M. le
duc d'Orléans eut à soutenir. L'abbé Dubois, tout absorbé dans sa
fortune ecclésiastique, qui courait enfin à grands pas à lui, avait été
la dupe de l'arrêt, puis n'osa soutenir Law contre l'universalité du
monde. Il se contenta de demeurer neutre et inutile ami, sans que Law
encore osât s'en plaindre. D'un autre côté, Dubois n'avait garde de se
brouiller avec un homme dont il avait si immensément tiré, et qui,
n'ayant plus d'espérance, se pouvait dépiquer à le dire. Dubois aussi
n'avait garde de le protéger ouvertement contre un public entier aux
abois et déchaîné. Tout cela tint encore quelque temps Law comme
suspendu par les cheveux, mais sans avoir pied nulle part, ni
consistance, jusqu'à ce que, comme on le verra bientôt, il fallut céder
et changer encore une fois de pays.

Cet arrêt fut donné et rétracté pendant une courte vacance du conseil de
régence, que j'allai passer à la Ferté. La veille de mon départ, étant
allé prendre congé de M. le duc d'Orléans, je le trouvai dans sa petite
galerie avec peu de monde. Il nous tira à part, le maréchal d'Estrées,
moi et je ne sais plus qui encore, et nous apprit cet arrêt qu'il avait
résolu. Je lui dis qu'encore que je me donnasse pour n'entendre rien en
finance, cet arrêt me semblait fort hasardeux\,; que le public ne se
verrait pas tranquillement frustrer de la moitié de son bien, avec
d'autant plus de raison qu'il craindrait tout pour l'autre\,; qu'il n'y
avait si mauvaise emplâtre\footnote{L'auteur fait emplâtre du féminin.}
qui ne valut mieux que celle-là, dont sûrement il se repentirait. On
voit, par bien des endroits de ces Mémoires, que je disais souvent bien
sans en être cru, et sans que les événements que j'avais prédits et qui
arrivent corrigeassent pour d'autres fois. M. le duc d'Orléans me
répondit d'un air serein en pleine sécurité. Les deux autres parurent de
mon avis, sans dire grand'chose. Je m'en allai le lendemain, et il
arriva ce que je viens de raconter.

Des que M. le duc d'Orléans eut vu Law, comme il vient d'être dit, il
travailla souvent avec lui, et le mena même, le samedi 25, dans sa
petite loge de l'Opéra, où il parut fort tranquille. Toutefois les
écrits séditieux et les mémoires raisonnés et raisonnables pleuvaient de
tous côtés, et la consternation était générale.

Le parlement s'assembla le lundi 27 mai au matin, et nomma le premier
président, les présidents Aligre et Portail, et les abbés Pucelle et
Menguy pour aller faire des remontrances. Sur le midi du même jour, M.
le duc d'Orléans envoya La Vrillière dire au parlement qu'il révoquait
l'arrêt du mercredi 22 mai, et que les actions et les billets de banque
demeureraient comme ils étaient auparavant. La Vrillière, trouvant la
séance levée, alla chez le premier président lui dire ce dont il était
chargé. L'après-dînée, les cinq députés susdits aillèrent au
Palais-Royal, furent bien reçus\,; M. le duc d'Orléans leur confirma ce
qu'il leur avait mandé par La Vrillière, leur dit de plus qu'il voulait
rétablir des rentes sur l'hôtel de ville à deux et demi pour cent. Les
députés lui répondirent qu'il était de sa bonté et de sa justice de les
mettre au moins à trois pour cent. M. le duc d'Orléans leur répondit
qu'il voudrait non seulement les mettre à trois, mais à quatre et à cinq
pour cent\,; mais que les affaires ne permettaient pas qu'on put passer
les deux et demi. Le lendemain 28 mai on publia l'arrêt qui remit les
billets de la banque au même état ou ils étaient avant l'arrêt du 22
mai, qui fut ainsi révoqué au bout de six jours, après avoir fait un si
étrange effet.

Le mercredi 29, La Houssaye et Fagon, conseillers d'État et intendants
des finances, furent, avec Trudaine, prévôt des marchands, visiter la
banque\,; en même temps Le Blanc, secrétaire d'État, alla chez Law, à
qui il dit que M. le duc d'Orléans le déchargeait de l'emploi de
contrôleur général des finances et le remerciait des soins qu'il s'y
était donnés, et que, comme bien des gens ne l'aimaient pas dans Paris,
il croyait devoir mettre auprès de lui un officier de mérite et connu,
pour empêcher qu'il ne lui arrivât quelque malheur. En même temps
Beuzwaldt\footnote{On écrit ordinairement ce nom Besenval ou Beseval,
  comme on l'a déjà fait observer.}, major du régiment des gardes
suisses, qui avait été averti, arriva avec seize Suisses pour rester
jour et nuit dans la maison de Law. Il ne s'attendait à rien moins qu'à
sa destitution ni à cette garde\,; mais il parut fort tranquille sur
l'une et sur l'autre, et ne sortit en rien de son sang-froid accoutumé.
Ce fut le lendemain que le duc de La Force mena Law chez M. le duc
d'Orléans par la porte ordinaire, qui ne voulut pas le voir, et qui le
vit le lendemain, conduit par Sassenage, par les derrières\,; depuis
quoi il continua de travailler avec lui, sans s'en cacher, et à le
traiter avec sa bonté ordinaire. J'ai rapporté plus haut cette comédie
que donna le régent, mais d'avance et en gros, pour mettre toute la
scène sous un même coup d'oeil. Le dimanche 2 juin, Beuzwaldt et ses
seize Suisses se retirèrent de chez Law. On ôta l'agiotage qui se
faisait dans la rue Quincampoix, et on l'établit dans la place Vendôme.
Il y fut en effet plus au large et sans empêcher les passants. Ceux qui
demeuraient dans cette place ne l'y trouvèrent pas si commode. Le roi
abandonna à la banque les cent millions d'actions qu'il y avait.

Pendant tous ces embarras, M. le duc d'Orléans, piqué contre Argenson,
auteur de l'arrêt du 22 mai, qui les avait causés, et dont les suites
avaient conduit nécessairement à la destitution de Law malgré Son
Altesse Royale, voulut ôter les sceaux à Argenson. Il m'en parla une
après-dînée que j'étais venu de Meudon travailler avec lui, m'expliqua
ses raisons en homme qui avait pris son parti, et tout de suite me
proposa de me les donner. Je me mis à rire\,; il me dit qu'il n'y avait
point à rire de cela, qu'il ne voyait que moi qu'il put en charger. Je
lui témoignai ma surprise d'une idée qui me paraissait si étrange, comme
s'il ne se pouvait trouver personne dans ce grand nombre de magistrats,
qui put en faire dignement les fonctions, à leur défaut par impossible,
par un prélat, et avoir recours à un homme d'épée qui ne savait ni ne
pouvait savoir un mot de lois, de règles et des formes pour
l'administration des sceaux. Il me répondit\,: qu'il n'y avait rien de
plus simple ni de plus aisé\,; que cette administration n'était qu'une
routine que j'apprendrais en moins d'une heure, et qui s'apprenait toute
seule en tenant le sceau. J'insistai à lui faire chercher quel qu'un. Il
prit donc l'Almanach royal, et eut la patience de me lire nom par nom la
liste de tous les magistrats principaux par leurs places ou par leur
simple réputation, et de me détailler sur chacun ses raisons
d'exclusion. De la, il passa au conseil de régence avec les mêmes
raisons d'exclusion sur chacun\,; enfin aux prélats, mais légèrement,
parce qu'en effet il n'y en avait point sur qui on put s'arrêter.

Je lui contestai plusieurs exclusions de magistrats, celle surtout, du
chancelier. J'insistai même sur quelques-uns du parlement, comme sur
Gilbert de Voisins, mais sans pouvoir nous persuader l'un l'autre. Je
lui dis que je comprenais que les sceaux étaient pour un magistrat une
fortune par l'autorité, le rang, la décoration pour leur famille à
laquelle ils ne pouvaient résister\,; que je ne pouvais être touché de
pas une de ces raisons, parce qu'aucune ne pouvait me regarder\,; que
les sceaux ne décoreraient point ma maison, qu'ils n'apporteraient aucun
changement à mon rang, à mon habit, à mes manières, mais qu'ils
m'exposeraient à la risée de ceux qui me verraient tenir le sceau, et à
me casser la tête à apprendre un métier que je cesserais de faire avant
que d'en savoir à peine l'écorce\,; que de plus je ne voulais hasarder
ni ma conscience, ni mon honneur, ni le bien précieux de son amitié, en
scellant ou refusant bien ou mal à propos des édits et des déclarations
qu'il m'enverrait ou des signatures à faire d'arrêts du conseil rendus
sous la cheminée. Le régent ne se paya d'aucune de ces raisons. Il
essaya de m'exciter par la singularité de la chose et par les exemples
du premier maréchal de Biron et du connétable de Luynes\footnote{Voy. la
  liste des gardes des sceaux, t. X, p.~447. On y trouve, en effet, ce
  connétable de Luynes.}. Ils ne m'ébranlèrent point, de sorte que la
discussion dura plus de trois grosses heures. Je voulus m'en aller
plusieurs fois sous prétexte qu'il y avait loin à Meudon, et toujours je
fus retenu. À la fin, de guerre lasse, il me permit de m'en aller, mais
à condition qu'il m'enverrait le lendemain deux hommes à Meudon, qu'il
ne me nomma point, qui peut-être me persuaderaient, et qu'il me
demandait instamment d'entretenir et d'écouter tant qu'ils voudraient\,;
il fallut bien y consentir, et cerne fut encore après qu'avec peine
qu'il me laissa aller.

Le lendemain matin je ne vis point de harangueur arriver\,; mais à la
moitié du dîner, où j'avais toujours bien du monde, je vis entrer le duc
de La Force et Canillac. Ce dernier me surprit fort. Je n'avais jamais
eu de commerce avec lui que de rencontres rares, je l'avais vu chez moi
et chez lui quatre ou cinq fois dans la première quinzaine de la
régence\,; oncques depuis nous ne nous étions vus que d'un bout de table
à l'autre, au conseil de régence, depuis qu'il y fut entré, et sans nous
approcher devant ni après, ni nous rencontrer ailleurs. On a vu ici
qu'il s'était livré à l'abbé Dubois, au duc de Noailles, à Stairs, et
qu'il l'était totalement au parlement, et on y a vu aussi son caractère.
Leur arrivée n'allongea pas le repas. Ils mangèrent en gens pressés de
finir, et à peine le café pris ils me prièrent de passer dans mon
cabinet. Ils étaient venus ferrés à glace, et je ne pus douter que M. le
duc d'Orléans ne leur eût rendu tout le détail de la si longue
discussion que j'avais eue avec lui sur les sceaux, l'après-dînée de la
veille. M. de La Force ouvrit non pas la conférence, mais le plaidoyer
qui ne fut pas court. Canillac ensuite, qui se plaisait à parler et qui
parlait bien, mais sans cesse, se donna toute liberté. Leur grand
argument fut\,: l'absolue nécessité de se défaire entièrement du garde
des sceaux, dont l'infidélité causée par sa jalousie de Law, avait
produit ce fatal arrêt du 22 mai, uniquement pour perdre Law, sans se
soucier du péril où il jetait M. le duc d'Orléans, en mettant au net ce
qui ne pouvait être tenu trop caché, et qui de plus était en partie le
fruit de toutes les entraves qu'il avait jetées sans cesse à toute
l'administration de Law et a ses opérations\,; les menées du parlement
plus envenimées que jamais contre M. le duc d'Orléans, et plus
organisées, devenu plus habile en ce genre et plus précautionné, en même
temps plus furieux par la leçon que lui avait donné le lit de justice
des Tuileries, qu'il ne pardonnerait jamais\,; l'impossibilité, par
conséquent, de choisir qui que ce put être de cette compagnie pour les
sceaux, exclusion qui regardait également le chancelier par son
attachement extrême et irrémédiable pour ce corps dont il sortait et
dont il faisait sa divinité\,; qu'il fallait dans les conjonctures
présentes un garde des sceaux dont l'attachement à M. le duc d'Orléans
fut tel, qu'il n'en put jamais douter, que rien ne put ébranler, qui fut
connu pour tel, et qui imposât par la une crainte et un embarras qui
troublât la cabale et ses résolutions. Avec cela ils me faisaient
beaucoup d'honneur\,; mais rien ne coûte quand on veut persuader avec
des propos tels qu'ils me dirent, un homme de tête, d'esprit, de
courage, de réputation intacte sur l'honneur, la vérité, l'intérêt\,;
surtout connu pour n'en avoir jamais voulu avoir avec les actions ni la
banque\,; intact sur les finances dont il ne se serait jamais voulu
mêler, qui eut de la dignité, qui la connut, qui fut jaloux de
l'autorité royale, enfin qui eut la parole à la main et qui fut
incapable de crainte pour savoir soutenir les remontrances et les divers
efforts du parlement, le contenir par ses réponses et préserver le
régent de faiblesse qui lui serait soufflée de toutes parts, à laquelle
il n'était que trop naturellement enclin, et qui serait sa perte
certaine et bien projetée dans les circonstances présentes. Qu'il ne
fallait point se flatter de trouver dans le conseil aucun magistrat
capable de ce poids, qui ne sentît la robe, qui n'aimât ou ne craignît
le parlement, qui ne fut entraîné à mollir à l'aspect de l'état des
finances, qui fut bien supérieur au plaisir de voir l'embarras ou on
était tombé pour s'être opiniâtrement écarté de toutes les routes
connues et battues\,; qui ne fut affaibli par les cris que les menées du
parlement et de ses adjoints aigrissaient et augmentaient sans cesse\,;
qui par-dessus tout ne songeât a sa conservation et qui ne fut effrayé
de ce qu'on lui ferait envisager au bout de la régence, qui ne le fut
même des hasards de l'intérieur du régent avant même la fin de la
régence. Qu'il était également inutile de rien espérer d'aucun de ceux
qui composaient le conseil de régence, presque tous incapables, faibles,
effrayés, entraînés, le reste ou ignorants ou plus que très suspects, et
dont l'esprit et la capacité serait extrêmement dangereuse. M. de La
Force reprit la parole, mais je leur, proposai alors d'aller achever la
conversation qui avait déjà duré près de trois heures, en prenant l'air
sur la terrasse qui mène aux Capucins.

Chemin faisant M. de La Force essaya de me tenter tout bas par le
plaisir de mortifier le parlement et le premier président par moi-même,
après tout ce qui s'était passé sur le bonnet, et de me montrer à eux
sous le visage sévère, et supérieur que j'emprunterais des sceaux dont
il m'étala les occasions continuelles et la satisfaction que j'aurais
d'en profiter en servant bien l'État et M. le duc d'Orléans. Canillac
s'était peu à peu écarté en sorte qu'il ne pouvait entendre, jeu ne sais
si ce fut de hasard ou de concert, mais il se rapprocha et il fut de la
fin de cette sorte de conversation avec la légèreté d'un homme d'esprit
qui, sans s'éloigner de ses préjugés, ne laisse pas de profiter de tout
pour arriver au but qu'il s'était proposé à mon égard. Le beau temps et
la belle vue de cette terrasse firent quelques moments de trêve au
sérieux que nous traitions\,; nous gagnâmes ainsi le bout de la terrasse
et ce qu'on appelle le bastion des Capucins\,; là nous nous assîmes, et
quoique la vue y soit encore plus admirable, la conversation se reprit
incontinent.

On peut juger que jusqu'alors ils n'avaient pas parlé seuls et que
j'avais pris quelquefois la parole, quoique avec Canillac il fut aisé de
la laisser reposer. Ce fut ici ou ils m'exposèrent le plus au long le
péril dont M. le duc d'Orléans était menacé, les vues et les menées du
parlement appuyé de beaucoup de gens considérables, du mécontentement,
public, du désordre des affaires, de la perspective de la majorité, qui
n'était plus éloignée que de trois ans moins quelques mois. L'exposé fut
long et vif, les noms des gens considérables suspects et plus que
suspects\,; leurs intrigues, leurs vues, leurs intérêts n'y furent pas
oubliés\,; j'y admirai souvent que Canillac consentît a tout ce qui
était allégué là-dessus par le duc de La Force, et que lui-même,
protecteur public du parlement, du premier président, lui, ami du
maréchal de Villeroy, qui a force de recherches l'avait gagné, et si
enclin au duc du Maine, chargeât encore le tableau sur leur compte. Je
ne pus m'empêcher de lui dire quelquefois que, si j'en avais été cru, et
si je n'avais pas trouvé des contrebatteries si fortes, qui avaient fait
jouer tant de ressorts en tout temps auprès de M. le duc d'Orléans, ni
le parlement, ni pas un de tous ceux dont ils me parlaient et dont ils
ne me cachaient pas les noms, ne serait pas maintenant en situation de
se faire considérer, ni de causer la moindre réflexion à faire, et je
regardais Canillac qui baissait les yeux. Il était vrai que le
parlement, et tous ceux qui, avec M. et M\textsuperscript{me} du Maine,
avaient été si déconcertés et si effrayés, avaient enfin peu à peu
repris leurs esprits et travaillèrent de nouveau, fondés sur le mépris
de la mollesse qui avait suivi tant d'éclat de si près. Mais je ne
voyais pas en quoi les sceaux entre mes mains pouvaient remédier à ces
menées dont le décri et le dévoilement des affaires était le trop
apparent fondement, la légèreté et la faiblesse naturelles de M. le duc
d'Orléans, l'appui\,: ce fut là tout l'argument de ma défense. Je leur
fis les mêmes réponses que j'avais faites la veille a M. le duc
d'Orléans, et les priai de remarquer que les cris publics sur l'état des
finances, démasqué par l'arrêt du 22 mai, éclataient principalement
contre les routes détournées de la conduite des finances, que ce n'était
donc pas le temps d'en prendre une autre, pour une autre partie du
ministère et de l'administration, qui, pour n'avoir pas le même danger
ni la même conséquence, n'en paraîtrait pas moins extraordinaire et
insolite, et ne ferait qu'augmenter le murmure contre ce goût du
nouveau, quand on verrait un homme d'épée avoir les sceaux, et son
ignorance à les tenir exposée aux brocards du dépit de toute la robe de
les voir hors de ses mains.

Je ne finirais point si je voulais rapporter tout ce qui fut dit et
discuté de part et d'autre. Je me contenterai de dire que je fus pressé
par ces deux hommes, qui y employèrent tout leur esprit, comme si
d'accepter ou de refuser les sceaux, la fortune, le salut, la vie de M.
le duc d'Orléans eut été entre mes mains, et n'eut dépendu que du parti
qu'à cet égard j'allais prendre\,; je n'en pus être persuadé, et je ne
me rendis point. Enfin la nuit nous gagnant, et il faut remarquer que
c'était dans la fin de mai, par le plus beau temps du monde, je leur
proposai le retour. Tout le chemin fut encore employé de leur part au
pathétique, à la fin aux regrets, à m'annoncer ceux que les événements
que j'aurais empêchés me causeraient, et à tous les propos de gens qui
s'étaient promis de réussir, et qui s'en voyaient déçus. En arrivant au
château neuf, je me gardai bien d'entrer chez moi\,; je les conduisis où
était la compagnie, avec laquelle je me mêlai pour me défaire de mes
deux hommes, qui près de sept heures durant m'avaient fatigué à l'excès.
Leur voiture les attendait depuis longtemps, ils causèrent un peu debout
avec le monde, enfin me dirent adieu et s'en aillèrent.

Je n'ai jamais compris cette fantaisie de M. le duc d'Orléans, encore
moins l'acharnement de Canillac à me persuader. J'ai toujours cru que,
M. le duc d'Orléans y allait de bonne foi, pour avoir dans la place des
sceaux un homme parfaitement sur et ferme qui l'aiderait et le
fortifierait à se débarrasser des menées et des entreprises du
parlement, et qui toutefois par ce qu'il en avait expérimenté sur
l'affaire du duc du Maine lors du lit de justice des Tuileries, et sur
la personne aussi du premier président, ne le mènerait pas trop loin\,;
M. de La Force aussi, ravi d'être chargé de quelque commission que ce
fut, bien aise de voir ôter les sceaux à la robe, et d'y voir un duc
ulcéré contre le premier président et le parlement, en place de les
barrer et de les mortifier. L'abbé Dubois, avec qui je n'étais pas bien,
et que j'avais depuis outré par l'aventure que j'ai racontée sur son
sacre, sans lequel rien d'important ne se faisait alors, aurait, je
crois, voulu m'embarquer dans quelque ânerie, me commettre avec le
parlement, et le raccommoder avec le régent à mes dépens, pour de pique
me faire abandonner la partie, et me retirer tout à fait. Law, de son
côté, qui m'avait toujours courtisé, et qui savait qu'il ne lui en avait
rien coûté, quelque presse qu'il m'en eut faite et fait faire par M. le
duc d'Orléans, et qui était bien sur que je ne voulais en aucune sorte
me mêler de finance, me voulait aux sceaux comme un homme sûr et ferme
qui ne mollirait point, qui ne le barrerait et ne le tracasserait point,
qui tiendrait en bride ceux des départements des finances qui le
voudraient faire, quand je verrais la raison de son côté, qu'il serait à
portée de me faire entendre\,; de qui il n'aurait à craindre ni la
haine, ni la jalousie, ni l'envie auprès de M. le duc d'Orléans, et qui
donnerait du courage et de la dignité à ce prince à l'égard du parlement
et de la cabale qui lui était unie. Ces réflexions ne me vinrent
qu'après cette conférence si longue de Meudon, dont la persécution les
produisit le lendemain. Canillac me haïssait de jalousie de la confiance
de M. le duc d'Orléans, et de ricochet du duc de Noailles, du premier
président, etc. Son ambassade et la prodigalité de son éloquence à me
persuader ne pouvaient venir de sa part que de l'espérance de me jeter
dans quelque sottise dans l'administration des sceaux, dont lui et ses
amis pussent profiter avec avantage. Mais rien de tout cela n'eut part à
mon refus. Ces raisonnements ne se présentèrent à moi qu'après coup\,:
faire un métier important et fort éclairé dont j'ignorais les premiers
éléments, m'exposer à expédier des édits, déclarations, arrêts, mauvais,
iniques, peut-être pernicieux, sans en connaître la force, le danger,
les suites, ou les refuser nettement, voila les raisons qui me
frappèrent d'abord, et dont rien ne put me faire revenir. Une autre
raison, mais qui aurait cédé à de meilleures, fut d'éviter de me donner
une singularité passagère qui ferait encore raisonner sur le goût des
choses inusitées, laquelle ne me donnait ni rang, ni illustration, ni
rien, dont je susse que faire, et qui ne m'apportait qu'un travail
aveugle par mon ignorance en ce genre, et fort ingrat d'ailleurs.

Mon refus, sans plus d'espérance de me persuader, rapporté à M. le duc
d'Orléans dans ces moments critiques où il n'en fallait perdre aucun
pour prendre un parti, devint la matière d'une délibération subite où je
ne fus point appelé, et qui ne se prit qu'entre M. le duc d'Orléans,
l'abbé Dubois et Law. Le résultat fut que Law irait trouver le
chancelier qu'on savait qui se mourait d'ennui d'être à Fresnes\,; que
le chevalier de Conflans, cousin germain, ami intime du chancelier, et
raisonneur fort avec beaucoup d'esprit, l'accompagnerait de la part de
M. le duc d'Orléans, dont il était premier gentilhomme de la chambre\,;
que Law expliquerait l'état présent des affaires, sonderait si le
chancelier se rendrait traitable, et si on pouvait compter que la cire
deviendrait molle entre ses mains, ses dispositions pour lui Law\,;
enfin si on pourrait se fier à lui à l'égard du parlement, non sur sa
probité dont on ne pouvait être en peine, mais bien de son goût, de son
affection et de son espèce de culte à l'égard de cette compagnie.
Conflans devait essayer de l'effrayer par la menace d'une continuation
d'exil sans fin et sans terme, même après la régence, que la fin de tout
crédit de M. le duc d'Orléans, et lui en faire briller aux yeux les
grâces, la confiance, le retour actuel avec les sceaux, s'il se voulait
résoudre de bonne grâce à ce qu'on désirait de lui. Trois ans et demi de
séjour à Fresnes avaient adouci les moeurs d'un chancelier de cinquante
ans, qui avait compté que, parvenu de si bonne heure à la première
place, il en jouirait et avancerait sa famille. Ces espérances se
trouvaient ruinées par l'exil, et il se trouvait beaucoup plus éloigné
de l'avancer et d'accommoder ses affaires domestiques que s'il fut
demeuré procureur général. Conflans profita de ces dispositions qui ne
lui étaient pas inconnues, et que l'ennui de l'exil grossissait. Le beau
parler de Law trouva des oreilles bien disposées. Le chancelier
s'accommoda à tout, et le publie, quand il fut informé, le reçut
froidement et s'écria\,: Et homo factus est\footnote{Il suffira
  d'indiquer une fois pour toutes, le Journal de l'avocat Barbier, comme
  rempli de détails sur ce qui touche au parlement et au chancelier. Au
  mois d'août 1720, il s'exprime ainsi\,: «\,On a fait une
  plaisanterie\,; on dit que M. le régent a mal choisi Pontoise pour
  transférer le parlement\,; qu'il fallait l'envoyer à Fresnes, qui est
  la terre de M. d'Aguesseau\,; que c'est là où l'on change de
  sentiments à cause du parti que prend le chancelier. Si tu veux de ton
  parlement / Punir l'humeur hautaine, / De Pontoise, trop doux régent,
  / Fais le sauter à Fresnel\,! / C'est un lieu de correction / La
  Faridondaine, la Faridondon, / Où d'Aguesseau s'est converti, etc.}

M. le duc d'Orléans, certain du bon succès du voyage, envoya, le
vendredi 7 juin, l'abbé Dubois demander les sceaux à Argenson, qui les
rapporta à M. le duc d'Orléans l'après-dînée du même jour, et comme il
les avait non en commission à l'ordinaire, mais en charge, enregistrée
au lit de justice des Tuileries, il en remit en même temps sa démission.
Il ne jouit donc pas longtemps du fruit de son insigne malice. Les amis
de Law après le premier feu passé la firent sentir au régent, tirèrent
sur le temps et culbutèrent le garde des sceaux sans que l'abbé Dubois,
qui, entre lui et Law, nageait entre deux eaux, osât soutenir son ancien
ami. Le chancelier arriva dans la nuit qui suivit la remise des sceaux,
alla sur le midi au Palais-Royal, suivit M. le duc d'Orléans aux
Tuileries où le roi lui remit les sceaux\,; mais comme il les dut à Law,
qui le ramena de Fresnes, ce retour fit la première brèche à une
réputation jusque-là la plus heureuse, et qui n'a cessé de baisser
depuis et de tomber tout à fait par divers degrés et par différents
événements. Argenson n'avait pas perdu son temps\,; il était né pauvre,
il se retira riche, ses enfants tous jeunes bien pourvus, en place avant
l'âge, son frère chargé de bénéfices. Il témoigna une grande
tranquillité, qui dans peu lui coûta la vie, sort ordinaire de presque
tous ceux qui se survivent à eux-mêmes. Sa retraite fut sans exemple. Ce
fut dans un couvent de filles dans le faubourg Saint-Antoine, qui
s'appelle la Madeleine de Tresnel\footnote{Communauté de femmes de
  l'ordre de Saint-Benoît, fondée à Tresnel, ou Traisnel, en Champagne,
  au XIIe siècle. Ces religieuses s'étaient établies à Paris, en 1654.
  On trouvera, dans le Journal de Barbier (juin 1720), quelques-unes des
  chansons et autres plaisanteries, auxquelles donna lieu la retraite de
  l'ancien garde des sceaux.}, ou il s'était accommodé depuis longtemps
un appartement dans le dehors qu'il avait rendu beau et complet, commode
comme une maison, ou il allait tant qu'il pouvait depuis longues années.
Il avait procuré, même donné beaucoup à ce couvent, à cause d'une
M\textsuperscript{me} de Veni, qui en était supérieure, qu'il disait sa
parente, et qu'il aimait beaucoup. C'était une personne fort attrayante,
et qui avait infiniment d'esprit, dont on ne s'est pas avisé de mal
parler. Tous les Argenson lui faisaient leur cour\,; mais ce qui était
étrange, c'est qu'étant lieutenant de police, elle sortait lorsqu'il
était malade pour venir chez lui et demeurer auprès de lui. Il conserva
le rang, l'habit et toutes les marques de garde des sceaux, mais pour sa
chambre\,; car il n'en sortit plus que deux ou trois fois pour aller
voir M. le duc d'Orléans par les derrières, qui lui continua toujours
beaucoup de considération\,; l'abbé Dubois aussi qui le fut voir
plusieurs fois. Hors deux ou trois amis particuliers et sa plus étroite
famille, il ne voulut voir personne, et s'ennuya cruellement\,; c'est ce
même couvent dont après sa mort, et cette même M\textsuperscript{me} de
Veni, dont M\textsuperscript{me} la duchesse d'Orléans a depuis fait ses
délices.

\hypertarget{chapitre-ii.}{%
\chapter{CHAPITRE II.}\label{chapitre-ii.}}

1720

~

{\textsc{Conférence de finance singulière au Palais-Royal.}} {\textsc{-
Création de rentes à deux et demi pour cent enregistrées.}} {\textsc{-
Diminution des espèces.}} {\textsc{- Des Forts presque contrôleur
général.}} {\textsc{- Les quatre frères Pâris exilés.}} {\textsc{-
Papiers publics solennellement brûlés à l'hôtel de ville.}} {\textsc{-
Caractère de Trudaine, prévôt des marchands.}} {\textsc{- M. le duc
d'Orléans m'apprend sa résolution d'ôter le prévôt des marchands, de
mettre Châteauneuf en sa place, de chasser le maréchal de Villeroy et de
me faire gouverneur du roi\,; à quoi je m'oppose avec la dernière force,
et je l'emporte\,: mais il ne me tient parole que sur le dernier.}}
{\textsc{- Trudaine remercié.}} {\textsc{- Châteauneuf prévôt des
marchands.}} {\textsc{- Trudaine et le maréchal de Villeroy sont tôt
informés au juste de tout ce tête-à-tête, sans qu'on puisse imaginer
comment, et avec des sentiments bien différents l'un de l'autre.}}
{\textsc{- Conduite étrange du maréchal de Villeroy.}} {\textsc{- Il est
visité par les harengères dans une attaque de goutte.}} {\textsc{-
Emplois des enfants d'Argenson.}} {\textsc{- Baudry lieutenant de
police.}} {\textsc{- M. le duc d'Orléans renvoie gracieusement les
députés du parlement au chancelier.}} {\textsc{- Arrêt célèbre sur les
pierreries.}} {\textsc{- Sutton succède à Stairs.}} {\textsc{- Courtes
réflexions.}} {\textsc{- Continuation de la brûlerie par le nouveau
prévôt des marchands.}} {\textsc{- Édit pour rendre la compagnie des
Indes, connue sous le nom de Mississipi, compagnie exclusivement de
commerce.}} {\textsc{- Effets funestes de cet édit.}} {\textsc{- Gens
étouffés à la banque.}} {\textsc{- Le Palais-Royal menacé.}} {\textsc{-
Law insulté par les rues\,; ses glaces et ses vitres cassées.}}
{\textsc{- Il est logé au Palais-Royal.}} {\textsc{- Le parlement refuse
d'enregistrer l'édit.}} {\textsc{- Ordonnance du roi étrange.}}
{\textsc{- Précautions\,; troupes approchées de Paris.}} {\textsc{-
Conférence au Palais-Royal entre M. le duc d'Orléans et moi.}}
{\textsc{- Petit conseil tenu au Palais-Royal.}} {\textsc{- Impudence de
Silly.}} {\textsc{- Translation du parlement à Pontoise.}} {\textsc{-
Effronterie du premier président, qui tire plus de trois cent mille
livres de la facilité de M. le duc d'Orléans, pour le tromper, s'en
moquer, et se raccommoder avec le parlement à ses dépens.}} {\textsc{-
Le parlement refuse d'enregistrer sa translation, puis l'enregistre en
termes les plus étranges.}} {\textsc{- Arrêt de cet enregistrement.}}
{\textsc{- Conduite du premier président.}} {\textsc{- Dérision du
parlement à Pontoise, et des avocats pareille.}} {\textsc{- Foule
d'opérations de finance.}} {\textsc{- Des Forts en est comme contrôleur
général.}} {\textsc{- Profusion de pensions.}} {\textsc{- Maréchal de
Villars cruellement hué dans la place de Vendôme.}} {\textsc{-
L'agiotage qui y est établi transporté dans le jardin de l'hôtel de
Soissons.}} {\textsc{- Avidité sans pareille de M. et de
M\textsuperscript{me} de Carignan.}} {\textsc{- Law, retourné du
Palais-Royal chez lui, fort visité.}} {\textsc{- Les troupes approchées
de Paris renvoyées.}} {\textsc{- Peste de Marseille.}}

~

L'après-dînée du jour que les sceaux furent rendus au chancelier
d'Aguesseau, il assista a une assemblée fort singulière qui fut tenue
par M. le duc d'Orléans, ou se trouvèrent le maréchal de Villeroy, seul
du conseil de régence, des Forts, Ormesson, beau-frère du chancelier, et
Caumont, tous trois conseillers d'État, et ayant des départements de
finance de la dépouille de Law, les cinq députés du parlement susdits
pour les remontrances qui étaient\,: le premier président, les
présidents Aligre et Portail, et deux conseillers clercs de la
grand'chambre, les abbés Pucelle et Menguy, et La Vrillière, en cas
qu'on eut besoin de plume et qu'il y eut des ordres a donner ou des
expéditions a faire. Le fruit de cette conférence fut l'enregistrement
de l'édit de création de rentes sur l'hôtel de ville à deux et demi pour
cent, qui fut fait au parlement le surlendemain lundi 10 juin, qui fut
publié le lendemain\,; on publia en même temps un arrêt pour la
diminution des monnaies à commencer au 1er juillet suivant. Par la
retraite d'Argenson, des Forts, sans en avoir le titre ni la fonction
précise devint comme contrôleur général. À l'égard de force arrêts et
autres opérations de finance, et de mutations de départements et de
bureaux, c'est de quoi je continuerai a ne pas charger ces Mémoires. Je
dirai seulement que les quatre frère Paris, dont j'ai parlé ailleurs,
furent exilés en Dauphiné. Ils ont depuis été les maîtres du royaume
sous M. le Duc, et ils le sont à peu près redevenus aujourd'hui,
c'est-à-dire les deux qui sont demeurés en vie\footnote{Les deux Pâris,
  qui avaient encore une grande influence en 1751, époque où Saint-Simon
  écrivait cette partie de ses Mémoires, étaient Pâris-Duverney et
  Pâris-Montmartel.}.

On cherchait depuis quelque temps à ranimer quelque confiance, et on
crut qu'un des plus utiles moyens d'y parvenir serait d'anéantir si
authentiquement les papiers publics acquittés, qu'il ne pût rester le
moindre soupçon qu'on en pût remettre aucun dans le commerce et gagner
dessus de nouveau. On prit donc le parti de les remettre toutes les
semaines par compte au prévôt des marchands, qui les brûlait
solennellement à l'hôtel de ville en présence de tout le corps de ville
et de quiconque y voulait assister, même bourgeois et peuple. Trudaine,
conseiller d'État, était prévôt des marchands\,: c'était un homme dur,
exact, sans entregent et sans politesse, médiocrement éclairé, aussi peu
politique, mais pétri d'honneur et de justice, et universellement
reconnu pour tel\,: il devait tout ce qu'il était au feu chancelier
Voysin, mari de sa soeur, et il n'avait pas pris d'estime, ni encore
moins d'affection dans ce tripot-là pour M. le duc d'Orléans, ni pour
son gouvernement. Il ne s'était point caché de toute l'horreur qu'il
avait pour le système et pour tout ce qui s'était fait en conséquence.
Ce magistrat s'expliqua si crûment à l'occasion de ce brûlement de
billets et de quelques méprises qui s'y commirent de la part de ceux
dont il les recevait, que ces messieurs offensés aigrirent M. le duc
d'Orléans, et lui persuadèrent qu'au temps scabreux ou on était du côté
de la confiance et du peuple, l'emploi de prévôt des marchands ne
pouvait être en de plus dangereuses mains. À cette disposition, Trudaine
mit le comble par un propos imprudent qui lui échappa de surprise en
public à un brûlement de billets, comme si quelques-uns de ceux-là lui
eussent déjà passé par les mains. Tout aussi {[}tôt{]} M. le duc
d'Orléans en fut informé, et il est vrai que ce discours fut promptement
débité et commenté, et qu'il ne fit pas un bon effet pour la confiance.
Un jour ou deux après, je vins de Meudon travailler avec M. le duc
d'Orléans à mon ordinaire\,; dès que je parus (et le premier président
était seul dans une grande pièce du grand appartement qui donne dans le
petit)\,: «\,Je vous attends avec impatience, me dit le régent, pour
vous parler de choses importantes\,;» et s'enfonçant dans cette autre
vaste pièce où était l'estrade et le dais, se mit à se promener avec moi
et me conta toute l'affaire de l'hôtel de ville comme on la lui avait
rendue, ajouta tout de suite que c'était un complot du maréchal de
Villeroy et du prévôt des marchands, et qu'il avait résolu de les
chasser tous deux.

Je lui laissai jeter son feu, puis j'essayai à lui ôter ce complot de la
tête, en lui faisant le portrait de Trudaine. Je condamnai sa rusticité,
je blâmai surtout son imprudence, en remontrant qu'elle ne méritait ni
un éclat ni un affront tel que de l'ôter de place avant la fin de sa
prévôté, mais bien un avertissement un peu ferme d'être plus mesuré dans
ses paroles. Pour donner plus de poids aux miennes, je lui dis que ce
n'était point par amitié pour Trudaine que je lui parlais, puisqu'il
pouvait se souvenir qu'il m'avait accordé son agrément d'une place
d'échevin de Paris pour Boulduc, apothicaire du roi, très distingué dans
son métier, et que j'aimais mois beaucoup de tout temps\,; que là-dessus
je l'avais demandée à Trudaine, qui me l'avait refusée avec la dernière
brutalité. Le régent s'en souvint très bien, mais insista toujours, et
moi aussi. L'altercation fut encore plus vive sur le maréchal de
Villeroy. Je lui représentai le double danger, dans un temps aussi
critique, de toucher pour la seconde fois à l'éducation du roi, après
l'avoir ôtée au duc du Maine, et quels affreux discours cela ferait
renouveler dans un public outré du désespoir de sa fortune pécuniaire et
parmi un peuple qu'on cherchait à soulever\,; à l'égard du prévôt des
marchands, que ce serait confirmer toute l'induction que les
malintentionnés voudraient tirer de son imprudence, et perdre toute
confiance et tout crédit à jamais que d'ôter à cette occasion un homme
de cette réputation d'honneur, de probité, de justice et d'amour pour la
droiture\,; qu'on ne manquerait pas d'en conclure qu'on avait voulu
jouer encore des gobelets et imposer au monde en brûlant de faux
papiers, et remettre les véritables dans le public\,; enfin, que c'était
une violence sans exemple d'ôter un prévôt des marchands avant
l'expiration de son temps, parce que celui-ci n'avait pu se prêter à une
si indigne supercherie.

M. le duc d'Orléans, résistant à toutes ces remontrances par la
persuasion du danger encore plus grand ou il s'exposait en laissant ces
deux hommes en place, me déclara que son parti était pris, et de me
faire gouverneur du roi, et Châteauneuf prévôt des marchands. Je
m'écriai que jamais je n'accepterais la place de gouverneur du roi, que
plus je lui étais attaché, à lui régent, moins j'en étais susceptible\,;
qu'il devait se souvenir qu'il en était convenu, lorsque, avant la mort
du roi, nous traitions cette matière\,; qu'il ne pouvait pas avoir
oublié tout ce que je lui en avais dit encore, il n'y avait pas si
longtemps, quand il avait voulu alors ce qu'il voulait de nouveau
aujourd'hui. Venant après à l'autre point, je le priai de considérer que
Châteauneuf était Savoyard de famille, né en Savoie, où il avait été
président de la cour supérieure de Chambéry, étranger par conséquent, et
bien que naturalisé, ci-devant ambassadeur à la Porte, en Portugal, en
Hollande, conseiller au parlement et maintenant conseiller d'État, il
était exclu par les lois municipales de la ville de Paris\,; que quelque
justice et bon et sage devoir qu'il eut fait à Nantes, à la tête de la
commission du conseil, cette commission était en gros triste et fâcheuse
pour servir de degré à revêtir les dépouilles d'un magistrat populaire,
cher par sa vertu, et {[}que c'était{]} offenser doublement Paris en le
lui ôtant, pour mettre un étranger à sa place, contre toutes les règles
et les lois de la ville et contre tout exemple. M. le duc d'Orléans,
demeurant ferme sur tous les points, et avec une vivacité qui m'effraya,
je me jetai à ses genoux, je les embrassai de mes deux bras, je le
conjurai par tout ce qui me vint de plus touchant, tandis qu'il
trépignait d'embarras pour me faire quitter prise\,; je protestai que je
ne me relèverais point qu'il ne m'eût donné sa parole de ne pas toucher
au maréchal de Villeroy et à Trudaine et de les laisser dans leurs
places. Enfin, il se laissa toucher ou arracher, et il me le promit à
plusieurs reprises, que j'exigeai avant de me vouloir relever. Quoique
j'abrège fort ici le récit de cette longue scène, j'en rapporte tout
l'essentiel. Nous travaillâmes ensuite assez longtemps et je m'en
retournai à Meudon, où je passais tous les étés en bonne compagnie et ne
venais à Paris que pour les affaires, sans y toucher.

Le lendemain, sans aller plus loin, le prince de Tingry entre autres
vint dîner à Meudon, qui d'abordée nous dit la nouvelle qui s'était
répandue comme il allait partir, que Trudaine était remercié et
Châteauneuf mis en sa place. Je cachai ma surprise autant qu'il me fut
possible et mon trouble secret sur le maréchal de Villeroy. Je compris
bien qu'il n'y avait rien encore à son égard, puisqu'on n'en parlait
point\,; mais un manquement de parole si prompt sur l'un m'inquiéta fort
pour l'autre, non par estime ni par amitié, non pour moi, qui étais bien
résolu à refuser très nettement et constamment la place de gouverneur du
roi, mais pour M. le duc d'Orléans et toutes les suites que je prévoyais
de l'ôter de cette place. Mais heureusement il n'en fut plus question
pour lors. Je ne sais si la parole que j'avais moins obtenue qu'arrachée
ne fut donnée que pour se dépêtrer de moi, ou si les mêmes qui lui
avaient fait prendre ces résolutions le poussèrent de nouveau depuis que
je l'eus quitté. Je croirais plutôt le premier, et que, si M. le duc
d'Orléans avait eu un successeur tout prêt pour le maréchal de Villeroy
comme il en avait un pour Trudaine, le maréchal eut sauté avec lui.
L'abbé Dubois aimait Châteauneuf depuis qu'il l'avait pratiqué en
Hollande, quoiqu'il y fut peu au gré des Anglais. Il était pauvre et
mangeur\,; ses ambassades l'avaient incommodé, malgré celle de la
Porte\,; il avait {[}des besoins{]} \footnote{Il y a ici dans le
  manuscrit une phrase tellement irrégulière qu'il a fallu la modifier.
  Saint-Simon a écrit\,: «\,Il avait besoin\,; la prévôté des marchands
  était propre à les remplir.\,»}\,; la prévôté des marchands était
propre à les remplir, et M. le duc d'Orléans avait toujours eu du goût
pour lui.

À quatre jours de là, il y eut conseil de régence et j'étais de mois
pour les placets. J'allai donc aux Tuileries un peu avant le conseil me
mettre dans la pièce qui précédait celle où on le tenait, derrière le
fauteuil du roi et la table des placets, entre deux maîtres des requêtes
pour les recevoir, c'est-à-dire pour les voir jeter sur la table et les
voir prendre après par les maîtres des requêtes et m'en rendre compte,
et après tous trois à M. le duc d'Orléans, après les avoir entièrement
dégrossis. L'un de ces deux maîtres des requêtes se trouva être Bignon,
mort jeune depuis conseiller d'État, fils du conseiller d'État intendant
de Paris, ami intime de M\textsuperscript{lle} Choin, duquel j'ai parlé
à l'occasion du mariage de M\textsuperscript{me} la duchesse de Berry,
ou on a vu ma liaison avec les Bignon et son ancienne cause. Il était
neveu de Bignon, aussi conseiller d'État, qui avait été prévôt des
marchands. Il me dit que son oncle ne se portait pas bien, mais qu'il ne
laisserait pas de m'aller chercher à Meudon s'il pouvait, qu'il avait à
me parler, qu'il en était même pressé, et qu'il l'avait chargé de savoir
de moi si et quand il me pourrait trouver chez moi à Paris. Je le priai
de dire à son oncle que je passerais chez lui au sortir du conseil avant
de retourner à Meudon. J'y allai donc. Des que Bignon me vit, il me dit
que, si Trudaine avait osé aller à Meudon, il y aurait couru me
témoigner toute sa reconnaissance\,; que, ne pouvant la contenir, il
l'avait chargé de m'assurer que je m'étais acquis en lui un serviteur à
jamais, et de là un torrent de louanges et de remerciements\,; moi, qui
de ma vie n'avais eu le moindre commerce avec Trudaine, et qui
n'imaginais pas ce que Bignon me voulait dire, je demeurai fort surpris.
Il me dit que je ne devais pas être si réservé, qu'ils savaient tout, et
de la me raconta de mot à mot toute la conversation entière que j'avais
eue avec M. le duc d'Orléans tête à tête, et que je viens de rapporter
en gros\,; alors mon étonnement fut extrême. Je niai d'abord tant que je
pus, mais je n'y gagnai rien. Le récit de tout fut exact, et pour
l'ordre jusque pour la plupart des termes\,; enfin, l'action de la fin,
tout me fut rendu par Bignon dans une si étrange justesse que je ne pus
malgré moi désavouer, et que je fus réduit à lui demander et à Trudaine
le secret pour toute reconnaissance. Ils me le gardèrent sur le maréchal
de Villeroy, dont Bignon sentit la conséquence\,; mais ils ne s'y purent
soumettre sur l'autre point\,; ils publièrent ce que Trudaine me devait.
Il me vint voir au bout de quelque temps et m'a cultivé toute sa vie. Il
faut dire, à l'honneur de son fils, que jusqu'à aujourd'hui il ne l'a
pas oublié. D'imaginer après comment cela s'est su\,: si un valet
relaissé entre deux portes où M. le duc d'Orléans lui-même aurait rendu
la conversation et avec cette longueur et cette justesse, c'est ce que
je n'ai jamais pu démêler. Je ne voulus pas en parler à M. le duc
d'Orléans, et je n'ai pu tirer de Bignon ni de Trudaine comment ils
l'avaient sue quoi que j'aie pu faire. Comme elle vint à eux, il n'est
pas surprenant qu'elle ne transpirât jusqu'au maréchal de Villeroy. Ce
que j'y gagnai fut rare\,: sa malveillance, qui ne put me pardonner
d'avoir pu remplir sa place, non pas même en faveur de ce que je l'avais
refusée et que je la lui avais fait conserver. Il avait déjà eu la même
crainte à mon égard, car ceci était une récidive\,; mais il n'en avait
eu que le soupçon et non la certitude, comme en celle-ci qui produisit
en lui ce sentiment bas à force d'orgueil et d'insolence, et si opposé a
celui d'un honnête homme. On le lui verra bien renouveler dans quelque
temps.

Ce n'était pas sans raison, comme on a déjà vu en bien des endroits,
mais raison toute récente, que le maréchal de Villeroy pesait rudement à
M. le duc d'Orléans dans la place de gouverneur du roi. Il n'y avait
rien qu'il n'eût mis en usage depuis la régence pour se rendre agréable
au parlement et au peuple. M. de Beaufort lui avait tourné la tête. Il
crut qu'avec la confiance que le feu roi lui avait marquée dans les
derniers temps de sa vie, ce qu'il pouvait penser attendre des troupes
qu'il avait si longtemps commandées, se trouvant doyen des maréchaux de
France, et le roi entre ses mains, le gouvernement de Lyon, où il était
de longue main maître absolu et son fils entièrement dans sa dépendance
capitaine des gardes du corps, c'était de quoi balancer l'autorité du
régent et faire en France le premier personnage. Par cette raison il
affecta de s'opposer à tous les édits bursaux\footnote{On appelait édits
  bursaux les édits qui établissaient de nouveaux impôts ou avaient pour
  but de tirer, par toute espèce de moyens, de l'argent des sujets.}, à
Law, aux divers arrangements de finances, à tout ce que le parlement
répugnait à enregistrer. Il rendit, tant qu'il put, la vie dure au duc
de Noailles tant que celui-ci eut les finances, quoique encore plus
indécent et bas valet du parlement que lui, quoiqu'il ne s'en mêlât que
bien superficiellement, ainsi que de toutes autres affaires. On a vu son
attachement au duc du Maine, le désespoir qu'il marqua quand l'éducation
lui fut ôtée, son engagement et ses frayeurs quand ce bâtard fut arrêté,
avec quelle bassesse et quelle importunité pour le roi il en faisait les
honneurs et le montrait aux magistrats à toutes heures qu'ils se
présentaient, comme il les distinguait sur qui que ce put être,
l'affectation avec laquelle il faisait voir le roi au peuple qui s'en
était pris de passion à proportion qu'il s'était pris de haine contre le
feu roi, et que les ennemis de M. le duc d'Orléans le décréditaient
parmi ce même peuple.

Ce fut aussi de ce dernier article que le maréchal se servit le plus
dangereusement. Il portait sur lui la clef d'une armoire ou il faisait
mettre le pain et le beurre de la Muette dont le roi mangeait, avec le
même soin et bien plus d'apparat que le garde des sceaux celle de la
cassette qui les renferme, et fit un jour une sortie d'éclat parce que
le roi en avait mangé d'autre, comme si tous les vivres dont il usait
nécessairement tous les jours, la viande, le potage, le poisson, les
assaisonnements, les légumes, tout ce qui sert aux fruits, l'eau, le vin
n'eussent pas été susceptibles des mêmes soupçons. Il fit une autre fois
le même vacarme pour les mouchoirs du roi, qu'il gardait aussi\,; comme
si ses chemises, ses draps, en un mot, tout son vêtement, ses gants,
n'eussent pas été aussi dangereux, que néanmoins il ne pouvait avoir
sous clef et les distribuer lui-même. C'était ainsi des superfluités
d'impudentes précautions vides de sens, pleines de vues les plus
intéressées et les plus noires, qui indignaient les honnêtes gens, qui
faisaient rire les autres, mais qui frappaient le peuple et les sots, et
qui avaient ce double effet de renouveler sans cesse les dits horribles
qu'on entretenait soigneusement contre M. le duc d'Orléans, et que
c'était aux soins et à la vigilance d'un gouverneur si fidèle et si
attaché qu'on était redevable de la conservation du roi et dont
dépendait sa vie. C'est ce qu'il voulait bien établir dans l'opinion du
parlement et du peuple, et peu à peu dans l'esprit du roi, et c'est à
quoi il s'en fallut bien peu qu'il ne parvînt parfaitement. C'est ce qui
lui attachait tellement ce peuple, qu'ayant eu tout nouvellement une
violente attaque de goutte qu'il avait toujours fort courtes, le peuple
en fut en émoi, et les halles lui députèrent les harengères qui
voulurent le voir. On peut juger comment ces ambassadrices furent
reçues. Il les combla de caresses et de présents, et il en fut comblé de
joie et d'audace, et c'était la ce qui avait ranimé dans M. le duc
d'Orléans la volonté et la résolution de l'ôter d'auprès du roi. Le
maréchal de Villeroy comptait encore s'attacher le roi et le public par
ces odieuses précautions de manière à se persuader que, quoi qu'il put
faire, jamais le régent n'oserait le chasser, et que, s'il
l'entreprenait, le roi, tout enfant qu'il était, l'empêcherait par ses
cris, dans la conviction qu'il lui inspirait que sa vie était attachée à
ses soins et que ce ne serait que pour se procurer les moyens d'y
pouvoir attenter qu'on l'éloignerait de sa personne. On verra en son
temps que ce raisonnement infernal n'était pas mal juste, et qu'il fut
fort près de lui réussir.

Le fils aîné d'Argenson, qui tout jeune avait eu sa place de conseiller
d'État, était intendant à Maubeuge, où il ne demeura pas longtemps. Le
cadet était lieutenant de police, il en fut remercié\,; Baudry eut cette
place et le jeune Argenson eut tôt après l'intendance de Tours, ou il
demeura peu. Les deux frères sont depuis parvenus au ministère, et
{[}à{]} être secrétaires d'État\footnote{Il a été question, t. XVII,
  p.~219, des deux fils du garde des sceaux appelés l'un le marquis
  d'Argenson, et l'autre le comte d'Argenson. Le premier fut ministre
  des affaires étrangères de 1744 à 1747, et le second ministre de la
  guerre de 1743 à 1757.}.

M. le duc d'Orléans reçut les remontrances du parlement le mieux du
monde. Elles ne furent que générales, sur la situation des finances\,;
il les renvoya au chancelier pour voir avec lui ce qu'il serait de plus
à propos à faire.

Il y eut le 5 juillet, un arrêt du conseil, portant défense d'avoir des
pierreries, d'en garder chez soi, ni d'en vendre qu'aux étrangers. On
peut juger du bruit qui en résulta. Cet arrêt\footnote{Le pluriel est
  dans le manuscrit et s'explique par le commencement de la phrase où
  Saint-Simon parle de plusieurs arrêts.}, enté sur tant d'autres,
allaient trop visiblement tous à s'emparer de tout l'argent pour du
papier décrié, et auquel on ne pouvait plus avoir la moindre confiance.
En vain M. le duc d'Orléans, M. le Duc, et M\textsuperscript{me} sa
mère, voulurent-ils persuader qu'ils en donnaient l'exemple, en se
défaisant de leurs immenses pierreries dans les pays étrangers\,; en
vain y en envoyèrent-ils en effet, mais seulement en voyage\,; qui que
ce soit ne fut la dupe, et qui ne cachât bien soigneusement les siennes,
qui en avait\footnote{Voici la construction directe de cette phrase\,:
  «\,Quiconque en avait, cacha soigneusement les siennes, ce qui se
  put.\,»}, ce qui se put par le petit volume, bien plus aisément que
l'or et l'argent. Cette éclipse de pierreries ne fut pas de longue
durée.

Stairs enfin prit congé après avoir régné ici sans voile avec une
domination absolue, dont le commerce et la marine de France et d'Espagne
se ressentiront longtemps, et même l'Angleterre, par la supériorité que
son roi a acquise sur la nation, moyennant les subsides immenses qu'il à
tirés de nous, qui l'ont mis en état de se rendre le maître de ses
parlements, et de n'y trouver plus de barrière à ses volontés, grâces à
l'ambition de l'abbé Dubois, à l'aveuglement de Canillac, à la perfidie
politique personnelle du duc de Noailles, et à l'entraînement de M. le
duc d'Orléans. Stairs se pressa de passer la mer des que le chevalier
Sutton, son successeur, fut arrivé, pour trouver le roi d'Angleterre,
qui s'en allait dans ses États d'Allemagne. Jamais l'audace,
l'insolence, l'impudence ne furent portées en aucun pays au point ou cet
ambassadeur les porta, ni avec tant de succès\,; malheureusement il ne
savait que trop à qui il avait affaire. Encore une fois, voila le fruit
de se livrer à un seul, à un seul de l'espèce de l'abbé Dubois encore,
enfin à un premier ministre qui veut être cardinal.

Le nouveau prévôt des marchands continua à brûler publiquement à l'hôtel
de ville les actions et les billets de banque, jusqu'à la réduction
qu'on avait résolue.

Tandis que les députés du parlement travaillaient souvent chez le
chancelier sans conclure, on projeta un édit pour rendre la compagnie
des Indes compagnie de commerce, laquelle s'obligeait, ce moyennant, à
rembourser dans un an, pour six cents millions de billets de banque, en
payant cinquante millions par mois\,: telle fut la dernière ressource de
Law et de son système. Aux tours de passe-passe du Mississipi il avait
fallu chercher à substituer quelque chose de réel, surtout depuis
l'événement de l'arrêt du 22 mai dernier, si célèbre et si funeste au
papier. On voulut donc substituer aux chimères une compagnie réelle des
Indes, et ce fut ce nom et cette chose qui succéda, et qui prit la place
de ce qui ne se connaissait auparavant que sous le nom de Mississipi. On
avait eu beau donner à cette compagnie la ferme du tabac et quantité
d'autres revenus immenses, ce n'était rien pour faire face au papier
répandu dans le public, quelques soins qu'on eût pris de le diminuer à
tous hasards, à toutes restes\footnote{Saint-Simon à déjà employé cette
  locution dans le sens de à toutes forces. On a donc eu tort de
  remplacer, dans les précédentes éditions, ces mots par à toutes
  ruines.}.

Il fallut chercher d'autres expédients. Il ne s'en trouva point que de
rendre cette compagnie de commerce\,; c'était sous un nom plus doux,
mais obscur et simple, lui attribuer le commerce exclusif en entier. On
peut juger comment une telle résolution put être reçue dans le public,
poussé à bout de la défense sévère, sous de grandes peines, d'avoir plus
de cinq cents livres en argent chez soi, d'y être visité et fouillé
partout, et de ne pouvoir user que de billets de banque pour payer
journellement les choses les plus médiocres et les plus nécessaires à la
vie. Aussi opéra-t-elle deux choses\,: une aigreur qui s'aigrit
tellement par la difficulté de toucher son propre argent, jour par jour,
pour sa subsistance journalière, que ce fut merveille comment l'émeute
s'apaisa et que tout Paris ne se révoltât pas tout à la fois\,; l'autre,
que le parlement, prenant pied sur cette émotion publique, tînt ferme
jusqu'au bout contre l'enregistrement de l'édit. Le 15 juillet, le
chancelier montra chez lui le projet de l'édit aux députés du parlement,
qui furent chez lui jusqu'à neuf heures du soir sans s'être laissé
persuader. Le lendemain 16, le projet de l'édit fut montré au conseil de
régence. M. le duc d'Orléans, soutenu de M. le Duc, y parla bien, parce
qu'il ne pouvait parler mal, même dans les plus mauvaises thèses.
Personne ne dit mot, et on ploya les épaules. Il fut résolu de la sorte
d'envoyer le lendemain, 17 juillet, l'édit au parlement.

Ce même jour 17, au matin, il y eut une telle foule à la banque et dans
les rues voisines pour avoir chacun de quoi aller au marché, qu'il y eut
dix ou douze personnes étouffées. On porta tumultuairement trois de ces
corps morts à la porte du Palais-Royal, ou le peuple voulait entrer à
grands cris. On y fit promptement marcher un détachement des compagnies
de la garde du roi, des Tuileries. La Vrillière et Le Blanc haranguèrent
séparément ce peuple. Le lieutenant de police y accourut\,; on fit venir
des brigades du guet\footnote{Le guet était la garde qui veillait à la
  sûreté de Paris. On distinguait, au XVIIIe siècle, le guet à cheval et
  le guet à pied\,: le premier se composait de cent soixante cavaliers,
  et le second de quatre cent soixante-douze fantassins.}. On fit après
emporter les corps morts, et par douceur et cajoleries on vint enfin à
bout de renvoyer le peuple, et le détachement de la garde du roi s'en
retourna aux Tuileries. Sur les dix heures du matin, que tout cela
finissait, Law s'avisa d'aller au Palais-Royal\,; il reçut force
imprécations par les rues, M. le duc d'Orléans ne jugea pas à propos de
le laisser sortir du Palais-Royal, ou, deux jours après, il lui donna un
logement. Il renvoya son carrosse, dont les glaces furent cassées à
coups de pierres. Son logis en fut attaqué aussi avec grand fracas de
vitres. Tout cela fut su si tard dans notre quartier des Jacobins de la
rue Saint-Dominique\footnote{Saint-Simon a déjà indiqué, dans plusieurs
  passages de ses Mémoires, qu'il demeurait rue Saint-Dominique, près
  des Jacobins (noviciat des dominicains réformés, aujourd'hui
  Saint-Thomas d'Aquin et Musée d'artillerie).}, qu'il n'y avait plus
apparence de rien quand j'arrivai au Palais-Royal, où M. le duc
d'Orléans, en très courte compagnie, était fort tranquille et montrait
que ce n'était pas lui plaire que de ne l'être pas. Ainsi je n'y fus pas
longtemps, n'y ayant rien à faire ni à dire. Ce même matin l'édit fut
porté au parlement\,; il refusa de l'enregistrer et envoya les gens du
roi à M. le duc d'Orléans pour lui rendre compte de leurs raisons,
lequel demeura fort piqué de ce refus. On publia le lendemain par la
ville une ordonnance du roi, portant défense au peuple de s'assembler,
sous de grandes peines, et qu'à cause des inconvénients arrivés la
veille à la banque, on n'y donnerait point d'argent et qu'elle serait
fermée jusqu'à nouvel ordre. On fut plus heureux que sage\,; car, de
quoi vivre en attendant\,? et si rien ne branla, ce qui marque bien la
bonté et l'obéissance de ce peuple qu'on mettait à tant et de si
étranges épreuves. On fit néanmoins venir des troupes auprès de
Charenton, qui étaient à travailler au canal de Montargis, quelques
régiments de cavalerie et de dragons à Saint-Denis, et le régiment du
roi sur les hauteurs de Chaillot. On envoya de l'argent à Gonesse, pour
faire venir les boulangers comme à l'ordinaire, de peur de leur refus de
prendre des billets, comme faisaient presque tous les marchands et les
ouvriers de Paris, qui ne voulaient plus recevoir de papier. Le régiment
des gardes eut ordre de se tenir prêt, et les mousquetaires de ne
s'éloigner point de leurs deux hôtels et de tenir leurs chevaux bridés.

Ce même jour du refus du parlement d'enregistrer l'édit, je fus mandé au
Palais-Royal sur les cinq heures après midi. M. le duc d'Orléans
m'apprit la plupart des choses faites ou résolues qui viennent d'être
rapportées, se plaignit fort de la mollesse du chancelier avec le
parlement et dans les conférences chez lui avec les députés de cette
compagnie\,; et de la force reproches de l'embarras où je le mettais par
mon opiniâtreté à ne vouloir point des sceaux. Je lui répondis qu'avec
sa permission je pensais tout autrement. «\,Comment, m'interrompit-il
vivement, me ferez-vous accroire que vous auriez été aussi mou que le
chancelier, et que vous ne leur eussiez pas fait peur\,? --- Ce n'est
pas cela, repris-je\,; mais vous n'ignorez pas à quel point je suis avec
le premier président et que je ne suis pas agréable au parlement depuis
la belle affaire du bonnet, où votre mollesse et votre peur du
parlement, vous qui aujourd'hui la reprochez aux autres, nous à mis dans
la fange, et vous dans le bourbier, par l'audace et l'intérêt du
parlement, du premier président et de leur cabale, après qu'ils ont eu
reconnu par là, dès l'entrée de votre régence, à qui ils avaient affaire
et comment vous manier\,; aussi s'y sont-ils donné ample carrière\,;
vous les aviez abattus par le lit de justice des Tuileries, vous ne
l'avez pas soutenu\,; cette conduite leur à remis les esprits, et la
cabale tremblante à repris force et vigueur. Cette courte récapitulation
ne serait pas inutile, si à la fin vous en pouviez et saviez profiter.
Mais revenons à moi et aux sceaux. Persuadez-vous, monsieur, que, si ces
gens-là se montrent si revêches à un magistrat nourri dans leur sein,
qui est leur chef et leur supérieur naturel, qu'ils aiment et dont ils
se savent aimés, persuadez-vous, dis-je, qu'ils se seraient montrés
encore plus intraitables avec un supérieur précaire, regardé par eux
comme un supérieur de violence, sans qualité pour l'être, revêtu d'une
dignité qu'ils haïssent et qu'ils persécutent avec la dernière audace et
la plus impunie\,; homme d'épée, qui est leur jalousie et leur mépris
tout à la fois\,; et homme que personnellement ils haïssent et dont ils
se croient haïs. Ils auraient pris pour une insulte d'avoir à traiter
avec moi\,; leur cabale aurait répandu cent mauvais discours\,; les
députés, par leurs propos, auraient exprès excité les miens, et tout le
monde vous aurait reproché et la singularité d'un garde des sceaux
d'épée, et le mauvais choix d'une manière d'ennemi pour travailler à une
conciliation. Voila ce qui en serait résulté, c'est-à-dire un bien plus
grand embarras pour vous, et un très désagréable pour moi. Ainsi, n'ayez
nul regret à mon refus. Tenez-le, au contraire, pour un avantage, qui
vous est clairement démontré par l'occasion présente, et ne regrettez
que de n'avoir pas eu sous la main un magistrat estimé royaliste et non
parlementaire à faire garde des sceaux\,; mais cela ne s'étant pu
trouver, vous avez fait la seule chose naturelle à faire, en rappelant
et rendant les sceaux au chancelier, et à un homme de ce mérite et de
cette réputation, puisque, pour d'autres raisons, vous les avez voulu
ôter à celui qui les avait, et qui était votre vrai homme tel qu'il vous
le fallait dans les circonstances présentes, et, pour le bien dire, au
vol que le parlement à pris et veut rendre de plus en plus, l'homme pour
qui les sceaux étaient le plus faits pendant une régence\,; mais il faut
partir d'ou on est\,: avez-vous quelque plan formé pour sortir bien du
détroit ou vous êtes\,? Il faut laisser le passé, et voir ce qu'il y a à
faire.\,»

M. le duc d'Orléans demeura muet sur les sceaux, se rabattit encore sur
le chancelier, et me dit qu'il ne voyait autre chose à faire que
d'envoyer le parlement à Blois. Je lui dis que cela était bon faute de
mieux, non que j'imaginasse ce mieux, mais que je voyais avec peine que,
par cet exil, le parlement était puni, mais n'était ni ramené ni dompté.
Le régent en convint, mais il espéra que ces magistrats, accoutumés à
Paris dans leurs maisons, leurs familles, leurs amis, se lasseraient
bientôt d'en être séparés, se dégoûteraient de n'être plus qu'entre eux,
s'ennuieraient encore plus de la dépense, de l'éloignement de chez eux
et de la diminution du sac par celle des affaires qui suivrait
nécessairement leur transplantation. Cela était vrai, et comme on ne
pouvait autre chose, il fallait bien s'en contenter. Je lui proposai
ensuite de bien examiner tout ce qui pouvait arriver, les remèdes
prompts et sûrs à y apporter, parce qu'il valait sans comparaison mieux
ne rien entreprendre que demeurer court et avoir le démenti de ce qu'on
aurait entrepris, qui serait la perte radicale de toute l'autorité. Il
me dit qu'il y avait déjà pensé, qu'il y réfléchirait encore, qu'il
comptait tenir un petit conseil le lendemain au Palais-Royal, où il
voulait que j'assistasse, où tout serait discuté. Il se mit après sur
les maréchaux de Villeroy, Villars, Huxelles et sur quelques autres
moins marqués, et ces propos terminèrent cette conversation.

J'allai donc le lendemain jeudi 18 juillet, sur les quatre heures, au
Palais-Royal. Ce conseil fut tenu dans une pièce du grand appartement,
la plus proche du grand salon, avec M. le Duc, le duc de La Force, le
chancelier, l'abbé Dubois, Canillac, La Vrillière et Le Blanc. On était
assis vers une des fenêtres, presque sans ordre, et M. le duc d'Orléans
sur un tabouret comme nous et sans table. Comme on commençait à
s'asseoir, M. le duc d'Orléans dit qu'il allait voir si quelqu'un
n'était point là auprès, qu'il ne serait pas fâché de faire venir, et
l'alla chercher. Ce quelqu'un était Silly, de la catastrophe duquel j'ai
parlé ailleurs, d'avance ami intime de Law, de Lassai, de
M\textsuperscript{me} la Duchesse, qui le fit chevalier de l'ordre
depuis, et qui était fort intéressé avec eux. Il entra donc à la suite
de M. le duc d'Orléans qui l'avait relaissé dans son petit appartement
d'hiver, et vint jusque tout contre nous. Je ne sais, et j'ai depuis
négligé d'apprendre ce qu'il avait contre Le Blanc. Mais dès qu'il
l'avisa\,: «\,Monseigneur, dit-il en haussant la voix à M. le duc
d'Orléans, je vois ici un homme, en regardant Le Blanc, devant qui on ne
peut parler, et avec lequel Votre Altesse Royale trouvera bon que je ne
demeure pas. Elle m'avait fait la grâce de me dire que je ne le
trouverais pas ici.\,» Nôtre surprise à tous fut grande, et Le Blanc
fort étonné. «\,Bon\,! bon\,! répondit M. le duc d'Orléans, qu'est-ce
que cela fait\,? Demeurez, demeurez. --- Non pas, s'il vous plaît,
monseigneur,\,» reprit Silly, et s'en alla. Cette incartade nous fit
tous regarder l'un l'autre. L'abbé Dubois courut après, le prit par le
bras pour le ramener. Comme la pièce est fort grande, nous voyions Silly
secouer Dubois et continuer son chemin, enfin passer la porte, et Dubois
après lui. «\,Mais quelle folie\,!» disait M. le duc d'Orléans, qui
avait l'air embarrassé, et qui que ce soit qui dît un mot, excepté Le
Blanc, qui offrit à M. le duc d'Orléans de se retirer, qui ne le voulut,
pas à la fin M. le duc d'Orléans alla chercher Silly\,; son absence dura
près d'un quart d'heure apparemment à catéchiser Silly, qui méritait
mieux pour cette insolence d'être jeté par les fenêtres, comme lui-même
s'y jeta depuis. Enfin M. le duc d'Orléans rentra, suivi de Silly et de
l'abbé Dubois.

Pendant l'absence personne n'avait presque rien dit que s'étonner un peu
de l'incartade et de la bonté de M. le duc d'Orléans. M. le Duc ne
proféra pas un mot. Silly se mit donc dans le cercle au plus loin qu'il
put de Le Blanc, et en s'asseyant combla l'impudence par dire à M. le
duc d'Orléans que c'était par pure obéissance, mais qu'il ne dirait
rien, parce qu'il ne le pouvait devant M. Le Blanc. M. le duc d'Orléans
ne lui répondit rien, et tout de suite ouvrit la conférence par
expliquer ce qui la lui avait fait assembler par un récit fort net de
l'état des choses, de la nécessité de prendre promptement un parti, de
celui qui paraissait le seul à pouvoir être pris, et finit par ordonner
au chancelier de rendre compte à l'assemblée de tout ce qui s'était
passé chez lui avec les cinq députés du parlement susdits. Le chancelier
en fit le rapport assez étendu avec l'embarras d'un arrivant d'exil qui
n'y veut pas retourner, et d'un protecteur secret, mais de cœur et de
toute son âme, du parlement qu'il voyait bien ne pouvoir sauver. Ce ne
fut donc qu'en balbutiant qu'il conclut la fin de son discours\,: que
les conjonctures forcées où on se trouvait jetaient dans une nécessité
triste et fâcheuse, sur quoi il n'avait qu'à se rapporter à la prudence
et à la bonté de Son Altesse Royale. Tous opinèrent à l'avis de M. le
duc d'Orléans qui s'était ouvert sur envoyer le parlement à Blois. M. le
Duc, le duc de La Force et l'abbé Dubois parlèrent fortement\,; les
autres, quoique de même avis, se mesurèrent davantage et furent courts.
Je crus ne devoir dire que deux mots sur une affaire résolue qui
regardait le parlement. Silly tint parole, et ne fit qu'une inclination
profonde quand ce fut à lui à opiner\,; de là on parla sommairement des
précautions à prendre pour être sûrement obéi, puis on se leva. Alors le
chancelier s'approcha de M. le duc d'Orléans et lui parla quelque temps
en particulier. L'abbé Dubois s'y joignit sur la fin, et cependant
chacun s'écoulait. M. le Duc fut appelé, enfin je sus qu'il s'agissait
de Pontoise au lieu de Blois, et cela fut emporté le lendemain matin.
Ainsi le châtiment devint ridicule et ne fit que montrer la faiblesse du
gouvernement, et encourager le parlement qui s'en moqua. Néanmoins ce
qui s'était passé en ce petit conseil demeura tellement secret, que le
parlement n'eut pas la plus légère connaissance de ce qui y fut résolu
que par l'exécution.

Le dimanche 21 juillet, des escouades du régiment des gardes avec des
officiers à leur tête se saisirent à quatre heures du matin de toutes
les portes du palais. Des mousquetaires des deux compagnies avec des
officiers s'emparèrent en même temps des portes de la grand'chambre,
tandis que d'autres investirent la maison du premier président qui eut
grand'peur pendant la première heure, et cependant d'autres
mousquetaires des deux compagnies aillèrent séparément quatre à quatre
chez tous les officiers du parlement leur rendre en main propre l'ordre
du roi de se rendre à Pontoise dans deux fois vingt-quatre heures. Tout
se passa poliment de part et d'autre, en sorte qu'il n'y eut pas la
moindre plainte\,; plusieurs obéirent dès le même jour et s'en aillèrent
à Pontoise. Le soir assez tard, M. le duc d'Orléans fit porter au
procureur général cent mille francs en argent, et autant en billets de
banque de cent livres et de dix livres pour en donner à ceux qui en
auraient besoin pour le voyage, mais non en don. Le premier président
fut plus effronté et plus heureux\,: il fit tant de promesses, de
bassesses, employa tant de fripons pour abuser de la faiblesse et de la
facilité de M. le duc d'Orléans, dont il sut bien se moquer, que ce
voyage lui valut plus de cent mille écus, que le pauvre prince lui fit
compter sous la cheminée à deux ou trois diverses reprises, et trouva
bon que le duc de Bouillon lui prêtât sa maison de Pontoise toute
meublée, dont le jardin est admirable et immense au bord de la rivière,
chef-d'oeuvre en son genre, qui avait fait les délices du cardinal de
Bouillon, et qui fut peut-être la seule chose qu'il regretta en France.
Avec de si beaux secours, le premier président, mal avec sa compagnie
qui le méprisait ouvertement depuis quelque temps, se raccommoda
parfaitement avec elle. Il y tint tous les jours table ouverte pour tout
le parlement qu'il mit sur le pied d'y venir tous les jours en foule, en
sorte qu'il y eut toujours plusieurs tables servies également
délicatement et splendidement, et envoyait, à ceux qui voulaient envoyer
chercher chez lui, tout ce qu'ils pouvaient désirer de vin, de liqueurs
et de toutes choses. Les rafraîchissements et les fruits de toutes
sortes étaient servis abondamment tant que les après-dînées duraient, et
il y avait force petits chariots à un et à deux chevaux toujours prêts
pour les dames et les vieillards qui voulaient se promener, et force
tables de jeu dans les appartements jusqu'au souper. Mesmes, sa, soeur
et ses filles faisaient les honneurs, et lui, avec cet air d'aisance, de
magnificence, de politesse, de prévenance et d'attention, en homme qui
saisissait l'occasion de regagner ainsi ce qu'il avait perdu, en quoi il
réussit pleinement\,; mais ce fut aux doubles dépens du régent, de
l'argent duquel il fournissait à cette prodigieuse dépense, et se
moquait encore de lui avec messieurs du parlement, tant en brocards
couverts ou à l'oreille, qu'en trahissant une confiance si chèrement et
si indiscrètement achetée, dont il leur faisait sa cour, tant en la leur
sacrifiant en dérision qu'en s'amalgamant à eux, à tenir ferme, et
faisant tomber le régent dans tous leurs panneaux par la perfidie du
premier président, à qui M. le duc d'Orléans croyait finement se pouvoir
fier à force d'argent, et de cacher cette intelligence dont le secret
servait à ce scélérat de couverture aux insolentes plaisanteries qu'il
faisait du régent et du gouvernement avec ses confrères, qui ne
pouvaient pas toutes échapper à M. le duc d'Orléans, et que le premier
président et ses traîtres de protecteurs donnaient au régent comme
nécessaires à cacher leur intelligence. Lui vouloir ouvrir les yeux sur
une conduite si grossière eût été temps perdu, de sorte que je ne lui en
dis pas une parole. Je lui aurais été suspect plus que personne sur le
premier président qui se joua de lui de la sorte, et qui, sans le
moindre adoucissement dans la roideur du parlement, le fit revenir à
Paris quand, pour son intérêt personnel, et après s'être pleinement
rétabli avec sa compagnie, et mieux avec elle qu'il n'y eut jamais été,
et maître de la tourner à son gré, il jugea à propos de procurer ce
retour. Quelques principaux magistrats du parlement firent demander à
voir M. le duc d'Orléans avant Paris, et en furent refusés.

Le parlement avait refusé l'enregistrement de l'édit de sa translation à
Pontoise. On lui en envoya de nouveau une déclaration dans laquelle on
osa avoir le courage de laisser échapper quelques expressions qui ne
devaient pas lui plaire. Néanmoins il l'enregistra, mais avec la
dérision la plus marquée et la plus à découvert. Comme cet
enregistrement ne contient pas un seul mot qui ne la porté avec le ton
et les termes du plus parfait mépris et de la résolution la plus ferme
de ne reculer pas d'une ligne, j'ai cru devoir l'insérer ici.

«\,Registrées, ouï ce requérant le procureur général du roi, pour
continuer par la cour ses fonctions ordinaires, et être rendu au roi le
service accoutumé tel qu'il à été rendu jusqu'à présent, avec la même
attention et le même attachement pour le bien de l'État et du public
qu'elle à eu dans tous les temps\,; continuant ladite cour de donner au
roi les marques de la même fidélité qu'elle a eue pour les rois ses
prédécesseurs et pour ledit seigneur roi, depuis son avènement à la
couronne jusqu'à ce jour, dont elle ne se départira jamais. Et sera
ledit seigneur roi très humblement supplié de faire attention à tous les
inconvénients et conséquences de la présente déclaration, et de recevoir
le présent enregistrement comme une nouvelle preuve de sa profonde
soumission. Et seront copies collationnées de la présente déclaration et
du présent enregistrement envoyées aux bailliages et sénéchaussées du
ressort, pour y être lues, publiées et enregistrées. Enjoint aux
substituts du procureur général du roi d'y tenir la main et d'en
certifier la cour dans un mois, suivant l'arrêt de ce jour à Pontoise,
en parlement y séant, le 27 juillet 1720. Signé Gilbert.\,»

Les paroles et le tour de cet arrêt sont tellement expressifs et
frappants, que ce serait les affaiblir qu'en faire le commentaire. Le
régent n'en parut pas touché ni y faire la moindre attention. Je suivis
la résolution que j'avais prise, je ne pris pas la peine de lui en dire
un mot. Tout se soutint en conséquence à Pontoise. Les avocats, de
concert avec le parlement, ne feignirent point de répandre qu'ils
étaient gens libres, qu'ils profiteraient de cette liberté pour aller à
la campagne se reposer, au lieu d'aller dépenser leur argent à Pontoise,
ou ils seraient mal logés et fort mal à leur aise. En effet aucun bon
avocat n'y mit le pied\,; il n'y eut que quelques jeunes d'entre eux et
en fort petit nombre, destinés à monter cette garde de fatigue\,; parce
qu'encore que le parlement eut résolu de ne rien faire de sérieux, il ne
voulut pas toutefois, après avoir enregistré sa translation, n'entrer
point du tout, et pour entrer il fallait bien quelque pâture légère
comme quelque défaut, quelque appointé\footnote{Appointer un procès,
  c'était décider que les parties produiraient leurs pièces, sur le vu
  desquelles il serait jugé. On avait recours à ce moyen quand une
  affaire paraissait trop compliquée pour être jugée immédiatement, ou
  lorsqu'on voulait l'ajourner indéfiniment.} à mettre et autres
bagatelles pareilles qui les tenaient assemblés une demi-heure, rarement
une heure et souvent ils n'entraient pas. Ils en riaient entre eux, et
malheur à qui avait des procès\,; quelque peu de présidents riches
tinrent quelquefois des tables. En un mot on n'y songea qu'à se
divertir, surtout à n'y rien faire, à le montrer même et à s'y moquer du
régent et du gouvernement.

Cette translation fut suivie de différentes opérations de finance et de
plusieurs changements dans les emplois des finances. Des Forts en eut le
principal, il exerça le contrôle général en toute autorité sans en avoir
le nom. Je n'entrerai point, selon ma coutume, dans tout ce nouveau
détail de finances. Leur désordre n'arrêta point les étranges
libéralités, ou pour mieux dire facilités de M. le duc d'Orléans à
l'égard de gens ou sans mérite ou sans besoin, et de pas un desquels il
ne pouvait se soucier\,; il donna à M\textsuperscript{me} la
grande-duchesse une augmentation de quarante mille livres de ses
pensions, une de huit mille livres à Trudaine, une de neuf mille livres
à Châteauneuf, qu'il venait de faire prévôt des marchands, une de huit
mille livres à Bontems, premier valet de chambre du roi, une de six
mille livres à la maréchale de Montesquiou, une de trois mille livres à
Foucault, président du parlement de Toulouse, une de neuf mille livres à
la veuve du duc d'Albemarle, remariée secrètement au fils de Mahoni,
dont il à été fort parlé ici à propos de l'affaire de Crémone, où le
maréchal de Villeroy fut pris. Cette femme était fille de Lussan, dont
il a été fait aussi mention ici à propos du procès que me fit sa mère,
qui me brouilla pour toujours avec M. le Duc et M\textsuperscript{me} la
Duchesse.

L'agiotage public était toujours établi dans la place de Vendôme, où on
l'avait transporté de la rue Quincampoix. Ce Mississipi avait tenté tout
le monde\,: c'était à qui en remplirait ses poches à millions par M. le
duc d'Orléans et par Law. Les princes et les princesses du sang en
avaient donné les plus merveilleux exemples. On ne comptait de gens à
portée d'en avoir tant qu'ils en auraient voulu, que le chancelier, les
maréchaux de Villeroy et de Villars, et les ducs de Villeroy, de La
Rochefoucauld et moi qui eussions constamment refusé d'en recevoir quoi
que ce fut. Ces deux maréchaux et La Rochefoucauld étaient frondeurs de
projet et d'effet, et le duc de Villeroy suivait le bateau de sel. Ils
étaient liés ensemble pour leur fronde, pensant mieux faire leurs
affaires par là, et devenir de plus des personnages avec qui le
gouvernement serait forcé de compter. Ce n'était pas que La
Rochefoucauld eut par soi, ni par sa charge, de quoi arriver à ce but,
mais riche à millions, fier de son grand-père dans la dernière minorité,
plus étroitement et de tout temps uni au duc de Villeroy, que parleur
proximité de beaux-frères, il suivait les Villeroy en tout\,; et cet air
de désintéressement et d'éloignement du régent, sans toutefois cesser
d'être devant lui ventre à terre, leur donnait dans le parlement et
auprès du peuple, les plus vastes espérances.

Un jour que le maréchal de Villars traversait la place de Vendôme dans
un beau carrosse, chargé de pages et de laquais, où la foule d'agioteurs
avait peine à faire place, le maréchal se mit à crier par la portière
contre l'agio, et avec son air de fanfaron à haranguer le monde sur la
honte que c'était. Jusque-là on le laissa dire, mais s'étant avisé
d'ajouter que pour lui il en avait les mains nettes, qu'il n'en avait
jamais voulu\,; il s'éleva une voix forte qui s'écria\,: «\,Eh\,! les
sauvegardes\footnote{Les sauvegardes étaient des soldats envoyés par un
  général pour mettre une maison ou une terre à l'abri du pillage.}\,!»
Toute la foule répéta ce mot, dont le maréchal honteux et confondu,
malgré son audace ordinaire, s'enfonça dans son carrosse, et acheva de
traverser la place au petit pas, au bruit de cette huée qui le suivit
encore au delà, et divertit Paris plusieurs jours à ses dépens sans être
plaint de personne.

À la fin on trouva que cet agiotage embarrassait trop la place de
Vendôme et le passage public\,; on le transporta dans le vaste jardin de
l'hôtel de Soissons\footnote{L'hôtel de Soissons a été démoli en 1750.
  L'emplacement est aujourd'hui occupé par la halle au blé.}. C'était en
effet son lieu propre. M. et M\textsuperscript{me} de Carignan qui
occupaient l'hôtel de Soissons à qui il appartenait, tiraient à toutes
mains de toutes parts. Des profits de cent francs, ce qu'on aurait peine
à croire s'il n'était très reconnu, ne leur semblaient pas au-dessous
d'eux, je ne dis pas pour leurs domestiques, mais pour eux-mêmes, et des
gains de millions dont ils avaient tiré plusieurs de ce Mississipi, sans
en compter d'autres pris d'ailleurs, ne leur paraissaient pas au-dessus
de leur mérite, qu'en effet ils avaient porté au dernier comble dans la
science d'acquérir avec toutes les bassesses les plus rampantes, les
plus viles, les plus continuelles. Ils gagnèrent en cette translation un
grand louage\footnote{L'avocat Barbier (Journal, août 1720) donne des
  détails sur le grand louage que le prince de Carignan tira de ses
  jardins\,: «\,Tout autour {[}de l'hôtel de Soissons{]}, on a fait des
  loges, toutes égales, propres et peintes, ayant une porte et une
  croisée avec le numéro au-dessus de la porte. C'est de bois\,; il y en
  à cent trente-huit avec deux entrées, l'une dans la rue de Grenelle,
  et l'autre dans la rue des Deux-Écus. Des Suisses de la livrée du roi
  aux portes, et des corps de garde avec une ordonnance du roi pour ne
  laisser entrer ni artisans, ni laquais, ni ouvriers. Ce sont deux
  personnes qui ont entrepris cela, peut-être au profit de la banque.
  Ils donnent cent cinquante mille livres à M. le prince de Carignan\,;
  il leur en coûte encore cent mille livres pour l'accommodement, et
  chaque loge est louée cinq cents livres par mois.\,»}, de nouvelles
facilités et de nouveaux tributs. Law, leur grand ami, qui avait logé
quelques jours au Palais-Royal, était retourné chez lui où il recevait
force visites. Le roi alla voir à diverses reprises les troupes qu'on
avait fait approcher de Paris, après quoi elles furent renvoyées. Celles
qui avaient formé un petit camp à Charenton retournèrent au leur de
Montargis travailler au canal qu'on y faisait.

Law avait obtenu depuis quelque temps par des raisons de commerce que
Marseille fut port franc. Cette franchise qui y fit abonder les
vaisseaux, surtout les bâtiments du Levant, y apporta la peste faute de
précaution, qui dura longtemps, et qui désola Marseille, la Provence, et
les provinces les plus voisines \footnote{La peste sévit à Marseille et
  dans toute la Provence pendant les années 1720 et 1721. Voy. Lemontey,
  Histoire de la Régence, t. I, p.~360 et suiv.}. Les soins et les
précautions qu'on prit la restreignirent autant qu'il fut possible, mais
ne l'empêchèrent pas de durer fort longtemps, et de faire d'affreux
désordres. Ce sont des détails si connus qu'on se dispensera d'y entrer
ici.

\hypertarget{chapitre-iii.}{%
\chapter{CHAPITRE III.}\label{chapitre-iii.}}

1720

~

{\textsc{Déclaration pour recevoir la constitution Unigenitus lue au
conseil de régence sans prendre là-dessus les avis de personne.}}
{\textsc{- Mort, fortune et caractère du chevalier de Broglio.}}
{\textsc{- Comte de Saxe entre au service de France\,; fait presque
aussitôt maréchal de camp.}} {\textsc{- Mariage d'Alincourt et de
M\textsuperscript{lle} de Boufflers.}} {\textsc{- Cellamare, ou le duc
de Giovenazzo, disgracié depuis son retour, rappelé à la cour d'Espagne
et bien traité.}} {\textsc{- La place du parlement absent laissée vide
par les autres cours à la procession de l'Assomption.}} {\textsc{- Le
parlement refuse d'enregistrer la déclaration en faveur de la
constitution Unigenitus.}} {\textsc{- Le régent la porte au grand
conseil, y fait trouver les princes du sang, les ducs et pairs et
maréchaux de France, me prie de ne m'y point trouver, et l'y fait
enregistrer à peine.}} {\textsc{- Nullité de cet enregistrement.}}
{\textsc{- Mort et caractère de La Brue, évêque de Mirepoix\,; de
l'évêque-comte de Châlon, frère du cardinal de Noailles\,; de Heinsius,
pensionnaire de Hollande.}} {\textsc{- Hoornbeck, pensionnaire de
Rotterdam, fait pensionnaire de Hollande.}} {\textsc{- Mort de
Saint-Olon.}} {\textsc{- Mort de M\textsuperscript{me} Dacier.}}
{\textsc{- Mort, extraction, fortune, famille, caractère et Mémoires de
Dangeau.}} {\textsc{- Raisons de s'y étendre.}} {\textsc{- Duc de
Chartres grand maître des ordres de Notre-Dame du mont Carmel et de
Saint-Lazare.}} {\textsc{- Mort du duc de Grammont\,; son nom et ses
armes.}} {\textsc{- Mort de M\textsuperscript{me} de Nogent, soeur du
duc de Lauzun.}} {\textsc{- Réflexion.}}

~

L'abbé Dubois qui ne pensait qu'à faciliter sa promotion au cardinalat,
et qui y sacrifiait l'État, le régent, et toutes choses, fit si bien,
que nous fumes tous surpris qu'au conseil de régence tenu l'après-dînée
du dimanche 4 août, M. le chancelier tira de sa poche des lettres
patentes pour accepter la constitution Unigenitus, et les lut par ordre
de M. le duc d'Orléans, qui ne prit les voix de personne, dont je fus
aussi aise que surpris. Cette nouveauté de ne prendre point les avis
frappa tout le monde, et marqua bien solennellement qu'ils n'auraient
point été pour la déclaration et le tour de passe-passe et de violence
d'en user hardiment de la sorte pour les faire passer pour approuvées,
dans la certitude que personne n'oserait réclamer. Ce fut un grand
mérite que Dubois s'acquit auprès des jésuites et de toute la cabale de
la constitution.

Le chevalier de Broglio, frère du premier maréchal, oncle de l'autre,
mourut fort vieux en ce temps-ci, et aurait été bien étonné s'il eut vu
leur fortune. C'était un homme très bien fait, qui avait passé les trois
quarts de sa vie dans le subalterne de la guerre, l'extrême pauvreté,
assez pourtant dans la bonne compagnie, entretenu par les dames, vivant
sur le commun, qui presque tout à coup perça jusqu'à devenir lieutenant
général, grand'croix de Saint-Louis et riche par la mort de son frère
Revel et par un mariage dont il ne laissa qu'une fille qui est morte
sans s'être mariée.

Ce fut en ce temps-ci que le comte de Saxe, bâtard du roi de Pologne,
électeur de Saxe, et de M\textsuperscript{lle} de Konigsmarck, qui s'est
fait depuis un si grand nom à la tête de nos armées, vint se mettre au
service de France, et fut fait maréchal de camp parce qu'il l'était dans
les troupes de Saxe. Alincourt, second fils du duc de Villeroy et le
favori du maréchal son grand-père, épousa la fille de la maréchale de
Boufflers dont le fils était gendre du duc de Villeroy. Cela devint donc
un double mariage où la magnificence du maréchal de Villeroy fut
déployée.

En ce même temps, Cellamare, qui fut arrêté ici pendant son ambassade,
et qui, après la mort de son père, avait pris le nom de duc de
Giovenazzo, eut permission de venir saluer le roi d'Espagne à l'Escurial
qui, depuis son retour de France, n'avait pas voulu le voir, et l'avait
tenu exilé, mais dans son gouvernement. Il fut bien reçu, et peu après
fit sa couverture comme grand d'Espagne après son père, et demeura en
cette cour, faisant les fonctions de sa charge de grand écuyer de la
reine.

La procession accoutumée de la Notre-Dame d'août se fit à l'ordinaire,
ou le cardinal de Noailles officia. La chambre des comptes et la cour
des aides y laissèrent vides les places que le parlement a coutumé d'y
remplir, qui était lors à Pontoise.

Le parlement ne voulant point enregistrer la déclaration du roi pour
l'acceptation de la constitution Unigenitus, et l'abbé Dubois, pressé
par l'intérêt de son chapeau de donner des marques éclatantes de son
zèle à Rome et aux jésuites, fit prendre la résolution à M. le duc
d'Orléans de la faire enregistrer au grand conseil, et pour n'y point
trouver les obstacles qu'il y craignait, d'y aller lui-même et d'y mener
tous les princes du sang, autres pairs et maréchaux de France, parce
qu'en ce tribunal tous les officiers de la couronne y ont séance et voix
délibérative, à la différence des parlements où ils ne l'ont que quand
le roi y va et qu'il les y mène. Arrivant de Meudon au Palais-Royal pour
travailler avec M. le duc d'Orléans, je le trouvai seul dans son grand
appartement, donnant des ordres à des garçons rouges pour aller avertir
et convier ces messieurs pour le lendemain matin. J'ignorais
parfaitement de quoi il s'agissait. Dubois avait peur que je n'eusse
fait manquer la chose et persuadé M. le duc d'Orléans de la faiblesse et
de l'indécence d'une démarche si solennelle, si nouvelle et si inutile.
Je demandai donc à M. le duc d'Orléans de quoi il s'agissait\,; il me le
dit et tout de suite souriant et étendant ses bras vers moi, il me pria
de ne me trouver point au grand conseil. Je me mis à rire aussi, et lui
répondis qu'il ne pouvait me donner un ordre plus agréable et que
j'exécutasse plus volontiers, parce qu'il m'épargnait la douleur de
m'élever publiquement contre sa volonté et d'opiner de toute ma force
contre elle. Il me dit qu'il s'en doutait bien et que c'était pour cela
qu'il m'avait prié de n'y point venir. Je ne laissai pas, quoique de
chose faite, de lui dire en deux mots qu'on lui faisait faire un pas de
clerc, afficher son impuissance pour un enregistrement valable in loco
majorum dans le seul tribunal, j'entends les autres parlements comme
celui de Paris pour leur ressort, en caractère d'enregistrer les édits
et les déclarations et de les faire enregistrer par ses arrêts dans les
tribunaux inférieurs ressortissant à lui\,; conséquemment que le grand
conseil, et tout tribunal non parlement, n'en avait le pouvoir que pour
des choses intérieures à sa juridiction qui n'est pas universelle pour
les choses publiques et générales, par la non obligatoires à personne,
nouveauté étrangère au grand conseil et qui ne lui donnait ni droit ni
puissance par soi-même de tenir la main à l'exécution de son
enregistrement. Je me contentai de ces deux mots parce qu'il n'était pas
question d'espérer de rompre un parti pris si avancé, qui se devait
exécuter le lendemain matin, et que l'abbé Dubois regardait comme sa
propre et plus capitale affaire. Je fis ensuite ce que j'avais à faire
avec M. le duc d'Orléans, et je m'en retournai à Meudon, fâché de ce
qu'on lui faisait faire, mais très soulagé d'être dispensé, et, sans
l'avoir demandé, d'aller au grand conseil. Le lendemain, 23 septembre,
le régent s'y rendit en pompe et y trouva les princes du sang, les
autres pairs et les maréchaux de France en aussi grand nombre qu'il s'en
trouva à Paris.

L'affaire ne se passa pas sans bruit. Plusieurs magistrats du grand
conseil opinèrent contre avec beaucoup de lumière, de force et
d'étendue, et ne s'étonnèrent point de quelques interruptions que leur
fit le régent, auquel ils répondirent avec respect, mais avec encore
plus de raisons et de nerf, et il fut avéré par le compte des voix que
la chose ne fut emportée que par le nombre de pairs et de maréchaux, qui
tous avec très peu de magistrats du grand conseil emportèrent la
balance. Je sus que mon absence fut extrêmement remarquée, et que
beaucoup de gens aillèrent et envoyèrent visiter l'amas de carrosses
pour voir si le mien y était. Je n'ose dire que le monde applaudit à mon
absence, et qu'elle fâcha fort l'abbé Dubois, quoiqu'il ne m'en eut
point parlé, et qu'il fut fort surpris quand il sut de M. le duc
d'Orléans que c'était lui qui m'avait prié de n'y point aller, en
m'apprenant la chose. Le succès fut tel que je le lui avais prédit. On
se moqua et de la chose et de son appareil\,; on la regarda comme un
épouvantail inutile, une faiblesse avouée, une bassesse pour Rome. On ne
s'y méprit pas à l'intérêt de l'abbé Dubois, et il n'y eut personne qui
ne regardât cet enregistrement comme sans aucune force ni autorité dans
le royaume, à commencer par le grand conseil même.

La Brue, évêque de Mirepoix, mourut dans ces entrefaites. C'était un
excellent évêque, résidant, aumônier, édifiant, instruisant, prêchant
ses ouailles, dont il était adoré et de tout le pays, et d'ailleurs très
savant et fort éloquent. Il fut l'un des quatre évêques qui firent leur
appel en Sorbonne, et qui en furent chassés de Paris.

L'évêque comte de Châlon mourut en même temps d'une si courte maladie,
que le cardinal de Noailles son frère, parti, dès qu'il le sut malade,
pour l'aller trouver, apprit sa mort en chemin. C'était un prélat d'un
grand exemple, d'une rare piété et d'une grande fermeté contre la bulle
Unigenitus. Son savoir et ses lumières étaient médiocres.

La France perdit aussi un de ses plus implacables ennemis, mais dans un
temps où il ne pouvait plus lui nuire, par la mort du célèbre Heinsius,
pensionnaire de Hollande, duquel il à souvent été fait mention. Il avait
quatre-vingt-un an, la tête et le sens comme à quarante, la santé ferme.
Il fut emporté par une maladie de peu de jours, à la Haye, à quoi le
chagrin eut grande part. Créature, puis confident intime, conseiller le
plus accrédité du prince d'Orange, et l'instrument de l'autorité et du
pouvoir sans bornes qu'il s'était acquis dans les Provinces-Unies, il en
avait épousé tous les intérêts, ses affections et ses haines. On a vu
ici ailleurs, et pourquoi, le prince d'Orange était devenu l'ennemi
personnel du roi, et le plus grand ennemi de la France. Heinsius succéda
non à ses charges et à l'autorité qu'elles donnent, mais à tout son
crédit sur les esprits et à son art de gouverner et de devenir le
premier mobile et comme le maître de toutes les délibérations
importantes de sa république. Entraîné par son grand objet d'humilier la
France et la personne du roi, flatté par la cour rampante que lui
faisaient sans ménagement le prince Eugène et le duc de Marlborough,
jusqu'à attendre quelquefois deux heures dans son antichambre, il ne
voulut jamais la paix, et tous trois ne visèrent pas à moins, au milieu
de leurs énormes succès, qu'à réduire la France au-dessous de la paix de
Vervins.

Les finances de l'empereur, quoique le plus intéressé, étaient toujours
fort courtes. Quelque animés que fussent les Anglais, leur parlement
sentait avec peine le poids d'une distribution si inégale, et n'allait
pas à beaucoup près à ce qu'il était nécessaire d'en tirer. Ce fut donc
à la Hollande à suppléer pour ces deux puissances. La haine d'Heinsius,
et les cajoleries des deux héros du temps l'aveuglèrent, acheva de
ruiner sa république, que son crédit et son autorité entraîna. Il fut
trente ans pensionnaire, et jamais pensionnaire n'a été si maître de
toutes les affaires, on pourrait dire si absolu, si la forme du
gouvernement n'eût demandé des insinuations lumineuses et adroites, mais
qui avaient toujours un plein succès. On peut juger par là de la
capacité, des connaissances, de la dextérité, de l'éloquence, de
l'expérience et de la force de tête de ce ministre, qui, n'{[}y{]} ayant
point de stathouder depuis la mort du roi Guillaume, se trouvait en tout
genre le chef et le premier homme de sa république, de longue main si
accoutumée du temps du roi Guillaume, et depuis, à suivre comme
aveuglément ses impulsions et ses sentiments. Mais la paix faite, la
république, désenivrée d'espérances fondées sur une guerre heureuse
jusqu'au prodige, et ramenée sur elle-même, aperçut enfin jusqu'où la
passion d'Heinsius l'avait menée, et vit avec horreur la profondeur des
engagements où il l'avait jetée et l'immensité de dettes dont elle se
trouva accablée. Les yeux s'ouvrirent donc sur la conduite d'Heinsius,
le mécontentement ne se contraignit pas, le crédit du ministre tomba,
ses embarras à se défendre d'avoir précipité la république dans cet
abîme se multiplièrent, les dégoûts devinrent fréquents, puis
continuels, qui le conduisirent amèrement au tombeau. Outre la place de
pensionnaire, il avait aussi les sceaux pour que rien ne manquât à son
autorité. Les États généraux séparèrent ces deux grands emplois, et,
après avoir délibéré six semaines et davantage, ils donnèrent, le 20
septembre, la garde du grand sceau au baron de Wassenaer-Stattenberg, et
l'importante place de pensionnaire de Hollande et de West-Frise à
Hoornbeck, pensionnaire de la ville de Rotterdam.

Saint-Olon mourut fort vieux. Son nom était Pidou, et de fort bas aloi.
Il était gentilhomme ordinaire chez le roi\,; on n'en parle ici que
parce qu'il avait été longtemps employé en des voyages en pays étranger
avec confiance et succès, et avait été aussi envoyé du roi à Maroc et à
Alger, où il vint à bout d'affaires difficiles et même fort périlleuses
pour lui, avec une grande fermeté et beaucoup d'adresse et de capacité,
d'ailleurs fort honnête homme, et qui ne s'en faisait point accroire.

La mort de M\textsuperscript{me} Dacier fut regrettée des savants et des
honnêtes gens. Elle était fille d'un père qui était l'un et l'autre, et
qui l'avait instruite. Il s'appelait Lefèvre, était de Caen et
protestant. Sa fille se fit catholique après sa mort, et se maria à
Dacier, garde des livres du cabinet du roi, qui était de toutes les
académies, savant en grec et en latin, auteur et traducteur. Sa femme
passait pour en savoir plus que lui en ces deux langues, en antiquités,
en critique, et a laissé quantité d'ouvrages fort estimés. Elle n'était
savante que dans son cabinet ou avec des savants, partout ailleurs
simple, unie, avec de l'esprit, agréable dans la conversation, où on ne
se serait pas douté qu'elle sût rien de plus que les femmes les plus
ordinaires. Elle mourut dans de grands sentiments de piété, à
soixante-huit ans\,; son mari, deux ans après elle, à soixante-onze ans.

Philippe de Courcillon, dit le marquis de Dangeau, mourut à Paris à
quatre-vingt-quatre ans, le 7 septembre, ce fut une espèce de personnage
en détrempe, sur lequel, à l'occasion de ses singuliers
Mémoires\footnote{Le Journal de Dangeau n'avait été publié jusqu'ici que
  par fragments. MM. Soulié, Dussieux, de Chennevières, Mantz et de
  Montaiglon, en ont entrepris, en 1854, une édition complète qu'ils
  continuent avec la plus louable persévérance. M. Feuillet de Conches y
  a joint les notes de Saint-Simon, que l'on peut considérer comme une
  première ébauche de ses Mémoires.}, la curiosité engage à s'étendre un
peu ici. Sa noblesse était fort courte, du pays Chartrain, et sa famille
était huguenote. Il se fit catholique de bonne heure, et s'occupa fort
de percer et de faire fortune. Entre tant de profondes plaies que le
ministère du cardinal Mazarin a faites et laissées à la France, le gros
jeu et ses friponneries en fut une à laquelle il accoutuma bientôt tout
le monde, grands et petits. Ce fut une des sources où il puisa
largement, et un des meilleurs moyens de ruiner les seigneurs qu'il
haïssait et qu'il méprisait, ainsi que toute la nation française, et
dont il voulait abattre tout ce qui était grand par soi-même, ainsi que
sur ses documents on y a sans cesse travaillé depuis sa mort jusqu'au
parfait succès que l'on voit aujourd'hui, et qui présage si sûrement la
fin et la dissolution prochaine de cette monarchie. Le jeu était donc
extrêmement à la mode à la cour, à la ville et partout, quand Dangeau
commença à se produire.

C'était un grand homme, fort bien fait, devenu gros avec l'âge, ayant
toujours le visage agréable, mais qui promettait ce qu'il tenait, une
fadeur à faire vomir. Il n'avait rien, ou fort peu de chose\,; il
s'appliqua à savoir parfaitement tous les jeux qu'on jouait alors\,: le
piquet, la bête, l'hombre, grande et petite prime, le hoc, le reversi,
le brelan, et à approfondir toutes les combinaisons des jeux et celles
des cartes, qu'il parvint à posséder jusqu'à s'y tromper rarement, même
au lansquenet et à la bassette, à les juger avec justesse et à charger
celles qu'il trouvait devoir gagner. Cette science lui valut beaucoup,
et ses gains le mirent à portée de s'introduire dans les bonnes maisons,
et peu à peu à la cour, dans les bonnes compagnies. Il était doux,
complaisant, flatteur, avait l'air, l'esprit, les manières du monde, de
prompt et excellent compte au jeu, où, quelques gros gains qu'il ait
faits, et qui ont fait son grand bien et la base et les moyens de sa
fortune, jamais il n'a été soupçonné, et sa réputation toujours entière
et nette. La nécessité de trouver de fort gros joueurs pour le jeu du
roi et pour celui de M\textsuperscript{me} de Montespan, l'y fit
admettre\,; et c'était de lui, quand il fut tout à fait initié, que
M\textsuperscript{me} de Montespan disait plaisamment qu'on ne pouvait
s'empêcher de l'aimer ni de s'en moquer, et cela était parfaitement
vrai. On l'aimait parce qu'il ne lui échappait jamais rien contre
personne, qu'il était doux, complaisant, sûr dans le commerce, fort
honnête homme, obligeant, honorable\,; mais d'ailleurs si plat, si fade,
si grand admirateur de riens, pourvu que ces riens tinssent au roi ou
aux gens en place ou en faveur\,; si bas adulateur des mêmes, et depuis
qu'il s'éleva, si bouffi d'orgueil et de fadaises, sans toutefois
manquer à personne, ni être moins bas, si occupé de faire entendre et
valoir ses prétendues distinctions, qu'on ne pouvait pas s'empêcher d'en
rire.

Établi dans les jeux du roi et de sa maîtresse, il en profita pour se
décorer, et comprit qu'il ne le pouvait qu'à force d'argent. Il en donna
donc à M. de Vivonne, à ce qu'il me semble, car ce fait est de 1670,
tout ce qu'il voulut du gouvernement de Tours et de Touraine, et il
acheta, peu de mois après, une des deux charges de lecteur du roi, parce
qu'elles donnent les entrées, si rares et si utiles sous Louis XIV. Son
argent commença donc à en faire un homme du petit coucher, un gouverneur
de province, et un familier dans les parties du roi et de
M\textsuperscript{me} de Montespan, qui jouaient presque tous les jours.
Avec peu d'esprit, mais celui du grand monde et de savoir être toujours
dans la bonne compagnie, il ne laissait pas de rimailler. Le roi
s'amusait quelquefois alors à donner des bouts-rimés à remplir. Dangeau
souhaitait ardemment un logement qui étaient rares dans les premiers
temps que le roi s'établit à Versailles.

Un jour qu'il était au jeu avec M\textsuperscript{me} de Montespan,
Dangeau soupirait fadement en parlant de son désir d'un logement à
quelqu'un, assez haut pour que le roi et M\textsuperscript{me} de
Montespan le pussent entendre\,; ils l'entendirent effectivement et s'en
divertirent, puis trouvèrent plaisant de mettre Dangeau sur le gril, en
lui composant sur-le-champ les bouts-rimés les plus étranges qu'ils
pussent imaginer, les donnèrent à Dangeau, et comptant bien qu'il ne
pourrait jamais en venir à bout, lui promirent un logement s'il les
remplissait sans sortir du jeu et avant qu'il finît. Ce fut le roi et
M\textsuperscript{me} de Montespan qui en furent les dupes. Les muses
favorisèrent Dangeau, il conquit un logement, et en eut un sur-le-champ.
Il avait été capitaine de cavalerie\,; il obtint le régiment du roi\,;
puis la guerre étant moins son fait que la cour, non qu'il ait été
accusé de poltronnerie, il fut employé auprès de quelques princes en
Allemagne, puis en Italie. Au mariage de Mgr le Dauphin, il fit si bien
qu'il fut un de ses menins, quoique tous les autres fussent de qualité
distinguée. On a pu voir ici que M\textsuperscript{me} de Maintenon, qui
voulait environner la Dauphine de gens à elle, fit passer la duchesse de
Richelieu, dame d'honneur de la reine, à M\textsuperscript{me} la
Dauphine, et que, pour adoucir cette complaisance, elle fit donner la
charge de chevalier d'honneur de cette princesse au duc de Richelieu,
avec promesse qu'après l'avoir gardée quelque temps, il la vendrait tout
ce qu'il la pourrait vendre à qui il voudrait qui serait agréé, Il
s'était étrangement incommodé au jeu. Dangeau, déjà menin et gouverneur
de province, fut son homme\,; il en tira cinq cent mille livres. Dangeau
devint ainsi chevalier d'honneur de M\textsuperscript{me} la Dauphine,
et nécessairement par là chevalier de l'ordre, en la grande promotion,
trois ans après, le premier jour de l'an 1689\footnote{Voy, le Journal
  de Dangeau (édit Didot), t. II, p.~284-285, et la lettre de
  M\textsuperscript{me} de Sévigné du 3 janvier 1689.}.

Il avait épousé en 1682 une fille fort riche, d'un partisan qu'on
appelait Morin le Juif, qui le fit beau-frère du maréchal d'Estrées,
mari de l'autre. Dangeau en eut une fille unique, qu'il maria au duc de
Montfort, fils aîné du duc de Chevreuse, dont il se bouffit fort. Étant
devenu veuf, il se trouva assez riche pour se remarier à une comtesse de
Loewenstein, fille d'honneur de M\textsuperscript{me} la Dauphine, et
fille d'une soeur du cardinal de Fürstemberg, laquelle avait des soeurs
grandement mariées en Allemagne, et des frères en grands emplois. On a
vu ailleurs quels sont les Loewenstein, et le bruit que fit Madame, et
même M\textsuperscript{me} la Dauphine, de voir les armes palatines
accolées à celles de Courcillon, à la chaise de M\textsuperscript{me} de
Dangeau, et combien il fut avec raison inutile. M\textsuperscript{me} de
Dangeau n'avait rien vaillant, mais elle était charmante de visage, de
taille et de grâces. On en a parlé souvent ici ailleurs. C'était un
plaisir de voir avec quel enchantement Dangeau se pavanait en portant le
deuil des parents de sa femme, et en débitait les grandeurs. Enfin, à
force de revêtements l'un sur l'autre, voilà un seigneur, et qui en
affectait toutes les manières à faire mourir de rire. Aussi La Bruyère
disait-il, dans ses excellents Caractères de Théophraste\footnote{Le
  titre de la première édition des Caractères de La Bruyère est les
  Caractères de Théophraste, traduits du grec, avec les caractères ou
  les moeurs de ce siècle (Paris, 1687, in-12). Le passage auquel
  Saint-Simon Lait allusion se trouve dans les Caractères de La Bruyère
  (chap.~des grands)\,: «\,Un Pamphile, en un mot, veut être grand\,; il
  croit l'être\,; il ne l'est pas\,; il est d'après un grand.\,»}, que
Dangeau n'était pas un seigneur, mais d'après un seigneur.

Je fus brouillé avec lui longtemps, pour un fou rire qui partit malgré
moi, et que j'ai eu lieu de croire qu'il ne m'a jamais bien pardonné. Il
faisait magnifiquement les honneurs de la cour, où sa maison et sa
table, tous les jours grande et bonne, était ouverte à tous les
étrangers de considération. Il m'avait prié à dîner. Plusieurs
ambassadeurs et d'autres étrangers s'y trouvèrent, et le maréchal de
Villeroy, qui était fort de ses amis, et chez qui sa noce s'était faite.
Il fit peu à peu tomber à table la conversation sur les gouvernements et
les gouverneurs de province\,; puis, se balançant avec complaisance, se
mit à dire à la compagnie\,: «\,Il faut dire la vérité\,: de tous nous
autres gouverneurs de provinces, il n'y a que M. le maréchal, en
regardant Villeroy, qui soit demeuré maître de la sienne.\,» Les yeux de
M\textsuperscript{me} de Dangeau et les miens se rencontrèrent dans cet
instant\,; elle sourit, et moi je fis pis, quelque effort que je pusse
faire, car il était bon homme, et je ne voulais pas le fâcher, mais
cette fatuité fut plus forte que moi. Un an après la mort de M. de
Louvois, le roi se lassa d'être grand maître des ordres de Saint-Lazare,
et de Notre-Dame du mont Carmel, dont Louvois avait toute la gestion en
qualité de grand vicaire, et donna cette grande maîtrise à Dangeau.
L'envie de s'en divertir eut grande part à ce choix. Il traitait bien
Dangeau, mais il s'en moquait volontiers. Il connaissait ses fadeurs, sa
vanité, sa fatuité. Cette grâce en devint une source. On a vu ici
ailleurs avec quelle dignité il tâcha d'imiter le roi donnant l'ordre du
Saint-Esprit, en donnant celui de Saint-Lazare, combien le prie-Dieu
était bien imité dans Saint-Germain des Prés, comment ses prêtres de
l'ordre, placés comme le sont les évêques et les abbés au prie-Dieu du
roi, représentaient bien les cardinaux avec leurs soutanes et leurs
camails rouges\,; avec quelle grâce et quel air de satisfaction et de
bonté Dangeau faisait la roue au milieu de cette pompe et de toute la
cour, hommes et femmes, qui y allaient sur des échafauds parés, et y
riaient scandaleusement. Le roi après s'amusait du récit qu'il lui en
faisait faire chez M\textsuperscript{me} de Maintenon, et il était ou,
se montrait transporté de la privance de ces conversations et des
applaudissements qu'il en recevait. Il est pourtant vrai qu'il faisait
un très noble usage de sa commanderie magistrale, qui était bonne, et
qu'il abandonna tout entière, pour y élever de pauvres gentilshommes,
qui y apprenaient gratuitement tout ce qui peut convenir à leur état, et
y étaient fort honnêtement nourris et entretenus.

On a vu ici en son temps ce qui regarde le fils unique qu'il eut de sa
seconde femme, qu'il maria à la fille unique du dernier de la maison de
Pompadour et d'une fille de M. et de M\textsuperscript{me} de Navailles,
par conséquent soeur de la duchesse d'Elboeuf, mère de la dernière
duchesse de Mantoue. Je ne fais ici que renouveler le souvenir de toutes
ces alliances de sa femme et de son fils, nécessaires à savoir avant de
parler de ses Mémoires. En 1696 il fut conseiller d'État d'épée, et on a
vu ici en son lieu qu'au mariage de Mgr le duc de Bourgogne, le roi lui
rendit la charge de chevalier d'honneur qu'il avait perdue à la mort de
la Dauphine, et fit sa femme dame du palais, dont elle fut la première
par la charge de son mari, n'y ayant point eu alors de duchesse, et on
n'a pas oublié de remarquer les privances et la faveur de
M\textsuperscript{me} de Dangeau auprès de M\textsuperscript{me} de
Maintenon, qui lui attirèrent celles du roi. Tout cela enfla Dangeau et
en augmenta merveilleusement les ridicules. Il adorait le roi et
M\textsuperscript{me} de Maintenon\,; il adorait les ministres et le
gouvernement\,; son culte, à force de le montrer, s'était glissé jusque
dans ses moelles. Leurs goûts, leurs affections, leurs éloignements, il
se les adaptait entièrement. Tout ce que le roi faisait, en quelque
genre que ce fût, et quelquefois de plus étrange, transportait Dangeau
d'admiration, qui passait du dehors jusqu'à l'intérieur. Il en était de
même de tout ce qu'il voyait que M\textsuperscript{me} de Maintenon
aimait, avançait ou écartait, et il s'incrusta si bien de tout cela
qu'il en fit sa propre chose, même après leur mort. De là vient la
partialité que toute sa tremblante politique n'a pu cacher dans ses
Mémoires contre M. le duc d'Orléans et pour les bâtards en général, et
spécialement pour la personne du duc du Maine, {[}pour{]} tout ce que
l'ambition, ou le mécontentement, ou l'aveuglement lui avait attaché, et
pour tout ce qui se montrait ou était contraire à M. le duc d'Orléans.

Par même raison, et par plusieurs autres, il était grand partisan du
parlement, des bâtards et des princes étrangers, vrais et faux\,; grand
ennemi de la dignité des ducs, avec l'ignorance la plus profonde jusqu'à
être surprenante dans un homme qui avait passé sa vie à la cour, en
sorte qu'il n'a pu se retenir là-dessus dans ses Mémoires, jusqu'à y
avoir sacrifié la vérité bien des fois à cet égard, et d'autres fois
passé grossièrement à côté, n'osant hasarder les négatives, et d'autres
fois omettant ce qui s'était passé sous ses yeux. Cette aversion des
ducs lui venait de celle de M\textsuperscript{me} de Maintenon, la mie
ancienne et la protectrice des bâtards, qui, pour leur ranger tout
obstacle, eût voulu anéantir la première dignité du royaume. Ainsi, tout
ce qui s'opposait à elle, en tout genre, pour nouveau et pour étrange
qu'il fût, trouvait appui en elle. Dangeau ne pouvait se consoler de
l'inutilité de tout ce qu'il avait tenté pour se faire faire duc, et en
avait pris une haine particulière contre la dignité à laquelle il
n'avait pu atteindre\,; il croyait ainsi s'en dédommager. Les alliances
de sa femme qui, en vraie Allemande, croyait que rien ne pouvait égaler
un prince ni même un ancien comte de l'empire\,; l'alliance de son fils,
si proche avec les duchesses d'Elboeuf et de Mantoue, lui avaient tout à
fait tourné la tête là-dessus. On a vu en son lieu l'étroite liaison de
la comtesse de Fürstemberg avec M\textsuperscript{me} de Soubise et la
cause de cette union, et quelle était M\textsuperscript{me} de Soubise à
l'égard du roi et même de M\textsuperscript{me} de Maintenon. On a vu
aussi quelle était cette comtesse de Fürstemberg à l'égard du cardinal,
frère du père de son mari et de la mère de M\textsuperscript{me} de
Dangeau, qui vivait avec eux en intimité de famille. Il n'en fallut pas
davantage à Dangeau pour être comme à genoux devant les Rohan, et, par
concomitance, devant les Bouillon, en ce que ces deux maisons avaient de
commun ensemble. C'est ce qui paraît par sa partialité extrême dans ses
Mémoires, par ses louanges ou son aridité, enfin par ses méprises ou
d'ignorance ou de pis, et par ses réticences. Après ses remarques
nécessaires, venons aux Mémoires qu'il a laissés, qui le peignent si
parfaitement lui-même, et si fort d'après nature.

Dès les commencements qu'il vint à la cour, c'est-à-dire vers la mort de
la reine mère\footnote{La reine mère, Anne d'Autriche, mourut le 20
  janvier 1666. Le Journal de Dangeau ne commence qu'en avril 1684.}, il
se mit à écrire tous les soirs les nouvelles de la journée, et il a été
fidèle à ce travail jusqu'à sa mort. Il le fut aussi à les écrire comme
une gazette sans aucun raisonnement, en sorte qu'on n'y voit que les
événements avec une date exacte, sans un mot de leur cause, encore moins
d'aucune intrigue ni d'aucune sorte de mouvement de cour ni d'entre les
particuliers. La bassesse d'un humble courtisan, le culte du maître et
de tout ce qui est ou sent la faveur, la prodigalité des plus fades et
dés plus misérables louanges, l'encens éternel et suffoquant jusque des
actions du roi les plus indifférentes, la terreur et la faveur suprême
qui ne l'abandonnent nulle part pour ne blesser personne, excuser tout,
principalement dans les généraux et les autres personnes du goût du roi,
de M\textsuperscript{me} de Maintenon, des ministres, toutes ces choses
éclatent dans toutes les pages, dont il est rare que chaque journée en
remplisse plus d'une, et dégoûtent merveilleusement. Tout ce que le roi
a fait chaque jour, même de plus indifférent, et souvent les premiers
princes et les ministres les plus accrédités, quelquefois d'autres
sortes de personnages, s'y trouvent avec sécheresse pour les faits, mais
tant qu'il se peut avec les plus serviles louanges, et pour des choses
que nul autre que lui ne s'aviserait de louer.

Il est difficile de comprendre comment un homme a pu avoir la patience
et la persévérance d'écrire un pareil ouvrage tous les jours pendant
plus de cinquante ans, si maigre, si sec, si contraint, si précautionné,
si littéral, à n'écrire que des écorces de la plus repoussante aridité.
Mais il faut dire aussi qu'il eût été difficile à Dangeau d'écrire de
vrais Mémoires qui demandent qu'on soit au fait de l'intérieur et des
diverses machines d'un cour. Quoiqu'il n'en sortit presque jamais, et
encore pour des moments, quoiqu'il y fût avec distinction et dans les
bonnes compagnies, quoiqu'il y fût aimé, et même estimé du côté de
l'honneur et du secret, il est pourtant vrai qu'il ne fut jamais au fait
d'aucune chose ni initié dans quoi que ce fût. Sa vie frivole et
d'écorce était telle que ses Mémoires\,; il ne savait rien au delà de ce
que tout le monde voyait\,; il se contentait aussi d'être des festins et
des fêtes, sa vanité a grand soin de l'y montrer dans ses Mémoires, mais
il ne fut jamais de rien de particulier. Ce n'est pas qu'il ne fût
instruit quelquefois de ce qui pouvait regarder ses amis, par eux-mêmes,
qui, étant quelques-uns des gens considérables, pouvaient lui donner
quelques connaissances relatives, mais cela était rare et court. Ceux
qui étaient de ses amis de ce genre, en très petit nombre, connaissaient
trop la légèreté de son étoffe pour perdre leur temps avec lui.

Dangeau était un esprit au-dessous du médiocre, très futile, très
incapable en tout genre, prenant volontiers l'ombre pour le corps, qui
ne se repaissait que de vent, et qui s'en contentait parfaitement. Toute
sa capacité n'allait qu'à se bien conduire, ne blesser personne,
multiplier les bouffées de vent qui le flattaient, acquérir, conserver
et jouir d'une sorte de considération, sans vouloir s'apercevoir qu'à
commencer par le roi, ses vanités et ses fatuités divertissaient souvent
les compagnies, ni des panneaux où on le faisait tomber souvent
là-dessus. Avec tout cela, ses Mémoires sont remplis de faits que
taisent les gazettes, gagneront beaucoup en vieillissant, serviront
beaucoup à qui voudra écrire plus solidement, pour l'exactitude de la
chronologie, et pour éviter confusion. Enfin ils représentent, avec la
plus désirable précision, le tableau extérieur de la cour, des journées,
de tout ce qui la compose, les occupations, les amusements, le partage
de la vie du roi, le gros de celle de tout le monde, en sorte que rien
ne serait plus désirable pour l'histoire que d'avoir de semblables
Mémoires de tous les règnes, s'il était possible, depuis Charles V, qui
jetteraient une lumière merveilleuse parmi cette futilité sur tout ce
qui a été écrit de ces règnes.

Encore deux mots sur ce singulier auteur. Il ne se cachait point de
faire ce journal, parce qu'il le faisait de manière qu'il n'en avait
rien à craindre\,; mais il ne le montrait pas\,; on ne l'a vu que depuis
sa mort. Il n'a point été imprimé jusqu'à présent, et il est entre les
mains du duc de Luynes, son petit-fils, qui en a laissé prendre quelques
copies. Dangeau, qui ne méprisait rien, et qui voulait être de tout,
avait brigué et obtenu de bonne heure une place dans l'Académie
française, dont il est mort doyen, et une dans l'Académie des sciences,
quoiqu'il ne sût rien du tout en aucun genre, quoiqu'il s'enorgueillît
d'être de ces compagnies et de fréquenter les illustres qui en étaient.
Il se trouve dans ses Mémoires des grossièretés d'ignorance sur les
duchés et sur les dignités de la cour d'Espagne qui surprennent au
dernier point. Il essuya la grande opération de la fistule, dont il
pensa mourir, et fut taillé d'une fort grosse pierre. Il a vécu depuis
sans aucune incommodité de la première, et longues années, parfaitement
guéri et sans aucune suite de l'autre.

Deux ans avant sa mort, il fut taillé pour la seconde fois\,; la pierre
n'était pas grosse, à peine eut-il quelques heures de fièvre\,; il fut
guéri en un mois, et s'en est bien porté depuis. À la fin, le grand âge,
et peut-être l'ennui de ne voir plus de cour ni de grand monde, termina
sa vie par une maladie de peu de jours.

N'attendons pas le temps de la mort de l'abbé de Dangeau son frère, qui
arriva le 1er janvier 1723, pour parler de lui tout de suite. Il naquit
huguenot, il y persévéra plus longtemps que son frère, et je ne sais
s'il y a jamais bien renoncé. Il avait plus d'esprit que son aîné, et
quoiqu'il eût assez de belles-lettres qu'il professa toute sa vie, il
n'eut ni moins de fadeur ni moins de futilité que lui\,; il parvint de
bonne heure à être des académies. Les bagatelles de l'orthographe et de
ce qu'on entend par la matière des rudiments et du Despotère\footnote{Despotère,
  ou plus correctement Despautère, avait composé une grammaire latine
  dont on se servit longtemps dans les écoles. Le nom du grammairien
  sert ici à désigner la grammaire elle-même.} furent l'occupation et le
travail sérieux de toute sa vie \footnote{L'abbé de Dangeau a laissé un
  grand nombre de manuscrits qui sont conservés à la Bib. Imp. On y
  trouve des renseignements curieux sur les diverses parties de
  l'administration à l'époque de Louis XIV.}. Il eut plusieurs
bénéfices, vit force gens de lettres et d'autre assez bonne compagnie,
honnête homme, bon et doux dans le commerce, et fort uni avec son frère.
Il avait été envoyé étant jeune en Pologne, et il avait trouvé le moyen
de se faire décorer d'un titre de camérier d'honneur par Clément X qu'il
avait connu en Pologne, non à Rome où il n'alla jamais, et de se le
faire renouveler par Innocent XII\,; il avait aussi acheté une des deux
charges de lecteur du roi pour en conserver les entrées, et y venait de
temps en temps à la cour\,; il y était peu\,; n'y sortait guère de chez
son frère, et y avait peu d'habitude.

Je ne sais de quoi M. le duc d'Orléans s'avisa de faire donner à M. son
fils la grande maîtrise de Saint-Lazare. On lui fit sans doute accroire
que cela donnerait des créatures à ce jeune prince. Ceux qui prenaient
cet ordre si dégradé de biens et d'honneurs n'étaient pas pour lui en
faire. Le régent ne m'en parla point, et la chose faite, je ne lui en
dis rien non plus.

Le duc de Grammont mourut en même temps à Paris\footnote{Antoine-Charles
  de Grammont, ou Gramont, mourut le 25 octobre 1720. Le nom de cette
  maison de Béarn s'écrit plus correctement Gramont, pour la distinguer
  des Grammont de Franche-Comté.}, à près de quatre-vingts ans\,; il en
est tant parlé ici à l'occasion de son étrange et second mariage, et de
son ambassade en Espagne, qu'il n'y a rien à y ajouter. Il était frère
cadet du célèbre comte de Guiche, qui a tant fait parler de lui, et fils
et père des deux maréchaux de Grammont. Leur nom est Aure, connus par la
possession de plusieurs fiefs et du vicomté d'Arboust, vers 1380\,;
Sauce Garcie d'Aure servit le roi en 1405, sous J. de Bourbon, à la
conquête de Guyenne, avec dix-neuf écuyers. Menaud d'Aure, fils d'une
bâtarde de Béarn, épousa en 1523 Claire de Grammont, qui était de cette
maison de Grammont si illustre en Béarn, Gascogne, Navarre et Aragon, et
par les guerres qu'elle y soutint si longtemps contre la maison de
Beaumont, bâtards de la maison de France, qui s'étaient grandement
élevés en ces pays-là. Cette Claire de Grammont, lorsqu'elle fut mariée,
avait des frères et des neveux desquels tous elle devint héritière.
Antoine d'Aure, son fils, vicomte d'Aster, prit gratuitement le nom et
les armes de Grammont, car, quoi qu'en dise le Moréri, il le fit sans
aucune obligation, et il composa son écusson d'une manière à montrer
qu'il ne faisait pas grand cas de ses armes. Il porta au premier
quartier d'or un lion d'azur qui est Grammont, au second et troisième
les trois flèches en pal\footnote{On appelle pal, en termes de blason,
  une bande ou pièce perpendiculaire sur l'écu.}, la pointe en bas,
d'Aster, et d'Aure au quatrième qui est d'argent à la levrette de
sable\footnote{Le mot sable, dans le blason, désigne la couleur noire.},
à la bordure de sable chargée de huit besants d'or. L'héritière d'Aster
était la grand'mère paternelle de ce Mahaut d'Aure qui quitta son nom
pour prendre le nom de Grammont. Son mariage est de 1525, et sa mort est
de 1534\,; sa femme Claire de Grammont le survécut plus de vingt ans.
Antoine d'Aure qui, comme on vient de le dire, prit volontairement le
nom de Grammont et abandonna le sien, comme fit sa postérité après lui,
eut un fils aîné, dit Antoine de Grammont, qui épousa Hélène de
Clermont, dame de Traves et de Toulongeon. Leur fils aîné, Philibert,
dit de Grammont, épousa la fille unique de Paul d'Andouins, vicomte de
Louvigny et seigneur de Lescun. C'est la belle Corisande dont Henri IV
en sa jeunesse fut si amoureux, qu'il disparut aussitôt après sa
victoire de Coutras, et, suivi d'un seul page, alla lui présenter son
épée, ce qui lui fit perdre tous les avantages qu'il pouvait tirer de ce
grand succès, où le duc de Joyeuse, général de l'armée catholique, et
tant d'autres gens de marque avaient été tués, {[}lui{]} qui avait
défait cette armée et en avait mis les restes en désarroi. Celle des
huguenots, quoique victorieuse, demeura sans rien faire dans
l'étonnement de la disparition du roi de Navarre aussitôt après le
combat, ne sachant s'il était tué, pris ou ce qu'il était devenu pendant
six ou sept jours qu'il revint après ce fatal tour de jeunesse. Cet
amour valut au mari de la belle le gouvernement de Bayonne et la charge
de sénéchal de Béarn. Il s'était marié en 1567, et il fut tué à
vingt-six ans devant la Fère, en 1580. Sa femme le survécut longtemps et
rendit des services considérables à son royal amant, pendant les guerres
de religion. De son mariage vint la grand'mère paternelle du duc de
Lauzun et le père du premier maréchal de Grammont.

M\textsuperscript{me} de Nouent mourut aussi à quatre-vingt-huit ans.
Elle était soeur du duc de Lauzun. Elle était fille de la reine, et
n'avait rien, lorsqu'en 1663, elle épousa Bautru, dit le comte de
Nogent, capitaine de la porte, puis maître de la garde-robe du roi, qui
fut tué lieutenant général au passage du Rhin, 12 juin 1672, dont elle
porta le premier grand deuil le reste de sa vie. Son fils est mort sans
enfants, et sa fille épousa Biron, devenu enfin duc, pair et maréchal de
France, qui, du chef de cette Bautru par sa mère, a hérité de plus de un
million deux cent mille livres des ducs de Foix et de Lauzun. Autre
exemple terrible des mariages de filles de qualité pour rien avec des
gens aussi de rien et qui deviennent héritières. Heureusement que c'est
Biron et non pas un Bautru qui en a profité, mais par le plus grand
hasard du monde.

\hypertarget{chapitre-iv.}{%
\chapter{CHAPITRE IV.}\label{chapitre-iv.}}

1720

~

{\textsc{Lede, fait grand d'Espagne, est victorieux en Afrique.}}
{\textsc{- Mortification du cardinal del Giudice à Rome dépouillé de la
protection d'Allemagne en faveur du cardinal d'Althan, qu'il courtise
bassement.}} {\textsc{- Princesse des Ursins à Rome pour toujours, où
elle est considérée.}} {\textsc{- Barbarigo, Borgia et Cienfuegos faits
cardinaux, quels.}} {\textsc{- Saint-Étienne de Caen au cardinal de
Mailly.}} {\textsc{- La survivance des gouvernements du duc d'Uzès à son
fils.}} {\textsc{- Voyage et retour à Paris de la duchesse d'Hanovre.}}
{\textsc{- Sa nullité à Vienne\,; son changement de nom\,; son état
ambigu et délaissé à Paris.}} {\textsc{- Nouveautés étranges, mais sans
suite à son égard.}} {\textsc{- La Houssaye contrôleur général\,;
quel.}} {\textsc{- Triste fin et mort de Guiscard.}} {\textsc{- Mort et
caractère de Caumartin.}} {\textsc{- Époque du velours en habits
ordinaires pour les gens de robe.}} {\textsc{- Le parlement enregistre
la déclaration pour recevoir la constitution, et revient à Paris.}}
{\textsc{- Chambre établie aux Grands-Augustins pour vider force
procès.}} {\textsc{- Mariage du duc de Lorges avec
M\textsuperscript{lle} de Mesmes.}} {\textsc{- Mariage du duc de Brissac
avec M\textsuperscript{lle} Pécoil.}} {\textsc{- Mort étrange du vieux
Pécoil.}} {\textsc{- Ambassadeur du Grand Seigneur en France.}}
{\textsc{- Congrès de Cambrai inutile.}} {\textsc{- Saint-Contest et
Morville y vont ambassadeurs plénipotentiaires.}} {\textsc{- Sage pensée
du cardinal Gualterio.}} {\textsc{- Maulevrier-Langeron en Espagne.}}
{\textsc{- Law sort enfin du royaume.}} {\textsc{- Son caractère\,; sa
fin\,; sa famille.}}

~

On a vu ici en son lieu que l'extrême supériorité des Anglais par mer et
des Impériaux par terre, joints à eux, avaient fait avorter les grands
desseins de l'Espagne sur l'Italie et le traité qui s'ensuivit. Le
marquis de Lede, tout faible qu'il fût à la tête de l'armée d'Espagne,
s'y était montré grand, vaillant et habile capitaine. Le roi d'Espagne,
qui aimait à faire la guerre, ne voulut pas laisser ses troupes inutiles
ni les licencier. Il était avec raison fort content du marquis de Lede.
Il le fit grand d'Espagne et le fit passer en Afrique avec l'armée qu'il
commandait. Il fit lever aux Mores le siège de Ceuta qu'ils faisaient
depuis longtemps, reprit Oran, gagna plusieurs victoires et revint en
Espagne avec la plus grande réputation, où il reçut l'ordre de la Toison
d'Or. J'aurai occasion de parler de lui si j'ai le temps d'écrire mon
ambassade en Espagne où je l'ai beaucoup vu.

Le cardinal del Giudice, dont il a été tant parlé ici, reçut en ce
temps-ci une grande mortification. Transfuge forcé par Albéroni du
service du roi d'Espagne, il s'était jeté dans celui de l'empereur, dont
il n'avait pas honte d'être chargé des affaires à Rome où il se baignait
d'aise de l'état d'Albéroni, vagabond caché et accusé juridiquement
devant le pape, depuis qu'il avait été chassé d'Espagne. L'empereur
avait un favori. C'était le comte d'Althan qui était devenu le martre de
son coeur et de son esprit. Il avait fait son frère cardinal, et ce
nouveau cardinal arriva à Rome pour prendre le chapeau, et être chargé
en même temps des affaires de l'empereur, dont il dépouilla Giudice avec
toute la hauteur d'un favori allemand. Giudice, qui n'avait plus de
ressource ni de nouveau maître à prendre, ploya les épaules, et eut la
bassesse de donner chez lui une fête magnifique au cardinal d'Althan.
Cette douleur fut incontinent suivie d'une petite consolation. Il vit
arriver à Rome la princesse des Ursins, qui, lassée enfin du séjour de
Gênes, s'était déterminée à venir fixer son séjour dans son ancienne
demeure, où elle fut reçue avec beaucoup de considération du pape et de
sa cour, du roi et de la reine d'Angleterre, à qui elle s'attacha, du
sacré collège, et de tout ce qu'il y avait de principal et de plus grand
à Rome\,; mais Giudice ne la vit pas. Le pape fit presque en même temps
trois cardinaux\,: Barbarigo, Vénitien, évêque de Brescia, réservé in
petto de la dernière promotion\,; Borgia, Espagnol, patriarche des
Indes, que j'ai fort vu en Espagne, et dont j'espère parler, et le
fameux jésuite espagnol Cienfuegos, homme de tant d'esprit et
d'intrigue, qui débaucha l'amirante de Castille, dont il était
confesseur, et qui l'accompagna dans sa fuite en Portugal, comme il a
été dit ici en son temps. Il s'était depuis retiré à Vienne où
l'empereur l'employait en beaucoup d'affaires. Ces trois cardinaux
étaient de la nomination de l'empereur, du roi d'Espagne et de la
république de Venise.

J'obtins l'abbaye de Saint-Étienne de Caen pour le cardinal de Mailly,
et la survivance des gouverneurs de Saintonge et d'Angoumois du duc
d'Uzès pour son fils.

On a vu, vers les commencements de ces Mémoires, que la duchesse de
Hanovre était depuis longtemps en France avec ses deux filles sans
aucune sorte de distinction, la mortifiante aventure qui, de dépit, la
fit se retirer en Allemagne, d'où elle fit le mariage de son aînée avec
le duc de Modène, qui, par la mort de son neveu aîné, avait eu sa
succession, et quitté le chapeau de cardinal, et c'est de ce mariage
qu'est venu le duc de Modène, gendre de M. le duc d'Orléans. On y a vu
en même temps par quel bonheur de conjonctures et d'intrigues sa seconde
fille épousa l'empereur Joseph. On y a vu encore que, arrivée peu après
à Vienne dans l'espérance d'y recevoir les plus grands honneurs, elle y
fut tellement trompée qu'elle ne put jamais se montrer à la cour, ni
voir sa fille, ni les personnes impériales que par un escalier secret,
en particulier, et cela encore rarement et courtement, tant qu'enfin,
dépitée de ne réussir en pas une de ses prétentions, et de n'être même
visitée de personne, elle prit assez promptement le parti de se retirer
à Modène auprès de son autre fille, qui, au bout de quelques années,
mourut entre ses bras en septembre 1710. La duchesse de Hanovre, qui ne
savait où se retirer, demeura à Modène, sous prétexte d'y élever ses
deux petites-filles\,; elle avait aussi deux petits-fils. Mais, lasse au
bout de dix ans des caprices de son gendre, elle résolut de tenter
encore une fois fortune à Vienne, et, si elle n'y réussissait pas, de
venir en France, où elle n'ignorait pas que tout avait changé de face,
les prétentions les plus absurdes bien reçues, tout désordre et toute
confusion protégée, tout ordre, toute règle, tout droit proscrit\,; elle
espéra donc tout du crédit de M. le Duc, par sa soeur,
M\textsuperscript{me} la Princesse, et s'achemina lentement en
Allemagne, où elle n'avait point de demeure que triste et solitaire, où
elle ne put se résoudre d'habiter. En approchant de Vienne, elle apprit
qu'elle n'y pouvait aller. On s'y souvenait avec dégoût des prétentions
qu'elle y avait montrées, et quoiqu'elles n'eussent eu aucun succès, la
cour de Vienne aima mieux ne l'y point voir que de les voir
renouveler\,; on la fit donc demeurer à Aschau à quelques journées de
Vienne, où l'impératrice sa fille l'alla voir, et l'y fit recevoir par
ses officiers. Elle n'y demeura que quelques jours avec elle, et s'en
retourna à Vienne. L'empereur offrit à la duchesse de Hanovre la demeure
du château et de la ville de Lintz, ou dans tel autre appartenant à la
maison d'Autriche qu'elle aimerait le mieux\,; mais les espérances de
France la touchèrent davantage. Elle partit d'Aschau le même jour que
l'impératrice, et prit le chemin de France par Munich à petites
journées, pour s'assurer en chemin de ce qu'elle espérait.

Elle crut faire oublier la façon dont elle y avait été traitée, en
changeant de nom, et prit en chemin celui de duchesse de Brunschweig,
que les François prononcent Brunswick. M\textsuperscript{me} la
Princesse obtint pour elle l'un des deux grands appartements de
Luxembourg, avec les logements nécessaires pour sa suite et son service,
parce que, depuis la mort de M\textsuperscript{me} la duchesse de Berry,
les deux grands appartements étaient vides, et les autres n'étaient
occupés que par des particuliers, dont plusieurs furent délogés peu de
jours après son arrivée. On vit une chose sans exemple, que l'abbé
Dubois, pour l'intérêt de son chapeau, arracha de M. le duc d'Orléans,
dans la pensée d'en faire bien sa cour au roi d'Angleterre, qui était de
la maison de Brunswick, mais d'une branche fort éloignée de celle du
mari de cette prétendue nouvelle hôtesse de la France. Le roi l'alla
voir, à l'étonnement public et quelque chose de plus. La visite se passa
debout et fut de peu de moments, puis alla voir Madame nouvellement
revenue de Saint-Cloud. Deux jours après, la duchesse de Brunswick eut
la bonté de faire l'honneur au roi de lui rendre sa visite. Elle se
passa comme l'autre, et depuis elle ne le vit plus chez elle, et une ou
deux fois l'année au plus chez lui.

Ce début lui fit prendre de grands airs et vouloir se donner tous les
avantages dont jouissent les princesses du sang, et même en usurper
davantage. Soutenue de la maison de Condé, de la faiblesse et de
l'indifférence de M. le duc d'Orléans, et de la chimère de l'abbé Dubois
de plaire au roi d'Angleterre, qui pourtant ne montra jamais prendre le
plus léger intérêt en ceux de cette cousine, elle se mit sur le pied
qu'elle voulut\,; mais elle n'y put mettre le monde, malgré la sottise
si ordinaire en ce genre aux François. Qui que ce soit, hommes ni
femmes, ne lui donna signe de vie\,; elle ne put apprivoiser que des
gens de rien et des bourgeoises inconnues, ravies de se croire admises à
une petite cour où elles faisaient bonne chère et jouaient un petit jeu
à leur portée. Force étrangers y fréquentèrent aussi\,; d'autres gens,
pas un. M\textsuperscript{me} la Princesse, qui logeait au petit
Luxembourg qu'elle avait acheté et magnifiquement rebâti, lui était de
quelque ressource\,; elle était sa plus proche voisine\,; mais elles ne
se voyaient qu'en particulier et ne mangeaient jamais l'une chez
l'autre. Pour les enfants et petits-enfants de M\textsuperscript{me} la
Princesse, ils ne la voyaient que fort rarement et courtement en
particulier\,; mais elle était riche, se repaissait de ses chimères et
vivait contente dans sa petite et mauvaise compagnie, où elle jouait la
petite souveraine. Elle vit aussi Madame fort rarement, et comme point
M. {[}le duc{]} et M\textsuperscript{me} la duchesse d'Orléans.

Tout à la fin de l'année, Pelletier de La Houssaye fut contrôleur
général. Il n'était pas de la même famille que Pelletier des Forts, fils
de Pelletier de Sousy, qui était du conseil de régence, lequel était
frère de Pelletier qui avait été contrôleur général après M. Colbert, et
ministre d'État, père et grand-père de deux premiers présidents du
parlement de Paris. La Houssaye était frère de la femme d'Amelot, si
estimé dans ses ambassades, duquel il a été souvent parlé ici. Ce La
Houssaye étant conseiller d'État et intendant d'Alsace, est le même qui
fut nommé troisième ambassadeur avec le maréchal de Villars et le comte
du Luc, pour aller signer la paix à Bade, qui se fit moquer de lui en
refusant de céder au comte du Luc, et comme il n'y a en France qu'à
prétendre et entreprendre pour réussir, pourvu qu'on ait tort, fit la
planche par ce refus que les conseillers d'État ne veulent plus céder
qu'aux ducs et aux officiers de la couronne. On tortille depuis
là-dessus, on le trouve ridicule, mais on le souffre. La Houssaye avait
fort réussi en Alsace, il en écrivait des lettres de sa main et des
mémoires, dont la netteté et la capacité étaient merveilleuses. Cette
réputation l'en fit rappeler pour le mettre dans les grandes commissions
des finances. C'était un grand homme, très bien fait, de fort bonne
mine, dont l'air et le ton était imposant. Mais à travers cette écorce
et la réputation qu'il avait usurpée, il montra bientôt le tuf. On
découvrit qu'il avait un secrétaire extrêmement capable qui lui était
fort attaché, qui contrefaisait son écriture, à ne les pouvoir
distinguer, qui envoyait d'Alsace ces lettres et ces mémoires, qu'on
admirait comme étant de la main de La Houssaye qui se divertissait
pendant que {[}son{]} secrétaire travaillait pour lui, car il était
homme de plaisir en tout genre, et qui ne se contraignait pas, sans même
en trop craindre l'indécence. Cela même suppléa à sa capacité. Il plut à
M. le duc d'Orléans, il s'attacha à l'abbé Dubois, et fut ainsi
contrôleur général, où il prit beaucoup de morgue et d'insolence, et
montra l'épaisseur de son esprit et de sa compréhension, jusqu'à
n'entendre pas la moindre affaire.

Guiscard mourut en ce temps-ci d'une manière étrange. Il était
gouverneur de Sedan, et l'avait été de Dinan et de Namur, dont la
défense sous le maréchal de Boufflers lui valut le collier de l'ordre.
On a souvent ici parlé de lui. Il avait été après d'Avaux ambassadeur de
Suède, et il avait marié sa fille unique, qui était très riche, à
Villequier, fils aîné du duc d'Aumont\,; il avait eu plus de malheur que
de part à la défaite du maréchal de Villeroy à Ramillies, mais il ne put
revenir sur l'eau, comme il lit. Il était fort des amis du maréchal de
Villeroy, qui, après son retour dans la faveur du roi par
M\textsuperscript{me} de Maintenon, eut grand'peine à obtenir qu'il
revint à la cour. Le roi l'y reçut mal, et ne put revenir sur son
compte. Il était frère de ces deux scélérats de La Bourlie, dont il a
été parlé ici, où leur naissance et leur fortune a été expliquée.
Guiscard était bon homme, honnête homme, doux et d'un commerce agréable
et fort honorable. Avec ses biens, son cordon bleu, ses amis, car il en
avait, l'alliance de sa fille, il se pouvait passer de la cour et mener
une vie agréable\,; mais il avait de l'honneur et de l'ambition. Sa
disgrâce et plus encore la cause de sa disgrâce troublait tout son repos
et tous les agréments de l'état où sa fortune l'avait mis. La mort du
roi et le brillant du maréchal de Villeroy dans la régence avaient fait
renaître ses espérances. Il se flatta longtemps, je ne sais de quoi ni
pourquoi. Voyant enfin qu'on ne songeait à lui pour rien, il se retira
tout à fait en Picardie auprès de Chaulnes, dans une terre qui
s'appelait Magny, à qui il avait fait donner le nom de Guiscard, dont il
avait rendu la demeure fort agréable. La mélancolie l'y gagna de plus en
plus. Au bout de dix-huit mois, il eut un peu de goutte légère. Sa fille
l'alla voir\,; il quitta son appartement sans cause de caprice,
peut-être pis, et s'alla mettre dans une tour à l'autre bout de la cour.
Il y fut quelques jours sans sortir de sa chambre, où il ne se laissa
voir qu'à sa fille et aux valets purement nécessaires. Il ne lui
paraissait ni fièvre ni aucun autre mal, et cependant gardait son lit.
Sa fille, au bout de quelques jours, le pressa de se lever. Il lui
répondit que ce n'était plus la peine, et lui tint quelques discours
ambigus. La conclusion fut que, sans nul accident qui parût, il mourut
le soir de ce même jour à soixante-onze ou douze ans.

Caumartin, conseiller d'État et intendant des finances, mourut aussi en
ce même temps à soixante-cinq ou six ans. C'était un grand homme très
bien fait et de fort bonne mine\,; on voyait bien encore qu'il avait été
beau\,; il avait pris tous les grands airs et les manières du maréchal
de Villeroy, et s'était fait par là un extérieur également ridicule et
rebutant. Il avait l'écorce de hauteur d'un sot grand seigneur, il en
avait aussi le langage, et le ton d'un courtisan qui se fait parade de
l'être\,; ces façons lui aliénèrent beaucoup de gens. Il était fort
proche parent et ami intime du chancelier de Pontchartrain\,; il eut
toute sa confiance\,: tant qu'il fut contrôleur général toute la finance
passait par ses mains. C'est ce qui gâta encore ses façons. Le dedans
était tout autre que le dehors\,; c'était un très bon homme, doux,
sociable, serviable, et qui s'en faisait un plaisir, qui aimait la règle
et l'équité, autant que les besoins et les lois financières le pouvaient
permettre\,; et au fond honnête homme, fort instruit dans son métier de
magistrature et dans celui de finance, avec beaucoup d'esprit, et d'un
esprit accort, gai, agréable. Il savait infiniment d'histoire, de
généalogie, d'anciens événements de la cour. Il n'avait jamais lu que la
plume ou un crayon à la main\,; il avait infiniment lu, et n'avait
jamais rien oublié de ce qu'il avait lu, jusqu'à en citer le livre et la
page. Son père, aussi conseiller d'État, avait été l'ami le plus
confident et le conseil du cardinal de Retz. Le fils, dès sa première
jeunesse, s'était mis par là dans les compagnies les plus choisies et
les plus à la mode de ces temps-là. Cela lui en avait donné le goût et
le ton, et de l'un à l'autre il passa sa vie avec tout ce qu'il y avait
de meilleur en ce genre. Il était lui-même d'excellente compagnie, et
avait beaucoup d'amis à la cour et à la ville. Il se piquait de
connaître, d'aimer, de servir les gens de qualité, avec lesquels il
était à sa place, et point du tout glorieux, et parfaitement libre des
chimères de la robe, avec cela très honorable et même magnifique, point
conteur, mais très amusant, et quand on voulait un répertoire, le plus
instructif et le plus agréable. Il aimait et faisait fort bonne chère,
et il n'avait pas été indifférent pour les dames. C'est le premier homme
de robe qui ait hasardé de paraître en justaucorps et manteau de velours
dans les dernières années du roi. Ce fut d'abord une huée à
Versailles\,; il la soutint, on s'y accoutuma\,; nul autre n'osa
l'imiter de longtemps, et puis peu à peu ce n'est plus que velours pour
les magistrats, qui d'eux a gagné les avocats, les médecins, les
notaires, les marchands, les apothicaires et jusqu'aux gros procureurs.

L'abbé Dubois et M. le duc d'Orléans, celui-ci par faiblesse, l'autre
pour son chapeau, avaient toujours en tête leur déclaration pour faire
recevoir la constitution Unigenitus. Ils ne furent pas longtemps à
s'apercevoir de l'inutilité et du ridicule effet d'avoir avec tant de
pompe et de seigneurs bas et flatteurs, forcé le grand conseil à
l'enregistrer\,; ils se mirent bientôt après à reprendre leurs
négociations avec le parlement\,; elles durèrent trois mois, et ces
trois mois furent une mine et une abondante veine d'or pour le premier
président, qui vendait le régent à sa compagnie, pour s'y réaccréditer,
et qui enfin la vendit au régent. Quand il se crut au point qu'il
désirait avec le parlement aux dépens du régent, qui fournissait à ses
profusions et à ses brocards, et qu'il comprit qu'il était temps de
finir l'affaire, pour ne pas tarir cette veine, et ne pas passer l'hiver
à Pontoise, au hasard, s'il poussait le régent à bout, de lui fermer la
main, de se voir forcé à mettre bas sa table, et à tomber de l'énorme
splendeur qu'il avait soutenue jusqu'alors, il se fit valoir à sa
compagnie, fort lasse de l'éloignement de ses foyers qu'il la
ramenait\footnote{Le manuscrit porte ramenait\,; mais le sens demande
  ramènerait.} à Paris, si elle voulait enregistrer une déclaration
qu'ils sauraient toujours bien expliquer dans la pratique, et qui au
fond ne donnerait guère plus à la constitution, qui avait un si nombreux
parti dans l'Église, et toute l'autorité du gouvernement pour elle. Il
en vint à bout\,; le parlement l'enregistra le 4 décembre, et deux jours
après il eut son rappel à Paris, où il revint incontinent reprendre sa
séance ordinaire, et se remettre tout de bon à écouter et à juger les
procès.

Quelque temps avant le retour du parlement à Paris, on établit aux
Grands-Augustins une chambre pour juger en dernier ressort quantité de
procès restés depuis longtemps aux rôles et divers autres encore restés
en arrière. Armenonville fut choisi pour y présider, avec six autres
conseillers d'État ses cadets, dix maîtres des requêtes et un onzième
pour servir de procureur général. On douta si les parties s'y
présenteraient volontiers dans la crainte que le parlement de retour
prétendit invalider tout ce qui y aurait été instruit et jugé.
Néanmoins, peu à peu les affaires s'y portèrent. Le parlement de retour
consentit à cette juridiction extraordinaire, pour un temps, parce qu'il
sentit qu'il était si chargé et si arriéré de procès, à force de s'être
abandonné aux affaires publiques et à ne rien faire à Pontoise, qu'il
était indispensable d'y pourvoir autrement. Ce nouveau tribunal, qui
dura assez longtemps, se rendit recommandable par son équité, son
travail et son expédition\,; il vida tout ce qui y fut porté, et
Armenonville en particulier s'y acquit beaucoup d'honneur.

Vers le milieu du séjour du parlement à Pontoise, travaillant, une
après-dînée, seul avec M. le duc d'Orléans, il m'apprit que le premier
président lui avait demandé son agrément pour le mariage de sa fille
aînée arrêté avec le duc de Lorges. Ma surprise et ma colère me firent
lever brusquement et jeter mon tabouret à l'autre bout du petit cabinet
d'hiver où nous étions. Il n'y avait sorte de plaisirs essentiels que je
n'eusse faits toute ma vie à ce beau-frère, non pour l'amour de lui, car
je le connaissais bien, mais par rapport à M\textsuperscript{me} de
Saint-Simon. On a vu en son lieu que je l'avais fait capitaine des
gardes et ce qu'il m'en arriva, et comme j'obtins pour rien un régiment
pour son fils aîné à qui il n'en eût jamais acheté, et combien peu il en
fut touché. J'ajouterai ici qu'à la mort de M. le maréchal de Lorges, je
lui quittai près de dix mille écus qui, sans dispute ni difficulté,
revenaient à M\textsuperscript{me} de Saint-Simon, sur le brevet de
retenue de la charge de capitaine des gardes qu'eut le maréchal
d'Harcourt\,; et malgré une conduite étrange et misérable, j'avais
toujours très bien vécu avec lui. Je n'avais donc garde de m'attendre
qu'il choisît la fille d'un homme que je traitais en ennemi déclaré, à
qui je refusais publiquement le salut, duquel je parlais sans aucune
mesure et à qui je faisais des insultes publiques tout autant que
l'occasion s'en présentait, ce qui arrivait le plus ordinairement au
Palais-Royal, n'ayant guère ou point d'occasion de le rencontrer
ailleurs. Je ne me contraignis donc pas avec M. le duc d'Orléans sur un
mariage qui m'offensait si vivement. M. le duc d'Orléans n'osa trop rire
du torrent que je débondai, me voyant si outré\,; il trouva pourtant que
j'avais raison.

Je venais nouvellement de sauver une cruelle affaire au duc de Lorges.
Il avait une maison dans le village de Livry où il se croyait tout
permis. Non content de désoler Livry sur les chasses, et Livry en était
capitaine et seigneur du lieu avec qui je le raccommodai bien des fois,
il s'avisa d'ouvrir, devant une grille de son jardin, une route
prodigieusement large tout à travers de la forêt de Livry et de faire
cette expédition avec tant d'ouvriers qu'elle fut achevée avant qu'on
s'en fût aperçu. On peut juger des cris des officiers des eaux et forêts
et de l'intendant des finances qui les avait dans son département, et
des suites ruineuses et même personnelles de leurs procédures, si la
bonté de M. le duc d'Orléans pour moi ne leur eût imposé silence tout
aussitôt et fait rendre un arrêt du conseil antidaté qui ordonnait cette
ouverture et cette coupe de bois du roi. De cela et de tant d'autres
bottes que j'avais parées au duc de Lorges, et de tant d'autres choses
faites pour lui, tel fut le salaire. Je retournai à Meudon où j'appris
ce beau mariage à M\textsuperscript{me} de Saint-Simon qui en fut
consternée. Je lui déclarai qu'elle ni moi ne verrions jamais son frère,
ni celle qu'il allait épouser, et qu'elle fit savoir à
M\textsuperscript{me} la maréchale de Lorges et à M. et à
M\textsuperscript{me} de Lauzun que, s'ils signaient le contrat de
mariage ou s'ils assistaient à cette noce, nous ne les verrions de notre
vie. Dans le public, je m'expliquai sans aucune sorte de ménagement ni
en choses ni en termes. Le contrat ne fût point signé de
M\textsuperscript{me} la maréchale de Lorges ni de M. et de
M\textsuperscript{me} de Lauzun, et ils n'allèrent point à ce mariage
qui se fit à Pontoise avec toute la magnificence du premier président
qui y convia tout le parlement, lequel il fit signer au contrat de
mariage.

Parmi tout ce vacarme que je fis, rien n'échappa au premier président ni
aux siens. Au contraire, force regrets de ma colère, force désirs de
l'apaiser, force respects, malgré toute leur gloire. Il faut achever cet
épisode tout de suite. Après quelque temps et qu'ils se flattèrent que
leur conduite à mon égard, tandis que je ne me refusais rien, aurait pu
émousser ma colère, ils me firent parler par plusieurs de mes amis dans
les termes les plus propres à se faire écouter. Cela dura longtemps sans
autre réponse que mes propos accoutumés sur le beau-père et le gendre. À
la fin ce fut quelque chose de plus intime et de plus cher qui m'abattit
plutôt qu'il ne me gagna. M\textsuperscript{me} de Saint-Simon ne
cessait de répandre des larmes en silence\,; elle ne mangeait et ne
dormait plus\,; sa santé délicate s'altérait visiblement. Cet état, qui
ne pouvait se changer que par une réconciliation, fit en moi un combat
intérieur, dont les fougues et les élans ne se peuvent décrire entre ce
que je respectais et que j'aimais le plus tendrement, entre une douleur
continuelle qui la minait et qui me perçait le coeur, et de me
réconcilier avec deux hommes qui avec tant de raison m'étaient si
démesurément odieux, et qui ne m'étaient pas moins méprisables. Enfin,
pour abréger, je fis à la conservation de M\textsuperscript{me} de
Saint-Simon un sacrifice vraiment sanglant, et au bout de six ou sept
mois, la réconciliation se fit en cette sorte. Je consentis que le
contrat fût signé, et de voir la duchesse de Lorges à l'hôtel de Lauzun,
sans personne que la duchesse de Lauzun. Cela se passa debout en un
moment, et fort cavalièrement de ma part. Le lendemain le premier
président vint chez moi en robe de cérémonie, où il m'accabla de
compliments et de respects. Je fus sec, mais poli, comme je m'y étais
engagé. Les jours suivants M\textsuperscript{me} de Fontenilles sa
soeur, le bailli de Mesmes et leurs plus proches vinrent au logis où je
les reçus civilement, mais très froidement\,; le premier président y
revint encore sur ce que j'avais déclaré que je ne voulais point voir
son gendre. C'était lui pourtant qu'il fallait que je revisse pour
essuyer les larmes de M\textsuperscript{me} de Saint-Simon\,; et enfin
j'y consentis. Il vint chez moi, conduit par elle. Je le reçus fort mal,
quoique le moins mal que je pus gagner sur moi. J'allai après chez le
premier président qui me reçut avec des empressements et des civilités
extrêmes. Il n'épargna ni le terme de respect ni celui de
reconnaissance\,; en un mot, il continua d'oublier sa morgue, et se
répandit en bien dire.

M\textsuperscript{me} de Lorges et sa soeur étaient venues chez moi,
menées par M\textsuperscript{me} de Lauzun, dès que j'eus vu la duchesse
de Lorges à l'hôtel de Lauzun\,; puis peu à peu j'allai voir la soeur,
le frère et la belle-mère du premier président. Il désira avec grande
ardeur donner une espèce de repas de noce où je voulusse bien être avec
M\textsuperscript{me} de Saint-Simon, qu'il avait visitée dans son
appartement toutes les fois, et dès la première qu'il était venu chez
moi, et mes enfants aussi\,; enfin j'y consentis encore\,; le repas fut
excellent et magnifique, et accompagné, de la part du premier président
et des siens, de tout ce qui me pouvait plaire en façons et en discours.
De l'un à l'autre on se laisse conduire à tout. M\textsuperscript{me} de
Saint-Simon désira si fort que nous leur donnassions un repas aussi
comme de noce, qu'il fallut bien y consentir. Le premier président ne
l'osait espérer, et en parut transporté de joie. Il fut des mêmes
personnes qui avaient été de celui du premier président, et je m'y
donnai la torture pour y faire médiocrement bien. Ainsi finit la
division atroce qui me séparait du premier président, avec tant d'éclat
si continuellement soutenu depuis l'affaire du bonnet, et que ce mariage
avait comblée de nouveau. Dans la suite le premier président vint de
temps en temps chez moi, puis plus souvent, moi quelquefois chez lui,
jusqu'à la fin de sa vie\,; on peut croire qu'il n'y eut que de la
civilité et que la conversation n'était pas intéressante. Mais pour
M\textsuperscript{me} de Fontenilles nous nous accommodâmes d'elle et
elle de nous peu à peu, en telle sorte que nous sentîmes tout son
mérite, sa vertu, son esprit, les agréments et la sûreté de son
commerce, et que la liaison et l'amitié se forma étroite et a toujours
duré depuis.

Le duc de Brissac épousa en même temps M\textsuperscript{lle} Pécoil,
très riche héritière, dont le père était mort maître des requêtes, et la
mère était fille de Le Gendre, très riche négociant de Rouen. Le père de
Pécoil était un bourgeois de Lyon, gros marchand et d'une avarice
extrême. Il avait un grand coffre-fort rempli d'argent dans un fond de
cave, fermé d'une porte de fer à secret où on n'arrivait qu'en passant
d'autres portes. Il disparut un jour si longtemps que sa femme et deux
ou trois valets ou servantes qu'ils avaient le cherchèrent partout. Ils
savaient bien qu'il avait une cache, parce qu'ils l'avaient quelquefois
surpris descendant dans sa cave un martinet à la main, mais jamais
personne ne l'y avait osé suivre. En peine de ce qu'il était devenu, ils
y descendirent, enfoncèrent les dernières portes et trouvèrent enfin
celle de fer. Il fallut des ouvriers pour l'enfoncer ou l'ouvrir, en
attaquant les côtés de la muraille où elle tenait. Après un long travail
ils entrèrent et trouvèrent le vieil avare mort auprès de son coffre
fort, qui apparemment n'avait pu retrouver le secret de la serrure après
s'être enfermé en dedans, et n'avait pu l'ouvrir\,: fin bien horrible en
toutes manières\footnote{On a déjà vu cette anecdote plus haut.}. MM. de
Brissac ne sont pas délicats depuis longtemps en alliances, et toutefois
n'en paraissent pas plus riches. Les écus s'envolent, la crasse demeure.

Le Grand Seigneur avait nommé et fait partir un ambassadeur pour venir
complimenter le roi sur son avènement à la couronne. Comme c'est une
chose fort peu usitée à l'orgueil de la Porte, notre cour en fut
extrêmement flattée. Outre l'honneur et la considération des lieux
saints de la Palestine, l'intérêt du commerce et de la bannière de
France dans la Méditerranée, ne contribua pas moins à en être touché\,;
il débarqua à Toulon, et à cause de la peste on l'obligea à la
quarantaine, et on le fit venir par Toulouse à Bordeaux et de là à
Paris.

On était près d'ouvrir le congrès de Cambrai dont l'objet était de
régler ce qui ne l'avait pu être entre l'empereur et l'Espagne et
quelques suites de ce qui l'avait été à Bade. Saint-Contest, qui, comme
on l'a vu et pourquoi, avait été troisième ambassadeur plénipotentiaire
à la paix de Bade, le fut en premier à Cambrai avec Morville, fils
d'Armenonville, ambassadeur en Hollande. Toutes les puissances de
l'Europe y envoyèrent. Cette assemblée dura longtemps, où les cuisiniers
eurent plus d'affaires que leurs maîtres. Elle se sépara à la fin sans
avoir rien fait. Le cardinal Gualterio, avec qui j'étais en commerce
réglé toutes les semaines, m'écrivit pendant ce congrès une chose très
sensée\,: c'était de profiter de cette assemblée des ministres de toutes
les grandes puissances de l'Europe, pour convenir entre elles des
entrées et de la suite de leurs ambassadeurs dans toutes les cours, dont
la dépense toujours plus grande croissant toujours, à qui aura plus de
carrosses et d'équipages les plus magnifiques et le plus de
gentilshommes de suite, de riche et nombreuse livrée de toutes façons,
ruine les ambassadeurs en coûtant fort cher à leurs maîtres, de mettre
ainsi des bornes à l'émulation et à la dépense.

L'abbé de Maulevrier qui avait été aumônier du roi, dont il a été parlé
plus d'une fois ici, fit tant qu'il persuada à l'abbé Dubois d'envoyer
en Espagne Maulevrier, son neveu, qui était lieutenant général. Leur nom
est Andrault, fort léger\,: ils sont du Bourbonnais, originaires
d'autour de Lyon, très attachés de tout temps aux Villeroy, domestiques
de l'hôtel de Condé, et celui qui était mort lieutenant général des
armées navales et sa famille tout à M. et à M\textsuperscript{me} du
Maine. Ce n'était pas là des titres à faire valoir à M. le duc d'Orléans
pour être envoyé du roi en Espagne\,; néanmoins il le fut. On lui
joignit, mais sans titre, une espèce de financier marchand qui
s'appelait Robin, pour les affaires du commerce. On verra dans la suite
si j'ai le temps d'écrire mon ambassade en Espagne, qu'il lui en aurait
fallu encore un autre pour la négociation.

La maladie du pape, qu'on crut trop tôt désespérée, attira l'ordre à nos
cardinaux de se préparer diligemment à partir, et le retour du cardinal
de Polignac de son abbaye d'Anchin en Flandre, où on a vu qu'il était
exilé. L'alarme cessée suspendit leur départ, et le cardinal de Polignac
eut permission de saluer le roi et M. le duc d'Orléans, et de demeurer à
Paris en attendant des nouvelles de Rome plus pressantes.

L'année finit par le départ subit et secret de Law, qui n'avait plus de
ressources, et qu'il fallut enfin sacrifier au public. On ne le sut que
parce que le fils aîné d'Argenson, intendant à Maubeuge, eut la bêtise
de l'arrêter\footnote{Le marquis d'Argenson parle de ce fait dans ses
  Mémoires (édit. de 1825, p.~179)\,: «\,J'étais intendant de
  Valenciennes\,; je fis grand'peur à Law comme il traversait mon
  intendance pour fuir à l'étranger. Je le fis arrêter et le retins deux
  fois vingt-quatre heures à Valenciennes, ne le laissant partir que sur
  des ordres formels que je reçus de la cour.\,»}. Le courrier qu'il
envoya pour en donner avis lui fut redépêché sur-le-champ avec une forte
réprimande de n'avoir pas déféré aux passeports que M. le duc d'Orléans
lui avait fait expédier. Son fils était avec lui\,; ils allèrent à
Bruxelles, où le marquis de Prié, gouverneur des Pays-Bas impériaux, le
reçut très bien, et le régala\,; il s'y arrêta peu, gagna Liège et
l'Allemagne, où il alla offrir ses talents à quelques princes qui tous
le remercièrent. Après avoir ainsi rôdé, il passa par le Tyrol, vit
quelques cours d'Italie, dont pas une ne l'arrêta, et enfin se retira à
Venise, où cette république n'en fit aucun usage. Sa femme et sa fille
le suivirent quelque temps après\,; je n'ai point su ce qu'elles sont
devenues, ni même son fils. Law était Écossais, fort douteusement
gentilhomme, grand et fort bien fait, d'un visage et d'une physionomie
agréables, galant et fort bien avec les dames de tous pays où il avait
fort voyagé. Sa femme n'était point sa femme, elle était de bonne maison
d'Angleterre et bien apparentée, qui avait suivi Law par amour, en avait
eu un fils et une fille, et qui passait pour sa femme et en portait le
nom sans l'avoir épousé. On s'en doutait sur les fins\,: après leur
départ cela devint certain. Cette femme avait un oeil et le haut de la
joue couverts d'une vilaine tache de vin, du reste bien faite, haute,
altière, impertinente en ses discours et en ses manières recevant les
hommages, rendant peu ou point, et faisant rarement quelques visites
choisies, et vivait avec autorité dans sa maison. Je ne sais si son
crédit était grand sur son mari\,; mais il paraissait plein d'égards, de
soins et de respect pour elle. Tous deux avaient lors de leur départ
entre quarante-cinq et cinquante ans. Law laissa en partant sa
procuration générale au grand prieur de Vendôme et à Bully, qui avaient
bien gagné avec lui. Il avait fait force acquisitions de toutes sortes,
et encore plus de dettes, de façon que ce chaos n'est pas encore
débrouillé par une commission du conseil nommé pour régler ses affaires
avec ses créanciers. J'ai dit ici ailleurs, et je le répète, qu'il n'y
eut ni avarice ni friponnerie en son fait. C'était un homme doux, bon,
respectueux, que l'excès du crédit et de la fortune n'avait point gâté,
et dont le maintien, l'équipage, la table et les meubles ne purent
scandaliser personne. Il souffrit avec une patience et une suite
singulière toutes les traverses qui furent suscitées à ses opérations,
jusqu'à ce que vers la fin, se voyant court de moyens, et toutefois en
cherchant et voulant faire face, il devint sec, l'humeur le prit, et ses
réponses furent souvent mal mesurées. C'était un homme de système, de
calcul, de comparaison, fort instruit et profond en ce genre, qui, sans
jamais tromper, avait partout gagné infiniment au jeu à force de
posséder, ce qui me semble incroyable, la combinaison des cartes.

Sa banque, comme je l'ai dit ailleurs, était une chose excellente, dans
une république ou dans un pays comme l'Angleterre, où la finance est en
république. Son Mississipi, il en fut la dupe, et crut de bonne foi
faire de grands et riches établissements en Amérique. Il raisonnait
comme un Anglais, et ignorait combien est contraire au commerce et à ces
sortes d'établissements la légèreté de la nation, son inexpérience,
l'avidité de s'enrichir tout d'un coup, les inconvénients d'un
gouvernement despotique, qui met la main sur tout, qui n'a que peu ou
point de suite, et où ce que fait un ministre est toujours détruit et
changé par son successeur. Sa proscription d'espèces, puis de
pierreries, pour n'avoir que du papier en France, est un système que je
n'ai jamais compris ni personne, je pense, dans tous les siècles qui se
sont écoulés depuis celui d'Abraham, qui acheta un sépulcre en argent
pour Sara quand il la perdit, pour lui et pour ses enfants. Mais Law
était un homme à système, et si profond, qu'on n'y entendait rien,
quoique naturellement clair et d'une élocution facile, quoiqu'il y eût
beaucoup d'Anglais dans son français. Il vécut plusieurs années à Venise
avec fort peu de bien, et y mourut catholique, ayant vécu honnêtement,
quoique fort médiocrement, sagement et modestement, et reçut avec piété
les sacrements de l'Église. Ainsi se termina l'année 1720.

\hypertarget{chapitre-v.}{%
\chapter{CHAPITRE V.}\label{chapitre-v.}}

1721

~

{\textsc{1721.}} {\textsc{- Chaos des finances.}} {\textsc{- Retraite de
Pelletier de Sousy.}} {\textsc{- Conseil de régence curieux sur les
finances et la sortie de Law du royaume.}} {\textsc{- Réflexions sur ce
conseil de régence.}} {\textsc{- Prince de Conti débanque Law.}}
{\textsc{- Continuation de ce conseil de régence, orageux entre le
régent et M. le Duc, à l'occasion de la retraite de Law.}} {\textsc{- M.
le duc d'Orléans veut de nouveau ôter au maréchal de Villeroy la place
de gouverneur du roi et me la donner.}} {\textsc{- Il s'y associe M. le
Duc.}} {\textsc{- Je refuse.}} {\textsc{- Le combat dure plus d'un
mois.}} {\textsc{- Je demeure si ferme, que le maréchal de Villeroy
conserve sa place auprès du roi, faute de qui la remplir.}} {\textsc{-
Sa misère là-dessus.}} {\textsc{- Le maréchal de Villeroy découvre le
péril qu'il a couru pour sa place.}} {\textsc{- Il ne me pardonne pas
d'avoir pu la remplir, si je l'avais voulu.}} {\textsc{- Je le
méprise.}}

~

Depuis le changement du ministère des finances et la disjonction de tous
les droits et revenus royaux d'avec la compagnie des Indes, excepté la
ferme du tabac qui lui demeura unie, tout était resté dans l'inaction
qui, jointe au défaut de confiance, achevait de perdre le crédit du roi
et laissait une incertitude extrême dans la fortune des particuliers.
Tout en ce genre se passait entre le régent et La Houssaye, nouveau
contrôleur général qui, outre le chaos des finances, n'y avait trouvé ni
registres, ni notions, ni qui que ce fût en aucune place, ni personne
qui s'y présentât, parce qu'avec Law étaient tombés ceux qu'il y avait
mis. Toute circulation se trouvait arrêtée, enfin un épuisement et une
confusion au delà de tout ce qu'il s'en {[}peut{]} imaginer. Le duc de
Noailles, lorsqu'il était chargé des finances, avait montré l'exemple
d'en communiquer les affaires tout le moins qu'il le pouvait au conseil
de régence, quoique vrai conseil alors, surtout dans la fin de son
administration que ce conseil commençait à tomber. Argenson qui lui
succéda, avec l'autorité des sceaux, l'imita par une soustraction
entière qui fut incontinent suivie de celle de toutes les autres
véritables matières. Law, qui dans la suite administra les finances en
diverses façons, passa jusqu'à ne donner pas même connaissance au
conseil de régence des édits, des déclarations ni des arrêts qui étaient
affichés en foule par les rues. La Houssaye commença son administration
de la môme manière, et notamment par disjoindre de la compagnie des
Indes tout ce qui y avait été uni des droits et revenus royaux.

Résolu d'aller plus avant, il crut apparemment devoir s'appuyer du nom
du conseil de régence, quelque vain que ce conseil fût devenu, tellement
que la première fois qu'il y entra en qualité de contrôleur général des
finances, ce fut un jour ou il se passa des choses qui méritent bien
d'être rapportées, que j'écrivis dès que j'en fus sorti pour n'en pas
perdre une exacte mémoire, le voici\,:

CONSEIL LE RÉGENCE TENU AUX TUILERIES LE DIMANCHE 24 JANVIER 1721, À
QUATRE HEURES APRÈS-MIDI\,; PRÉSENTS ET SÉANTS EN CETTE SORTE\,:

LeRoi.

M. le duc d'Orléans, régent.

M. le prince de Conti.

M. le Duc, chef du conseil de régence.

M. le chancelier.

M. le comte de Toulouse.

M. le duc de La Force.

M. le duc de Saint-Simon.

M. le maréchal duc de Villeroy.

M. le maréchal duc de Grammont.

M. le duc de Noailles.

M. le duc de Saint-Aignan.

M. le duc d'Antin.

M. le maréchal duc de Villars.

M. le maréchal d'Estrées.

M. le maréchal duc de Tallard.

M. le maréchal de Bosons était malade et absent.

M. le maréchal d'Huxelles.

M. l'ancien évêque de Troyes (Bouthillier).

M. de Torcy.

M. de La Vrillière, secrétaire d'État.

M. l'archevêque de Rouen (Besons).

M. l'archevêque de Cambrai (Dubois), secrétaire d'État.

M. de La Houssaye, contrôleur général, mandé.

M, d'Armenonville, secrétaire d'État.

M. le marquis de Canillac.

M. Le Blanc, secrétaire d'État.

M. le duc de Chartres.

M. Le Pelletier de Sousy, doyen du conseil, qui était aussi du conseil
de régence, avait obtenu depuis quatre jours la permission de ne plus
faire aucune fonction de ses emplois, à cause de son âge, qui passait
quatre-vingts {[}ans{]}, mais avec la tête bonne et la santé aussi,
chagrin contre des Forts\footnote{Pelletier ou Le Pelletier des Forts
  devint contrôleur général des finances en 1726.}, son {[}frère{]},
avec qui il logeait, et alla se retirer à Saint-Victor, où l'ennui le
gagna bientôt et peut-être le repentir.

Tout le monde assis, M. le duc d'Orléans dit au roi qu'il y avait une
affaire fort importante à délibérer qui regardait la compagnie des
Indes, et qui concernait les papiers royaux, laquelle méritait toute
l'attention du conseil, dont M. de La Houssaye allait rendre compte. Il
ajouta vaguement deux périodes, après quoi M. le comte de Toulouse
rapporta une bagatelle concernant une augmentation à la ville de
Saint-Malo, laquelle finie, le régent donna la parole à La Houssaye.

En cet instant, M. le Duc se leva, contre l'usage de ceux qui opinent ou
qui veulent parler, fit signe à La Houssaye d'attendre, se rassit et dit
au roi qu'il n'était informé que de ce matin même de ce qui se devait
présentement proposer au conseil\,; qu'intéressé comme il l'était avec
la compagnie des Indes, il s'était d'abord proposé de ne point opiner,
pour éviter que ce qu'il dirait pût être interprété d'intérêt
particulier\,; mais que depuis il avait estimé plus convenable de se
mettre en liberté pour pouvoir dire ce qu'il croyait utile pour le bien
de l'État\,; qu'il avait eu et déposé quinze cents actions\,; qu'en
outre il en avait encore quatre-vingt-quatre sous son nom qui ne lui
appartenaient pas\,; que, si celui qui en était chargé se fût trouvé
chez lui, il aurait déjà porté les siennes à M. le duc d'Orléans pour
qu'il eût la bonté de les remettre à Sa Majesté, ou à la compagnie, ou
bien de les brûler, comme il aurait voulu\,; que ce qu'il n'avait pu
exécuter cejourd'hui il le ferait le lendemain dans la matinée, et que,
le déclarant en si bonne compagnie, il se croyait dès lors pouvoir
compter hors d'intérêt et en état de pouvoir dire son sentiment sur la
matière qu'on avait à traiter, d'autant plus qu'il n'avait jamais été
pour la compagnie qu'autant qu'il avait cru le devoir pour le service de
Sa Majesté et pour le bien de ses sujets.

M. le prince de Conti prit alors la parole et dit que tout le monde
savait bien que depuis longtemps il n'avait point d'actions, que ce
qu'il en avait eu il l'avait rendu à Law, et qu'il offrait de remettre
le duché de Mercœur qui en était le bénéfice. M. le Duc répondit assez
bas que des offres vagues ne suffisaient pas, qu'il en fallait la
réalité et l'exécution.

La Houssaye commença son discours sur les comptes de la compagnie avec
le roi\,: tout son rapport fut parfaitement beau. Il conclut que la
compagnie fût déclarée redevable de tous les billets de banque, et que
ceux qui ne seraient point éteints par les quinze cents millions de
récépissés retirés par la compagnie, elle devrait au roi l'excédant,
attendu que le roi s'en charge\,; que c'était une suite naturelle de
l'union qui avait été faite de la banque à la compagnie des Indes au
mois de février dernier, où le roi avait donné à la compagnie le
bénéfice et la charge de la banque.

M. le Duc prit alors la parole, et dit que, par la même assemblée de la
compagnie, il avait été réglé qu'on ne ferait plus d'achats d'actions,
et qu'il ne serait point fait de billets de banque, sinon par une
assemblée générale\,; qu'il n'y en a point eu\,; que, s'il a été fait
des achats d'actions et de billets, ç'a été par ordres du roi et arrêts
du conseil du propre mouvement, qu'ainsi c'est le roi qui en doit être
tenu.

M. le duc d'Orléans a répliqué que M. Law était l'homme de la compagnie,
aussi bien que celui du roi\,; que ce qu'il avait fait, il le croyait du
bien de la compagnie\,; que cela est si vrai que dans l'arrêt qui
ordonne l'achat des actions, il est dit que la dividende\footnote{Saint-Simon
  fait ce mot féminin, sous-entendant le substantif partie.} accroîtra
aux autres actionnaires\,; que c'était aussi Law qui avait fait faire
des billets de banque pour cet emploi, afin de faire valoir les actions.

M. le Duc a répondu que M. Law ne pouvait pas engager la compagnie,
puisqu'il était l'homme du roi comme contrôleur général\,; qu'il n'y
avait d'arrêts que pour douze cents millions de billets de banque\,;
qu'il avait même été dit dans l'assemblée générale qu'on supprimerait
les billets de banque de dix livres\,; que, loin de cela, on en avait
fait pour plus de cent millions des mêmes, et qu'il y avait dans le
public pour plus de deux milliards sept cents millions de billets de
banque\,; que cela ne pouvait jamais être regardé comme un fait de la
compagnie.

M. le duc d'Orléans expliqua que l'excédant des billets de banque avait
été fait par des arrêts du conseil, rendus sous la cheminée\,; que le
grand malheur venait de ce que M. Law en avait fait pour douze cents
millions au delà de ce qu'il en fallait\,; que les premiers six cents
millions n'avaient pas fait grand mal, parce qu'on les avait enfermés
dans la banque\,; mais qu'après l'arrêt du 21 mai dernier, lorsqu'on
donna des commissaires à la banque, il se trouva pour autres six cents
millions de billets de banque que Law avait fait faire et répandu dans
le public, à son insu de lui régent, et sans y être autorisé par aucun
arrêt, pour quoi M. Law méritait d'être pendu\,; mais que, lui régent
l'ayant su, il l'avait tiré d'embarras par un arrêt qu'il fit expédier
et antidater, qui ordonnait la confection de cette quantité de billets.

Là-dessus M. le Duc dit à M. le régent\,: «\,Mais, monsieur, comment,
sachant cela, l'avez-vous laissé sortir du royaume\,? --- C'est vous,
monsieur, répliqua le régent, qui lui en avez fourni les moyens. --- Je
ne vous ai jamais demandé, répondit M. le Duc, de le faire sortir du
royaume. --- Mais, insista le régent, c'est vous-même qui lui avez
envoyé les passeports. --- Il est vrai, monsieur, répondit M. le Duc,
mais c'est vous qui me les avez remis pour les lui envoyer\,; mais je ne
vous les ai jamais demandés, ni qu'il sortit du royaume. Je sais qu'on
m'a voulu jeter le chat aux jambes dans le public là-dessus, et je suis
bien aise d'expliquer ici ce qui en est puisque j'en ai l'occasion. Je
me suis opposé qu'on mit M. Law à la Bastille, ou dans quelque autre
prison, comme on le voulait, parce que je ne croyais pas qu'il fût de
votre intérêt de l'y laisser mettre après vous en être servi comme vous
avez fait\,; mais je ne vous ai jamais demandé qu'il sortît du royaume,
et, je vous prie, monsieur, de vouloir bien dire en la présence du roi,
et devant tous ces messieurs, si je vous l'ai jamais demandé. --- Il est
vrai, répondit M. le régent, que vous ne me l'avez pas demandé\,; je
l'ai fait sortir, parce que j'ai cru que sa présence en France nuirait
au crédit publie et aux opérations qu'on voulait faire. --- Je suis,
reprit M. le Duc, si éloigné, monsieur, de vous l'avoir demandé, que, si
vous m'aviez fait l'honneur de m'en demander mon avis, je vous aurais
conseillé de vous bien garder de le laisser sortir du royaume.\,»

La Houssaye continua ensuite son rapport. Il lut la requête de la
compagnie à ce que la banque lui fût unie, et que tous les profits
d'icelle lui fussent donnés. On lut aussi les deux articles de l'arrêt
du conseil qui intervint le lendemain de la requête qui faisaient à la
question, et La Houssaye conclut que la compagnie serait débitrice
envers le roi des billets de banque.

Armenonville proposa là-dessus une opinion que la compagnie fût
entendue. Le maréchal d'Estrées appuya cet avis\,; le régent y fit des
objections très fortes, et tout le conseil, excepté ces deux, furent de
l'avis de M. de La Houssaye.

Ensuite il proposa que, comme il y avait plusieurs particuliers qui
avaient mis tout leur bien dans les actions sur la foi publique, il
n'était pas juste que par la dette immense de la compagnie envers le roi
ils se trouvassent ruinés, et que réciproquement ceux qui étaient sortis
de la compagnie dans le bon temps, qui avaient converti leurs actions en
billets ou qui les avaient achetées à vil prix sur la place, ou
employées en rentes perpétuelles ou viagères, ou en comptes en banque,
profitassent du malheur des actionnaires de bonne foi\,; qu'ainsi il
fallait nommer des commissaires pour liquider tous ces papiers et
parchemins, et annuler ceux qui ne procéderaient point de biens réels.

M. le Duc dit à cela\,: «\,Il y a quatre-vingt mille familles au moins
dont tout le bien consiste en ces effets\,: de quoi vivront-elles
pendant cette liquidation\,?» La Houssaye répondit qu'on nommerait tant
de commissaires, que cela serait bientôt fait.

M. le Duc dit ensuite que, s'il y avait des gens à liquider, ce
n'étaient pas ceux qui étaient anciens porteurs des effets publics\,;
que le discrédit les ruinerait assez\,; mais qu'il fallait chercher ceux
qui avaient réalisé en argent ou en terres ou en maisons, ou qui avaient
vendu leurs meubles à des prix exorbitants, ou qui avaient arrangé leurs
affaires aux dépens de leurs créanciers.

La Houssaye dit qu'on les taxerait aussi par rapport à ceux qui avaient
des immeubles, mais que, par rapport à ceux qui avaient réalisé en
argent, c'était une chose fâcheuse par la peine qu'il y avait à les
connaître\,; qu'il arriveront cependant un bien de l'arrangement qu'on
proposait aujourd'hui, parce que le roi reprenant un nouveau crédit par
la liquidation, et absorbant une partie des dettes, les réaliseurs en
argent le mettraient au jour pour le prêter au roi, vu la facilité des
billets payables au porteur.

M. de La Houssaye continua son discours. Après qu'il fut fini, il fut
arrêté tout d'une voix qu'il serait nommé des commissaires pour liquider
les rentes sur le roi tant perpétuelles que viagères, les actions
rentières et intéressées, les comptes en banque et les billets de
banque.

M. le duc d'Orléans dit qu'il fallait faire un règlement qui serait
porté au premier conseil de régence pour prescrire aux commissaires les
règles qu'on devait tenir, après quoi il ne s'en mêlerait en aucune
façon, renverrait tout aux commissaires, et ne ferait grâce à personne.

M. le Duc lui dit là-dessus que ce serait le moyen que tout se passât
dans la règle\,; sur quoi le régent, s'adressant au roi, le supplia de
lui permettre de dire qu'il lui avait défendu de s'en mêler, et ordonné
de laisser tout faire par les commissaires.

Le maréchal de Villeroy s'écria, en s'adressant à M. le duc d'Orléans\,:
«\, N'êtes-vous pas revêtu de toute son autorité, parlant de celle du
roi, et n'en avez-vous pas aussi toute la confiance\,?» et à l'instant
on leva le conseil.

On a omis plusieurs propos de ceux qui n'ont aucune importance, mais il
ne faut pas oublier que le comte de Toulouse offrit ses actions, que le
régent ne voulut pas accepter, comme provenant effectivement des
remboursements qu'il avait reçus.

Le duc d'Antin déclara aussi qu'il en avait quatre cents qu'il
rapporterait le lendemain.

L'étonnement fut grand dans tous ceux qui se trouvèrent à ce conseil.
Personne n'ignorait en gros le désordre des finances\,; mais le détail
de tant de millions factices, qui ruinaient le roi ou les particuliers,
ou pour mieux dire l'un et l'autre, effraya tout le monde. On vit alors
à découvert où avait conduit un jeu de gobelets, dont toute la France
avait été séduite, et quelle avait été la prodigalité du régent, par la
facilité de battre monnaie avec du papier, et de tromper ainsi l'avidité
publique. Il y fallait un remède, parce que les choses étaient arrivées
à un dernier période, et ce remède, qui allait au dernier détriment des
actionnaires et des porteurs des billets de banque, ne se pouvait
trouver que par le dévoilement de tout le mal, si longtemps tenu caché,
autant qu'il avait été possible, pour que chacun vît enfin où on en
était au vrai, et la nécessité pressante aussi bien que les difficultés
du remède.

Depuis l'arrêt du 22 mai, qui fut l'époque de la décadence de ce qui
était connu sous les noms de Mississipi et de banque, et la perte de
toute confiance par la triste découverte qu'il n'y avait plus de quoi
faire face au payement des billets, par leur excédant prodigieux au delà
de l'argent, chaque pas n'avait été qu'un trébuchement, chaque opération
qu'un palliatif très faible. On n'avait pu chercher qu'à gagner des
jours et des semaines, dans des ténèbres qu'on épaississait à dessein,
dans l'horreur qu'on avait de laisser voir au jour tant de séduction et
de monstres de ruine publique. Law ne pouvait se laver à la face du
monde d'en avoir été l'inventeur et l'instrument, et il aurait couru
grand risque, au moment de ce terrible et public dévoilement\,; et M. le
duc d'Orléans, qui, pour suffire à sa propre facilité et prodigalité, et
satisfaire à l'avidité prodigieuse de chacun, avait forcé la main à Law
et l'avait débanqué de tant de millions, au delà de tous moyens d'y
faire face, et l'avait précipité dans cet abîme, ne pouvait se mettre au
hasard de l'y laisser périr, et moins encore, pour le sauver, se
déclarer le vrai coupable. Ce fut donc pour se tirer de ce premier et si
mauvais pas, qu'il fit sortir Law du royaume, lorsqu'il se vit acculé et
forcé de montrer à la lumière l'état des finances et de cette énorme
gestion qui n'était que tromperie. Cette manifestation qui intéressait
si fort les actionnaires et les porteurs de billets de banque en
général, mais bien plus vivement ceux qui les tenaient de leur autorité
ou de leur faveur, et qui n'en pouvaient montrer d'autre origine, les
mit tous au désespoir. Les plus importants, comme les princes du sang,
les plus avant dans ces affaires, comme d'Antin, le maréchal d'Estrées,
Lassai, M\textsuperscript{me} la Duchesse, M\textsuperscript{me} de
Verue et d'autres en petit nombre, qui y avaient si gros, et dont les
profits jusqu'alors avaient été immenses, avaient, de force ou
d'industrie, arrêté cette manifestation tant qu'ils avaient pu, soutenu
ce puissant mur qui s'écroulait malgré eux, et suspendu le moment si
funeste pour eux. Comme ils savaient à peu près le fond des choses, ils
voyaient que le moment qu'elles seraient connues finirait ces gains
prodigieux et mettrait à néant les papiers dont ils s'étaient farcis à
toutes mains et pur profit, sans y avoir mis un sou du leur pour les
acquérir. C'est ce qui engagea M. le duc d'Orléans à leur cacher le jour
de cette manifestation, pour éviter d'être importuné d'eux pour différer
ce qui ne pouvait plus l'être, et pour, en les surprenant, leur ôter le
temps de se préparer à former des difficultés et des réponses aux
opérations que La Houssaye avait à proposer à leurs dépens. C'est aussi
ce qui mit M. le Duc en fureur, et qui causa cette scène étrange entre
lui et M. le duc d'Orléans, qui scandalisa et qui effraya tous ceux qui
dans ce conseil en furent témoins\,; tous deux y firent un mauvais
personnage.

M. le Duc débuta par une vaine parade de la remise de ses actions, qu'il
ne pouvait plus garder, parce qu'elles étaient sans origine, et il ne
fit qu'en manifester l'énorme quantité. Il crut par là imposer et se
mettre en liberté de protéger la compagnie de toutes ses forces, parce
qu'il y avait le plus gros intérêt personnellement, ainsi que
M\textsuperscript{me} la Duchesse sa mère. Personne ne l'ignorait, aussi
n'imposa-t-il à personne. Il haïssait et méprisait le prince de Conti au
dernier point. Il est vrai qu'en cela il était du sentiment unanime.
Aussi ne put-il pas s'empêcher de relever l'offre de la remise du duché
de Mercœur, volé à Lassai par un retrait\footnote{Action en justice, par
  laquelle on retirait un héritage qui avait été vendu.} et un procès
indigne, offre qu'il était bien sûr qui ne serait pas acceptée. Ce
prince avait raison d'avancer que tout le monde savait bien qu'il
n'avait point d'actions. Mais un peu de jugement l'aurait retenu de
faire une protestation qui faisait souvenir tout le monde qu'il avait
porté le premier et le plus mortel coup à la banque, en se faisant tout
à coup rembourser en argent de tout son papier, dont Law ne s'est pu
relever depuis. On vit arriver publiquement à l'hôtel de Conti quatre
surtouts\footnote{Charrettes qui servaient à porter les bagages.}
chargés d'argent, et le prince de Conti pendu à ses fenêtres pour les
voir entrer chez lui.

M. le duc d'Orléans, qui de goût et depuis par nécessité vivait de ruses
et de finesses, crut avoir fait merveilles d'avoir chargé M. le Duc des
passeports de Law, et d'avoir caché ce qui se devait traiter dans ce
conseil de régence. Il voulait affubler M. le Duc de la retraite de Law
hors du royaume, et le prendre au dépourvu en ce conseil, pour lui ôter
les moyens de contredire. Il en fut cruellement la dupe\,; la matière
touchait à M. le Duc d'un si grand intérêt, qu'il était par lui, et par
d'autres principaux intéressés, continuellement alerte sur ce qui devait
se proposer, et il arriva qu'il fut assez tôt averti pour bien apprendre
sa leçon. La hardiesse et la fermeté ne lui manquaient pas\,; il n'avait
rien à craindre, il connaissait d'ailleurs par une expérience
continuelle l'extrême faiblesse de M. le duc d'Orléans, il en voulut
profiter, et puisque tout ce mystère d'iniquité se devait enfin révéler
en présence du roi et du conseil (et nombreux comme il l'était c'était
dire au public), il se proposa de ne garder aucun ménagement pour tirer
son épingle du jeu, faire retomber tout sur M. le duc d'Orléans, et se
montrer soi comme le beau personnage, piqué de plus du secret qui lui
avait été fait de ce qui se devait proposer en ce conseil, plus encore
peut-être de la proposition même si contraire à la compagnie, et au
grand intérêt qu'il y avait\,; piqué de plus de ce que M. le duc
d'Orléans avait adroitement fait passer à Law ses passeports par lui,
pour donner lieu au monde de se persuader que M. le Duc les avait
demandés, conséquemment que c'était lui qui avait obtenu de M. le duc
d'Orléans sa sortie du royaume. Aussi fut-ce là-dessus qu'il pressa
impitoyablement M. le duc d'Orléans, qu'il l'interpella, et qu'il le
força d'avouer qu'il ne lui avait jamais demandé cette sortie, qu'il
protesta que, s'il en avait été consulté, il n'en aurait jamais été
d'avis, et qu'il reprocha si durement à M. le duc d'Orléans d'avoir
laissé sortir Law du royaume, après avoir fait de son chef pour six
cents millions de billets de banque contre les défenses si expresses de
les multiplier davantage. Ce conseil donc nous apprit deux choses\,: que
Law était mis à la Bastille sans M. le Duc, et qu'à l'insu du régent Law
avait fait et répandu dans le public pour six cents millions de billets
de banque, non seulement sans y être autorisé par aucun arrêt, mais
contre les défenses expresses.

Pour la première, je ne sais qui avait pu donner un conseil si dangereux
à M. le duc d'Orléans, qui au ton qu'il avait laissé prendre au
parlement, et que le parlement ne quittait point malgré le lit de
justice et son voyage de Pontoise, aurait profité du désordre connu des
finances et de leur incroyable déprédation, et plus encore du
mécontentement public pour en prendre connaissance et se venger enfin de
Law, qui depuis si longtemps était sa bête, et par lui de M. le duc
d'Orléans, qui se serait trouvé bien empêché, et peut-être hors d'état
de le tirer de prison, après l'y avoir mis, et de l'arracher au
parlement qui se serait fait honneur et délice de le faire pendre malgré
le régent. Il y avait bien de quoi, puisque le régent acculé par M. le
Duc, l'avoua en plein conseil, et que, pour le tirer de péril, il avait
fait rendre un arrêt du conseil antidaté, qui ordonnait cette confection
si prodigieuse de billets de banque faits et répandus par Law de sa
propre autorité. Mais quel aveu d'un régent du royaume, en présence du
roi et d'un si nombreux conseil, dont la plupart ne lui étaient rien
moins qu'attachés\,! Et à qui espéra-t-il avec quelque raison de
persuader que Law eût fait un coup si hardi, et de cette importance, à
l'insu de lui régent, son seul appui contre le public ruiné, et contre
le parlement, qui ne cherchait qu'à le perdre, et cela, pour la première
opération qu'il eût jamais faite, sans l'aveu et l'approbation du
régent\,? Voilà pourtant oh les finesses dont ce prince se repaissait le
conduisirent, et le dépit et la férocité de M. le Duc le forcèrent à un
si étonnant aveu, et si dangereux, en présence du roi et d'une telle
assemblée. J'en frémis en l'entendant faire, et il est incroyable que ce
terrible aveu n'ait pas eu la moindre des suites que j'en craignis.

Pour la personne de Law, M. le Duc, tout bouché qu'il fût de soi-même,
était trop éclairé par le grand intérêt qu'il avait au papier, et trop
bien conseillé par les siens qui n'y en avaient pas un moindre, qui
étaient habiles et avaient les yeux bien ouverts, pour laisser mettre
Law en prison, exposé à des suites aisément funestes, à tout le moins
destructives de ce qu'ils comptaient bien sauver du naufrage et que par
l'événement ils en sauvèrent en effet. À l'égard de la sortie de Law
hors du royaume, c'est une obscurité entre M. le duc d'Orléans et M. le
Duc, que je n'ai pu démêler. Bien ai-je expliqué ci-dessus les raisons
qui m'ont paru celles qui engagèrent M. le duc d'Orléans à faire sortir
Law du royaume, et sa petite finesse de lui en faire mettre les
passeports entre les mains par M. le Duc, pour se décharger sur lui de
cette sortie\,: car de tout cela M. le duc d'Orléans ne m'en dit rien,
et la chose faite, je ne cherchai pas à en rien apprendre de lui\,; mais
que M. le Duc, qui avait pour ses trésors de lui et des siens le même
intérêt de ne pas exposer Law, non seulement à sa perte, mais encore à
la nécessité de répondre juridiquement, et de parler, comme on dit des
criminels, fût contraire à sa sortie du royaume, j'avoue que c'est ce
que je n'entends pas\,; moins encore qu'y étant si contraire, il ne
l'ait pas témoigné à M. le duc d'Orléans, et fait effort pour l'empêcher
lorsqu'il reçut de lui les passeports pour les remettre à Law, dont
l'occasion était si naturelle, puisqu'il savait bien que ces passeports
étaient pour sortir du royaume\,; qu'il ne l'ait pas fait alors, cela
est clair, puisqu'il ne s'en serait pas tu en ce conseil, et d'autre
part, que M. le duc d'Orléans, si malmené par lui sur cette sortie, ne
lui ait pas reproché ce silence en lui remettant les passeports, c'est
encore ce que je ne puis comprendre.

Autre chose encore difficile à entendre. Quelque bouché et peu préparé
que pût être M. le Duc à cette remise des passeports entre ses mains
pour les donner à Law, comment voulut-il s'en charger, et comment ne
sentit-il pas le but de ce passage par ses mains\,? Quelle autre raison
de ce passage put-elle se présenter à lui\,? et tout homme en place de
finance, ou Le Blanc, ou un autre secrétaire d'État, n'étaient-ils pas
aussi bons et bien plus naturels que non pas M. le Duc, pour remettre à
Law ses passeports\,? En un mot, ce sont des ténèbres que j'avoue que je
n'ai pu percer. Du reste, M. le Duc était venu bien préparé pour
soutenir la compagnie en laquelle lui et les siens se trouvaient si
grandement intéressés. Aussi faut-il convenir qu'il plaida bien cette
cause, et qu'il n'obtint rien de plausible de tout ce qu'il se pouvait
dire en sa faveur. Le rare est qu'après une scène si forte, si poussée,
si scandaleuse, si publique, il n'y parut pas entre M. le Duc et M. le
duc d'Orléans. Le régent sentait le poids énorme dont sa gestion était
chargée par la confiance aveugle jusqu'au bout, et la protection si
déclarée qu'il avait donnée à Law envers et contre tous. Il était
faible, je le dis à regret\,; il craignait M. le Duc, ses fougues, sa
férocité, son peu de mesure, quoique d'ailleurs il connût bien le peu
qu'il était. Cette débonnaireté, que je lui ai si souvent reprochée, lui
fit avaler ce calice comme du lait, et le porta à vivre à l'ordinaire
avec M. le Duc pour ne le point aigrir davantage, et à ne l'aliéner pas
de lui. À l'égard de M. le Duc, ce n'était pas à lui à se fâcher, il
avait poussé M. le duc d'Orléans à bout sans le plus léger ménagement,
toujours l'attaquant, toujours le faisant battre en retraite, jusqu'à
lui avoir arraché l'aveu le plus étonnant et le plus dangereux. Il était
donc content de l'issue de ce combat d'homme à homme, mais il n'avait
garde de l'être des résolutions prises au conseil, quoi qu'il eût pu
dire en faveur de la compagnie, et par là il sentit le besoin qu'il
aurait de M. le duc d'Orléans pour soi et pour les siens, pour n'être
pas enveloppés dans la fortune commune des porteurs de papiers, et pour
sauver les leurs du naufrage, comme il arriva en effet\,; car ces quinze
cents actions de la remise desquelles il fit tant de parade, quelque
énorme qu'en fit le nombre, n'étaient rien en comparaison de celles qui
lui restaient sous d'autres formes, et pareillement à
M\textsuperscript{me} la Duchesse, à Lassai, à M\textsuperscript{me} de
Verue, et à d'autres des siens, et qui profitèrent depuis si
furieusement et pour longtemps encore. Ce n'est donc pas merveilles si,
après une si étrange scène où il avait eu tout l'avantage sur M. lé duc
d'Orléans, il ne chercha depuis qu'à la lui faire oublier.

La fin de ce conseil ne fut pas plus heureuse pour M. le duc d'Orléans.
Il s'y montra battu de l'oiseau, en protestant, je n'oserais dire
bassement, qu'il laisserait faire aux commissaires la liquidation dont
ils seraient chargés, en pleine liberté, sans s'en mêler\,; encore pis,
quand M. le Duc lui fit comme une nouvelle injure par la façon dont il
l'approuva et l'y exhorta en deux mots si énergiques, de se tourner au
roi, et lui demander permission de publier que Sa Majesté lui avait
défendu de se mêler des liquidations. C'était avouer le peu de confiance
que le public pouvait prendre en lui et s'en moquer en même temps, en
demandant cette permission ridicule à un roi sans pouvoir, par le défaut
de son âge, d'ordonner ni de défendre rien d'important, et moins encore
qu'à qui que ce fût, au dépositaire de toute son autorité\footnote{Il
  faut entendre par cette phrase un peu obscure, que le roi ne pouvait
  donner aucun ordre important, et moins encore qu'à personne, au duc
  d'Orléans, dépositaire de toute son autorité.}. Aussi le maréchal de
Villeroy ne put-il contenir cette exclamation également ironique et
satirique qui marquait combien il trouvait l'autorité du roi mal
déposée, et le ridicule d'une confiance que le roi n'était pas en état
d'accorder ni de refuser.

Je ne sais si cette dérision du maréchal de Villeroy, si impertinente et
si publique, réveilla dans M. le duc d'Orléans le désir de le déplacer,
mais peu après il me fit en général ses plaintes de la conduite du
maréchal de Villeroy à son égard, de ses liaisons, de ses vues folles,
mais dangereuses, et du péril pour lui régent de laisser croître le roi
entre ses mains, et les conclut par me déclarer résolument qu'il me
voulait mettre en sa place. Je lui opposai les mêmes raisons que je lui
avais alléguées les autres fois que cette même tentation l'avait
surpris. Je le fis souvenir combien il avait approuvé le conseil que je
lui avais donné vers la fin de la vie du feu roi, qu'au cas qu'avant sa
mort, ou par testament, il ne disposât pas de la place de gouverneur de
son successeur, lui, M. le duc d'Orléans, après toutes les horreurs
qu'on avait eu tant de soin de répandre partout, devait se garder sur
toutes choses de mettre en une place si immédiate à la personne du jeune
roi aucun de ceux qui étaient publiquement ses serviteurs particuliers,
moi moins que pas un, qui, dans tous les temps, ne m'étais jamais caché
de l'être, et le seul qui eût continué à le voir hardiment, publiquement
et continuellement dans l'abandon général où il s'était trouvé.
J'insistai que ces mêmes raisons qui m'avaient engagé à le remercier
avec opiniâtreté les autres fois qu'il m'avait pressé d'accepter cette
place, subsistaient toutes pour me la faire encore refuser. J'ajoutai
que, convenant avec lui de tout sur le maréchal de Villeroy ces mêmes
raisons qui m'éloignaient de lui vouloir succéder, militaient toutes
pour l'y faire conserver\,; que, de plus, le désordre dévoilé des
finances, et la sortie de Law du royaume, auquel le maréchal de Villeroy
s'était opposé dans tous les temps avec éclat, n'était pas le moment de
l'ôter d'auprès du roi, et qu'il serait tôt ou tard trop dangereux,
après avoir renvoyé le duc du Maine, de réunir en faveur du maréchal de
Villeroy et contre Son Altesse Royale le renouvellement des plus affreux
soupçons, et le spécieux martyr du bien public, et de l'ennemi de Law et
des ruines dont il avait accablé l'État, mettre en furie Paris qui
croyait la vie du roi attachée à sa vigilance, le parti du duc du Maine
caché sous la cendre, tout ce qui s'appelait la vieille cour,
c'est-à-dire presque tous les plus grands seigneurs, enfin le parlement
et toute la robe que le maréchal de Villeroy avait toujours bassement
courtisée, et qui l'aimait et le considérait comme un protecteur.

Quelque fortes que fussent ces raisons, elles ne persuadèrent point M.
le duc d'Orléans\,: il ne sut trop que répondre, parce qu'elles étaient
péremptoires, mais le maréchal de Villeroy était une guêpe qui
l'infestait et que la vue du futur auprès du roi lui rendait encore plus
odieuse. Voir, par rapport à Son Altesse Royale, ce jeune monarque entre
les mains du maréchal de Villeroy ou entre les miennes, était un
contraste si puissant sur lui qu'il ne s'en put déprendre, et qui forma
deux longues conversations fort vives entre lui et moi. Depuis le lit de
justice des Tuileries, j'étais demeuré en grande familiarité, et même
fort en confiance avec M. le Duc. Le régent en était bien aise, et tous
deux se servaient de moi l'un envers l'autre assez souvent. M. le duc
d'Orléans espéra apparemment plus de force sur moi en joignant M. le Duc
à lui\,; car je vis entrer Millain chez moi un matin deux jours après,
qui, à ma grande surprise, me dit que M. le Duc l'avait chargé de me
dire que M. le duc d'Orléans ne lui avait pas caché son désir de me
faire gouverneur du roi, et ma résistance\,; qu'il trouvait que M. le
duc d'Orléans avait toutes sortes de raisons les plus solides d'ôter le
maréchal de Villeroy d'auprès du roi, et n'avait pas un meilleur choix,
ni un autre choix à faire que de moi pour mettre en cette place, ni de
qui que ce pût être que lui, M. le Duc, désirât davantage. Là-dessus,
Millain se mit sur son bien-dire, tant pour l'expulsion du maréchal de
Villeroy que pour me cajoler, m'enivrer, s'il avait pu, de louanges et
de persuasions, sans avoir pu faire ni l'un ni l'autre ne le priai
d'abord de témoigner à M. le Duc combien j'étais sensible à une si
grande marque de son estime et de sa bienveillance, et que, si quelque
chose, après la volonté de M. le duc d'Orléans et son service, me
pouvait tenter d'accepter la place de gouverneur du roi, {[}ce{]} serait
d'avoir à compter d'une éducation si importante avec un surintendant,
non bâtard, mais prince du sang, et tel que M. le Duc\,; mais que je le
suppliais de considérer toutes les raisons que j'avais alléguées à M. le
duc d'Orléans, tant contre le déplacement du maréchal de Villeroy que
contre le choix à faire de moi pour remplir sa place. Je les détaillai
toutes à Millain, je n'oubliai ni force ni étendue, et je conclus par le
prier de faire observer à M. le Duc que je méritais d'autant plus d'être
cru, qu'il n'ignorait pas que, si je m'opposais au déplacement du
maréchal de Villeroy, ce n'était ni par estime ni par amitié, et que, si
je tenais ferme au refus, ce n'était pas que je ne sentisse tout
l'honneur du choix des deux princes, et tout l'avantage et la
considération que cette grande place, et si importante, apporterait à
moi et aux miens.

Millain, bien instruit par M. le Duc\,; qui m'aimait depuis que je
l'avais connu chez le chancelier de Pontchartrain, et qui, depuis le lit
de justice des Tuileries, était demeuré dans l'habitude de suppléer,
tant que cela se pouvait, aux conférences entre M. le Duc et moi,
contesta mes raisons plus de deux grosses heures sans me faire perdre
une ligne de terrain. Les deux princes furent étonnés et fâchés de cette
résistance, tous deux me le témoignèrent. La dispute recommença, M. le
duc d'Orléans s'y prit de toutes les façons et à force reprises\,;
Millain m'assiégeait sans cesse chez moi. Enfin, ils me déclarèrent
qu'ils ne quitteraient point prise que je n'eusse accepté, et que cette
lutte durerait tant qu'il me plairait, et jusqu'à ce que je la voulusse
finir de la sorte elle dura ainsi cinq semaines. J'en étais excédé, et
en même temps peiné de répondre si durement à l'amitié, à la confiance,
à leur sentiment intime de la nécessité, surtout pour l'avenir si
délicat et si important pour M. le duc d'Orléans. Ces considérations
toutefois, quelque fortes qu'elles fussent, n'ébranlèrent aucune de mes
raisons\,: elles ne faisaient qu'accroître mon malaise, et l'importunité
que je recevais d'entendre et de répéter les mêmes raisons presque tous
les jours.

À la fin je voulus terminer une contestation si journalière et si
longue, et finir par Millain pour finir avec plus de mesure et moins
durement. Je dis donc à Millain que, sans me départir d'aucune des
raisons que j'avais si souvent alléguées aux deux princes et à lui, tant
contre le déplacement du maréchal de Villeroy que contre le choix à
faire de moi pour remplir sa place auprès du roi, que je croyais
péremptoires et sans réplique devant tout homme éclairé et indifférent,
je lui en dirais une autre, à moi plus personnelle et plus intime, que
j'avais expliquée à M. le duc d'Orléans, et qu'il fallait donc aussi que
M. le Duc sût, puisqu'il me pressait avec tant de force et de
persévérance. C'était en deux mots que, quelque attaché que je fusse à
M. le duc d'Orléans, et quelque serviteur que je fusse de M. le Duc, mon
honneur m'était plus cher que l'un ni l'autre, et que tout ce que la
plus grande fortune me pourrait présenter\,; qu'il savait lui Millain,
que personne n'ignorait ce que de tout temps j'étais à M. le duc
d'Orléans\,; qu'il n'ignorait pas aussi les horreurs si souvent
renouvelées et répandues contre ce prince depuis leur première
invention\,; que, mis par lui en la place du maréchal de Villeroy,
l'effroi factice des joueurs de ressorts de ces horreurs éclaterait de
plus belle contre le régent, et le contre-coup sur moi\,; que nul ne
pouvait me garantir que le roi fût exempt de tout accident et de toute
maladie tant qu'il serait entre mes mains\,; que cette garantie se
pouvait étendre aussi peu sur sa vie, puisqu'il était mortel comme tous
les autres hommes de son âge\,; que, s'il lui arrivait accident ou
maladie, je me sentais incapable de soutenir tout ce qui se répandrait
sur M. le duc d'Orléans, et qui en plein rejaillirait sur moi\,; que, si
malheur arrivait au roi, je courais toutes sortes de risques d'entendre
publier qu'il n'aurait été mis entre mes mains que pour avoir plus de
liberté de s'en défaire, soit par ma négligence, soit par ma connivence,
à quoi je me sentais radicalement incapable de survivre un moment\,; par
conséquent qu'il voyait, et que M. le Duc verrait à plein par le compte
qu'il allait lui rendre, combien radicalement aussi j'étais incapable de
me laisser vaincre par quoi que ce pût être pour accepter la place de
gouverneur du roi, même quand elle vaquerait par mort.

Millain, tout consterné qu'il me parût d'une résistance si ferme et si
bien causée, ne se tint point battu\,; il se mit à tâcher de m'éblouir,
à vanter ma réputation, qui ne pouvait être attaquée\,; à m'alléguer
qu'elle était demeurée intacte à la mort de nos princes, lors de la plus
grande fureur et des discours les plus horribles répandus contre M. le
duc d'Orléans\,; et lorsqu'il avait été si longtemps dans le décri et
dans un abandon si général, que qui que ce soit, sans exception, n'osait
le voir ni même lui parler, tandis que moi, unique, n'avoir jamais cessé
un moment de le voir et de l'entretenir chez lui et jusque sous les yeux
du roi, dans le salon et dans les jardins de Marly, à Versailles, et
partout, sans que pas un de ceux qui m'aimaient le moins ait jamais ni
dit ni laissé entendre quoi que ce pût être qui pût m'intéresser. Il
pressa tant qu'il put cet argument qu'il trouvait si fort. En effet, ce
qu'il disait était vrai, et j'eus ce rare bonheur que les inventeurs,
les instigateurs, les prôneurs de ces horreurs contre M. le duc
d'Orléans, qui d'ailleurs et de plus, par mon attachement pour lui,
étaient mes ennemis, n'imaginèrent jamais de laisser tomber sur moi
l'ombre du soupçon le plus léger, ni le public à qui ils donnaient
l'impulsion. Je convins avec Millain de cette vérité, mais je pus être
persuadé que cette vérité, pour flatteuse qu'elle pût être, me mit à
couvert sur ce qui pouvait arriver du roi entre mes mains. Raisonnant un
moment comme les inventeurs et les semeurs des bruits horribles si
étrangement répandus contre M. le duc d'Orléans à la mort de nos
princes, M. le duc d'Orléans non seulement n'avait aucun besoin de moi
pour l'exécution de tels crimes, mais au contraire grand besoin de s'en
cacher de moi. «\,Je laisse, dis-je à Millain, la religion, l'honneur,
la probité\,; je ne toucherai que l'intérêt.\,»

Monseigneur était mort\,: le roi avait pris toute confiance dans le
nouveau Dauphin, il lui renvoyait les ministres et les affaires, il
donnait les plus grandes charges à son choix, témoin le duc de Charost.
Ce prince par ses vertus, son application, l'autorité que le roi lui
faisait prendre\,; la Dauphine par ses charmes envers tout le monde,
qu'elle animait partout, était l'objet de la tendresse de son époux, de
celle du roi, de celle de tout le monde. Le duc de Beauvilliers se
trouvait dans la plus brande splendeur, par l'influence entière qu'il
avait conservée sur son ancien pupille. Personne n'ignorait à la cour,
et M. le duc d'Orléans moins qu'aucun, que le duc de Beauvilliers
m'aimait plus qu'un fils et me confiait presque toutes choses, depuis
bien des années que sa confiance allait toujours croissant. Il avait
transpiré malgré toutes nos précautions qu'il m'avait initié dans celle
du Dauphin, que la Dauphine voulait que M\textsuperscript{me} de
Saint-Simon succédât à la duchesse du Lude, fort âgée déjà, et accablée
de goutte. La couronne ne pouvait tarder longtemps à tomber sur la tète
du Dauphin. Que n'avais-je donc point à perdre en le perdant, comme j'y
ai tout perdu en effet, sans compter ce qui est mille fois plus cher que
les fortunes. C'était cette perspective charmante que le monde voyait
s'ouvrir devant moi, qui m'en attirait l'envie et la jalousie, et qui
était incompatible avec le partage ou la confidence des crimes dont on
accablait la réputation de M. le duc d'Orléans, dont le règne, s'il fut
arrivé même sans trouble, quelque favorable qu'il me pût être, ne
pouvait jamais me dédommager du personnel incomparable du Dauphin, ni
pour la fortune de ce que j'en pouvais attendre, sans compter ce que
m'eût été de voir la couronne sur la tête d'une bâtarde de
M\textsuperscript{me} de Montespan, au lieu de cette Dauphine si
aimable, et de là sur les petits-fils de cette Montespan. Par conséquent
quel rejaillissement sur ses frères, sur ses neveux, et quel éternel
désespoir pour l'antipode si déclaré de la bâtardise\,! M. le Duc était
trop éloigné de la couronne, pour que ce propos fût déplacé, et M. le
duc d'Orléans, trop frivole, trop peu touché par soi-même de la
possibilité de régner, enfin trop accoutumé à moi, à mes sentiments, à
mes manières pour en être embarrassé avec lui. J'ajoutai à Millain qu'il
prît garde à la différence des temps et des circonstances pour en faire
la comparaison, et porter un jugement sain de mon refus\,; qu'il était
clair que j'avais tout à perdre en perdant le Dauphin et la Dauphine\,;
qu'il ne l'était guère moins, pour continuer à ne traiter que l'intérêt
et faire abstraction de toute autre considération, {[}que{]} je n'avais
rien à perdre que de commun avec toute la France, si le roi lui était
ravi, tandis qu'en mon particulier je ne perdrais que l'espérance très
légère du crédit, qu'un gouverneur nouveau venu pourrait fonder de
s'acquérir auprès d'un enfant qui avant quatorze ans serait son maître,
environné de gens qui ne songeraient qu'à l'entraîner, et à lui rendre
son gouverneur odieux, tout au moins contraignant, importun et ridicule,
tandis que j'avais tout à me promettre de M. le duc d'Orléans devenu
roi. J'insistai avec raison et force sur cette si extrême différence des
temps et des circonstances\,; d'où je conclus que si ma réputation était
demeurée intacte à la mort de nos princes, j'avais tout lien de craindre
qu'elle ne la demeurât pas si, étant gouverneur du roi, j'avais le
malheur de le perdre de quelque accident et de quelque maladie que ce
pût être, pour palpablement naturelle qu'elle fût et qu'elle parût.
Enfin qu'il fit considérer à M. le Duc une raison si touchante, que rien
dans le monde ne me ferait passer pardessus.

Millain, étourdi de la solidité de cette raison finale, ne laissa pas de
se reprendre aux branches et d'insister sur ma réputation, qui ne
pouvait jamais être tant soit peu attaquée. Je lui répondis que je m'en
flattais parce que je m'étais conduit toute ma vie principalement vers
ce but, mais que le moyen le plus certain de la conserver entière, sans
tache et sans rides, était de ne l'exposer pas à aucun des cas qui
pourrait la gâter quelque injustement que ce pût être, et de n'être ni
assez présomptueux à cet égard, ni assez ambitieux pour risquer quoi que
ce pût être, qui pût entraîner sur elle le doute le plus léger, quoique
le plus visiblement mal fondé. Je finis une conversation qui consomma
presque toute cette matinée, par l'assurer que je ne serais ébranlé par
rien, que j'étais las de tant de redites, sur une matière plus
qu'épuisée\,; que je conjurais M. le Duc que je n'en entendisse plus
parler et que je ferais la même déclaration à M. le duc d'Orléans\,; je
la lui fis en effet deux jours après, sur ce qu'il me pressa encore.
Néanmoins, il se fonda encore en raisonnements, c'est-à-dire que les
mêmes sur le maréchal de Villeroy et sur moi furent amplement rebattus,
parce qu'il n'y avait plus rien de nouveau à en dire. Il me demanda
plusieurs fois si je le voulais livrer en proie au maréchal de Villeroy,
et je vis combien il était touché et frappé de la différence, pour lui,
de voir le roi entre de telles mains ou entre les miennes. En cela il
n'avait pas tort\,; mais, comme je l'ai déjà dit, d'autres
considérations plus fortes par un grand malheur devaient l'emporter pour
conserver le maréchal de Villeroy dans sa place\,; et quoique
véritablement sensible à la peine de M. le duc d'Orléans de mon refus,
ma réputation et mon honneur m'étaient trop chers pour les exposer le
moins du monde, outre mes autres raisons, qui ont été expliquées.

Je comptai donc l'affaire finie à mon égard, et que faute de trouver
quelque autre bien à point, le maréchal de Villeroy conserverait sa
place, comme en effet il arriva. Mais à mon égard, la persécution, si
j'ose me servir de ce terme, n'était pas finie. Millain eut ordre de
revenir encore à la charge, et il s'en acquitta si bien qu'il me mit
enfin en colère\,; je lui dis que c'était une tyrannie qu'exiger d'un
serviteur, sur qui on a raison de compter, d'exposer son honneur et sa
réputation, au hasard d'un futur contingent que j'espérais bien qui
n'arriverait pas, mais qui n'était que trop possible par les accidents
communs à tous les hommes, et par la rougeole et la petite-vérole que le
roi n'avait point eues, et qui tourneraient la tête aux médecins.
Qu'outre un si cher intérêt que celui de mon honneur et de ma
réputation, j'avais allégué plusieurs fois à ces princes des raisons qui
regardaient M. le duc d'Orléans, si péremptoires pour laisser le
maréchal de Villeroy dans sa place, et pour, quoi qu'il arrivât de lui,
ne me la jamais donner, que je ne pouvais attribuer cette opiniâtreté
qu'à une espèce d'ensorcellement\,; mais qu'en un mot, je l'avertissais
pour le rendre à M. le Duc, et M. le Duc à M. le duc d'Orléans, si bon
lui semblait, que je ne me défendrais plus\,; que de mon silence, ils en
inféreraient tout ce qu'il leur plairait\,; que, si le maréchal de
Villeroy était ôté d'auprès du roi, je ne dirais pas une parole, mais
que, si j'étais nommé pour la remplir, je refuserais ferme et net\,; que
ce refus m'attirerait les applaudissements de tout le monde aux dépens
de M. le duc d'Orléans, et peut-être de M. le Duc, qui pourraient bien
m'envoyer à la Bastille et me retirer l'honneur de leurs bonnes
grâces\,; que je serais au désespoir d'être loué à leurs dépens, mais
que, ne me restant plus que ce moyen pour me garantir d'une place qui
pouvait devenir funeste à mon honneur et à ma réputation, quelque
faussement et injustement que ce pût être, je l'embrasserais comme un
fer rouge, plutôt que de m'y exposer, que je ne les trompais point en
cela, puisque je le lui disais à lui, pour qu'ils en fussent avertis,
après quoi je n'ouvrirais plus la bouche sur une affaire si longuement
rebattue, et qui aurait dû être finie et abandonnée depuis longtemps.
Cela dit avec quelque force, je me levai, et par ma contenance, je fis
entendre à Millain que tout était épuisé, et civilement qu'il n'avait
qu'à s'en aller. Telle fut la fin finale de cette affaire dont les deux
princes ni Millain ne me parlèrent plus. M. le duc d'Orléans fut un peu
fâché\,; mais avec moi surtout ses fâcheries étaient légères et courtes.
Pour M. le Duc, il me parut qu'il se paya, quoique à regret, de raison.
Mon refus opéra la conservation du maréchal de Villeroy auprès du roi,
faute, comme je l'ai dit, de trouver de qui la remplir.

M. le duc d'Orléans conta tout cela à l'abbé Dubois\,; je l'appelle
toujours ainsi, quoique sacré archevêque de Cambrai. On a vu ailleurs
ici que souvent les choses intérieures les plus secrètes transpiraient
du Palais-Royal et se savaient au dehors. Le maréchal de Villeroy apprit
le risque qu'il avait couru, et qu'il n'avait tenu qu'à moi d'avoir sa
place. Tout autre que lui aurait pu en être piqué contre M. le duc
d'Orléans et contre M. le Duc, mais m'aurait su gré de mon refus et de
ma conduite qui l'avait conservé, d'autant que ce n'était pas pour la
première fois, ni même pour la seconde, que pareil cas était arrivé,
comme on, l'a pu voir ici en son temps, quoique avec moins de dispute et
de longueur.

Ce sentiment à mon égard ne fut pas celui du maréchal de Villeroy. Trop
fâché pour se contenir, trop bas et trop timide pour s'en prendre au
régent, quoique si hardi en d'autres choses, mais qui allaient à ses
projets, dont la cheville ouvrière était sa place auprès du roi, qu'il
ne voulait pas hasarder par une scène avec M. le duc d'Orléans, des
intentions duquel et de celles de M. le Duc il ne pouvait douter, il
s'en prit honteusement à la partie faible, dont pourtant l'opiniâtre
refus l'avait sauvé. Il renouvela donc ses anciennes plaintes là-dessus
et son ancien dépit contre moi. Malheureusement pour lui il ne sut et ne
put par ou me prendre. Il eut recours à de misérables généralités et à
aboyer à la lune. Cela me revint bientôt et de plusieurs côtés. Je ne
voulais pas avouer, non plus que les précédentes fois, que la place de
gouverneur du roi m'avait été offerte\,; je ne crus pas aussi devoir,
comme la dernière fois, rassurer le maréchal de Villeroy, qui payait si
mal le service si essentiel que je lui avais rendu, et dont la basse
jalousie allumait l'ingratitude. Je pris le parti de mépriser ses
discours, comme je faisais de tout temps sa personne, mais sans me
lâcher sur lui en rien. Je me contentai d'en hausser les épaules et de
traiter de radotage ce qu'on m'en contait. Je n'avais jamais eu de
commerce avec lui que de rare et légère bienséance pendant et depuis le
dernier règne, excepté les derniers temps de la vie du feu roi, qu'on a
vu en son lieu qu'il se jeta à moi pour essayer de me pomper avec une
importunité extrême. J'allais peu chez le roi, dont l'âge ne comportait
pas l'assiduité du mien, et où encore je ne le rencontrais presque
point, tellement que je ne le voyais qu'au conseil, où nous ne nous
abordions guère, au plus que des moments, et où il était difficile, par
l'ordre de la séance, que nous nous trouvassions l'un auprès de
l'autre\,; je n'eus donc rien à changer dans ma conduite à son égard, et
je me contentai de piquer de plus en plus, par mon parfait silence, son
orgueil et sa vanité blessée.

\hypertarget{chapitre-vi.}{%
\chapter{CHAPITRE VI.}\label{chapitre-vi.}}

1721

~

{\textsc{Forte conversation entre M. le duc d'Orléans et moi, qui
ébranle l'abbé Dubois fortement, mais inutilement.}} {\textsc{-
Faiblesse étrange de M. le duc d'Orléans, qui dit tout à l'abbé Dubois,
se laisse irriter contre moi jusqu'à me faire de singuliers reproches,
dont à la fin il demeure honteux\,; m'avoue sa faiblesse et défend à
l'abbé Dubois de lui jamais parler de moi.}} {\textsc{- Étrange trait
sur le chapeau de Dubois entre M. le duc d'Orléans et Torcy.}}
{\textsc{- Naissance du prince de Galles à Rome.}} {\textsc{- Sentiments
anglais sur cette naissance.}} {\textsc{- Mort du comte de Stanhope et
de Craggs, secrétaires d'État d'Angleterre, succédés (sic) par Townsend
et Carteret.}} {\textsc{- Leur caractère.}} {\textsc{- Mort du docteur
Sachewerell.}} {\textsc{- Mort et caractère de Huet, ancien évêque
d'Avranches\,; de la duchesse de Luynes\,; de la duchesse de Sully
(Coislin)\,; de la duchesse de Brissac (Vertamont).}} {\textsc{-
Embrasement de Rennes.}} {\textsc{- Cailloux singuliers.}}

~

Quoique M. le duc d'Orléans ne me mit plus au fait de tout comme avant
que l'abbé Dubois se fût entièrement et ouvertement rendu le maître de
toutes les affaires du dehors et du dedans, et fût parvenu à tenir de
court son maître et à le resserrer avec ses plus sûrs serviteurs, avec
moi surtout dont il craignait la liberté et l'ancienne habitude avec ce
prince, il ne put néanmoins le tenir de si court à mon égard, que,
quelque réservé que je me rendisse depuis que j'avais aperçu la réserve
insolite de M. le duc d'Orléans avec moi, l'abbé Dubois, dis-je, ne put
si bien faire qu'il n'échappa toujours quelque chose à l'habitude et à
la confiance pour moi. Je l'ai déjà dit et il faut le répéter ici, les
petits chagrins que ce prince avait quelquefois contre moi, étaient
légers et courts. Ainsi celui qu'il avait pris de mon opiniâtre refus de
la place de gouverneur du roi tomba incontinent après. Une après-dînée
que je travaillais avec lui, seul à mon ordinaire, il me parla du traité
entre l'Espagne et l'Angleterre qui s'avançait fort, et m'en apprit les
détails qui donnaient les plus grands avantages au commerce de
l'Angleterre, aux dépens de l'Espagne qui avait grand'peine à y
consentir, et qui ruinaient celui de France, en transportant aux Anglais
tous les avantages que les François y avaient eus depuis l'avènement de
Philippe V à la couronne, la plupart conservés de façon ou d'autre
depuis la paix d'Utrecht. Nous y avions perdu à la vérité la traite des
nègres\,; mais le vaisseau de permission et beaucoup d'autres avantages
nous étoient restés, que l'Angleterre prétendait nous faire ôter et les
obtenir, et desquels l'abbé Dubois ne leur faisait pas moins litière
qu'il ne pressait l'Espagne de se couper la gorge à elle-même en faveur
des Anglais.

Dès les commencements de la régence, on a pu voir ici et plusieurs fois
depuis combien ce joug Anglais me pesait\,; plus il s'appesantissait,
plus il me devenait insupportable. Je ne pus donc tenir au récit que me
fit M. le duc d'Orléans. Je lui fis sentir le préjudice extrême que le
commerce de France allait recevoir et l'Espagne elle-même si elle se
laissait entraîner aux conditions qu'il m'exposait, et combien lui-même
serait un jour comptable au roi et à la nation d'avoir souffert que
l'abbé Dubois vendit des intérêts si grands et si chers à l'Angleterre,
qui saurait bien dans tous les temps se conserver ce qui lui serait
accordé. Je l'exhortai du moins à laisser traiter cette affaire au
congrès de Cambrai qui s'allait ouvrir, où presque tous les ministres
des premières puissances étrangères étaient arrivés, duquel l'objet
n'était pas moins de régler les difficultés entre l'Angleterre et
l'Espagne sur le commerce et avec nous-mêmes, que de tacher d'ajuster
l'Espagne avec l'empereur et de parvenir à une paix entre eux. Que là,
en présence de tant de ministres, des Hollandais surtout, quoique si
liés à l'Angleterre par terre, mais jaloux et si las de leurs progrès au
delà des mers, l'Espagne trouverait des secours et l'Angleterre des
embarras et des difficultés très profitables\,; à tout le moins lui,
régent, éviterait le blâme de s'être hâté d'égorger la France et
l'Espagne sous la cheminée, en procurant à l'Angleterre toutes ses
nouvelles et très injustes prétentions. Le détail fut long sur les
plaies qui étaient portées par les conditions demandées par les Anglais
à l'Espagne\,; et au commerce de France qu'elles ruinaient, et à celui
de toute l'Europe qu'elles attaquaient et qui en demeurerait extrêmement
affaibli si elles étaient accordées, et sur la certitude qu'elles
demeureraient toujours aux Anglais, si elles tombaient une fois entre
les serres d'une nation si avide, si avantageuse, si puissante par mer,
si fort née pour les colonies et pour le commerce, si jalouse d'y
dominer, si suivie, si pénétrée de son intérêt, du commerce, dis-je, qui
intéresse chaque particulier et qui est tout entier et dans toutes ses
parties entre les mains de la nation, dans les parlements et absolument
hors de prise à leur roi et à ses ministres. J'insistai donc sur le
grand intérêt de la France et de l'Espagne de laisser porter ces
prétentions au congrès de Cambrai, où l'intérêt palpable du commerce de
toute l'Europe tiendrait les yeux de tous les ministres ouverts, et
formerait des obstacles et des entraves aux Anglais, dont le régent
n'aurait point le démérite, tout au plus ne ferait que le partager avec
toutes les autres puissances, et sauverait ainsi en tout ou en la plus
grande partie le commerce de France, celui d'Espagne et le commerce de
toute l'Europe dont l'Angleterre se voulait emparer, et deviendrait
enfin la maîtresse de l'Europe, puisqu'elle en posséderait seule tout
l'argent, qui par le commerce s'est jusqu'ici distribué en toutes ses
parties plus ou moins inégalement à proportion du commerce de chacune.

Ce discours plus fort et bien plus détaillé, et plus long que je ne le
rapporte, fit une grande impression à M. le duc d'Orléans. Il entra en
discussion, il convint avec moi de beaucoup de choses, et peu à peu que
j'avais raison. Cela m'encouragea, de sorte qu'après l'avoir battu sur
ses objections par rapport à ses entraves avec l'Angleterre, je lui dis
qu'il n'avait qu'à voir où l'intérêt personnel de l'abbé Dubois l'avait
conduit\,; que je lui avais souvent dit qu'il ne songeait qu'à être
cardinal, et que toujours, lui régent, s'était récrié d'indignation,
vraie ou feinte, et qu'il le ferait mettre dans un cul de basse-fosse
s'il le surprenait dans une telle pensée\,; que néanmoins rien n'était
plus vrai\,; que je ne lui enviais le cardinalat en aucune sorte, qu'il
ne serait pas le premier cuistre ni le centième qui le serait devenu\,;
qu'un régent de France, tel qu'il l'était, devait assez se sentir et
être en effet assez considérable pour pouvoir récompenser d'un chapeau
qui que ce fût, surtout un homme qui avait le vernis d'avoir été son
précepteur, et acquis depuis le caractère épiscopal d'un grand siège et
celui de ministre très principal\,; mais qu'il était vrai que je ne
pouvais souffrir que l'abbé Dubois se fît cardinal par l'autorité que
l'empereur exerçait despotiquement à Rome, et par le crédit
tout-puissant du roi d'Angleterre sur l'empereur. Que pour se rendre le
roi d'Angleterre et ses ministres non seulement favorables à Vienne,
mais pour leur faire épouser son intérêt par le leur, il n'avait songé
qu'à lier lui régent à l'Angleterre, à se rendre nécessaire pour serrer
cette union, faire plusieurs voyages à Hanovre et à Londres parce qu'on
dit ce qu'on n'ose écrire, peu après engager la rupture, puis la guerre
entre la France et l'Espagne, sans autre intérêt que le sien, pour
flatter Londres et Vienne, non seulement contre l'intérêt de la France,
mais en exposant lui régent personnellement, aux derniers dangers, comme
je le lui avais prédit dans le temps, comme il en {[}a{]} éprouvé une
partie dans l'affaire de Cellamare, et comme il a hasardé bien pis, si
la guerre eût duré et se fût échauffée. Que lui seul n'avait pas voulu
voir ce qui fut clair alors à toute l'Europe, que cette guerre n'eut
jamais d'autre objet que de satisfaire la jalousie des Anglais sur la
marine renaissante d'Espagne dont le maréchal de Berwick eut l'ordre,
qu'il exécuta, de brûler tous les vaisseaux, tous les chantiers, tous
les magasins des ports du Ferrol et des autres voisins, ce qui anéantit
toute la marine d'Espagne\,; tout aussitôt après quoi l'abbé Dubois
termina cette déplorable guerre. «\,De là, ajoutai-je, il vous a fait
entièrement passer sous le joug des Anglais, a été leur homme auprès de
vous plus que ne le fut jamais l'impudent Stairs, son bon ami\,; et
maintenant il vend, pour son chapeau, la France, l'Espagne, le commerce
de toutes les nations de l'Europe à l'Angleterre sans le moindre
retour\,; se vend en même temps à eux et s'applaudit de sa trahison et
de sa ruse, qui lui va incessamment procurer le chapeau auquel votre
considération n'aura pas la moindre part, mais la seule autorité de
l'empereur, parla vive et pressante entremise du roi d'Angleterre, ou
plutôt en vertu du traité secret de ses ministres avec l'abbé Dubois.\,»

L'impression de ce vif et trop vrai raccourci de la conduite de l'abbé
Dubois, si pourpensée et si bien suivie, frappa le régent au delà de ce
que je l'ai jamais vu. Il s'appuya les coudes sur la table qui était
entre lui et moi, se prit la tète entre ses deux mains et y demeura
quelque peu en silence, le nez presque sur la table. C'était sa façon
quand il était assis et fort agité. Enfin il se leva tout à coup, fit
quelques pas sans parler, puis se prit à se dire à soi-même\,: «\,Il
faut chasser ce coquin. --- Mieux tard que jamais, repris-je\,; mais
vous n'en ferez rien.\,» Il se promena un peu en silence avec moi. Je
l'examinais cependant, et je lisais sur son visage et dans toute sa
contenance la vive persuasion de son esprit, même de sa volonté,
combattue par le sentiment de sa faiblesse, et de l'empire absolu qu'il
avait laissé prendre sur lui. Il répéta ensuite deux ou trois fois\,:
«\,Il faut l'ôter,\,» et comme l'habitude me le faisait connaître très
distinctement, je croyais à son ton et à son maintien entendre tout à la
fois l'expression la plus forte d'une nécessité instante et de
l'insurmontable embarras d'avoir la force de l'exécuter\,; dans cet
état, je vis clairement qu'il ne me restait plus rien à dire pour
arriver à la conviction parfaite de la nécessité urgente de chasser
l'abbé Dubois\,; mais que pour lui en inspirer la force, mes paroles
seraient inutiles, et ne feraient qu'affaiblir celles qui lui avaient
fait une si forte impression, parce qu'elles ne feraient que le dépiter
en lui faisant sentir plus fortement sa faiblesse, sans lui donner la
force de la surmonter. Cela m'engagea à me retirer pour le laisser à
lui-même, et le soulager de la peine et de la honte de me voir le témoin
de ce combat intérieur. Je lui dis donc que je n'avais plus rien à
ajouter à une matière si importante à l'État, à toute l'Europe,
singulièrement à lui-même, que je le laissais à ses réflexions, et qu'il
ne me restait qu'à désirer qu'elles eussent sur lui tout le pouvoir
qu'elles devaient avoir. Il était si occupé qu'à peine me répondit-il je
ne sais quoi, et me laissa aller sans peine contre son ordinaire toutes
les fois qu'il se trouvait fort agité. Je m'en allai content d'avoir
rempli mon devoir par une conversation si forte et si nécessaire, mais
avec peu d'espérance du fruit qu'elle devait si naturellement produire.

Achevons cette matière tout de suite trop intéressante et trop curieuse
pour être interrompue et en faire à deux fois\,; trois semaines à peu
près se passèrent sans que j'aperçusse rien que d'ordinaire en M. le duc
d'Orléans avec moi. Dans mes jours de travail, il ne me parla ni
d'affaires étrangères ni de l'abbé Dubois\,; de mon côté, je me gardai
bien de lui en ouvrir la bouche. Néanmoins, j'avais su que le lendemain
de la conversation que je viens de raconter, il y avait {[}eu{]} tant de
bruit et si long par reprises entre M. le duc d'Orléans et l'abbé
Dubois, que les chambres voisines s'en étaient fortement aperçues,
malgré des pièces vides entredeux, et je fus informé aussi que M. le duc
d'Orléans avait paru longtemps occupé et de mauvaise humeur, lui qui
n'en montrait et n'en avait même comme jamais\,; en même temps que
l'abbé. Dubois était plus furieux et plus intraitable qu'il ne l'avait
jamais paru. J'en conclus de plus en plus la volonté et la faiblesse\,;
qu'il y avait eu des reproches et des éclats qui ne menaient à rien, car
il n'y avait qu'à le chasser sans le voir et sans donner prise à la
faiblesse\,; enfin que cette faiblesse l'emporterait sur les plus
importantes considérations, et que l'abbé Dubois demeurerait le maître.
Je ne me trompai pas.

Vers la fin des trois semaines depuis la conversation, allant travailler
avec M. le duc d'Orléans, je le trouvai seul qui se promenait dans la
pièce de son grand appartement la plus proche du passage de son petit
appartement. Il me reçut contre son ordinaire d'un air si froid et si
embarrassé, qu'après quelque peu de mots indifférents je lui demandai
franchement à qui il en avait, et que je voyais bien qu'il y avait
quelque chose sur mon compte. Il balança, il tergiversa. Je le pressai,
l'apostume creva. Il me dit donc, puisque je voulais le savoir, qu'il
était fort peiné contre moi, et tout de suite me débagoula, car c'est le
terme qui convient à la façon dont il se déchargea, que je voulais qu'il
fit tout ce qu'il me plaisait, et que je refusais de faire tout ce qui
ne me plaisait pas\,; que j'avais refusé les finances, la place de chef
du conseil des affaires du dedans, depuis de me trouver avec lui et tous
les pairs et les maréchaux de France au grand conseil, les sceaux après,
et trois fois de le délivrer de la plus fâcheuse épine en refusant
autant de fois la place de gouverneur du roi. «\,N'y a-t-il que cela,
lui répondis-je, qui vous mette en cette humeur contre moi\,? --- Non,
reprit-il vivement, il me semble que c'est bien assez. --- Or bien,
monsieur, lui dis-je, il faut commencer par les refus que vous me
reprochez, parce que ce sont des faits\,; nous viendrons après à la
plainte vague de vouloir vous faire faire tout ce qu'il me plaît. Des
deux premiers refus, souvenez-vous s'il vous plaît qu'il n'y en a qu'un
qui porte, qui est celui des finances. Il est vrai que vous fûtes fâché,
il est plus vrai encore que vous l'auriez été davantage, si je les avais
acceptées\,; ma raison de les refuser fut mon incapacité et mon dégoût
naturel de ces matières, j'y aurais fait autant de fautes que de pas, et
en finances il n'y a point de petites fautes. Si je n'entends rien aux
finances ordinaires, comment aurais-je pu comprendre les diverses
opérations de Law, et tenir ce timon qui a enfin rompu entre vos mains à
vous-même\,; et si la souplesse et la bassesse du duc de Noailles pour
le parlement, jusqu'à rendre compte des finances à ses commissaires, n'a
pu émousser ses entreprises à cet égard, pensez-vous que ma conduite lui
eût été plus agréable avec l'affaire du bonnet et ma rupture sans nul
ménagement avec le premier président\,? Voilà donc, monsieur, pour les
finances. À quoi on n'a jamais imputé à mal à personne le refus d'une
place grande par son autorité, son importance et ce qu'elle vaut, ni
l'aveu d'une incapacité véritable. J'oserais dire, s'il s'agissait d'un
autre, que ce refus mériterait louange et estime, et qu'il n'est pas
commun. La place de président du conseil des affaires du dedans, il est
vrai que je la refusai, parce que je la trouvais trop forte et trop
laborieuse à me charger du détail de tout ce qui vient de procès, de
disputes, de règlements au conseil de dépêches, et de les rapporter au
conseil de régence\,; souvenez-vous du peu d'ambition que je témoignai
dans la formation des conseils\,: vous me demandâtes sur ces deux refus
ce que je voulais donc prendre, et j'eus l'honneur de vous répondre que
c'était à moi à vous laisser disposer de moi, mais que, si vous vouliez
m'employer à quelque chose, et me mettre à ce dont je croirais
m'acquitter le moins mal, ce serait de me donner une place dans ce même
conseil des affaires du dedans, sur quoi vous vous moquâtes de moi, et
me dites avec bonté, que, ne voulant ni des finances ni de la place de
chef de ce conseil du dedans, il n'y en avait point d'autre pour moi,
que dans le conseil où vous seriez vous-même. J'ai donc raison de dire
que ce refus-ci ne porte pas, puisque je me contentais de bien moins
dans le même conseil et que vous n'avez pas eu lieu de vous plaindre du
travail, de l'onction, de la capacité de d'Antin, que je vous proposai
pour chef de ce conseil, et que vous en chargeâtes. Quant au grand
conseil, dites-moi, monsieur, en avez-vous sitôt perdu la mémoire\,? Si
cela est, rappelez-vous, s'il vous plaît, que je ne savais pas un mot de
cette belle séance, lorsque j'arrivai de Meudon, pour travailler avec
vous\,; que je vous trouvai dans cette même pièce-ci, donnant vous-même
des commissions à des garçons rouges et à d'autres de vos gens\,; que je
vous demandai ce que c'était que tout cela que je n'entendais qu'à
bâtons rompus\,; que vous me l'expliquâtes, et tout de suite me dites en
souriant qu'à mon égard ce serait le contraire des autres pairs
mandés\,; que vous me priiez de ne me pas trouver au grand conseil,
parce que sûrement je ne serais pas de l'avis que vous vouliez qui y
passât et que je disputerais contre comme un diable\,; à quoi j'eus
l'honneur de vous répondre que je réputais à grâce très particulière
cette défense qui me délivrait de la nécessité de vous déplaire en
public, et peut-être de vous embarrasser beaucoup, pour suivre le
mouvement de ma conscience et de mon honneur pour le service de l'État,
et en particulier de l'Église et de la vérité. Vous vous mîtes à rire de
ma réponse avec votre légèreté ordinaire\,; là-dessus la conversation se
fit ensuite sur cette séance du lendemain, que je ne pus approuver\,;
j'eus ensuite l'honneur de travailler avec vous. Vous ne fûtes fâché ni
alors ni depuis, et aujourd'hui est la première fois que vous vous en
avisez\,: franchement, monsieur, pardonnez-moi si je vous le dis cela
est-il raisonnable\,? Passons maintenant aux sceaux, permettez-moi de
vous dire que je n'ai jamais compris quelle a été la fantaisie de me les
vouloir donner, et une fantaisie aussi opiniâtre\,: faire une sorte
d'insulte à toute la magistrature de les donner à un homme d'épée, à un
homme entièrement ignorant du sceau et de tout ce qui y a rapport, à un
homme pour être entre vous et le parlement, répondre à ses remontrances
et à ses entreprises, y présider, y parler, y prononcer, en cas de lit
de justice, toutes choses très difficiles à allier, pour ne pas dire
incompatibles, avec la séance et la fonction de pair\,; et de tous les
pairs choisir l'ennemi déclaré du premier président, avec qui, en tant
d'occasions, il faut conférer, et de plus des moins agréables au
parlement, et, par rapport à vous, montrer une légèreté singulière en
ôtant les sceaux au chancelier à qui vous veniez si nouvellement de les
rendre, et de le rappeler de Fresnes où vous l'aviez exilé. Mon refus,
que j'ose dire avoir été sage, fit laisser les sceaux au chancelier, et
vous avez vu qu'il ne vous en est pas arrivé le moindre inconvénient ni
le moindre embarras. Reste donc la place de gouverneur du roi\,; mais
cette place n'est-elle pas assez importante, assez brillante\,? ne
tire-t-elle pas naturellement d'assez grandes suites pour tenter un
homme de mon âge, qui a une famille, qui n'est revêtu que de sa dignité
de duc et pair, et qui n'a jamais été avec le maréchal de Villeroy sur
aucun pied de sentir le moindre embarras de recevoir sa place, avec la
satisfaction de ne l'avoir ni demandée ni désirée. Enfin, cette place,
en honneur, en confiance, en considération, en toutes sortes d'avantages
réels, peut-elle {[}être{]} refusée et refusée jusqu'à trois différentes
fois sans des considérations de contre-poids les plus fortes et les plus
démontrées\,? Leur base est une suite d'horreurs dont il a fallu vous
remettre trop souvent {[}le tableau{]} devant les yeux pour vous les
renouveler encore. Mais au nom de Dieu, monsieur, faites-y réflexion, et
je m'assure que vous me rendrez justice.\,»

Jusqu'ici M. le duc d'Orléans m'avait laissé parler sans m'interrompre.
Ou il n'avait pas trouvé de réplique à mes réponses, ou ces refus ne
l'avaient affecté que dans le moment que l'abbé Dubois l'avait poussé,
dont mes réponses effaçaient l'impression\,; mais l'importunité qu'il
recevait du maréchal de Villeroy, que rien de sa part n'avait pu gagner,
et ce qu'il en craignait auprès du roi dans les suites, lui tenaient au
coeur. Il ne put donc se satisfaire de mes réponses sur mon refus si
opiniâtre et si constant de la place de gouverneur du roi. Il m'en fit
des plaintes amères, et me contraignit de reprendre avec lui les raisons
de mon refus, qu'on a vues ici, avec beaucoup plus d'étendue. Comme
cette longue explication ne roula que sur les mêmes principes, tant à
l'égard des raisons de ne point ôter le maréchal de Villeroy de cette
place, quelque mal qu'il s'en acquittât, quelque incapable qu'il en
parût, et qu'il en fût, quelque dangereux qu'il y pût être au régent, et
sur celles de ne m'y point mettre quand même elle deviendrait vacante
par mort, je n'en allongerai pas ce récit. Je me contenterai de dire que
je mis enfin M. le duc d'Orléans à bout sur cet article, après une
longue et forte discussion, et que je le forçai de convenir que tous mes
refus ne méritaient point de reproches, et que j'avais eu raison de les
faire. De là, j'eus beau jeu sur le reproche général que je ne voulais
rien faire que ce qui me plaisait, et que je voulais lui faire faire
tout ce que bon me semblait.

Sur la première partie, je le fis souvenir de la façon dont je m'étais
conduit chez le chancelier dans ce comité de finances dont il voulut si
absolument que je fusse, quoi que j'eusse pu dire et supplier au
contraire plusieurs fois dans son cabinet de ma juste répugnance, par
mon incapacité sur les finances où je n'entendais rien, de mon ignorance
de la gestion du duc de Noailles qui en cachait tout au conseil de
régence, et sur le personnel du duc de Noailles, avec lequel j'étais
hors de toute mesure, qui avait apparemment ses raisons pour vouloir que
je fusse de ce comité, et que je ne me rendis qu'au commandement
inattendu et absolu qu'il m'en fit en nommant les commissaires de ce
comité au conseil de régence, dans lequel je protestai de mon incapacité
en cette matière, et de mon inutilité en choses où je n'entendais rien.
Je le priai encore de se souvenir de diverses autres choses qu'il avait
exigées de mon obéissance, à quoi je m'étais soumis malgré moi, et du
commerce qu'il avait si fortement voulu que j'eusse une fois au moins la
semaine avec Law sur sa banque et son Mississipi, auxquels il savait que
je m'étais si fort opposé dans son cabinet, et en plein conseil de
régence, lorsqu'il fut question de les établir. «\,Vous m'avez, malgré
tout ce que je pus faire, dire et prédire, forcé par une violence
d'autorité absolue d'aller apprendre à M\textsuperscript{me} la duchesse
d'Orléans la chute de son frère, au sortir du lit de justice des
Tuileries, ce qui depuis m'a brouillé entièrement avec elle, comme je le
prévis et ne pus vous en persuader. Enfin, monsieur, ajoutai-je, je n'ai
refusé rien de tout ce que vous avez désiré de moi, en choses générales
et faisables, tant qu'il m'a été possible, et vous ne m'en sauriez citer
une seule que j'aie refusée, sans que vous ayez trouvé que j'eusse
raison\,: voilà pour la première partie de votre reproche général. À
l'égard de la seconde, vous savez si je vous ai importuné pour moi ou
pour les miens. Pour ce qui est des autres, je ne vous ai jamais rien
demandé que de juste ou de convenable à votre réputation pour les choix,
et à votre intérêt, très souvent sans égard à mon amitié pour les
personnes, témoin les chefs des conseils et plusieurs membres que je
vous ai proposés et que vous avez faits. Si vous et moi pouvions nous
souvenir de quantité de grâces que j'ai procurées, par les
représentations que j'ai cru vous devoir faire, vous trouveriez que le
même principe m'a conduit, et que vous en trouveriez fort peu, et encore
de celles-là de conséquence indifférente, où mon amitié ou ma
considération pour les gens aient eu toute la part\,; si de là vous
passez à vous rappeler les affaires, vous trouverez que celles que j'ai
eues le plus à coeur ne sont pas celles qui ont réussi, comme le rang
des bâtards, l'affaire du bonnet, si criantes et si souvent et
solennellement promises, les autres querelles du parlement, ses
entreprises sur vous-même, les dangereuses et folles démarches de cette
prétendue noblesse, toutes choses où vous vous êtes laissé abuser, dont
vous vous êtes très mal tiré, qui en ont enfanté de pires, comme je vous
l'avais prédit, et dont vous ne sauriez me nier que vous ne vous soyez
repenti de la conduite que vous y avez tenue, puisque vous me l'avez
avoué vous-même, et, traité de fripons ceux qui vous y ont entraîné.
Souvenez-vous donc, s'il vous plaît, que rien ne m'a jamais si vivement
intéressé que ces choses-là, mais qu'après vous avoir pressé à mesure
sur chacune, et remontré tout ce que j'ai cru vous devoir être
représenté, j'ai embrassé tellement le parti du silence que je ne vous
en ai depuis ouvert la bouche une seule fois, et que, quand vous avez
voulu quelquefois me mettre sur ces chapitres, je n'y ai jamais pris, et
toujours détourné la conversation à autre chose sur-le-champ. Est-ce
donc là, monsieur, vouloir vous faire faire tout ce qui me plaisait, et
quand vous a plu à vous de faire si souvent tout l'opposé de ce qui
m'affectait le plus, m'avez-vous vu après moins attaché à vous et moins
occupé de votre intérêt et de votre avantage\,? Sur les affaires
publiques, vous m'avez trouvé également fidèle à ce que j'ai cru de
l'intérêt de l'État, à vous le représenter, tout le plus fortement de
raisons qu'il m'a été possible, à demeurer inébranlable dans mon avis
quand ce que vous ou vos ministres y ont opposé ne {[}m'a{]} pas paru
solide, à vous proposer de m'abstenir du conseil quand vous y craindriez
que mon opposition préjudiciât à ce que vous aviez à coeur d'y faire
passer, et à m'en abstenir en effet, sous prétexte de quelque
incommodité\,; toutes les fois que vous l'avez désiré\,; il me semble
donc, monsieur, que mes réponses à vos reproches, tant en gros qu'en
détail, sont catégoriques, plus que suffisantes et sans aucune sorte de
réplique. J'attends la vôtre, si tant est que vous en trouviez, et
cependant je n'en puis être en peine.\,»

M. le duc d'Orléans demeura quelque temps sans parler. Il était la tête
basse comme quand il se sentait embarrassé et peiné, tantôt marchant,
tantôt nous arrêtant pendant cette conversation. Rompant enfin le
silence, il se tourna à moi, et me dit en souriant que tout ce que
j'avais dit était vrai, et qu'il ne fallait plus penser à tout cela\,;
qu'il était vrai que ce groupe de refus s'était présenté à lui sous une
autre face, et l'avait fâché, et que je voyais qu'il n'avait pas été
longtemps sans me le dire franchement\,; mais qu'encore une fois il n'y
fallait plus penser et parler d'autre chose. «\,Très volontiers, lui
répondis-je, monsieur, mais qu'il me soit permis aussi de vous parler
franchement à mon tour. Vous avez été conter à l'abbé Dubois ce que je
vous dis dernièrement du traité d'Angleterre et d'Espagne, et de sa
conduite énorme pour obtenir un chapeau par le ricochet du roi
d'Angleterre à l'empereur et de l'empereur au pape, et de là cet honnête
prêtre et si désintéressé vous a mis dans la tête tous ces potages
réchauffés que vous venez si bien de m'étaler et que j'ai encore mieux
fait fondre. Avouez-moi la vérité. --- Mais, me répondit-il d'un air
honteux et embarrassé au dernier point, cela est vrai, c'est l'abbé
Dubois qui m'a rabâché tous ces refus, qui m'a poussé et qui m'a fâché
contre vous. --- Hé bien\,! monsieur, lui répliquai-je, mes réponses
vous ont-elles pleinement satisfait\,? --- Oui, me dit-il, il n'y a rien
à y répondre\,; je le savais bien, mais il m'a embrouillé l'esprit.\,»

La même faiblesse qui lui avait fait tout dire à l'abbé Dubois, et
recevoir de lui, malgré toute sa connaissance, les impressions qu'il
avait voulu lui donner contre moi, fit le même effet lorsqu'à mon tour
je le tins tête à tête, opéra le renouvellement de sa première
conviction sur ma conduite, dès que je la lui justifiai ainsi en détail,
enfin l'aveu implicite d'avoir révélé à l'abbé Dubois ce que je lui
avais dit de lui, et l'aveu formel que c'était l'abbé Dubois qui lui
avait aigri l'esprit contre moi et fourni les reproches qu'il m'avait
faits. Alors je le suppliai de réfléchir en quelles mains il s'était
livré, et si qui que ce soit leur pouvait échapper, si son plus ancien
et son plus assuré serviteur n'en était pas hors de prise, et sur choses
hors de toute sorte de raison et connues pour telles par Son Altesse
Royale, et ce que pourrait devenir tout homme hors de portée de sa
privance et d'explications avec elle, toutes les fois qu'il plairait à
l'abbé Dubois de l'écarter et de le perdre. «\,Vous avez raison, me
répondit M. le duc d'Orléans dans la dernière honte, à ce qu'il me
parut\,; je lui défendrai si bien et si sec de me parler de vous que
cela ne lui arrivera plus. Allons, qu'avez-vous pour aujourd'hui\,?»
J'eus pitié, si je l'ose dire, de l'état où je le vis. Je ne répondis
rien, et je me mis à lui rendre compte de ce que j'avais pour ce
jour-là. Peu après il entra dans son petit cabinet. J'y travaillai avec
lui assez courtement, parce que l'entretien que je viens de rapporter
avait été fort long\,; et sans plus en rien remettre en avant, nous nous
séparâmes le mieux du monde sans qu'il y ait du tout paru depuis, et
j'eus lieu de croire par la suite que M. le duc d'Orléans m'avait tenu
parole, et défendu à l'abbé Dubois de lui parler de moi. On peut juger
des dispositions de ce bon ecclésiastique à mon égard, après une
pareille confidence de son maître, de ce que je lui avais dit de lui,
entées sur tant d'autres choses, qui m'avaient mis fort mal avec lui. Le
récit simple, tel qu'on vient de le voir de cette dernière, supplée à
toute réflexion, et peint au naturel quels étaient le maître et le valet
à l'égard l'un de l'autre.

Mais, pour achever le coup de pinceau, je joindrai ici ce qui arriva peu
après à Torcy, et qu'il m'a conté lui-même. Quelques mesures que prit
Dubois pour cacher ses machines à Rome, Torcy vit tant de choses par le
secret de la poste, qu'il crut devoir avertir M. le duc d'Orléans des
menées de l'abbé Dubois à Rome. Il lui dit donc, avec sa mesure
accoutumée, que si cet abbé y travaillait pour son chapeau de l'aveu de
Son Altesse Royale, il n'avait rien à dire\,; mais que, dans
l'incertitude, il avait cru de son devoir de l'avertir de ce qu'il en
voyait. M. le duc d'Orléans se mit à rire. «\,Cardinal\,! répondit-il,
ce petit faquin\,! vous vous moquez de moi\,; il n'oserait y avoir
jamais songé.\,» Et sur ce que Torcy insista et montra les preuves, le
régent se mit en colère, et dit que, si ce petit impudent se mettait
cette folie dans la tête, il le ferait mettre dans un cul de
basse-fosse. Ce même propos fut répété à Torcy deux ou trois fois,
c'est-à-dire toutes celles que Torcy lui rendait un nouveau compte de ce
qu'il trouvait dans les lettres étrangères sur la continuation de
l'intrigue pour ce chapeau. Enfin, la dernière fois, qui fut proche du
temps que ce chapeau fut obtenu, Torcy reçut la même réponse avec la
même colère\,; mais le lendemain précis de cette réponse, Torcy étant
allé au Palais-Royal, M. le duc d'Orléans l'appela, le tira dans un coin
et lui dit\,: «\,A propos, monsieur, il faut écrire de ma part à Rome
pour le chapeau de M. de Cambrai\,; voyez à cela, il n'y a pas de temps
à perdre.\,» Torcy demeura sans parole comme une statue, et le régent le
quitta dès qu'il lui eut donné cet ordre avec le même sang-froid que
s'il ne se fût pas emporté là-dessus avec Torcy, la veille, et qu'il eût
toujours été question entre lui et Torcy de favoriser l'abbé Dubois à
Rome. C'est bien de ceci qu'on peut dire ce mauvais proverbe\,: Cela
lève la paille\footnote{On dit figurément et proverbialement de
  certaines choses qui excellent en leur genre, que cela lève la paille.}.
Aussi Torcy n'en pouvait-il revenir, non de la conduite actuelle de M.
le duc d'Orléans sur ce chapeau, non qu'il n'eût toujours soupçonné de
la comédie dans les réponses menaçantes de M. le duc d'Orléans
là-dessus, mais de la transition en vingt-quatre {[}heures{]} de ces
mêmes menaces de cul de basse-fosse, tout archevêque qu'il fût, à
ordonner à Torcy, qui ne lui en donnait aucune occasion, et qu'il appela
exprès, d'écrire à Rome en son nom, de lui régent, pour favoriser le
chapeau de l'abbé Dubois, avec la tranquillité la plus parfaite\,: tel
était le terrain d'alors.

Rome me fait souvenir qu'on apprit alors la naissance du prince de
Galles, le dernier décembre 1720. Les cardinaux Paulucci, secrétaire
d'État, Barberin, chef de l'ordre des cardinaux-prêtres, Sacripanti,
protecteur d'Écosse, Gualterio, protecteur d'Angleterre, Imperiali,
protecteur d'Irlande, Ottoboni, protecteur de France et vice-chancelier
de l'Église, n'y ayant point de chancelier, et Albane, neveu du pape et
camerlingue de l'Église, tous cardinaux des plus distingués du sacré
collège, se trouvèrent à ces couches, par ordre et de la part du pape.
Le sénat romain y fit assister de sa part les évêques de Segni et de
Monte-Fiascone, Falconieri, gouverneur de Rome, depuis cardinal,
Colligola et Ruspoli, protonotaires apostoliques\footnote{Officiers de
  la cour de Rome, qui ont un degré de prééminence sur tous les notaires
  de la même cour\,; ils reçoivent les actes des consistoires publics et
  les expédient en forme.}. Les ambassadeurs de Bologne et de Ferrare
s'y trouvèrent aussi. Les princesses des Ursins, Piombino, Palestrine et
Giustiniani, et les duchesses de Fiano et Salviati. Le prince fut
baptisé sur-le-champ par l'évêque de Monte-Fiascone, et nommé Charles.
Le pape envoya complimenter ces Majestés Britanniques, et porter au roi
d'Angleterre dix mille écus romains, un brevet à vie de jouissance de la
maison de campagne jusqu'alors prêtée à Albano, et deux mille écus pour
la meubler. On chanta un Te Deum dans la chapelle du pape, en sa
présence, et il y eut des réjouissances à Rome. Lorsque la reine
d'Angleterre vit du monde, le cardinal Tanara la fut complimenter en
cérémonie de la part du sacré collège. Le décanat vaquait alors,
contesté entre Tanara, qui l'emporta enfin, et Giudice, par un jugement
contradictoire du pape et du sacré collège. Cette naissance fut très
sensible à la cour d'Angleterre et aux papistes et jacobites de ce pays,
en sentiments fort différents\,: non seulement les catholiques et les
protestants, ennemis du gouvernement, en furent ravis, mais presque tous
les trois royaumes en marquèrent\,; de la joie autant qu'ils osèrent,
non par attachement pour la maison détrônée, mais par la satisfaction de
voir continuer une lignée dont ils pussent toujours menacer leurs rois
et leur famille, et la leur pouvoir opposer. On n'osa en France rien
marquer là-dessus, on y était trop sujet de l'Angleterre, et le régent
et Dubois trop grands serviteurs de la maison d'Hanovre, dans le point
surtout où Dubois en était pour son chapeau.

L'Angleterre perdit en ce même temps deux ministres, dont on a vu
ci-devant beaucoup de choses en rapportant les affaires étrangères, le
comte Stanhope et Craggs, les deux secrétaires d'État, qui moururent à
peu de jours l'un de l'autre. Craggs était violent et emporté\,;
Stanhope ne perdait point le sang-froid, rarement la politesse, avait
beaucoup d'esprit, de génie et de ressources. Ils furent remplacés par
Townsend et Carteret, deux grands ennemis de la France, indépendamment
de la raison d'État. Un autre personnage singulier qui avait fait grand
bruit en son temps, les suivit de fort près, le docteur Sachewerell qui,
par ses sermons sous la reine Anne, commença à attaquer le ministère et
le système d'alors, qui ne voulait que la guerre, dont la reine se défit
après.

En même temps, il y eut aussi en ce pays-ci plusieurs morts\,: Huet, si
connu de toutes sortes de savants, à quatre-vingt-huit ans, avec la tête
encore entière et travaillant toujours. Sa science vaste et nette, et sa
sage et sûre critique, avec de très bonnes moeurs, l'avaient fait
associer au célèbre Fléchier, depuis évêque de Nîmes, dans la place de
sous-précepteur de Monseigneur. Huet eut ensuite l'évêché de Soissons,
qu'il troqua pour celui d'Avranches avec Sillery, frère de Puysieux, qui
se voulait rapprocher de la cour. L'étude, qui était la passion
dominante d'Huet, comme la fortune était celle de Sillery, le fit
défaire enfin de son évêché d'Avranches pour une abbaye\,; il se retira
à Paris dans un appartement que lui donnèrent les jésuites, dans leur
maison professe, pour y jouir à son aise de leur belle bibliothèque et
de la conversation de leurs savants. Il y mourut après y avoir passé un
grand nombre d'années, toujours dans l'étude, sans presque sortir, et
menant une vie très frugale. Il y voyait beaucoup de savants, et n'avait
point d'autre plaisir ni de commerce.

La duchesse de Luynes à vingt-quatre ans, dont ce fut grand dommage, qui
laissa des enfants et beaucoup de regrets. Elle était fille unique d'un
bâtard obscur du dernier comte de Soissons, prince du sang, tué à la
bataille de Sedan ou la Maffée. M\textsuperscript{me} de Nemours,
irritée contre M. le prince de Conti et contre tous ses héritiers, fit
légitimer ce bâtard, lui donna tout ce qu'elle put, qui fut immense, et
lui fit épouser la fille du maréchal duc de Luxembourg.

La duchesse de Sully à cinquante-six ans\,: elle était fille et nièce du
duc et du cardinal de Coislin, la meilleure femme du monde, et qui
serait morte de faim sans son frère l'évêque de Metz. Sa mort ne
démentit point son nom\,: il lui vint un abcès en lieu que la modestie
ne lui permit pas de montrer à un chirurgien. Une femme de chambre la
pansa quelque temps en cachette, puis expliqua le mal aux chirurgiens.
Ce n'était rien s'ils eussent pu la traiter comme une autre\,; mais
jamais personne ne put gagner cela sur elle. La femme de chambre disait
l'état du mal à travers la porte aux chirurgiens, et faisait ce qu'ils
lui prescrivaient\,; mais cette manière de traiter par procureur la
conduisit bientôt au tombeau. Elle était veuve sans enfants.

La duchesse de Brissac à soixante-trois ans. C'était une petite bossue,
soeur de Vertamont, premier président du grand conseil, extrêmement
riche, que le duc de Brissac, frère de la dernière maréchale de
Villeroy, veuf sans enfants de ma soeur, avait épousée pour son bien,
qu'il mangea. Devenue veuve et parfaitement ruinée, son frère la prit
chez lui et lui donnait jusqu'à des souliers. Elle avait beaucoup de
vertu, infiniment d'esprit, de conversation agréable et de lecture. La
duchesse de Lesdiguières-Gondi, qui l'aimait fort, lui avait donné en
mourant une pension assez honnête.

On n'a jamais su par quel accident l'embrasement d'une maison d'artisan
embrasa toute la ville de Rennes\,; le malheur fut complet, pour la vie
et les biens. La ville a été rebâtie depuis beaucoup mieux qu'elle ne
l'était auparavant, et avec bien plus d'ordre et de commodités
publiques. Il se trouva parmi l'ancien pavé des cailloux précieux par
leurs couleurs et leur vivacité et variété, dont on f t beaucoup de
tabatières de différentes formes, qui égalèrent presque les plus belles
de ces sortes de beaux cailloux.

\hypertarget{chapitre-vii.}{%
\chapter{CHAPITRE VII.}\label{chapitre-vii.}}

1721

~

{\textsc{Affaire du duc de La Force.}} {\textsc{- Saint-Contest et
Morville, plénipotentiaires au congrès de Cambrai.}} {\textsc{- Mort,
fortune et caractère de Foucault, conseiller d'État.}} {\textsc{-
Méliant, Harlay, Ormesson, conseillers d'État.}} {\textsc{- Alliance des
Neuville et des Harlay.}} {\textsc{- Mort de Coettensao\,; de
Joffreville\,; d'Ambres\,; son caractère.}} {\textsc{- Mort de la
comtesse de Matignon.}} {\textsc{- Ambassadeur extraordinaire du Grand
Seigneur à Paris.}} {\textsc{- Son entrée.}} {\textsc{- Sa première
audience.}} {\textsc{- Vienne, en Autriche, archevêché.}} {\textsc{-
Mort de la reine de Danemark (Mecklembourg).}} {\textsc{- Dix-huit jours
après, le roi épouse la Rewenclaw, sa maîtresse.}} {\textsc{- Duperie
étrange du cardinal de Rohan par Dubois.}} {\textsc{- Mort de Clément XI
(Albane).}} {\textsc{- Innocent XIII (Conti) élu.}} {\textsc{- Condition
étrange de son exaltation.}} {\textsc{- Albéroni à Rome et rétabli.}}
{\textsc{- Intérêt des cardinaux.}} {\textsc{- Robert Walpole comme
grand trésorier d'Angleterre.}} {\textsc{- M. le duc de Chartres colonel
général de l'infanterie.}} {\textsc{- Survivance {[}de la charge{]} de
premier écuyer et du gouvernement de Marseille au fils de Beringhen, et
des bâtiments au fils de d'Antin.}} {\textsc{- Perfidie du maréchal de
Villeroy à Torcy et à moi.}}

~

En ce temps-ci commença une affaire si honteuse à la faiblesse de M. le
duc d'Orléans, si fort ignominieuse à celle des pairs, si scandaleuse au
parlement, à son animosité et à ses entreprises, si scélérate au premier
président, si abominable à l'avarice du prince de Conti, en un mot si
infâme en toutes ses parties, que je crois devoir me contenter de
l'énoncer et tirer le rideau sur les horreurs qui s'y passèrent pendant
le reste, de cette année. Les apparences très prochaines de la déroute
de Law et de ses suites nécessaires, hâtèrent ceux qui étaient le plus à
portée de les prévoir de réaliser promptement leurs papiers. Le prince
de Conti, qui en avait amassé à toutes mains, et à qui il en restait
encore après avoir asséché Law du plus gros par les quatre surtouts
d'argent en espèces qu'on a vu naguère qu'il se fit payer tout à la fois
à la banque et voiturer tout à la fois chez lui, cherchait à employer
encore des papiers qui lui restaient. Il sut que le duc de La Force
était prêt d'acheter une terre obscure, mais considérable pour sa
valeur\,; il courut sur son marché déjà conclu. Il trouva de la
résistance, et l'orgueil joint à l'avarice ne la put pardonner. Il avait
toujours fait une cour basse au parlement et au premier président de
Mesures, pour essayer de donner de l'ombrage à M. le Duc et à M. le duc
d'Orléans même, qui le méprisèrent trop pour en prendre jamais. Mesmes
et le parlement, bien aises d'avoir un client prince du sang, le
cultivaient\,; il se promettait tout d'eux. Law parti et la banque et la
compagnie en désarroi, le prince de Conti imagina de faire faire une
insulte juridique au duc de La Force, sous prétexte de monopole, bien
assuré que Mesmes et le parlement se porteraient de grand cour à faire
cet affront à un duc et pair. Il ne se trouva à la fin que de la Chine,
des paravents et quelques autres colifichets semblables, qui montrèrent
en plein l'iniquité, l'excès et l'abus de la passion. Il ne s'en fallut
rien dans le cours de l'affaire que le maréchal d'Estrées ne fût
attaqué\,; la prise y était tout entière, quoiqu'il n'y eût jamais pensé
mal\,; mais M. le duc d'Orléans imposa, et comme il n'était pas duc et
pair, et ne le fut qu'en juillet 1723, par la mort du dernier duc
d'Estrées, en directe\footnote{En ligne directe.} gendre du duc de
Nevers, le parlement ni le premier président ne se soucièrent pas de
cette poursuite.

Saint-Contest, qui avait été troisième ambassadeur plénipotentiaire à
Bade, et Morville, ambassadeur à la Haye, furent nommés
plénipotentiaires au congrès de Cambrai, et partirent incontinent pour
s'y rendre.

La mort de Foucault, qui avait été intendant de Caen et chargé des
affaires de Madame, fit vaquer une troisième place de conseiller d'État.
On a vu en son lieu combien j'avais été content de Méliant, maître des
requêtes, dans une grande affaire que je gagnai au conseil, contre le
duc de Brissac, la duchesse d'Aumont, etc., dont il était rapporteur, et
que je gagnai depuis au fond au parlement de Rouen. Je désirais depuis
longtemps qu'il fût conseiller d'État. Il avait été intendant de l'armée
en Espagne sous M. le duc d'Orléans, et l'était alors de Lille. Cette
place et son ancienneté l'y portaient naturellement. Il était, de plus,
sans aucun reproche. Il avait déplu en Espagne aux valets de M. le duc
d'Orléans, qui lui en avaient donné de mauvaises impressions, en sorte
que j'eus toutes les peines du monde à lui faire rendre cette justice.
Le maréchal de Villeroy, qui dans le mécontentement extrême dont était
M. le duc d'Orléans de lui, en obtenait d'autorité tout ce qu'il
voulait, fit donner la seconde de ces trois places à Harlay, fils du
premier ambassadeur plénipotentiaire à Ryswick. Celui-ci était un fou
plein d'esprit, plaisant, dangereux, et peut-être la plus indécente
créature qu'on pût rencontrer, de plus ivrogne, crapuleux et d'une
débauche débordée\,; il avait été intendant de Metz, puis d'Alsace\,; la
capacité ne lui manquait pas, mais il ne prenait pas la peine de rien
faire\,; ses secrétaires faisaient tout\,; il lui était arrivé partout
mille scandales publics, et il était si accoutumé et si heureux à s'en
tirer, et à monter toujours de place en place jusqu'à l'intendance de
Paris, qu'il disait\,: «\,Encore une sottise, et je serai secrétaire
d'État.\,» Le maréchal de Villeroy le protégeait hautement\,; il avait
été fort ami du premier président Harlay, et parent des Harlay, qui s'en
faisaient honneur réciproquement. Alincourt, fils de Villeroy,
secrétaire d'État, avait épousé la fille unique de Mandelot, gouverneur
de Lyon, etc., et d'une Robertet. La Ligue avait fait ce mariage, et
Alincourt eut la survivance du gouvernement de son beau-père. Il n'eut
qu'une fille unique de ce mariage, qui épousa le marquis de Courtenvaux,
chevalier du Saint-Esprit, premier gentilhomme de la chambre, fils du
maréchal de Souvré, dont une fille unique, que le premier maréchal de
Villeroy sacrifia à la faveur, et maria, étant son tuteur, à M. de
Louvois.

M. d'Alincourt, veuf de la Mandelot, épousa la fille aînée du célèbre
Harlay-Sancy, dont il eut le premier maréchal de Villeroy\,; enfin le
chancelier, à qui les sceaux avaient pensé être ôtés, comme on l'a vu,
depuis si peu de temps, ne laissa pas d'avoir le crédit de faire donner
la troisième place à d'Ormesson, intendant des finances, frère de sa
femme.

Foucault, conseiller d'État, qui venait de mourir, était un honnête
homme, savant en antiquités et en médailles, dont il avait un beau
cabinet. Ce goût commun avec le P. de La Chaise lui en acquit la
connaissance, puis l'amitié, qui l'avança et le protégea
toujours\footnote{Il a été question plusieurs fois de Joseph Foucault,
  et nous avons cité en note (t. XII) des extraits des Mémoires qu'il a
  laissés et qui sont encore inédits.}. Il était père de ce Magny, dont
il a été parlé en soin lieu, et qui passa en Espagne, où je le trouvai.

Je perdis en ce temps-là Coettenfao, brave gentilhomme et très galant
homme, fort mon ami, lieutenant général, que j'avais fait chevalier
d'honneur de M\textsuperscript{me} la duchesse de Berry. Il n'était
point vieux et n'eut point d'enfants.

Joffreville, lieutenant général distingué, mourut aussi. Il était fort
bien avec M. le duc d'Orléans et fort ami du maréchal de Berwick, sous
qui il avait servi en Espagne. Le feu roi l'avait nommé, par son
testament, sous-gouverneur du roi d'aujourd'hui\,; il était aussi fort
bien avec le duc du Maine\,; il vit promptement la difficulté de ce
double attachement dans cette place auprès du jeune roi. C'était un
honnête homme et sage\,; il refusa sous prétexte de sa santé\,; et
Ruffey, qui se disait Damas et ne l'était point, eut cette place\,: il
était du pays de Dombes, extrêmement attaché à M. du Maine.

Le marquis d'Ambres mourut en même temps à quatre-vingt-deux ans.
C'était un grand homme très bien fait, du nom de Gelas, très brave
homme, qui avait grande mine, de l'esprit, beaucoup de hauteur, qui
quitta le service pour ne pas écrire monseigneur à Louvois, qui ne lui
pardonna jamais, ni le roi non plus. Il avait de grandes terres, où il
fit le petit tyran de province, comme autrefois, s'y fit des affaires
désagréables, et eut force dégoûts dans sa charge de lieutenant général
de Guienne. Son père fut chevalier de l'ordre en 1633\,; il ennuyait
souvent le peu de monde qu'il voyait à la cour, où, quoique mal, il
allait souvent. Après la mort du roi, il tint chez lui, à Paris,
quelques jours de la semaine, une petite assemblée de vieux ennuyeux
comme lui, où se débitaient les nouvelles et la critique d'esprits
chagrins.

Le comte de Matignon, chevalier de l'ordre, dont le fils épousa
M\textsuperscript{lle} de Monaco, avec de nouvelles lettres de duc et
pair de Valentinois, comme on l'a vu en son lieu, promises par le feu
roi et depuis exécutées, perdit sa femme, fille aînée de son frère aîné,
qui lui en avait apporté tous les biens. C'était une femme peu propre au
monde, et qui vécut toujours fort retirée.

Paris vit un spectacle peu accoutumé, le dimanche 28 mars, qui donna
beaucoup de jalousie aux premières puissances de l'Europe. Le Grand
Seigneur, qui ne leur envoie jamais d'ambassades, sinon si rarement à
Vienne, à quelque grande occasion de traité de paix, en résolut une,
sans en être sollicité, pour féliciter le roi sur son avènement à la
couronne, et fit aussitôt partir Méhémet-Effendi Tefderdar, c'est-à-dire
grand trésorier de l'empire, en qualité d'ambassadeur extraordinaire,
avec une grande suite, qui s'embarquèrent sur des vaisseaux du roi, qui
se trouvèrent fortuitement dans le port de Constantinople. Il débarqua
au port de Sète, en Languedoc, parce que la peste était encore en
Provence. Il lit même quarantaine et le détour par Bordeaux pour venir à
Paris, défrayé de tout depuis son débarquement, où il fut reçu par un
gentilhomme ordinaire du roi et des interprètes de langues, qui
l'accompagnèrent jusqu'à Paris. Il y arriva le 8 mars, au faubourg
Saint-Antoine, où il demeura huit jours, complimenté de la part du roi,
etc., comme les ambassadeurs extraordinaires des monarques de l'Europe.

Le dimanche 16 mars, le maréchal d'Estrées et Rémond, introducteur des
ambassadeurs, l'allèrent prendre à une heure après midi. Dès qu'ils
furent arrivés, ils montèrent à cheval avec l'ambassadeur entre eux
deux. Deux carrosses du maréchal, force valets de pied, pages,
gentilshommes, chevaux de main, la police avec trompettes et timbales,
trois escadrons d'Orléans-Dragons, douze chevaux de main des écuries du
roi, trente-six Turcs à cheval deux à deux, portant des fusils et des
lances, Merlin, aide introducteur, à cheval, puis les principaux
officiers de l'ambassade, quatre trompettes de la chambre du roi, six
chevaux de main de l'ambassadeur, harnachés à la turque, et tout cela
extrêmement magnifique\,; enfin l'interprète du roi, précédant
immédiatement l'ambassadeur, dont le cheval était harnaché à la turque.
Il marchait de front avec le maréchal et l'introducteur, environnés de
leur livrée et de valets de pied turcs. L'écuyer de l'ambassadeur
marchait à cheval derrière lui, portant son sabre, et vingt martres du
Colonel-général les côtoyaient à droite et à gauche\,; venaient ensuite
les grenadiers à cheval, le régiment Colonel-général, puis les carrosses
du roi et les autres qui vont aux entrées, côtoyés par la
connétablie\footnote{Archers chargés de faire exécuter les sentences du
  tribunal des maréchaux de France appelé connétablie.}. Le régiment
d'infanterie du roi, la compagnie de la Bastille, celle des fusiliers,
se trouvèrent en haie jusqu'à la place Royale\,; l'ambassadeur fut
conduit par de longs détours à la rue Saint-Denis, Saint-Honoré, etc.,
et partout des pelotons, des escouades du guet. Il trouva la compagnie
du prévôt de la monnaie en haie dans cette rue, le guet à cheval sur le
pont Neuf bordé du régiment des gardes, et force trompettes et timbales
autour de la statue d'Henri IV. La compagnie du lieutenant de robe
courte\footnote{Lieutenant du prévôt de Paris qui poursuivait les
  vagabonds et meurtriers. Il commandait une compagnie d'archers. On
  l'appelait lieutenant de robe courte pour le distinguer des
  lieutenants de robe longue, dont la fonction se bornait à juger les
  procès.}, et celle du prévôt de l'île\footnote{Prévôt des maréchaux
  chargé de maintenir l'ordre dans toute l'étendue de l'Ile-de-France.},
se trouvèrent dans les rues Dauphine et de Vaugirard. Arrivés à l'hôtel
des ambassadeurs extraordinaires, rue Tournon, ils mirent pied à terre
dans la cour. Le maréchal accompagna l'ambassadeur jusque dans sa
chambre, qui aussitôt après, lui donnant la main, le conduisit à son
carrosse, et le vit sortir de sa cour. Tous les chevaux que montèrent
l'ambassadeur et sa suite étaient des écuries du roi, et les chevaux de
main de l'ambassadeur aussi, menés par des Turcs à cheval.

Le vendredi 21 du même mois, le prince de Lambesc et Rémond,
introducteur des ambassadeurs, allèrent dans le carrosse du roi prendre
l'ambassadeur à l'hôtel des ambassadeurs extraordinaires, où il fut
toujours logé et défrayé avec toute sa nombreuse suite, tant qu'il fut à
Paris, et aussitôt ils se mirent en marche pour aller à l'audience du
roi la compagnie de la police avec ses timbales et ses trompettes à
cheval, le carrosse de l'introducteur, celui du prince de Lambesc,
entourés de leur livrée, précédés de six chevaux de main, et de huit
gentilshommes à cheval, trois escadrons d'Orléans, douze chevaux de
main, menés par des palefreniers du roi à cheval, trente-quatre Turcs à
cheval, deux à deux, sans armes, puis Merlin, aide introducteur, et huit
des principaux Turcs à cheval, le fils de l'ambassadeur à cheval, seul,
portant sur ses mains la lettre du Grand Seigneur dans une étoffe de
soie, six chevaux de main, harnachés à la turque, menés par six Turcs à
cheval, quatre trompettes du roi à cheval\,; l'ambassadeur entre le
prince de Lambesc et l'introducteur, tous trois de front à cheval,
environnés de valets de pied turcs et de leurs livrées, côtoyés de vingt
maîtres du régiment Colonel-général\,; ce même régiment, précédé des
grenadiers à cheval, suivait\,; puis le carrosse du roi et la
connétablie. Les mêmes escouades et compagnies ci-devant nommées à
l'entrée se trouvèrent postées dans les rues du passage, dans la rue
Dauphine, sur le pont Neuf, dans les rues de la Monnaie et Saint-Honoré,
à la place de Vendôme, devant le Palais-Royal, à la porte Saint-Honoré,
avec leurs trompettes et timbales\,; depuis cette porte en dehors
jusqu'à l'esplanade, le régiment d'infanterie du roi en haie des deux
côtés, et dans l'esplanade les détachements des gardes du corps, des
gens d'armes, des chevau-légers, et les deux compagnies entières des
mousquetaires. Arrivés en cet endroit, les troupes de la marche et les
carrosses allèrent se ranger sur le quai, sous la terrasse des
Tuileries\,: l'ambassadeur, avec tout ce qui l'accompagnait et toute sa
suite à cheval, entra par le pont tournant dans le jardin des Tuileries,
depuis lequel jusqu'au palais des Tuileries, les régiments des gardes
françaises et suisses étaient en haie des deux côtés, les tambours
rappelant et les drapeaux déployés. L'ambassadeur et tout ce qui
l'accompagnait passa ainsi à cheval le long de la grande allée, entre
ces deux haies, jusqu'au pied de la terrasse, où il mit pied à terre, et
fut conduit dans un appartement en bas, préparé pour l'y faire reposer
en attendant l'heure de l'audience.

À midi, l'ambassadeur, accompagné du prince de Lambesc et de
l'introducteur, sortit de cet appartement avec tout son cortège, précédé
de son fils, qui portait la lettre du Grand Seigneur sur ses mains
élevées, et suivait l'aide introducteur. Il trouva, comme les autres
ambassadeurs extraordinaires, le grand maître et le maître des
cérémonies au bas de l'escalier, bordé jusqu'au haut par les
Cent-Suisses\,; il en trouva d'autres en haie dans leur salle, leur
drapeau déployé, et Courtenvaux à l'entrée pour le recevoir, qui faisait
la charge de leur capitaine pour son neveu enfant. Le duc de Noailles,
capitaine des gardes en quartier, le reçut à l'entrée de la salle des
gardes, en haie et sous les armes. Il traversa le grand appartement
jusqu'à la galerie. Elle était tendue des plus belles tapisseries de la
couronne\,; les dames fort parées remplissaient les gradins
magnifiquement ornés, et la galerie, couverte de beaux tapis de pied,
était fort remplie d'hommes. Au fond, elle était traversée de trois
marches, et au bout de quelque espace, de deux autres sur lesquelles
était le trône du roi\,; à ses côtés étaient, à droite et à gauche, M.
le duc d'Orléans et les princes du sang, debout et toujours découverts.
Le grand chambellan, le premier gentilhomme de la chambre, le grand
maître de la garde-robe et le maréchal de Villeroy, étaient tous quatre
derrière le roi\,; l'archevêque de Cambrai au bas des deux premières
marches\,; à droite et plus reculés, les trois autres secrétaires d'État
sur le même plain-pied.

Dès que l'ambassadeur put être aperçu du roi, il s'inclina très
profondément à l'orientale, sa main droite sur sa poitrine. Alors le roi
se leva sans se découvrir, et l'ambassadeur s'avança au pied des trois
premières marches, où il fit sa seconde révérence. Il monta ensuite ces
trois degrés, ayant à sa droite le prince de Lambesc et le duc de
Noailles ensemble de front, à gauche l'introducteur et l'interprète,
derrière lui son fils, portant la lettre du Grand Seigneur en la manière
qu'on a dit\,; l'ambassadeur fit là sa troisième révérence, prit des
mains de son fils la lettre du Grand Seigneur, qu'il éleva sur sa tête,
puis la remit à l'archevêque de Cambrai, comme secrétaire d'État des
affaires étrangères, lequel la posa sur une table près et à la droite du
trône, couverte de brocard d'or. L'ambassadeur fit au roi son compliment
de très bonne grâce, d'un air fort respectueux, mais point timide ni
embarrassé. L'interprète l'expliqua. Le roi ne parla point ni M. le duc
d'Orléans\,; le maréchal de Villeroy fit une courte réponse que
l'interprète rendit à l'ambassadeur. Alors il fit sa révérence et se
retira à reculons, sans tourner le dos tant qu'il put être vu du roi,
fit ses deux autres révérences où il les avait faites en venant, puis
s'en alla lentement, regardant fort et d'un air très assuré tout ce qui
s'offrait à sa vue. Le prince de Lambesc le conduisit à l'appartement où
il était entré d'abord et y prit congé de lui. L'ambassadeur s'y reposa
un peu\,; puis l'introducteur à côté de lui, à sa gauche, il traversa la
terrasse du palais des Tuileries, monta à cheval avec tout ce qui
l'accompagnait, trouva dans la grande allée, au pont tournant, à
l'esplanade, les mêmes troupes dans les mêmes postes et les mêmes
honneurs qu'en venant, le régiment du roi d'infanterie en haie jusqu'à
la porte de la Conférence, les troupes qui l'avaient accompagné rangées
sur le quai des Tuileries, et les carrosses, qui se remirent en marche
dans le même ordre qu'en venant. Il passa sur le pont Royal, le quai des
Théatins, devant le collège Mazarin, la rue Dauphine, et trouva partout,
jusqu'à la porte de l'hôtel des ambassadeurs extraordinaires, les mêmes
troupes et détachements, instruments de guerre qu'il avait trouvés
allant à l'audience, pendant laquelle elles s'étaient postées sur les
lieux de son retour. La singularité de la cérémonie m'a engagé à
l'insérer ici, quoiqu'elle se trouve dans les gazettes.

On approuva fort le chemin qu'on fit prendre à cet ambassadeur, surtout
celui du jardin des Tuileries, avec tout cet air si martial de ce grand
nombre des plus belles troupes, et de l'avoir fait retourner par le quai
des Tuileries et par celui des Théatins, qui sont les endroits où Paris
paraît le mieux. Que serait-ce si on dépouillait le pont Neuf de ces
misérables échoppes, et tous les autres ponts de maisons et les quais de
celles qui sont du côté de la rivière\,? Peu de jours après
l'ambassadeur turc fut au Palais-Royal, à l'audience de M. le duc
d'Orléans, mais tout simplement, et reçu comme les ambassadeurs
extraordinaires, conduit sans troupes et avec peu de cortège par
l'introducteur de M. le duc d'Orléans.

L'empereur obtint enfin l'érection de l'évêché de Vienne en archevêché,
avec un petit démembrement des diocèses de Passau et de Salzbourg. Ces
deux prélats et leurs chapitres s'y étaient longuement opposés à Vienne
et à Rome.

La reine de Danemark mourut à Copenhague d'une longue maladie, à
cinquante-quatre ans. Elle était fille de Gustave-Adolphe de
Mecklembourg-Gustrow et d'une Holstein-Gottorp. Elle avait épousé, en
décembre 1695, Frédéric IV, roi de Danemark, le même qui voyagea et vint
en France étant prince royal. Elle mourut le 15 mars de cette année
1721. Elle ne laissa que le feu roi de Danemark, Christian-Frédéric,
mort en 1746, père du régnant, gendre du roi d'Angleterre, et
Charlotte-Amélie, encore vivante sans alliance. Frédéric, amoureux
depuis longtemps de la fille du comte de Rewenclaw, chancelier de
Danemark, dont il avait eu une bâtarde en 1709, donna en 1712 le titre
de duchesse de Sleswig à cette maîtresse, et n'eut pas honte de déclarer
son mariage avec elle le 4 avril, c'est-à-dire dix-huit jours après la
mort de la reine sa femme, et l'épousa en effet publiquement à
Copenhague le même jour. Le 7 du même mois, c'est-à-dire trois jours
après, le prince et la princesse ses enfants se retirèrent à Jarespries
en Jutland. Tels sont les funestes effets des amours des rois\,; plût à
Dieu que ceux-ci fussent les plus grands\,!

Il y avait déjà quelque temps que l'abbé Dubois avait persuadé au
cardinal de Rohan qu'il le ferait premier ministre, s'il voulait aller à
Rome presser son chapeau, et Rohan se préparait au départ avec de
grandes sommes que Dubois lui faisait donner par M. le duc d'Orléans,
pour le défrai de son voyage, lorsqu'on apprit par un courrier du
jésuite Lafitau, évêque de Sisteron, que Dubois tenait à Rome avec
d'autres agents encore, la mort du pape Clément XI, le 19 mars, n'ayant
guère été que vingt-quatre heures malade, à soixante et onze ans, près
d'onze ans de cardinalat et un peu plus de vingt ans de pontificat. Il
était de Pezaro, où les Albani étaient peu de chose. La manière dont il
a gouverné se voit si bien dans ce qui a été rapporté ici des affaires
étrangères par Torcy, qu'il serait superflu de s'étendre sur son
caractère. Nos cardinaux se pressèrent d'arriver à Rome, où Rohan trouva
le pape fait \footnote{Michel-Ange Conti, né le 15 mai 1655, fut élu
  pape le 8 mai 1721. Il prit le nom d'Innocent XIII et mourut le 7 mars
  1724.}. Tencin et Lafitau avaient fait leur cabale et tiré un billet
de la main du cardinal Conti, par lequel il promettait, s'il était élu
pape, de faire incontinent après Dubois cardinal\,; ce billet fut donné
assez longtemps avant la maladie du pape pour avoir le loisir de former
la cabale.

Clément XI, qui avait plusieurs descentes, menaçait d'une fin prochaine
et prompte. Il était fort gros, rompu aussi au nombril, relié de partout
et soutenu par une espèce de ventre d'argent, en sorte que l'accident le
plus léger et le plus imprévu suffisait pour l'emporter brusquement,
comme il arriva en effet. Dubois, informé du billet et du succès de la
cabale, fut si transporté de joie de la mort du pape, qu'il ne la put
contenir ni l'imprudence de dire qu'il ne fallait point d'autre pape que
Conti. M. le duc d'Orléans m'en parla aussi comme d'un sujet qu'il
désirait passionnément, sur lequel il pouvait compter, et qui, selon
toutes les mesures et les apparences, serait élu, mais sans me rien dire
de la convention du cardinalat. Conti fut élu en effet le 8 mai au
matin, le trente-huitième jour du conclave. La joie de M. le duc
d'Orléans parut grande à cette nouvelle\,; Dubois ne se possédait pas,
et ne fut pas trois mois sans recevoir cette calotte si ardemment
désirée et si monstrueusement procurée.

La mort de Clément XI termina les affaires d'Albéroni à Rome, où on
travaillait à le priver juridiquement du chapeau. Il fut mandé au
conclave errant encore et caché en Italie. La voix au conclave, qui fait
la base de la grandeur et de l'importance des cardinaux, leur est trop
chère pour souffrir qu'aucun en soit privé pour quelque cause que ce
puisse être. Albéroni était l'opprobre du sacré collège qui le sentait
vivement\,; il était actuellement in reatu\footnote{En accusation.},
puisqu'à Rome son procès s'instruisait juridiquement pour le dépouiller
de la pourpre. Le roi et la reine d'Espagne poursuivaient publiquement
et ardemment cette affaire. Le pape, indignement outragé par Albéroni
dès qu'il eut son chapeau, et qu'il n'eut plus besoin de lui, le
poussait sous main de toutes ses forces\,; il n'était protégé d'aucune
couronne ni d'aucune puissance, qu'il avait toutes insultées\,; mais il
avait le chapeau, et ses collègues, devant qui son procès s'instruisait,
quelque indignés qu'ils fussent de sa promotion contre laquelle devant
et depuis ils avaient tous si fortement et si unanimement crié, excepté
les Espagnols et les Français par la crainte de leurs maîtres, mais qui
sous main l'avaient éloignée tant qu'ils avaient pu, ne s'accommodaient
point du dépouillement d'un cardinal de la pourpre. Ils en regardaient
l'exemple comme très funeste qui les rendait trop dépendants de leurs
rois et des papes.

L'indépendance est leur point capital\,; ils y étaient peu à peu
parvenus\,; ils n'avaient garde de contribuer à en déchoir pour quelque
considération que ce pût être. Qu'un cardinal prince ou fort grand
seigneur remette le chapeau pour se marier quand l'état de sa maison
l'exige, à la bonne heure\,; mais de voir un cardinal se priver du
chapeau par pénitence et comme mal acquis (comme le voulut faire le
cardinal de Retz, quand Dieu l'eut touché, et qu'il se retira), c'est ce
que les cardinaux ne veulent pas souffrir (comme il arriva au même
cardinal de Retz, dont la demande fut rejetée, et qui demeura cardinal,
malgré lui), beaucoup moins par privation du chapeau. C'est ce qui fit
marcher si lentement la congrégation établie pour le jugement d'Albéroni
qui, malgré tous les efforts de l'Espagne, secondés de toute la volonté
et de tout ce que le pape put faire, prolongea ce procès dans
l'espérance des futurs contingents, de la mort du pape surtout, comme il
arriva. Question se mut alors si Albéroni fugitif, caché, actuellement,
bien qu'absent, sur la sellette devant cette congrégation établie pour
le juger, le procès fort avancé, il pouvait être admis ou exclu du
conclave. Ce même intérêt des cardinaux les engagea tout aussitôt à
déclarer que la situation en laquelle il se trouvait ne pouvait
l'exclure du conclave\,; que, s'il en était déclaré exclu, il serait en
droit d'en appeler, et cependant de protester contre toute élection de
pape, faite sans lui\,; que cet acte rendrait l'élection irrégulière et
douteuse, et pouvait conduire à un schisme, tellement qu'il fut invité à
deux reprises de venir au conclave, et d'y donner sa voix. Il différa
pour éviter l'air d'empressement, et montrer la prétendue justice de sa
cause, en ne venant au conclave qu'après une invitation réitérée de
ceux-là même qui étaient naguère ses juges en privation du chapeau. Il
arriva donc à Rome, mais sans entrée, dans son propre carrosse, et fut
reçu dans le conclave avec les mêmes honneurs que tous les autres
cardinaux où il fit toutes les fonctions de sa dignité.

Peu de jours après l'élection, il s'absenta de Rome comme pourvoir s'il
serait encore question de son affaire, mais elle tomba d'elle-même. Le
nouveau pape n'y avait nul intérêt. Celui des cardinaux était tout
entier qu'il ne s'en parlât plus. L'Espagne comprit enfin l'inutilité
désormais de ses cris. Dubois sentait qu'il n'allait pas moins
déshonorer le sacré collège et le pape qui l'y allait mettre, qu'avait
fait Albéroni\,; avait intérêt que le rideau fût tiré sur ce confrère,
tellement qu'après une courte absence, Albéroni loua dans Rome un
magnifique palais, et y revint pour toujours avec une suite, une dépense
et une hauteur que lui fournissaient les dépouilles de l'Espagne. Il s'y
trouva donc vis-à-vis du cardinal del Giudice et tous deux vis-à-vis de
la princesse des Ursins, triangle rare qui fit souvent à Rome un
spectacle singulier. Dans les suites Albéroni qui les vit mourir tous
deux parvint à être légat de Ferrare, et s'y faire continuer longtemps,
toutefois peu compté et peu considéré à Rome, où il est encore vivant et
sain de tête et de corps à quatre-vingt-six ans\footnote{Albéroni est
  mort en 1752 à quatre-vingt-sept ans. Saint-Simon dit que ce cardinal
  avait quatre-vingt-six ans, au moment où il écrit\,; c'est donc en
  1751 qu'il a composé cette partie de ses mémoires.}.

Quant au nouveau pape, il avait soixante-six ans et quatorze de
cardinalat, avait été nonce en Suisse, puis en Portugal, pour lequel il
avait conservé un grand attachement. Il était d'une des quatre premières
maisons romaines, allant de pair sans difficulté avec les Ursins, les
Colonne et les Savelli\,; ces derniers sont éteints et ayant donné
beaucoup de papes et de cardinaux. Sa naissance avait un peu suppléé à
ses talents. C'était un homme doux, bon, timide, qui aimait fort sa
maison, et qui parut peu sur le siège apostolique. Tencin dès lors
pensait au cardinalat. Trop petit compagnon pour oser montrer y
prétendre, il se renferma dans les basses ruses qui l'avaient porté
jusqu'où il se trouvait. Il agit donc sous terre\,; il fut amusé\,; il
s'en aperçut enfin et menaça le pape, s'il ne le contentait, de rendre
public l'écrit qu'il avait de sa main, qui l'avait fait pape, par lequel
il s'engageait, s'il le devenait, de faire incontinent après Dubois
cardinal. Le pape se trouva donc dans de doubles horreurs, ou de faire
Tencin cardinal motu proprio sans qu'aucune puissance s'y intéressât,
sur l'autorité de laquelle il pût excuser une promotion de tous points
si indigne, ou de se voir déshonoré en plein par la publicité de ce
billet de sa main. L'embarras, le dépit, la douleur de se voir réduit en
de si cruelles extrémités, altérèrent tellement sa santé qu'il en
mourut, et finit ainsi sa vie sans être tombé dans aucune des deux
infamies, dont la juste frayeur et horreur le précipita dans le tombeau
un peu plus de deux ans après qu'il fut monté sur la chaire de saint
Pierre.

Ce fut vers ce temps-ci que Robert Walpole fut fait premier commissaire
de la trésorerie d'Angleterre et chancelier de l'Échiquier\,;
c'est-à-dire, grand trésorier sans en avoir le titre, et n'y en ayant
point. Ce ministre l'a été si longtemps\footnote{Robert Walpole devint
  ministre pour la seconde fois en 1721 et le resta vingt et un ans
  jusqu'en 1742.}, et a fait tant de bruit dans le monde par sa
capacité, que j'ai cru devoir marquer cette époque.

Le maréchal de Villeroy fit en ce temps-ci un tour de courtisan
supérieur à lui. Je ne sais qui lui en donna le conseil trop fort pour
que je l'aie cru pris de lui-même. Dans la situation où il se voyait
avec M. le duc d'Orléans et dans le mépris qu'il faisait de la timidité
et de la faiblesse de ce prince, qui, en même temps qu'il mourait
d'envie et d'impatience de le chasser, ne savait lui refuser aucune
chose et le recevait avec ouverture et respect, il l'entraîna dans la
plus grande faute qu'il pût faire, pour du même coup lui persuader son
attachement et le rendre odieux au roi et suspect à toute la France. Il
proposa à M. le duc d'Orléans de ressusciter le puissant office de la
couronne de colonel général de l'infanterie, en faveur de M. le duc de
Chartres, et l'assomma de tant d'autorité et d'exclamations qu'il en
vint à bout sur-le-champ, et dans le plus grand secret pour éviter que
quelqu'un n'ouvrît les yeux au régent, si, avant que cette affaire fût
faite, il venait à en parler à qui que ce fût. Parler au roi et
l'obtenir ne fut comme on peut le croire, que l'affaire d'un instant. Le
Blanc eut ordre d'en dresser l'édit et les patentes dans le même secret
et avec la même diligence. Personne ne le sut donc que par le
remerciement que M. le duc de Chartres en fit publiquement au roi, mené
par M. le duc d'Orléans en même temps que le parlement l'enregistrait.

Cette compagnie, conduite par le premier président, à qui sans doute le
maréchal de Villeroy avait parlé à l'oreille, n'eut garde de faire la
moindre difficulté et de ne pas faire sa cour au régent, d'une chose qui
pouvait si aisément servir dans la suite de matière à l'étrangler. En
effet on a vu quelle importante figure a su faire le fameux duc
d'Épernon, par cette charge qui dispose de tous les emplois de
l'infanterie, et des états-majors des places et des régiments
d'infanterie, seule alternativement avec le roi, même de celui des
gardes, qui décide souverainement de tous les détails des corps et des
garnisons et avec qui il faut que la cour compte sur tout ce qui regarde
l'infanterie. On laisse à penser ce qu'une telle charge pouvait devenir
entre les mains d'un premier prince du sang, fils unique du régent, et à
l'âge de l'un et de l'autre, avec le gouvernement du Dauphiné et la
parenté si proche de Savoie. Il est vrai que le régiment des gardes et
celui du roi furent soustraits à cet office par sa réérection. Mais cela
marquait plus la faiblesse du régent que la diminution d'un pouvoir
énorme sans cela, et que M. de Chartres serait toujours en état de
reprendre dans la suite sur ces deux corps exceptés sans droit de leur
part. La surprise générale fut grande, et les réflexions peu
avantageuses qui ne furent ni tues ni épargnées. Le maréchal de Villeroy
n'avait pas l'esprit d'en cacher sa maligne joie, et M. le duc d'Orléans
fut longtemps à s'apercevoir du tort extrême qu'il s'était fait. Il ne
me parla point de l'affaire avant qu'elle fût faite, parce qu'elle la
fut dans un tourne-main. Peut-être attendit-il après que je lui en fisse
mon compliment, comme tout le monde\,: s'il l'attendit, il se trompa\,;
je ne lui en dis jamais une parole, et je n'allai point chez M. son
fils. On a pu voir ici en plusieurs endroits que j'avais pour maxime de
ne lui parler jamais des choses qu'il avait mal faites, quand il ne m'en
parlait pas le premier. Je me contentai donc sur celle-ci de lui montrer
par mon silence combien je la désapprouvais. Ainsi nous ne nous en
sommes jamais parlé l'un à l'autre.

Ce prince donna en même temps à Beringhen la survivance de sa charge de
premier écuyer et de son gouvernement des forts et citadelle de
Marseille, pour son fils. D'Antin obtint en même temps pour le sien sa
survivance des bâtiments.

L'autorité de Dubois devenait tous les jours plus extrême. C'était un
premier ministre en plein, qui gardait même peu de bienséance pour son
maître. Tout le monde en souffrait et en gémissait\,; ceux qui voyaient
les choses de plus près, ceux qui aimaient l'État, ceux qui étaient
vraiment attachés à M. le duc d'Orléans, plus que les autres. Ce trait
de malice du maréchal de Villeroy, et d'autorité sur M. le duc
d'Orléans, frappa Torcy. Peu de jours après sortant du conseil de
récence, il me demanda une conversation particulière et prompte. J'allai
chez lui le lendemain, pour être moins interrompu que chez moi, ou {[}de
crainte{]} que fermant ma porte, ce tête-à-tête pût faire bruit. Torcy
me parla sur l'excès de l'abandon de M. le duc d'Orléans à Dubois, avec
cette sagesse, cette lumière, cette précision qui lui étaient si
naturelles, et m'en exposa tous les dangers pour les dehors et pour les
dedans. Je ne m'arrêterai point à ce qu'il m'en dit\,: cent endroits de
ces Mémoires marquent assez ce qu'il m'en put dire\,; nous ne nous
apprenions rien l'un à l'autre là-dessus, et nos avis étaient très
uniformes\,; mais la question fut du remède\,; nous nous contâmes
réciproquement ce qui nous était arrivé avec M. le duc d'Orléans, à
l'égard de Dubois, et nous conclûmes aisément qu'il n'y avait que
quelque chose de fort qui frappât M. le duc d'Orléans, non quant aux
choses, après toutes celles que je lui avais dites, mais quant au poids
des personnes réunies à lui en parler. Torcy s'étendit sur la faiblesse
du régent pour le maréchal de Villeroy, dont les preuves se voyaient
sans cesse et nouvellement par cette charge de l'infanterie, dont la
plus légère réflexion lui aurait fait sentir le piège, et sur la crainte
qu'il prenait si aisément de M. le Duc, témoin nouvellement l'étrange
scène qui se passa entre eux à ce conseil de régence, que j'ai rapportée
ci-dessus. M. de Torcy me proposa donc de nous concerter avec M. le Duc,
et avec le maréchal de Villeroy, pour parler tous quatre ensemble à M.
le duc d'Orléans sur l'abbé Dubois, pour essayer en dernier remède
l'impression que ce groupe ainsi réuni pourrait faire. Lui et moi étions
lors à portée de tout avec M. le Duc, lui anciennement par les liaisons
intimes, et de tout temps de M\textsuperscript{me} de Bouzols, sa sueur,
avec M\textsuperscript{me} la Duchesse mère, et avec les Lassai, moi par
les raisons qu'on a vues.

M. le Duc ne pouvait souffrir le grand vol que prenait Dubois, et d'être
obligé lui-même de compter sur toutes choses avec lui\,; et le maréchal
de Villeroy le haïssait à mort, et ne s'en cachait à personne. On a vu
que de tout temps j'étais peu à portée de lui, et nouvellement moins que
jamais, par le travers que son orgueil lui avait fait prendre, au lieu
de me savoir gré de n'avoir jamais voulu le déplacer ni être gouverneur
du roi. Je le dis alors à Torcy, pour éviter de fausses mesures. Cela ne
l'arrêta point, il trouvait le maréchal si frivole qu'il était persuadé
que cette aventure de gouverneur du roi ne ferait aucun obstacle quand
il s'agirait de servir sa haine contre Dubois, étayé du poids de M. le
Duc sur M. le duc d'Orléans, de ma privance avec ce prince et de la
confiance qu'il avait en moi, et de lui, Torcy, fondé sur les lettres
étrangères. Je ne pouvais me rendre à cette pensée\,; je lui représentai
fortement que je gâterais tout, et que le récent dépit de cette place de
gouverneur, qu'il rageait de devoir à mes refus, l'emporterait chez lui
sur toute autre considération. Je voulais donc qu'ils parlassent tous
trois, et n'en être pas avec eux\,; mais Torcy s'opiniâtra à contester
que tout échouerait sans moi, parce que M. le duc d'Orléans regarderait
cet effort comme venant de mains ennemies, et Torcy entraîné par elles,
bien de tout temps avec M. le Duc et avec le maréchal de Villeroy, ce
qui n'arriverait pas s'il me voyait avec eux, parce qu'il ne présumerait
jamais que j'eusse agi de concert avec eux à mauvaise intention ni par
entraînement, et qu'il ne pourrait méconnaître ce que je lui avais dit
souvent tête à tête, et récemment cette dernière fois si forte que j'ai
rapportée\,; qu'il ne pourrait dire méconnaître ces mêmes choses dans ce
que nous lui dirions ensemble, et qu'il verrait, au contraire, l'homme
du monde en moi, duquel il se pouvait le moins méfier, s'unir à eux pour
lui tenir le même langage, qui appuierait si fortement ce que le secret
de la poste avait fourni, à lui Torcy, de raisons qui lui seraient alors
étalées avec plus de force et moins de ménagement que Torcy n'avait osé
employer avec lui tête à tête.

Après un long débat, je me rendis, malgré moi, à l'autorité de Torcy,
l'homme du monde le plus sage, le plus prudent, le plus modéré, le plus
éloigné des partis forts tant qu'il en pouvait prendre d'autres, et par
lui-même naturellement fort retenu et timide\,; bref, je ne me rendis
point, mais je cédai. Il voulut commencer par le maréchal de Villeroy
pour entraîner plus facilement M. le Duc, dont la férocité n'empêchait
pas toujours la timidité, surtout dans un intérêt d'État général et non
un intérêt particulier fort grand. Nous convînmes donc que nous irions,
Torcy et moi, parler au maréchal de Villeroy au sortir du premier
conseil de régence, parce qu'il logeait aux Tuileries, et que cette
visite ensemble serait moins remarquée en y allant ainsi de plain-pied,
et nous trouvant tous deux naturellement ensemble. Nous nous amusâmes
donc tous deux exprès après le conseil de régence pour laisser écouler
le monde, et donner le temps au maréchal de rentrer dans son
appartement, avec convention que Torcy porterait la parole.

Le hasard fit que nous trouvâmes le maréchal de Villeroy seul dans sa
chambre. Dès qu'il nous vit il se douta de quelque chose
d'extraordinaire, et nous demanda ce qui nous amenait ainsi tous deux.
Nous avancions cependant vers lui\,; il répéta sa demande\,; le valet de
chambre qui nous avait ouvert la porte sortit, et avant de nous asseoir,
Torcy, comme pour lui répondre commença à lui faire entendre le sujet de
notre visite. Au premier mot que le maréchal en sentit\,: «\,Messieurs,
dit-il, je suis votre serviteur, mais point de cabale, vous ferez sans
moi tout ce que bon vous semblera. Mais d'aller ainsi en cohorte, c'est
ce que vous ne me persuaderez point, et je ne sais d'où cette idée vous
est entrée dans la tête. Je vois sur l'abbé Dubois tout ce qu'il y a à
voir, j'en parle peut-être autant, et plus fortement que vous au régent,
mais tète à tète, car autrement ce sont cabales que je n'entends point,
et où vous ne me ferez jamais entrer.\,» Delà, il se met en colère,
balbutie, interrompt, ne veut rien écouter, et nous éconduit avec
hauteur. Hors de sa chambre, nous nous regardâmes Torcy et moi,
confondus de la sottise et de l'impertinence de l'homme, et Torcy
découragé ne jugea pas à propos de voir M. le Duc, ni d'aller plus
loin\,; il convint que j'avais mieux jugé du maréchal que lui. «\,Mais
après tout, me dit-il, il n'y a rien de gâté, c'est un coup d'épée dans
l'eau.\,» Pour moi, je n'avais été qu'acolyte sans qu'il me fût sorti un
seul mot de la bouche.

Trois jours après, allant travailler avec M. le duc d'Orléans, je le
trouvai d'abordée, instruit par le maréchal de Villeroy qui, en vil
courtisan qu'il était, avec toute son arrogance et sa morgue, était allé
se faire un mérite de son refus et sacrifier son ancien ami Torcy, qui
toutefois le connaissait bien, et ne l'estimait guère, pour me nuire, et
me perdre s'il avait pu. Quelque surpris que je fusse d'une si basse et
si noire trahison, je dis à M. le duc d'Orléans qu'après tout ce que je
lui avais si souvent fait toucher au doigt de l'abbé Dubois sans aucun
fruit qu'une conviction inutile, et pénétré du tort extrême que cet
homme faisait à Son Altesse Royale et aux affaires pour son unique
intérêt, il était vrai que j e m'en étais ouvert à Torcy, qui, par ce
qu'il voyait du secret de la poste, en était encore plus touché et plus
convaincu que moi\,; que la raison d'État si manifeste, et notre
attachement particulier pour sa personne nous avait fait chercher
quelque moyen de lui faire enfin une impression utile dont il nous
devait savoir gré, et sentir la différence de gens qui comme Torcy et
moi lui disions ce que nous voyions sur l'abbé Dubois, sans jamais crier
contre l'autorité dont il abusait, et qui uniquement, poussés par
l'intérêt pressant de l'État et le sien, voulions lui faire une
impression plus forte, d'avec un chien enragé comme le maréchal de
Villeroy, qui criait à tout le monde contre le maître et le valet, ravi
du mécontentement public qu'il ne cherchait qu'à augmenter, et qui, au
lieu de chercher comme nous à y apporter un remède respectueux, secret,
utile, venait à lui faire le bon valet, et un infâme et misérable
rapport pour l'éloigner de ses vrais serviteurs, et en profiter s'il
pouvait à sa ruine.

Cette réponse ferme et sans balancer fit une si grande impression sur M.
le duc d'Orléans qu'il se rasséréna tout d'un coup, et me parla du
maréchal de Villeroy avec le dernier mépris, qui fut tout ce qu'il
remporta d'une délation si misérable. M. le duc d'Orléans n'en conserva
aucune mauvaise impression contre moi ni contre Torcy, à qui il parla la
première fois qu'il le vit en mêmes termes du maréchal de Villeroy. Je
ne fis jamais depuis aucun semblant au maréchal de sa perfidie ni Torcy
non plus, et il ne nous a jamais aussi reparlé de notre proposition. Au
sortir d'avec le régent, j'allai trouver Torcy, je lui rendis ce qui se
venait de passer entre ce prince et moi, et quoi que je lui pusse dire
pour le rassurer, il en demeura fort en peine, et s'exclama fort, tout
sage et tout mesuré qu'il fût, sur la trahison du maréchal de Villeroy.
À son tour, dès qu'il eût vu M. le duc d'Orléans, il me vint dire
combien cela s'était passé à souhait, et à cette fois, il demeura
parfaitement rassuré. Il faut convenir que voilà une étrange et bien
vilaine aventure, et qui ne se pouvait pas imaginer\,; mais ce qu'elle
eut de triste, c'est que Dubois contre qui elle devait porter en plein,
même manquée comme elle le fut, n'en diminua pas d'une ligne, et fut
sans doute instruit du fait par le régent qui lui disait tout aussi
verrons-nous bientôt qu'il la garda bonne à Torcy, que jusque-là il
avait fait profession d'estimer et de considérer, apparemment pour se
faire honneur à lui-même\,: quant à moi, on a pu voir que j'étais avec
lui de manière que cette façon de plus n'y pouvait guère ajouter.

\hypertarget{chapitre-viii.}{%
\chapter{CHAPITRE VIII.}\label{chapitre-viii.}}

1721

~

{\textsc{Le duc de Sully déclare son mariage secret avec
M\textsuperscript{me} de Vaux.}} {\textsc{- Leur caractère.}} {\textsc{-
Mort de Chamillart\,; raccourci de sa fortune et de son caractère.}}
{\textsc{- Mort de Desmarets\,; abrégé de son caractère.}} {\textsc{-
Mort d'Argenson\,; abrégé de son caractère.}} {\textsc{- Mort de
Maupertuis\,; abrégé de son caractère.}} {\textsc{- Mort de Mézières\,;
son caractère.}} {\textsc{- Mort de Serignan\,; de l'abbé de Mornay\,;
son caractère et sa fortune.}} {\textsc{- Mort de l'abbé de Lyonne\,; de
Bullion.}} {\textsc{- Le grand écuyer se sépare pour toujours de sa
femme, qu'il renvoie au duc de Noailles, son père.}} {\textsc{-
Breteuil, maître des requêtes, prévôt et maître des cérémonies de
l'ordre.}} {\textsc{- La Houssaye, contrôleur général, en a le râpé.}}
{\textsc{- Breteuil, frère du précédent, tué en duel par Gravelle.}}
{\textsc{- Traité d'Angleterre, à son mot, avec l'Espagne.}} {\textsc{-
M. le duc d'Orléans me confie le traité fait du mariage du roi avec
l'infante d'Espagne, et de sa fille avec le prince des Asturies.}}
{\textsc{- Conversation curieuse entre lui et moi là-dessus.}}
{\textsc{- J'obtiens l'ambassade d'Espagne pour faire mon second fils
grand d'Espagne.}} {\textsc{- J'obtiens pour ma dernière belle-soeur
l'abbaye de Saint-Amant de Rouen.}} {\textsc{- Audience de congé,
caractère et traitement de l'ambassadeur turc.}} {\textsc{- Prince de
Lixin fait grand maître de Lorraine en épousant une fille de M. et de
M\textsuperscript{me} de Craon.}} {\textsc{- Son caractère et sa fin.}}
{\textsc{- Mariage du marquis de Villars avec une fille du duc de
Noailles.}} {\textsc{- Caractère de cette dame.}} {\textsc{- Mariage du
duc de Boufflers avec une fille du duc de Villeroy.}}

~

Le chevalier de Sully, devenu duc et pair par la mort, sans enfants, de
son frère aîné, dont la veuve venait de mourir, était depuis bien des
années amoureux de la fille de la fameuse Guyon, dont il a été parlé ici
en son temps, qu'elle avait mariée à de Vaux, fils aîné de l'infortuné
surintendant Fouquet, dont elle était veuve sans enfants depuis
plusieurs années. Il y avait longtemps que la duchesse du Lude, veuve,
riche, sans enfants, qui avait été dame d'honneur de
M\textsuperscript{me} la duchesse de Bourgogne pressait et faisait
presser le duc de Sully, fils de son frère, de se marier. Son
attachement pour M\textsuperscript{me} de Vaux la désolait, elle en
craignait la vile alliance qui par l'âge, plus encore par l'excessif
embonpoint, ne promettait pas d'enfants, qu'elle souhaitait
passionnément de voir à son neveu. Elle lui promettait de lui donner
tout son bien par un mariage sortable, et le menaçait de l'en priver,
s'il poussait à bout un attachement si disproportionné et apparemment
stérile ; mais l'affaire en était faite dans le plus grand secret, pour
ne pas révolter la duchesse du Lude, et couler ainsi le temps en
écartant tous les mariages jusqu'à sa mort, que l'âge et une goutte
continuelle laissaient voir peu éloignée. Ce manège dura si longtemps,
qu'il les ennuya tous trois. Sully, plus attaché que jamais à celle
qu'il avait épousée, ne pouvait plus user sa vie dans la contrainte de
ce secret. L'épouse aimée l'y poussait dans l'extrême désir du rang et
de l'état qui serait la suite nécessaire et immédiate de la déclaration
du mariage. Enfin la duchesse du Lude, excédée de la fermeté de son
neveu, à esquiver et à rejeter tous les mariages, aima mieux savoir
enfin où elle en était là-dessus. Il fallut employer bien des amis, des
préparations, des motifs de conscience pour disposer la duchesse du Lude
à souffrir un aveu si amer. Toutefois on y parvint, elle prit la chose
en pénitence, reçut froidement son neveu, lui permit de déclarer son
mariage et ne lui fit point de mal.

On eut plus de peine à la résoudre de voir la nouvelle duchesse de
Sully, qui se hâta de prendre son tabouret, et qui prit sans peine tout
le maintien d'une grande dame avec assez d'esprit pour ne blesser
personne par un si grand changement. Elle en avait en effet beaucoup,
beaucoup de monde, de la lecture et de l'ornement, une beauté romaine,
de beaux traits, un beau teint, et la conversation très aimable, avec
beaucoup d'amis de tous les genres, et assez choisis en hommes et en
femmes. Sa réputation fut toujours sans reproche\,; elle n'eut jamais
d'autre attachement que celui qui fut couronné par la persévérance, et
depuis même que le mariage secret leur avait tout permis, les
bienséances et les dehors furent si exactement observés qu'il ne se put
rien apercevoir entre eux. Le commerce de l'un et de l'autre avec leurs
amis était honnête, et sûr\,; le duc de Sully en avait beaucoup et avait
toujours été fort au goût du monde, mais jamais de celui du roi. Quoique
gros, c'était le meilleur danseur de son temps, son visage et sa figure
étaient agréables, avec beaucoup de grâce et de douceur. Toujours
pauvre, toujours rangé, et se soutenant de peu avec honneur, peu
d'esprit mais sage, et avait servi toute sa vie avec beaucoup de valeur,
et peu de fortune. Je n'ai jamais su pourquoi le roi l'avait pris en une
sorte d'aversion, si ce n'est qu'il ne fut jamais fort assidu à la cour,
et qu'il était fort des amis de M. le prince de Conti. À la fin, les
respects, les mesures, la patience de la duchesse de Sully, gagnèrent la
duchesse du Lude, qui s'accoutuma à elle, et la vit chez elle avec une
sorte d'amitié.

Plusieurs personnages et quelques autres moururent cette année.
Chamillart commença, à soixante et dix ans. On a vu ailleurs sa fortune
et sa chute, et en plusieurs endroits son caractère. Il succéda à
Pontchartrain aux finances, lorsque ce dernier devint chancelier par la
mort de Boucherat en septembre 1699\,; ministre d'État, septembre 1700,
par la mort de Pomponne\,; secrétaire d'État au département de la
guerre, sans quitter les finances, en janvier 1701 par la mort de
Barbezieux, cinq ans après grand trésorier de l'ordre\,; remit les
finances en juin 1709 à Desmarest\,; fut congédié un an après, et sa
charge de secrétaire d'État donnée à Voysin. On a vu aussi avec quel
courage et quelle tranquillité il soutint sa disgrâce, et il la soutint
également jusqu'à sa mort. C'était un homme aimable, obligeant, modeste,
compatissant, doux dans le commerce et sur, jamais enflé, encore moins
gâté par la faveur et l'autorité, d'abord facile et honnête à tous, mais
à la vérité impar oneri, peu d'esprit et de lumière, peu de
discernement, aisé à prévenir, à s'entêter, à croire tout voir et
savoir, du plus parfait désintéressement, tenant au roi par attachement
de coeur en tous les temps, et point du tout à ses places. Depuis son
retour à Paris, il y vécut toujours en la meilleure compagnie de la cour
et de la ville\,; donnait tous les jours à dîner et à souper sans faste,
mais bonne chère\,; ne sortait presque point de chez lui, sinon
quelquefois pour venir chez moi, et chez un nombre fort étroit d'amis
particuliers\,; passait deux mois à Courcelles où toute la province
abondait, et sans rien montrer, pensait solidement à son salut. Toutes
les fois que je venais à Paris, je mangeais une fois chez lui et le
voyais tous les jours, que j'y demeurais, qui étaient toujours rares et
courts. J'étais à la Ferté lorsqu'il mourut à Paris, et je le regrettai
beaucoup.

Le 4 mai suivant, mourut à Paris Desmarets, à soixante-treize ans,
dix-huit jours après Chamillart. On a vu ailleurs ses revers et sa
fortune. Bon Dieu, dans quel étonnement serait-il de celle de son
fils\footnote{Jean-Baptiste-François Desmarets, marquis de Maillebois,
  né en 1682, lieutenant général en 1731, maréchal de France en 1741,
  gouverneur d'Alsace en 1748, mort en 1762. Le marquis de Pezay a
  publié l'Histoire des campagnes du maréchal de Maillebois en Italie,
  pendant les années 1745-46 (Paris, 1775, 3 vol.)}\,! Je le vis
toujours jusqu'à sa mort depuis que nous nous étions raccommodés, comme
on l'a vu en son lieu. C'était un homme qui avait plus de sens que
d'esprit, et qui montrait plus de sens qu'il n'en avait en effet\,;
quelque chose de lourd et de lent, parlant bien et avec agrément, dur,
emporté, dominé par une humeur intraitable, et l'antipode de Chamillart
en ce que ce dernier avait une qualité bien rare d'être excellent ami,
et point du tout ennemi. Desmarets n'était ami que par intérêt, et
souvent beaucoup moins que son intérêt le voulait. On a vu ici son
caractère en plusieurs endroits.

Deux jours après, le 6 mai, mourut d'Argenson dans sa singulière
retraite, au dehors de la maison des Filles de la Croix, au faubourg
Saint-Antoine. C'était un homme de beaucoup d'esprit, de connaissance du
monde, de nulle d'affaires d'État, de finances, de magistrature, qui
pensait noblement et honnêtement, et qui aurait été bon en grand s'il y
avait été élevé. Mais son esprit s'était rétréci et tellement accoutumé
au petit qu'il ne put jamais s'étendre et s'élever. Il avait passé sa
jeunesse dans le chétif exercice de la charge de lieutenant général
d'Angoulême qu'avait eue son père. Il était pauvre et de meilleure
condition que la plupart des gens de robe, aussi s'en piquait-il, aussi
respectait et aimait à obliger les gens de qualité et la noblesse dont
il se prétendait avant que ses pères eussent pris la robe. Devenu maître
des requêtes, il épousa une soeur de Caumartin qui s'en fit honneur, et
qui, par le chancelier de Pontchartrain, alors contrôleur général, le
fit lieutenant de police. C'est où il excella \footnote{Fontenelle a
  retracé avec l'ingénieuse et élégante précision de son style les
  services que rendit d'Argenson dans la charge de lieutenant de
  police\,: «\,Les citoyens d'une ville bien policée jouissent de
  l'ordre qui y est établi, sans songer combien il en coûte de peine à
  ceux qui l'établissent ou le conservent, à peu près comme tous les
  hommes jouissent de la régularité des mouvements célestes, sans en
  avoir aucune connaissance\,; et même plus l'ordre d'une police
  ressemble par son uniformité à celui des corps célestes, plus il est
  insensible\,; et par conséquent il est toujours d'autant plus ignoré
  qu'il est plus parfait. Mais qui voudrait le connaître, l'approfondir,
  en serait effrayé\,: entretenir perpétuellement dans une ville telle
  que Paris une consommation immense, dont une infinité d'accidents
  peuvent toujours tarir quelques sources\,; réprimer la tyrannie des
  marchands à l'égard du public, et en même temps animer leur
  commerce\,; empêcher les usurpations naturelles des uns sur les
  autres, souvent difficiles à démêler\,; reconnaître dans une foule
  infinie ceux qui peuvent aisément y cacher une industrie
  pernicieuse\,; en purger la société, ou ne les tolérer qu'autant
  qu'ils peuvent être utiles par des emplois dont d'autres qu'eux ne se
  chargeraient pas ou ne s'acquitteraient pas si bien\,; tenir les abus
  nécessaires dans les bornes précises de la nécessité, qu'ils sont
  toujours prêts à franchir\,; les renfermer dans l'obscurité à laquelle
  ils doivent être condamnés, et ne les en tirer pas même par des
  châtiments trop éclatants\,; ignorer ce qu'il vaut mieux ignorer que
  punir, et ne punir que rarement et utilement\,; pénétrer par des
  souterrains dans l'intérieur des familles et leur garder les secrets
  qu'elles n'ont pas confiés, tant qu'il n'est pas nécessaire d'en faire
  usage\,; être présent partout sans être vu\,; enfin mouvoir ou arrêter
  à son gré une multitude immense et tumultueuse, et être l'âme toujours
  agissante et presque inconnue de ce grand corps\,; voilà quelles sont
  en général les fonctions du magistrat de police. Il ne semble pas
  qu'un homme seul y puisse suffire ni par la quantité des choses dont
  il faut être instruit, ni par celle des vues qu'il faut suivre, ni par
  l'application qu'il faut apporter, ni par la variété des conduites
  qu'il faut tenir et des caractères qu'il faut prendre.\,»}, et où il
sauva bien des gens de qualité et des enfants de famille. Il était
obligeant, poli, respectueux, sous une écorce quelquefois brusque et
dure, et une figure de Rhadamante, mais dont les yeux pétillaient
d'esprit et réparaient tout le reste. Il ne put soutenir sa chute, et ne
sortit plus de sa chambre ou du parloir. On a suffisamment parlé de lui
ailleurs. Il commença sur les fins à signer de Voyer au lieu de Le
Voyer, qui est son nom. Ses enfants, qui ont depuis fait une si grande
fortune et qui veulent pousser leurs enfants dans une d'un autre genre,
imitent soigneusement la dernière façon de signer de leur père et de
faire appeler leurs enfants.

Maupertuis, des bâtards de Melun, mourut à quatre-vingt-sept ans,
jusqu'alors dans une santé parfaite. Il était lieutenant général,
grand'croix de Saint-Louis, gouverneur de Toul, et avait été longtemps
capitaine de la première compagnie des mousquetaires, où il était
parvenu rapidement de maréchal des logis. C'était un homme dont j'ai
parlé tout au commencement de ces Mémoires, plein d'honneur, de valeur
et de vertu\,; de petitesses aussi, d'exactitude et de pédanterie, fort
court d'esprit, par conséquent fort au goût du feu roi. Il ne laissa
point d'enfants.

Mezières, lieutenant général et gouverneur d'Amiens et de Corbie.
C'était un petit bossu devant et derrière à faire peur, avec un visage
très livide, qui ressemblait fort à une grenouille. De la valeur, assez
d'esprit, encore plus d'effronterie, de hardiesse, de confiance,
d'impudence, l'avaient poussé. Il s'ajustait et se regardait avec
complaisance dans les miroirs, était galant, attaquait les femmes, se
croyait digne et prétendait à toutes les fortunes de la guerre, de la
cour, même de la galanterie. Il était frère de la mère du marquis,
depuis duc de Lévi, et n'était pas éloigné de prétendre que cette
alliance honorait ce neveu. Boulainvilliers m'a pourtant dit que ces
Béthisy, c'était le nom de Mezières, étaient anoblis, mais pas trop
anciennement\,; lui et sa femme, maîtresse et dangereuse intrigante,
dont j'ai parlé lors de son mariage, s'étaient bien nantis au
Mississipi. Il laissa des fils et des filles, lesquelles n'ont pas été
moins intrigantes ni moins dangereuses que leur mère. Canillac,
lieutenant général et capitaine de la seconde compagnie des
mousquetaires, eut le gouvernement d'Amiens.

Sérignan, gouverneur de Ham, qui avait passé la plupart de sa vie
aide-major des gardes du corps, et qui fort au goût du roi avait eu le
secret de bien des choses, mourut à quatre-vingt-quatorze ans, depuis
longtemps retiré, ayant jusqu'au bout conservé sa tête et santé.

L'abbé de Mornay, passant à Madrid, revenant de Lisbonne, ou il était
ambassadeur depuis longtemps. Il était fils de M. et de
M\textsuperscript{me} de Montchevreuil, l'un et l'autre si favoris de
M\textsuperscript{me} de Maintenon et du roi, desquels j'ai parlé en
leur temps. Toutefois cette faveur si grande ne put faire leur fils
évêque\,; c'était pourtant un homme d'esprit et de mérite, sage et
capable, et qui n'avait point fait parler de ses moeurs\,; mais sa
figure le perdit, et le commerce ordinaire et tout simple des dames de
la cour comme des hommes. C'était un grand homme blond, fort bien fait,
de visage agréable, qui capriça le roi et que rien ne put vaincre. Cette
opiniâtreté d'une part, et la considération du père et de la mère de
l'autre, lui firent donner l'ambassade de Portugal, où il réussit très
bien et s'y fit fort estimer. M. le duc d'Orléans lui avait donné
l'archevêché de Besançon. Peu avant de partir de Lisbonne, il perdit
presque les yeux d'une fluxion, et en chemin il les perdit tout à fait.
Arrivant à Madrid il se trouva mal, et en peu de jours y mourut, dont ce
fut grand dommage. Son archevêché fut donné au frère du prince de
Monaco, qui avait été prêtre de l'Oratoire, puis jésuite, qui en était
sorti béat fort glorieux et très ignorant, qui n'était propre ni au
monde ni à l'Église.

L'abbé de Lyonne peu après, fils du célèbre ministre et secrétaire
d'État, auquel il ne ressembla en rien. Il avait les abbayes de
Marmoutiers, de Chalis et de Cercamp\,; avec le prieuré de Saint-Martin
des Champs dans Paris, où il avait passé sa vie, sans voir presque
personne, et où il mourut aussi obscurément qu'il avait vécu. Il avait
été débauché et accusé de vendre ses collations\footnote{Droit de
  conférer un bénéfice.}. J'en ai parlé ailleurs. Il buvait tous les
matins plus de vingt pintes d'eau de la Seine depuis fort longtemps.

Bullion, duquel j'ai parlé ailleurs. Il avait fait plusieurs folies à
Versailles, où on sut qu'il en était attaqué depuis longtemps. Il était
enfermé depuis quelques années dans une de ses maisons en Beauce, où
personne ne le voyait. Son fils aîné obtint, par la duchesse de
Ventadour, leur proche parente, son gouvernement du Perche et du Maine.
Un de ses cadets était dès lors prévôt de Paris sur sa démission.

Le grand écuyer, qui, dédaignant de s'appeler M. le Grand, comme son
père l'avait toujours été, se faisait nommer le prince Charles et sa
femme M\textsuperscript{me} d'Armagnac, se brouilla avec elle sur
quelque jalousie qu'il en prit à Saint-Germain, chez le duc de Noailles
son père, à qui, un beau matin, il la renvoya sans autre façon, sans en
avoir voulu ouïr parler depuis ni d'aucun Noailles. On prétendit que le
duc d'Elboeuf, à qui la soif de l'argent avait fait faire ce mariage, en
voyant la source tarie par le déplacement du duc de Noailles, contribua
fort à cet éclat. Il n'y avait guère qu'un an qu'elle était chez son
mari, parce qu'elle était fort jeune\,; personne ne la crut coupable, et
sa conduite y a fort bien répondu depuis. Elle voulut se retirer auprès
de sa tante, fille de Sainte-Marie, au faubourg Saint-Germain, où elle
est demeurée, sans en vouloir sortir, plusieurs années. Toute la maison
de Lorraine, jusqu'à M\textsuperscript{lle} d'Armagnac, soeur du prince
Charles et ses autres proches, le blâmèrent publiquement et virent
toujours sa femme, excepté le duc d'Elboeuf, ce qui les brouilla avec
lui. En sorte qu'il n'a pas vu depuis M\textsuperscript{lle} d'Armagnac,
avec qui il avait toujours été fort uni. Il faut pourtant dire que, sans
esprit du tout, le prince Charles est un très honnête homme, et dont
partout ailleurs les procédés ont toujours été fort bons et surtout fort
nobles dans sa charge.

Le Camus, premier président de la cour des aides, qui avait acheté, en
1709, de Pontchartrain fils, la charge de prévôt et maître des
cérémonies de l'ordre, eut permission en ce temps-ci de la vendre à
Breteuil, maître des requêtes, et de conserver le cordon bleu. La
Houssaye, contrôleur général des finances et surintendant des maisons,
affaires et finances de M. le duc d'Orléans, en eut le râpé. Breteuil
est celui qui fut depuis secrétaire d'État de la guerre à deux reprises.

Il avait un frère dans le régiment des gardes, avec qui Gravelle, autre
officier aux gardes, querelleur et fort en gueule, eut des paroles.
Breteuil en serait demeuré là sans ses camarades et sans sa famille qui
le forcèrent à se battre. Ils n'y firent pas grande façon, le combat se
fit en plein midi, dans la rue de Richelieu\,; en un tournemain Breteuil
fut tué, et il n'en fut pas autre chose. M. le duc d'Orléans, pour le
dire faiblement, ne haïssait pas les duels. Gravelle était capitaine aux
gardes\,; Breteuil, qui l'était aussi, venait de vendre sa compagnie.

Enfin l'Espagne, non seulement abandonnée par la France, mais pressée à
l'excès de signer son accommodement avec l'Angleterre, y consentit, ne
pouvant mieux, par lequel les Anglais obtinrent tous les avantages
qu'ils s'étoient proposés pour leur commerce et la ruine de celui de
toutes les autres nations, singulièrement de celui de France et au grand
détriment de l'Espagne. Les Anglais, en outre, eurent
l'asiento\footnote{Il a déjà été question de ce traité.} à leur mot, un
vaisseau de permission\,; conservèrent Port-Mahon et toute l'île avec
Gibraltar. Véritablement ils restituèrent quelques vaisseaux
nouvellement pris à l'Espagne, et la gratifièrent d'autres bagatelles.
Moyennant ce traité, l'empereur, à l'ardente prière du roi d'Angleterre,
redoubla ses instances à Rome, qui, aidées de l'étrange engagement qu'on
vient de voir qu'avait pris le pape pour son exaltation, mirent enfin
les choses au point où Dubois les désirait pour recevoir incessamment la
pourpre.

Ayant mis ainsi le couteau à la gorge de l'Espagne pour l'entière et
l'énorme satisfaction des Anglais, ou plutôt pour celle de Dubois,
j'avoue que je ne comprends pas comment le traité du double mariage
entre la France et l'Espagne put suivre si brusquement. Le secret en fut
si entier qu'aucune puissance ni aucun particulier ne s'en douta. Depuis
longtemps l'abbé Dubois avait fermé la bouche à mon égard à son maître
sur les affaires étrangères, et plus étroitement encore depuis ce que
j'ai raconté ici il n'y a pas longtemps. Cela n'empêchait pourtant pas
qu'il n'en échappât toujours à M. le duc d'Orléans quelque bribe avec
moi, mais avec peu de détail et de suite, et de mon côté je demeurais
fort réservé. Étant allé les premiers jours de juin pour travailler avec
M. le duc d'Orléans, je le trouvai qui se promenait seul dans son grand
appartement. Dès qu'il me vit\,: «\,Ho çà\,! me dit-il me prenant par la
main, je ne puis vous faire un secret de la chose du monde que je
désirais et qui m'importait le plus et qui vous fera la même joie\,;
mais je vous demande le plus grand secret.\,» Puis, se mettant à rire\,:
«\,Si M. de Cambrai savait que je vous l'ai dit, il ne me le
pardonnerait pas.\,» Tout de suite il m'apprit sa réconciliation faite
avec le roi et la reine d'Espagne\,; le mariage du roi et de l'infante,
dès qu'elle serait nubile, arrêté, et celui du prince des Asturies
conclu avec M\textsuperscript{lle} de Chartres.

Si ma joie fut grande, mon étonnement la surpassa. M. le duc d'Orléans
m'embrassa, et après les premières réflexions des avantages personnels
pour lui d'une si grande affaire, et sur l'extrême convenance du mariage
du roi, je lui demandai comment il avait pu faire pour la faire réussir,
surtout le mariage de sa fille. Il me dit que tout cela s'était fait en
un tournemain, que l'abbé Dubois avait le diable au corps pour les
choses qu'il voulait absolument\,; que le roi d'Espagne avait été
transporté que le roi son neveu demandât l'infante\,; et que le mariage
du prince des Asturies avait été la condition sine qua non du mariage de
l'infante qui avait fait sauter le bâton au roi d'Espagne. Après nous
être bien étendus et bien éjouis\footnote{Il y a dans le manuscrit
  éjouis et non réjouis.} là-dessus, je lui dis qu'il fallait que le
secret du mariage de sa fille fût entièrement gardé jusqu'au moment de
son départ, et celui du mariage du roi jusqu'au moment où les années
permettraient son exécution, pour empêcher la jalousie de toute l'Europe
de cette réunion si grande et si étroite des deux branches de la maison
royale, dont l'union avait toujours été {[}sa{]} terreur, et la désunion
l'objet de toute {[}sa{]} politique, à laquelle les souverains n'étoient
que trop et trop longtemps parvenus, et dans la confiance de laquelle il
les fallait laisser aussi longtemps qu'il serait possible, l'infante
surtout n'ayant que trois ans, car elle est née à Madrid le 30 mars 1718
au matin, ce qui donnait des années devant soi à laisser calmer les
inquiétudes de l'Europe sur le mariage de sa fille avec le prince des
Asturies, qui même par rapport à l'âge, se pouvait un peu différer, le
prince étant de 1707 en août, ce qui ne faisait que quatorze ans, et
M\textsuperscript{lle} de Chartres, car elle avait pris ce nom depuis la
profession de M\textsuperscript{me} de Chelles, n'en ayant pas douze,
étant de décembre 1709. «\,Vous avez bien raison, me répondit M. le duc
d'Orléans, mais il n'y a pas moyen, parce qu'ils veulent en Espagne la
déclaration tout à l'heure, et envoyer ici l'infante, dès que la demande
sera faite et le contrat de mariage signé. --- Quelle folie m'écriai-je,
et à quoi ce tocsin peut-il être bon qu'à mettre toute l'Europe en
cervelle et en mouvement\,? Il leur faut faire entendre cela, et y tenir
ferme, rien n'est si important. --- Tout cela est vrai, répliqua M. le
duc d'Orléans\,; je le pense tout comme vous, mais ils sont têtus en
Espagne, ils l'ont voulu de la sorte, on l'a accordé. C'est une chose
faite, convenue et arrêtée\,; l'affaire est si grande pour moi à tous
égards que vous ne m'auriez pas conseillé de rompre sur cette
fantaisie.\,» J'en convins en haussant les épaules sur une impatience si
à contre-temps.

Après quelques raisonnements là-dessus, je lui demandai ce qu'il
prétendait faire de cette enfant, quand elle serait ici. Il me dit qu'il
la mettrait au Louvre. Je lui répondis qu'à mon sens il fallait en faire
toute autre chose\,; qu'au Louvre, table, suite, etc., seraient d'une
grande dépense, et très inutile\,; qu'en croissant la dépense croîtrait,
et qu'elle verrait nécessairement des compagnies à éviter le plus
longtemps qu'il serait possible. Pis que tout cela, il faudrait que le
roi lui rendit des soins\,; qu'il en verrait des enfances\,; elle, en
croissant, en remarquerait de lui\,; qu'il y aurait entre eux ou trop de
familiarité, ou trop de contrainte, qu'ils se rebuteraient l'un de
l'autre, s'ennuyeraient, se dégoûteraient, le roi surtout, {[}ce{]} qui
serait le souverain malheur\,; qu'il serait de plus impossible que la
petite princesse, croissant au milieu du monde et de la cour, ne fût
gâtée\,; qu'il était bien difficile que tout cela ne causât de grands
maux\,; que pour moi, mon avis serait, puisque le sort en était jeté, et
qu'il fallait qu'elle arrivât bientôt, qu'on la mit au Val-de-Grâce,
dans le bel appartement de la reine mère qu'il connaissait et moi aussi,
pour y être entré allant y voir M\textsuperscript{me} de Chelles\,; que
le dedans et le dehors de ce monastère étaient magnifiques, le monastère
royal, fondé par la reine mère, et bâti par elle à plaisir\,; que le
jardin était beau, très grand, en très bon air\,; qu'il fallait mettre
auprès d'elle la duchesse de Beauvilliers, veuve et sans famille, dont
le mari avait été gouverneur du roi d'Espagne\,; que sa vertu, sa piété,
son esprit, sa connaissance\,; de la cour et du monde, où elle avait
passé sa vie, dans la plus haute considération et réputation, la
rendaient l'unique personne à choisir\,; que je croyais bien qu'elle
s'en défendrait tant qu'elle pourrait, mais qu'elle ne résisterait pas
aux instances du roi d'Espagne, à qui il fallait représenter toutes ces
choses, ne mettre personne en dames ni en officiers principaux, et
laisser la duchesse de Beauvilliers mettre et ôter les femmes de chambre
et celles-ci en petit nombre, être seule maîtresse de l'éducation en
tout genre, même de là cuisine. Ni chevaux, ni carrosses, ni gardes, ni
quoi que ce soit\,; une ou deux fois l'année une visite du roi d'un
quart d'heure, autant d'elle au roi, et alors lui envoyer des carrosses
et des gardes du roi, et lui faire faire quelques tours dans Paris, ou
au Cours, en allant ou revenant, et lorsque peu à peu elle sera en âge
de commencer à voir quelques dames, quelles soient du choix de la
duchesse de Beauvilliers, ainsi que pour le nombre et le temps\,; que de
cette manière elle recevra une éducation à souhait, en lieu digne et
décent, à couvert des mauvaises compagnies, sans dépense, en un lieu de
s'amuser, se promener, et faire des enfances qui ne porteront aucun
coup, et le roi et elle hors de portée de se familiariser ou de
s'ennuyer l'un de l'autre, de se mépriser par leurs enfances, de se
dégoûter\,; et ne la sortir du Val-de-Grâce que la veille de la
célébration de son mariage, où elle trouverait toute sa maison faite, et
toute, quant aux dames et aux femmes, de l'avis de la duchesse de
Beauvilliers.

M. le duc d'Orléans écouta tout fort tranquillement, me dit que j'avais
raison, que ce serait bien le mieux, mais que cette place ne se pouvait
ôter à la duchesse de Ventadour, gouvernante des enfants de France.
«\,Mais elle ne l'est pas des enfants d'Espagne, repris-je vivement. ---
Non, me dit-il, mais elle l'a été du roi, et l'infante élevée ici pour
l'épouser ne saurait être mise en d'autres mains, et
M\textsuperscript{me} de Ventadour n'est pas femme à s'enfermer au
Val-de-Grâce. --- C'est donc à dire, répliquai-je, qu'il faut sacrifier
l'infante, et tout ce qui en peut arriver, que je vous viens de
représenter, avec toute la dépense, à M\textsuperscript{me} de
Ventadour, à sa charge, à ses complexions, qui la gâtera et en fera tout
ce que l'enfant et les femmes qui l'obséderont en voudront être\,;
M\textsuperscript{me} de Ventadour votre ennemie, elle et tous ses
entours et son maréchal de Villeroy qui, de votre aveu à moi et du su de
chacun, vous ont fait et vous font encore tout du pis qu'ils ont pu et
qu'ils peuvent et sûrement qu'ils pourront. Je contestai encore un peu
et fort inutilement, puis je me tus, sentant bien que ce choix venait de
l'abbé Dubois, par rapport aux Rohan et à ce qu'il espérait du cardinal
de Rohan pour accélérer son chapeau, et qui lors était tout porté à
Rome.

Pendant tous ces raisonnements divers, je ne laissais pas de penser à
moi, et à l'occasion si naturelle de faire la fortune de mon second
fils. Je lui dis donc que, puisque les choses en étaient nécessairement
au point qu'il me les apprenait, il devenait donc instant d'envoyer
faire la demande solennelle de l'infante, et en signer le contrat de
mariage, qu'il y fallait un seigneur de marque et titré, et que je le
suppliais de me donner cette ambassade avec sa protection et sa
recommandation auprès du roi d'Espagne pour faire grand d'Espagne le
marquis de Ruffec\,; qu'il avait fait pair La Feuillade, son plus grand
et son plus insolent ennemi, parce qu'il l'avait plu ainsi à son ami
Canillac, au grand scandale de tout le monde, le seul homme contre qui
je l'avais jamais vu outré jusqu'à lui vouloir faire donner des coups de
bâton, dont il pouvait se souvenir que je l'avais empêché avec peine, et
de plus lui avait donné beaucoup d'argent sous le frivole prétexte de
l'ambassade de Rome où il ne fut jamais question de l'envoyer\,; qu'en
même temps il avait aussi fait pair le duc de Brancas\,; que je lui
avouais que ni du côté du monde ni par rapport à lui je n'avais pas
l'humilité de m'estimer de niveau ni du père ni du fils\,; que tout à
l'heure il venait de faire duc et pair M. de Nevers, à côté duquel je ne
croyais pas être\,; que j'omettais les grâces sans nombre qu'il avait
répandues à pleines mains, en particulier la capitainerie de
Saint-Germain et de Versailles, qu'avait eue mon père, au duc de
Noailles et à ses enfants\,; que revêtu de rien que de petits
gouvernements dont j'avais eu la survivance comme tout l'univers en
avait obtenues, je ne voyais pas ce qu'il me pourrait donner\,; que je
ne lui avais pas demandé de faire mon second fils duc, quoiqu'il ne
l'eût pas offensé en cent façons éclatantes comme La Feuillade, quoique
MM. de Brancas et de Nevers n'eussent que point ou peu, et comment,
servi\,; ce qui ne se pouvait reprocher à l'âge de mon fils\,: «\,Mais
je vous demande pour lui une chose sans conséquence pour qui que ce
soit, qui lui donne le rang et les honneurs de duc, qui est une suite
naturelle d'une ambassade pour faire le mariage du roi, et que personne
ne peut qu'approuver que vous me la donniez et en vue de cette
grandesse.\,» M. le duc d'Orléans eut peine à me laisser achever, me
l'accorda tout de suite et tout ce qu'il fallait de sa part pour obtenir
la grandesse pour le marquis de Ruffec, l'assaisonna de beaucoup
d'amitié, et m'en demanda un secret sans réserve et de ne rien montrer
par aucun préparatif qu'il ne m'avertit d'en faire.

J'entendis bien qu'outre le secret de l'affaire même il voulait avoir le
temps de tourner son Dubois et de lui en faire avaler la pilule. Mes
remerciements faits, je lui demandai deux grâces, l'une de ne me point
donner d'appointements d'ambassadeur, mais de quoi en gros en faire la
dépense sans m'y ruiner, l'autre de ne me charger d'aucune affaire, ne
voulant pas le quitter, et d'une affaire à l'autre prendre racine en
Espagne, d'autant que je n'y voulais aller que pour avoir la grandesse
pour mon second fils et revenir tout court après. C'est que je craignis
que Dubois, ne pouvant empêcher l'ambassade, m'y retînt en exil pour se
défaire de moi ici, sous prétexte d'affaires en Espagne, et je vis bien
par l'événement, que la précaution n'avait pas été inutile. M. le duc
d'Orléans m'accorda l'un et l'autre avec force propos obligeants sur ce
qu'il ne désirait pas que mon absence fût longue. Je crus ainsi avoir
fait une grande affaire pour ma maison et me retirai chez moi fort
content. Mais, mon Dieu, qu'est-ce des projets et des succès des
hommes\,!

Peu de jours après il m'accorda l'abbaye de Saint-Amand dans Rouen pour
la dernière soeur de M\textsuperscript{me} de Saint-Simon, religieuse du
même ordre à Conflans, très bonne religieuse, qui eut bien de la peine à
se résoudre à l'accepter, et qui tant qu'elle a eu quelque santé a été
une excellente abbesse, fille d'esprit et de sens, parfaitement bien
faite et d'un visage fort agréable.

Le 12 juillet l'ambassadeur turc eut son audience de congé.
L'après-dînée le prince de Lambesc et le chevalier Sainctot,
introducteur des ambassadeurs, l'allèrent prendre chez lui, dans le
carrosse du roi, dans lequel il monta, ayant le prince de Lambesc à sa
gauche, l'introducteur vis-à-vis de lui, le fils de l'ambassadeur
vis-à-vis du prince de Lambesc, et l'interprète à la portière, du côté
de l'ambassadeur. L'accompagnement fut comme à la première audience,
mais sans troupes qu'un détachement des dragons d'Orléans devant et
derrière le carrosse du roi entouré de la livrée de l'introducteur à
droite, et de celle du prince de Lambesc à gauche. Le carrosse de
l'ambassadeur suivait, puis la connétablie. La marche gagna le quai de
Conti jusqu'au pont Royal, puis le long des galeries du Louvre, passa
par le premier guichet et par la rue Saint-Nicaise aux Tuileries. Les
mêmes pelotons qui avaient garni les rues de son passage pour sa
première audience les garnirent de même pour celle-ci, les régiments des
gardes françaises et suisses tenaient le pont Royal, le quai des
galeries du Louvre, la rue Saint-Nicaise\,; la garde du roi à
l'ordinaire sous les armes, les tambours rappelant, les deux compagnies
des mousquetaires en bataille dans la place du Carrousel.

L'ambassadeur se reposa dans un appartement bas qu'on lui avait préparé
jusqu'à quatre heures et demie qu'il fut conduit à l'audience comme la
première fois. Il y fut reçu de même partout, et la galerie et le trône
du roi disposés comme ils l'avaient été et environnés de même des
princes du sang, etc.\,; et comme la première fois, le roi se leva sans
se découvrir et personne ne se couvrit. L'ambassadeur marcha, salua, se
plaça comme à sa première audience, fit son compliment, le maréchal de
Villeroy la réponse, le roi mot\,; après quoi le maréchal de Villeroy
prit, sur une table couverte de brocart d'or, la lettre du roi au Grand
Seigneur, enveloppée dans une étoffe d'or, et la présenta au roi, qui la
donna à l'archevêque de Cambrai, et celui-ci à l'ambassadeur, qui la
porta sur sa tête, la baisa et la donna à son fils à porter qui était
derrière lui, puis l'ambassadeur se retira à reculons, comme la première
fois, et retourna dans l'appartement où il était descendu, où le prince
de Lambesc prit congé de lui\,; un peu après l'ambassadeur monta dans le
carrosse du roi, l'introducteur à sa gauche, le fils de l'ambassadeur et
l'interprète sur le devant\,; il retourna chez lui par le même chemin
qu'il était venu, avec le même cortège, et trouva dans tous les lieux de
son passage les mêmes troupes et les mêmes pelotons qu'il y avait
trouvés en venant. Il fut encore un mois à Paris.

Pendant ces quatre mois de séjour il vit avec goût et discernement tout
ce que Paris lui put offrir de curieux et les maisons royales
d'alentour, où il fut magnifiquement traité et reçu. Il parut entendre
les machines, les manufactures, surtout les médailles et l'imprimerie\,;
il vit aussi avec grand plaisir les plans en relief des places du roi et
sa bibliothèque, où il parut savoir et avoir beaucoup de connaissance de
l'histoire et des bons livres. Il était ami particulier du grand vizir,
et se proposait à son retour d'établir à Constantinople une imprimerie
et une bibliothèque, malgré l'aversion des Turcs, et il y réussit. Les
dames de la cour et de la ville se familiarisèrent à l'aller voir\,; il
les régala souvent de café et de confitures, et, moyennant l'interprète,
il fournissait très galamment à la conversation. Il en visita aussi
quelques-unes. M. de Lauzun, qui aimait les choses singulières et tous
les étrangers, lui donna chez lui, à Paris, une grande collation avec un
biribi\footnote{Jeu de hasard où l'avantage du banquier est de six sur
  soixante-dix.}. Ce fut là où je le vis à mon aise. Il me parut au plus
de moyenne taille, gros et d'environ soixante ans, un beau visage et
majestueux, la démarche fière, le regard haut et perçant. Il entra où
était la compagnie comme le maître du monde\,; de la politesse, mais
plus encore de grandeur\,; il se mit sans façon à la première place, au
milieu des dames, qu'il sut fort bien entretenir, sans le moindre
embarras et l'air fort à son aise. Il ne savait ce que c'était que le
biribi et n'en avait jamais vu. Ces tableaux l'amusèrent fort\,; il se
divertit à voir jouer\,; on lui fit entendre ce jeu comme on put\,; il
voulut jouer après, il gagna deux ou trois pleins et en parut ravi. On
lui avait préparé un cabinet avec un tapis pour l'heure de sa prière.
Nous la lui vîmes faire très dévotement avec leurs prostrations et
toutes leurs façons. Elle fut courte\,; il but et mangea très bien, et
toute sa suite fut magnifiquement régalée. Tout cela dura bien deux
heures. Il s'en alla fort content de la réception et de la compagnie, et
la laissa très satisfaite de lui.

Il fut très exact à ne boire ni vin, ni liqueur\,; mais retiré dans sa
chambre, on dit qu'il ne se faisait faute de bien avaler du vin en
secret\,; son fils et sa suite en usaient avec moins de réserve. Sa
suite ne commit pas le plus léger désordre, et il se comporta en tout
très décemment et en homme d'esprit\,; quelques ministres le régalèrent.
La procession de la petite Fête-Dieu de Saint-Sulpice passa devant sa
porte. Il ne fit aucune difficulté de tendre tout le devant de sa
maison, et d'orner ses fenêtres de tapis d'où il vit passer la
procession. Pendant toute cette matinée, il tint tout son monde enfermé
chez lui et sa grande porte à la clef. Il eut, peu de jours après son
audience de congé du roi, celle de M. le duc d'Orléans, qui se passa
comme la première. Il ne vit point Madame, ni M\textsuperscript{me} la
duchesse d'Orléans, ni pas un prince ni princesse du sang. Comme il
n'avait vu le roi qu'à ses audiences, il eut grande envie de le voir
plus à son plaisir. On lui proposa d'aller voir les pierreries de la
couronne chez le maréchal de Villeroy. Il y alla, et sur la fin le roi y
vint et y demeura quelque temps, dont l'ambassadeur fut charmé. Il fut
reconduit à son embarquement, comme il en avait été amené. On lui donna
des fêtes dans les villes les plus considérables. Lyon s'y surpassa, où
il alla droit de Paris. Des vaisseaux du roi le portèrent avec sa suite
à Constantinople où il ne sut quelle chère faire et procurer à tous les
officiers de son passage et à tous les autres François. La fortune lui
rit tant que son ami demeura grand vizir\,; il eut part à sa disgrâce\,;
mais il se raccrocha, et a vécu plusieurs années depuis en place et en
considération, toujours ami des Français.

Le chevalier de Lorraine, frère du prince de Pons, quitta la croix de
Malte, pour épouser M\textsuperscript{lle} de Beauvau, fille de M. et de
M\textsuperscript{me} de Craon, qui pouvaient tout en Lorraine,
moyennant quoi M. de Lorraine le fit grand maître de sa maison, comme
l'avait été le feu prince Camille, son cousin germain, fils de M. le
Grand. Il prit le nom de prince de Lixin, et continua de servir en
France. C'était un homme très poli et fort brave, mais haut et
pointilleux à l'excès. Sur une dispute d'un point d'histoire fort
indifférent qu'il eut avec M. de Ligneville, frère de
M\textsuperscript{me} de Craon, sa belle-mère, aussi peu endurant que
lui, ils se battirent, et le prince de Lixin le tua. Il fut payé en même
monnaie pour s'être avisé seul, et dernier cadet de sa maison, de
trouver mauvais que le duc de Richelieu sur la naissance duquel il
s'espaça, eût épousé une fille de M. de Guise, soeur de la duchesse de
Bouillon. M. de Richelieu, après avoir fait tout ce qu'il avait pu pour
le ramener, se lassa enfin de ces procédés, se battit avec lui, et le
tua tout au commencement du siège de Philipsbourg par le maréchal de
Berwick, qui y fut tué lui-même.

Le maréchal de Villars maria son fils unique à une fille du duc de
Noailles, extrêmement jolie, et depuis dame du palais, et après dame
d'atours de la reine, femme de beaucoup d'esprit et d'agrément, devenue
dévote à ravir, et dans tous les temps intrigante et cheminant à
merveille.

Le duc de Boufflers épousa en même temps une fille du duc de Villeroy,
dont le maréchal de Villeroy fit magnifiquement la noce.

\hypertarget{chapitre-ix.}{%
\chapter{CHAPITRE IX.}\label{chapitre-ix.}}

1721

~

{\textsc{Dubois enfin cardinal.}} {\textsc{- Sa conduite en cette
occasion.}} {\textsc{- Conduite réciproque entre lui et moi.}}
{\textsc{- Il sort à merveille de ses audiences.}} {\textsc{- Croix
pectorale.}} {\textsc{- Embarras de M. de Fréjus.}} {\textsc{-
Imprudence de M\textsuperscript{me} de Torcy.}} {\textsc{- Dubois,
informé de mon ambassade, me rapproche par Belle-Ile pour me tromper et
me nuire.}} {\textsc{- Je le sens et ne puis l'éviter.}} {\textsc{-
Liaison plus qu'intime de Belle-Ile avec Le Blanc.}} {\textsc{- Leur
servitude sous Dubois.}} {\textsc{- Maladie du roi.}} {\textsc{- Audace
pestilentielle de la duchesse de La Ferté.}} {\textsc{- Conduite étrange
du maréchal de Villeroy.}} {\textsc{- Affectation de Te Deum sans fin.}}
{\textsc{- Instruction abominable et publique du maréchal de Villeroy au
roi.}} {\textsc{- Excellente conduite de M. le duc d'Orléans et des
siens dans la maladie du roi.}} {\textsc{- Mort de Trudaine\,; du duc de
Bouillon\,; son caractère.}} {\textsc{- Mort de Thury\,; son
caractère.}} {\textsc{- Mort du P. Lelong, de l'Oratoire.}} {\textsc{-
Armenonville obtient la survivance de sa charge de secrétaire d'État
pour son fils\,; la duchesse celle de gouvernante des enfants de France
pour M\textsuperscript{me} de Soubise, sa petite fille\,; Saumery, de la
sienne de sous-gouverneur du roi pour son fils aîné, chose sans
exemple.}} {\textsc{- Leur caractère.}} {\textsc{- Mort et caractère,
vie et conduite de M\textsuperscript{me} la grande-duchesse {[}de
Toscane{]}.}} {\textsc{- La conduite avec moi du cardinal Dubois
m'affranchit des conditions de notre raccommodement.}} {\textsc{-
Familiarité, liberté, confiance conservée entre M le Duc et moi, depuis
le lit de justice des Tuileries.}} {\textsc{- Conversation importante et
très curieuse entre M. le Duc et moi.}}

~

À mesure que le temps s'écoulait depuis l'exaltation du pape, et qu'il
était vivement pressé de tenir à l'abbé Dubois la parole qu'il lui avait
donnée par écrit au cas qu'il fût élu pape, l'impatience de Dubois
croissait avec ses espérances, et ne lui laissait plus de repos. Il se
trouva bien étourdi quand il apprit que le pape avait fait cardinal tout
seul, le 16 juin, son frère, évêque de Terracine depuis dix ans, moine
bénédictin du mont Cassin. Dubois s'attendait qu'il ne se ferait point
de promotion sans qu'il en fût, et jeta feu et flammes. Son attente ne
fut pas longue\,: un mois après, le 16 juillet, le pape le fit cardinal
avec don Alexandre Albane, neveu du feu pape et frère du cardinal
camerlingue. Il en reçut la nouvelle et les compliments avec une joie
extrême, mais qu'il sut contenir dans quelque décence, et en donner tout
l'honneur à la protection de M. le duc d'Orléans, qui, comme on l'a vu,
y eut peu ou point de part. Mais il ne se put empêcher de débiter à tout
le monde que ce qui l'honorait plus que la pourpre romaine était le voeu
unanime, et l'empressement de toutes les puissances à la lui procurer, à
en presser le pape, et à désirer que sa promotion fût avancée sans
attendre leur nomination ni la promotion des couronnes. Il s'éventait
là-dessus, et ne pouvait finir sur ce chapitre qu'il recommençait à tout
moment, et dont personne ne fut la dupe.

Quoique nous fussions au point où on l'a vu ici, je crus devoir mettre
M. le duc d'Orléans à son aise entre Dubois et moi, avec lequel j'allais
avoir un commerce nécessaire et forcé dans mon ambassade. J'allai donc
chez lui où il me combla de respects, de compliments, de protestations
de reconnaissance de l'honneur que je lui faisais, sans parler du passé.
Quoiqu'à la façon dont nous étions ensemble, et à l'occasion qui
m'amenait chez lui, la visite fût de cérémonie, et qu'il y eût un monde
infini, il en usa avec sa calotte rouge qu'il venait de recevoir des
mains du roi, comme si elle eût été encore noire, me fit litière de la
main, de termes de respect, de conduite jusque tout bout de son
appartement, et à la petite cour où il aboutissait. M. le duc d'Orléans
me témoigna beaucoup de gré de cette démarche de ma part, et je ne
rencontrai plus le nouveau cardinal chez ce prince qu'il ne vint à moi,
se reculât aux portes et ne me fît merveilles, auxquelles je n'avais
garde de me fier. En recevant sa calotte des mains du roi, il détacha de
son cou sa croix épiscopale, la présenta à l'évêque de Fréjus, lui dit
qu'elle portait bonheur, et que c'était pour cela qu'il le priait de la
porter pour l'amour de lui. Fréjus rougit et la reçut avec beaucoup
d'embarras. Cette croix, quoique faite comme toutes les autres avait
pourtant une façon très remarquable, et qui la faisait parfaitement
distinguer. Fréjus, exposé à rencontrer très fréquemment le cardinal
nouveau chez le roi, n'osa ne pas porter cette croix assez souvent.

Dînant dans ces premiers jours, ayant cette croix à son cou chez la
duchesse du Lude, avec M. et M\textsuperscript{me} de Torcy et bonne
compagnie, M\textsuperscript{me} de Torcy qui n'aimait pas Dubois, et
qui fort Arnauld était fort mécontente de l'ardente conduite de Fréjus
sur la constitution, et contre ce qu'on taxait de jansénisme, et
accoutumée à l'avoir vu si longtemps poirier\footnote{Expression
  proverbiale qui s'appliquait à un homme élevé en fortune, mais pour
  lequel on n'avait pas une grande considération, parce qu'on l'avait vu
  autrefois dans une position misérable. On prétend que cette expression
  vient de ce qu'un paysan ne voulait pas saluer la figure d'un saint de
  son village, parce qu'elle avait été faite avec un poirier de son
  jardin.}, commensal et complaisant de sa maison, l'entreprit sur cette
croix à table avec beaucoup d'esprit, de licence et d'aigreur, tombant
sur tous les deux avec une finesse aiguë, et mit Fréjus dans un tel
désordre qu'il ne savait plus où il en était, sans que la compagnie qui
s'en aperçut et qui souffrait de cette scène en pleine table, pût rompre
les chiens de cette chasse qui dura fort longtemps, et que Fréjus n'a
jamais pardonnée à M\textsuperscript{me} de Torcy, ni même à son mari,
quoiqu'il n'y eût rien mis du sien. Il était trop sage et trop mesuré
pour n'en avoir pas été très embarrassé lui-même, et à la vérité ce fut
une grande imprudence à M\textsuperscript{me} de Torcy.

L'abbé Passarini, camérier d'honneur du pape, étant arrivé avec le
bonnet, le nouveau cardinal le reçut des mains du roi, et fit ses
visites au sang royal avec les cérémonies accoutumées. Il avait eu près
de deux mois à s'y préparer, et il faut avouer qu'il en profita bien. Il
avait un compliment à faire à Madame et à M. {[}le duc{]} et à
M\textsuperscript{me} la duchesse d'Orléans, dans l'audience de
cérémonie qu'il en eut\,; car pour les visites aux princes et princesses
du sang, ce ne sont que visites et compliments en cérémonie, mais ce ne
sont pas des audiences avec un compliment en forme qui est une petite
harangue. Il devait bien s'attendre à ce que Madame souffrirait de le
recevoir en cérémonie, de le saluer et de lui donner un tabouret, et
M\textsuperscript{me} la duchesse d'Orléans, de lui donner un siège à
dos\,; après l'avoir vu si longuement si petit compagnon, et Madame qui
ne lui avait jamais pardonné le mariage de son fils, qui l'avait traité
toujours avec le plus grand mépris, parlé de lui sans mesure, et demandé
comme on l'a vu pour toute grâce à M. le duc d'Orléans, le jour de sa
régence de n'employer à rien ce petit fripon-là qui le vendrait et le
déshonorerait. Le cardinal Dubois se composa, parut devant Madame
pénétré de respect et d'embarras. Il se prosterna comme elle s'avança
pour le saluer, s'assit au milieu du cercle, se couvrit un instant de
son bonnet rouge qu'il ôta aussitôt, et fit son compliment. Il commença
par sa propre surprise de se trouver en cet état devant Madame, parla de
la bassesse de sa naissance et de ses premiers emplois, les employa avec
beaucoup d'esprit et en termes fort choisis à relever d'autant plus la
bonté, le coeur et la puissance de M. le duc d'Orléans, qui de si bas
l'avait élevé où il se voyait, se fit une leçon de n'oublier jamais ce
qu'il avait été, pour sentir toujours plus vivement ce qu'il devait à ce
prince, et employer tout ce qui pouvait être en lui, sans se louer ni
s'applaudir le moins du monde, pour le servir, car la modestie surnagea
toujours dans ses discours d'audiences, donna un encens délicat à
Madame, enfin se confondit en respects les plus profonds et en
reconnaissance. Il parla si judicieusement et si bien que quelque
indignation qu'on eût contre sa personne et sa fortune, tous ceux qui
l'entendirent en furent charmés, et Madame elle-même ne put s'empêcher,
après qu'il fut sorti, de louer son discours et sa contenance, tout en
ajoutant qu'elle enrageait de le voir où il était.

Ses audiences de M. le duc d'Orléans et de M\textsuperscript{me} la
duchesse d'Orléans se passèrent avec le même succès\,; ce fut le même
fond en d'autres termes. Je me suis étendu sur celle de Madame comme la
plus difficile et la plus curieuse, et j'ai voulu rapporter tout de
suite ce qui regarde cette réception du cardinalat.

Il ne fut pas longtemps sans que M. le duc d'Orléans lui apprit qu'il
m'avait promis l'ambassade d'Espagne et de me protéger pour une
grandesse pour mon second fils. À chose faite point de remède. Le
cardinal Dubois le comprit bien, il en fut outré et résolut bien de me
faire du pis qu'il pourrait en tous genres. Pour cela il fallut couvrir
son jeu, ne point montrer de mécontentement à M. le duc d'Orléans et me
combler de gentillesses pour me mieux tromper. Il n'était pas encore
cardinal lorsque cela arriva, mais il le fut tôt après. Il avait fait de
Le Blanc comme son secrétaire, pour ne pas dire comme son valet, l'avait
rendu assidu auprès de lui jusqu'à l'esclavage, tout secrétaire d'État
de la guerre qu'il était, et s'en servait à toutes mains, surtout depuis
l'affaire de M. et de M\textsuperscript{me} du Maine, dont il eut seul
tout le secret parce qu'il fut l'instrument dont il se servit
uniquement.

Belle-Ile était ami de Le Blanc. Le commerce des femmes et leur
attachement commun au char de M\textsuperscript{me} de Plénoeuf les
avait liés. Le Blanc était un esprit doux, fort inférieur à celui de
Belle-Ile, qui s'attacha de plus en plus à lui pour le gouverner et en
tirer, dès qu'il le vit en place, et qui en serra les liens à mesure
qu'il le vit dans tout ce qu'il était en Dubois de donner de confiance.
Par Le Blanc, il s'approcha de Dubois, et si bien que Dubois ne les
regarda plus que comme ne faisant qu'un et qu'il eut part à la même
confiance, jusque-là que tous les soirs ils entraient tous deux seuls
chez Dubois, et que, entre eux trois, il se disait et se passait bien
des choses. Dubois, qui n'ignorait rien en matière de commerce et de
liaisons, connaissait les miennes avec M\textsuperscript{me} de Lévi et
le duc de Charost, conséquemment avec Belle-Ile, tellement que ce fut de
lui qu'il se servit pour me rapprocher.

Je ne savais point encore que M. le duc d'Orléans eût parlé de mon
ambassade à Dubois, et je n'en avais moi-même ouvert la bouche à qui que
ce soit, lorsque je vis entrer Belle-Ile chez moi, qui, après un court
préambule, me parla de mon ambassade en homme qui n'en ignorait rien. Ma
surprise fut grande, elle ne m'empêcha pas de demeurer ignorant et
boutonné. Alors Belle-Ile me dit que je pouvais lui en parler
franchement, parce qu'il savait tout par l'abbé Dubois, à qui M. le duc
d'Orléans l'avait dit, et tout de suite me demanda comment j'entendais
me conduire là-dessus avec l'abbé Dubois, qui avait seul les affaires
étrangères, qui n'attendait que le moment de sa promotion, dont je ne
pouvais me dissimuler le crédit et l'ascendant entier sur M. le duc
d'Orléans, qui, après mon départ, demeurerait sans contre-poids le
maître de son maître, et qui me pouvait servir ou nuire infiniment\,;
qu'au demeurant il ne me dissimulerait pas qu'il m'apportait le choix de
la paix ou de la guerre\,; que Dubois était infiniment ulcéré de tout ce
que j'avais dit tant de fois à M. le duc d'Orléans contre lui\,; que,
malgré cela, il ne s'éloignerait pas de revenir à moi, et de se
raccommoder, d'y vivre sur l'ancien pied, mais à de certaines
conditions, et de me servir utilement et franchement dans le cours de
mon ambassade, et pour l'objet qui me l'avait fait désirer.
L'exhortation amicale suivit, et cependant je faisais mes réflexions.

Je connaissais trop le terrain pour ne pas sentir que Belle-Ile disait
vrai en tout, excepté sur la sincérité d'une âme si double et
offensée\,; mais que ne me pas prêter à un raccommodement offert
donnerait beau jeu à Dubois auprès de M. le duc d'Orléans, qui serait
également embarrassé et importuné de ce contraste, et qui surtout en mon
absence, je veux dire Dubois, {[}en{]} saurait bien profiter\,; de plus,
comment éviter le commerce réglé de lettres avec l'homme chargé seul des
affaires étrangères, et comment le soutenir avec un homme avec qui on
est brouillé et avec qui on n'a pas voulu se raccommoder\,? Ces
considérations si évidentes ployèrent ma raideur\,; mais je voulus
savoir ce que c'était que les conditions dont il m'avait parlé.
Belle-Ile me dit qu'elles n'étaient pas difficiles\,: d'oublier de part
et d'autre tout ce qui s'était passé, ne nous en jamais parler, promesse
de ne plus rien dire en public contre lui ni en particulier à M. le duc
d'Orléans, nous revoir et traiter ensemble à l'avenir avec ouverture et
liberté, et que je verrais que Dubois, ravi de n'avoir plus à me compter
au nombre de ses ennemis, irait au-devant de tout ce qui me pourrait
plaire. Belle-Ile, tout de suite, sans me laisser le temps de parler, me
fit l'analyse de ces conditions telle que je la sentais moi-même\,: la
nécessité du raccommodement avec un homme qui me l'offrait, avec qui il
fallait concerter tout ce qui pouvait regarder mon ambassade, et avoir
avec lui un commerce de lettres réglé toutes les semaines, tant qu'elle
durerait, sans possibilité de le faire passer par un autre\,; le
raccommodement fait, l'indécence de parler mal en public d'un homme avec
qui on s'est raccommodé, enfin d'en parler mal à M. le duc d'Orléans en
particulier\,; l'expérience de l'inutilité, même du danger, me devait
convaincre là-dessus et la raison me démontrer qu'il était déjà le
maître des affaires, des grâces de tout l'intérieur\,; combien plus
l'allait-il devenir quand il serait élevé à la pourpre, qui peut-être
était déjà en chemin par un courrier\,! À l'égard de la bonne foi,
quelque difficulté que je pusse avoir d'y prendre confiance, je lui
liais les bras par ce raccommodement, quitte à marcher avec les
précautions raisonnables, et à voir de jour à autre comment il se
conduirait avec moi, parti sage en tous ses points, dont je ne pourrais
jamais me faire de reproche dans ma position présente, et bien différent
d'une brouillerie ouverte dans la situation où je me trouvais.

Ces mêmes raisons m'avaient déjà sauté aux yeux, de sorte que je
renvoyai Belle-Ile content de sa négociation, qui, deux jours après, me
vint dire merveilles de la part de Dubois. Là-dessus sa calotte arriva.
Je fus le voir comme je l'ai dit, et le surlendemain il vint chez moi.
Sa barrette arrivée, il ne tarda pas à y revenir encore en habit long et
rouge. On peut juger quelle put être notre confiance réciproque\,: aussi
n'eûmes-nous pas sitôt entamé les propos de l'ambassade, et ils le
furent dès lors, que je vis clairement son venin et sa duplicité. Aussi
me crus je dispensé à son égard de tout ce que la prudence me pouvait
permettre. Pour ne point interrompre ce qui se passa sur mon ambassade,
avant mon départ, je le remettrai tout de suite au temps de mon départ
même, quoique les propos et la tyrannie en aient commencé dès ce
temps-ci, presque aussitôt que nous nous fûmes vus. Passons à un
événement qui fut court, mais qui effraya beaucoup.

Le dernier juillet, le roi, jusqu'alors dans une santé parfaite, se
réveilla avec mal à la tête et à la gorge\,; un frisson survint, et sur
l'après-midi, le mal de tête et de gorge ayant augmenté, il se mit au
lit. J'allai le lendemain, sur le midi, savoir de ses nouvelles. Je
trouvai que la nuit avait été mauvaise et qu'il y avait depuis deux
heures un redoublement assez fort. Je vis partout une grande
consternation. J'avais les grandes entrées, ainsi j'entrai dans sa
chambre. Je la trouvai fort vide, M. le duc d'Orléans, assis au coin de
la cheminée, fort esseulé et fort triste. Je m'approchai de lui un
moment, puis j'allai au lit du roi. Dans ce moment Boulduc, un de ses
apothicaires, lui présentait quelque chose à prendre. La duchesse de La
Ferté, qui, par la duchesse de Ventadour sa soeur, avait toutes les
entrées comme marraine du roi, était sur les épaules de Boulduc, et
s'étant tournée pour voir qui approchait, elle me vit, et tout aussitôt
me dit entre haut et bas\,: «\, Il est empoisonné, il est empoisonné.
--- Taisez-vous donc, madame, lui répondis-je, cela est horrible\,!»
Elle redoubla et si bien et si haut, que j'eus peur que le roi ne
l'entendit. Boulduc et moi nous nous regardâmes, et je me retirai
aussitôt d'auprès du lit et de cette enragée avec qui je n'avais nul
commerce. Pendant cette maladie, qui ne dura que cinq jours, mais dont
les trois premiers furent violents, j'étais fort fâché et fort en
peine\,; mais en même temps si aise d'avoir opiniâtrement refusé d'être
gouverneur du roi, et si agité en me représentant l'être, et en quel
état je serais, que je m'en réveillais la nuit en sursaut, et ces
réveils étaient pour moi de la joie la plus sensible de ne l'être pas.
La maladie ne fut pas longue et la convalescence fut prompte, qui rendit
la tranquillité et la joie, et causa un débordement de Te Deum et de
réjouissances. Helvétius en eut tout l'honneur, les médecins avaient
perdu la tête\,; il conserva seul la sienne, il opiniâtra une saignée au
pied dans une consultation où M. le duc d'Orléans fut présent\,; il
l'emporta\,: le mieux très marqué suivit incontinent et la guérison
bientôt après.

Le maréchal de Villeroy ne manqua pas cette occasion de signaler tout
son venin et sa bassesse\,; il n'oublia rien pour afficher des soupçons,
des soins, des inquiétudes extrêmes, et pour faire sa cour à la robe. Il
ne vint point si petit magistrat aux Tuileries qu'il ne se fit avertir
pour lui aller dire lui-même des nouvelles du roi et le caresser, tandis
qu'il était inaccessible aux premiers seigneurs. Les magistrats plus
considérables, j'entends toujours du parlement, ou les chefs des autres
compagnies, ou leurs gens du parquet, il les faisait entrer à toute
heure dans la chambre du roi et tout auprès de son lit pour qu'ils le
vissent, tandis qu'à peine ceux qui avaient les grandes entrées
jouissaient de la même privance. Il en usa de même dans la première
convalescence, qu'il prolongea le plus qu'il put pour donner la même
distinction aux magistrats à quelque heure qu'il en vint, et
privativement aux plus grands de la cour et aux ambassadeurs\,; il se
croyait tribun du peuple et aspirait à leur faveur et à leur dangereuse
puissance. De là il se tourna à une autre affectation, qui avait le même
but contre M. le duc d'Orléans. Il multiplia les Te Deum, qu'il incita
les divers états des petits officiers du roi de faire chanter en
différents jours et en différentes églises, assista à tous, y mena tout
ce qu'il put, et courut encore plus de six semaines les Te Deum qui se
chantèrent dans toutes les églises de Paris. Il ne parlait d'autre
chose, et sur sa joie véritable de la guérison, il en entait une fausse
qui puait le parti et le dessein à ne s'y pouvoir méprendre. Il fit
faire force fêtes à Lyon et à son fils l'archevêque, dont il eut soin de
faire répandre les relations.

Le roi alla en cérémonie remercier Dieu à Notre-Dame et à
Sainte-Geneviève. Ces momeries, ainsi allongées, gagnèrent la fin du
mois d'août et la Saint-Louis. Il y a tous les ans ce jour-là un concert
le soir dans le jardin. Le maréchal de Villeroy prit soin que ce concert
devint une manière de fête, à laquelle il fit ajouter un feu d'artifice.
Il n'en faut pas tant pour attirer la foule\,; elle fut telle, qu'une
épingle ne serait pas tombée à terre dans tout le parterre. Lés fenêtres
des Tuileries étaient parées et remplies, et tous les toits du Carrousel
pleins de tout ce qui put y tenir, ainsi que la place. Le maréchal de
Villeroy se baignait dans cette affluence, qui importunait le roi qui se
cachait dans des coins à tout moment\,; le maréchal l'en tirait par le
bras et le menait tantôt aux fenêtres d'où il voyait la cour et la place
du Carrousel toute pleine, et tous les toits jonchés de monde\,; tantôt
à celles qui donnaient sur le jardin, et sur cette innombrable foule qui
y attendait la fête. Tout cela criait vive le roi\,! à mesure qu'il en
était aperçu, et le maréchal retenant le roi qui se voulait toujours
aller cacher\,: «\,Voyez donc, mon maître, tout ce monde et tout ce
peuple, tout cela est à vous, tout cela vous appartient, vous en êtes le
maître\,; regardez-les donc un peu pour les contenter, car ils sont tous
à vous\,; vous êtes maître de tout cela.\,» Belle leçon pour un
gouverneur, qu'il ne se lassait point de lui inculquer à chaque fois
qu'il le menait aux fenêtres, tant il avait peur qu'il l'oubliât\,!
Aussi l'a-t-il très pleinement retenue. Je ne sais s'il en a reçu
d'autres de ceux qui ont eu la charge de son éducation. Enfin le
maréchal le mena sur sa terrasse, où dessous un dais il entendit la fin
du concert et vit après le feu d'artifice. La leçon du maréchal de
Villeroy si souvent et si publiquement répétée, fit grand bruit et à lui
peu d'honneur. Lui-même a éprouvé le premier effet de ses belles
instructions.

M. le duc d'Orléans se conduisit d'une manière si simple et si sage
qu'il y gagna beaucoup. Des soins et une inquiétude raisonnable mais
mesurée, une grande réserve dans ses discours, une attention exacte et
soutenue en propos et en contenance, qui {[}ne{]} laissa rien échapper
qui sentît le moins du monde qu'il était le successeur, surtout à ne
jamais montrer croire le roi trop bien ni trop mal, et laisser aucun
lieu qu'il le craignit trop bien et qu'il le souhaitât mal. Il ne
pouvait douter qu'une conjoncture si critique pour lui ne fixât sur lui
les regards les plus perçants et l'attention de tout le monde, et comme
dans la vérité il ne souhaita jamais la couronne, quelque peu
vraisemblable que cela paroisse, il n'eut besoin que de s'observer et
point du tout de se contraindre\,; aussi n'eut-il besoin d'aucun conseil
là-dessus, et son intérieur le plus libre et le plus familier, moi par
exemple, le vit toujours là-dessus tel que le publie le vit. Cela fut
aussi fort remarqué, et la cabale opposée fut entièrement réduite au
silence, qui se préparait bien à faire valoir jusqu'aux riens qu'elle
aurait aperçus. Il fut heureux que ceux qui lui étaient particulièrement
attachés et qui auraient pu se flatter le plus d'un événement sinistre
aient tous gardé toute la même conduite que lui, sans qu'aucun d'eux,
jusqu'aux valets, et c'est une merveille, aient laissé échapper de quoi
faire naître le plus léger soupçon.

Trudaine, conseiller d'État, à qui M. le duc d'Orléans avait fort mal à
propos ôté la prévôté des marchands, dont il a été parlé ici en son
lieu, mourut à soixante-deux ans. Ce n'était pas un aigle, mais un très
honnête homme, intègre, désintéressé, vertueux.

Le duc de Bouillon mourut en même temps, à quatre-vingt-deux ans,
s'étant démis, depuis la régence, de sa charge de grand chambellan et de
son gouvernement d'Auvergne en faveur du duc d'Albret, son fils aîné,
qui prit le nom de duc de Bouillon, à qui le feu roi ne les aurait
jamais laissé passer, et qui, comme on l'a vu ici en son temps, avait eu
de grands procès contre son père et avait été fort mal avec lui. Le père
était fort bon homme, prince tant qu'il pouvait, du reste fort valet,
mais du roi seulement, et d'une assiduité qui, jointe avec un esprit
extrêmement court, lui avait entièrement gagné le roi, quoique des
aventures de sa femme et du cardinal son frère l'eussent fait éloigner
plus d'une fois de la cour. On a vu ici en son lieu que beaucoup d'art,
quelque chose de pis de la part du procureur général d'Aguesseau, depuis
chancelier, l'habitude et l'affection du roi, sauvèrent sa prétendue
principauté, à l'évasion du cardinal de Bouillon du royaume.

Thury mourut aussi à soixante-deux ans, sans avoir été marié, ayant
donné ou plutôt trafiqué tout ce qu'il avait avec le maréchal
d'Harcourt. Ils étaient fils des deux frères, mais totalement
différents. Thury était noir, méchant, cynique, atrabilaire, avec
beaucoup d'esprit insolent et dangereux\,; et quoique avec méchante
réputation à la guerre et dans le monde, reçu en de bonnes compagnies.
Il est pourtant vrai qu'un soufflet que le duc d'Elboeuf lui appliqua à
table, avec une épaule de mouton, dont il ne fut autre chose, était
resté imprimé sur sa mauvaise physionomie.

Ils furent suivis du P. Lelong, prêtre de l'Oratoire, bibliothécaire de
leur maison de Saint-Honoré, à Paris, où il mourut, à cinquante-six ans,
regretté de tous les gens de bien, des savants et des hommes de lettres.
Il avait donné, sous le nom de Bibliothèque historique, {[}un ouvrage{]}
contenant avec une grande exactitude, une liste, en différentes classes,
de tous les ouvrages qui ont rapport à l'histoire de France, sacrée ou
profane\footnote{La Bibliothèque historique du P. Lelong parut en 1719
  en un vol.~in-fol.~Fevret de Fontette en a donné une édition beaucoup
  plus complète, en 5 vol.~in-fol.~7 1768.}, et un autre sous le titre
latin de Bibliotheca sacra, où il a donné le catalogue des manuscrits et
des éditions des textes originaux de la Bible et des versions, en toutes
sortes de langues, et des auteurs qui ont écrit sur la Bible.

Armenonville obtint pour son fils Morville la survivance de sa charge de
secrétaire d'État, et M\textsuperscript{me} de Ventadour celle de sa
charge de gouvernante des enfants de France, pour M\textsuperscript{me}
de Soubise, femme de son petit-fils, quoique très jeune, mais très sage
et très convenable à cette place.

Saumery, l'un des sous-gouverneurs du roi, dont il a été parlé ici en
plus d'un endroit, comblé déjà de grâces, avec tout ce qu'il fallait
pour n'en obtenir aucune en aucun temps, et qui en celui-ci était lié
avec toute la cabale opposée à M. le duc d'Orléans, en obtint de lui une
sans exemple ce fut la survivance de sa place de sous-gouverneur du roi
pour son fils aîné, qui valait en tout mieux que lui, car il était fort
honnête homme, avec du sens, avait bien servi et était envoyé du roi
quelque temps à Munich. C'était grossièrement lui faire passer les
entrées et les appointements de sous-gouverneur, parce que le père était
de santé à n'y avoir pas besoin d'aide, et à achever, et bien au delà,
comme il fit, le temps que le roi avait à être sous des gouverneurs.

M\textsuperscript{me} la grande-duchesse {[}de Toscane{]} mourut à
soixante-dix-sept ans, après plusieurs apoplexies, et fut enterrée,
comme elle l'avait ordonné, parmi les religieuses de Picpus, dans leur
cloître. Elle était fille aînée du second mariage de Gaston, frère de
Louis XIII, avec la soeur de Charles IV, duc de Lorraine.
M\textsuperscript{me} la grande-duchesse avait été fort belle, et très
bien faite et grande\,: on le voyait bien encore\,; bonne et peu
d'esprit, mais arrêtée en son sens sans pouvoir être persuadée. Elle
épousa, en 1661, Cosme de Médicis, grand-duc de Toscane, avec un esprit
de retour que rien ne put amortir. Elle vécut fort mal avec le
grand-duc, dont la patience et les soins pour la ramener furent
continuels, plus mal encore avec la grande-duchesse sa belle-mère, qui
était La Rovère-Urbin, morte en 1694, à soixante-douze ans.

Elle voulait vivre en liberté à la française, et se moquait de toutes
les manières italiennes. Elle eut assez promptement trois enfants\,:
l'aîné qui mourut longtemps avant son père, sans enfants de la soeur de
M\textsuperscript{me} la dauphine de Bavière\,; J. Gaston, marié à une
fille du dernier duc de Saxe-Lauenbourg, et dernière elle-même de cette
grande et si ancienne maison, avec qui il se brouilla, n'en eut point
d'enfants, succéda au grand-duc son père, mort à quatre-vingt-deux ans,
en 1723, et mourut sans postérité en {[}1737{]} et finit les Médicis,
grands-ducs de Toscane, après avoir vu souvent et diversement disposer,
pour après lui, de ses États, de son vivant\,; enfin l'électrice
Palatine, veuve sans enfants, et depuis son veuvage retirée à Florence.

Après avoir eu ces enfants, la grande-duchesse redoubla d'humeur exprès,
et de conduite étrange en Italie, avec tant d'éclat que le roi y mit la
main, par ses envoyés, diverses fois, et par les cardinaux d'Estrées et
Bonzi, allant et revenant de Rome, sans pouvoir lui rien persuader. Elle
en fit tant que le grand-duc consentit enfin à son retour en France,
mais sous des conditions qui lui donnèrent plus de contrainte qu'elle
n'en aurait eue à Florence en vivant bien avec son mari et sa
belle-mère, et que le roi lui fit scrupuleusement observer toujours,
parce qu'il était informé de sa conduite et très content de toute celle
que le grand-duc avait eue avec elle. Il lui assigna une pension telle
qu'il plut au roi, voulut qu'elle fût dans un couvent hors de Paris,
qu'elle ne couchât jamais à Paris et qu'elle y vint rarement, qu'elle
n'allât jamais à la cour que mandée ou pour quelque devoir très
nécessaire de famille, dont à chaque fois le roi déciderait, et sans y
coucher, à moins que cela ne fût indispensable, au jugement du roi, et
encore pour une seule nuit. Elle revint donc de la sorte, vers 1669,
fort peu accueillie, confinée au couvent de Picpus, où elle vit très peu
de monde. Après bien des années, elle se mit à venir souvent à Paris,
chez qui elle pouvait passer quelques heures, ou à quelques dévotions,
sans crédit et avec peu ou point de considération.

Sur la fin de la vie de Monsieur, qui en avait pitié, elle obtint la
liberté de passer à Saint-Cloud le temps qu'il y était. Madame, M. {[}le
duc{]} et M\textsuperscript{me} la duchesse d'Orléans lui firent
toujours fort bien. Mademoiselle, sa soeur de père, la méprisa toujours
parfaitement, et M\textsuperscript{me} de Guise, sa soeur de père et de
mère, n'en fit jamais grand cas\,; elle jouit de son rang de
petite-fille de France et de tous les honneurs qui y sont attachés. Sur
les fins, elle quitta Picpus pour le couvent de Saint-Mandé, et après la
mort du roi, le grand-duc son mari accorda à M. le duc d'Orléans qu'elle
pût loger à Paris. Elle y loua en très -simple particulière une maison à
la place Royale, où elle mourut dans une grande dévotion à sa manière
depuis longtemps, et, quoique avare, fort appliquée aux bonnes
oeuvres\,; elle était fort polie et bonne avec tout le monde.

J'étais alors aux prises avec le cardinal Dubois sur ce qui regardait
mon ambassade, et je voyais en plein ses bonnes intentions qui
n'allaient à rien moins qu'à me ruiner et me perdre, en me suscitant des
embarras en Espagne les plus ridicules, les plus fous et les plus
difficiles à. m'en tirer. Je ne dis que ce mot à cause de ce qui va
suivre, pour en raconter le détail de suite lors de mon départ, et ne
plus interrompre la matière de l'ambassade. Le cardinal, depuis fort peu
après que nous nous fûmes revus, comme je l'ai dit plus haut, me montra
à découvert ce que j'en devais attendre, et me délivra ainsi des
conditions de notre raccommodement, sur quoi néanmoins il fallut me
conduire avec la prudence que demandait la nécessité de passer sans
cesse par lui, jusqu'à mon départ, et dans tout le cours de mon
ambassade, et l'incroyable ascendant dont il était en pleine possession
sur M. le duc d'Orléans. Depuis le commerce étroit et plein de confiance
que l'affaire du lit de justice des Tuileries m'avait procuré avec M. le
Duc, il avait toujours duré le même. M. le duc d'Orléans et M. le Duc
l'avaient tous deux désiré, et j'étais souvent entre eux deux pour
conserver leur union nécessaire.

Un jour que je causais fort librement avec M. le Duc, il me parla fort
librement aussi de beaucoup de choses de sa famille. Nous avions souvent
traité ensemble le fameux chapitre de l'enfant de treize mois, dans les
temps que la duchesse du Maine ne se faisait faute d'en parler dans ses
grands éclats du procès de la succession de M. le Prince et des disputes
sur la qualité de prince du sang que la maison de Condé fit rayer au duc
du Maine, et lorsque les bâtards perdirent leur prétendue habilité de
succéder à la couronne, que le duc du Maine et Mine de Maintenon avaient
arrachée à la mourante faiblesse du feu roi. M. le Duc, à la mort de
M\textsuperscript{me} sa femme, arrivée dans les premiers mois de
l'année précédente, avait retenu des actions et force pierreries de sa
succession, malgré les plaintes de M\textsuperscript{lle} de La
Roche-sur-Yon, sa belle-soeur, qui avaient fait et faisaient encore
grand bruit dans le monde, et qu'il lui rendit longtemps après quand il
commença à songer à sa manière sérieusement à son salut. Ce chapitre
avait été effleuré entre lui et moi, et j'étais peiné qu'il se fit ce
tort dans le monde. Je lui proposai donc la nécessité de se remarier
pour avoir des enfants, puisque MM. ses frères n'y voulaient point
entendre, et pour couper court à toute cette affaire de la succession de
M\textsuperscript{me} sa femme, d'épouser M\textsuperscript{lle} de La
Roche-sur-Yon. Il se mit à sourire, et me répondit que, pour des Conti,
il en avait sa suffisance, et me parla de la conduite de feu
M\textsuperscript{me} la Duchesse, qui en effet ne s'était pas
contrainte sur les mesures, et qu'il avait soufferte avec une patience
qu'on n'aurait pas attendue de lui, et qu'il n'étendit pas depuis à
celle de sa seconde femme. De propos en propos, il me fit des plaintes
du peu de confiance de M. le duc d'Orléans, qui d'ordinaire ne lui
disait les choses que lorsqu'elles ne se pouvaient plus cacher.
J'excusai cela comme je pus, tant qu'enfin acculé par les faits qu'il
m'allégua, je me mis à sourire, et lui dis que, s'il me promettait de ne
le point trouver mauvais, je lui en dirais bien la raison, et le moyen
d'établir la plus entière confiance. Après quelques propos généraux et
réciproques là-dessus, et qu'il m'eut fort pressé de lui parler en ami,
et avec une franchise dont il n'aurait garde de se déplaire je lui dis
que, s'il voulait en user comme faisait M. le duc d'Orléans, ils
seraient bientôt contents l'un de l'autre. Après l'avoir un peu tenu
là-dessus, je lui dis qu'il avait une maîtresse la plus parfaitement
choisie pour les charmes du corps et de l'esprit\,; qu'à cela je n'avais
rien à lui dire\,; que c'était l'affaire de son confesseur\,; mais que
M. le duc d'Orléans était persuadé qu'il n'avait point de secrets pour
elle\,; que cela faisait qu'il en avait pour lui\,; que, s'il pouvait
être comme M. le duc d'Orléans, qui s'amusait avec ses maîtresses, avec
qui il ne lui échappait jamais rien de sérieux, je lui répondais qu'il
serait content de la confiance de ce prince. Il se défendit de ce
soupçon du régent assez mal, et avec un air peiné dit que c'était excuse
et prétexte, en sorte que je lui dis que, si je m'étais expliqué si
ouvertement avec lui, ce n'était que par le désir que j'avais de voir
leur union parfaite, si utile au bien de l'État, mais qui au fond lui
était bien plus nécessaire qu'à M. le duc d'Orléans. On verra dans la
suite qu'il rapporta ce point jaloux de notre conversation à
M\textsuperscript{me} de Prie, sa maîtresse, qui ne me le pardonna pas.
Revenu bien à lui de ce petit nuage, il jeta tout ce défaut de confiance
sur le cardinal Dubois, qui, tant qu'il pouvait, n'en permettait que
pour soi â son maître, et se mit à pleurer l'aveuglement et la faiblesse
de M. le duc d'Orléans pour ce valet indigne, qui en abusait sans cesse
si énormément. Ces propos me firent naître la pensée de revenir par un
autre biais à ce que Torcy avait pensé, et que la sottise du maréchal de
Villeroy avait fait manquer, comme je l'ai expliqué il n'y a pas
longtemps.

Il paraissait dans ce temps-là que le roi aimait M. le Duc. Je lui en
parlai comme en étant fort aise, et tout de suite je lui dis qu'il
devrait bien profiter de cette affection du roi pour le bonheur de
l'État et de M. le duc d'Orléans lui-même, en faisant bien connaître au
roi le danger de cette autorité que le cardinal Dubois avait usurpée\,;
la facilité que Sa\,: Majesté avait de montrer de l'aversion pour lui,
et d'engager M. le duc d'Orléans, qui avait si grandement fait pour lui,
de l'envoyer à Cambrai avec sa calotte rouge, et gorgé d'abbayes pour ne
plus revenir à la cour et n'avoir plus aucune part aux affaires. M. le
Duc se mit à rire à cette proposition. «\,Je suis bien aise, me dit-il,
qu'on croie que le roi a de l'amitié pour moi et de la confiance, et en
effet il m'en témoigne autant qu'il en est capable. Mais tout cela roule
sur des riens, et je le connais bien, sans se soucier de moi que par
l'habitude de me voir et de me parler, et je puis vous répondre que, si
je venais à mourir aujourd'hui, il ne s'en soucierait non plus que de
M\textsuperscript{me} la grande-duchesse, dont nous portons le deuil, et
ne parlerait que des causes de mort qu'on m'aurait trouvées avec la même
indifférence qu'il s'entretient de l'ouverture de cette princesse qu'à
peine avait-il vue.\,» Tout de suite il me parla de ce qu'il remarquait
du roi que son assiduité lui faisait sentir, quelque peu d'esprit qu'il
eût, ce qui n'est pas matière de ces Mémoires. Mais le résultat de la
conversation fut la parfaite et très certaine inutilité, peut-être même
le danger de cette tentative à laquelle le roi était radicalement
incapable de prendre, quoiqu'on vit bien qu'il avait une sorte
d'éloignement du cardinal Dubois.

\hypertarget{chapitre-x.}{%
\chapter{CHAPITRE X.}\label{chapitre-x.}}

1721

~

{\textsc{Mort, caractère, conduite du cardinal de Mailly.}} {\textsc{-
Il obtient que son neveu de Nesle porte la queue du grand manteau de
l'ordre du roi à Reims.}} {\textsc{- Il ne va point à Rome, arrêté par
une opération instante au moment de son départ.}} {\textsc{-
Réflexions.}} {\textsc{- Reims persévéramment offert à Fréjus,
obstinément refusé.}} {\textsc{- Motifs de l'un et de l'autre.}}
{\textsc{- Sa conduite à l'égard du roi, du régent, du maréchal de
Villeroy, du monde.}} {\textsc{- Raison à moi particulière de désirer
que Fréjus acceptât Reims.}} {\textsc{- Sagacité très singulière d'une
femme de chambre.}} {\textsc{- Fréjus accepte à grand'peine l'abbaye de
Saint-Étienne de Caen.}} {\textsc{- Fréjus point avide de biens.}}
{\textsc{- Fréjus, parfaitement ingrat, empêche que Reims soit donné à
Castries, archevêque d'Albi.}} {\textsc{- Abbé de Guéméné archevêque de
Reims.}} {\textsc{- Retraite et caractère du duc de Brancas.}}
{\textsc{- Mort, fortune et caractère de l'abbé de Camps.}} {\textsc{-
Mort de l'évêque-duc de Laon, Clermont-Chattes.}} {\textsc{- Ses deux
premiers successeurs.}} {\textsc{- Mort et caractère de l'archevêque de
Rouen, Besons.}} {\textsc{- Son successeur.}} {\textsc{- Mort du duc de
Fitz-James\,; de M\textsuperscript{lle} de La Rochefoucauld\,; de
M\textsuperscript{me} de Polignac, mère du cardinal\,; de Prior, à
Londres.}}

~

Le cardinal de Mailly était mort quatre jours avant
M\textsuperscript{me} la grande-duchesse dans l'abbaye de Saint-Thierry,
unie à l'archevêché de Reims, à soixante-trois ans. Cette mort était
bien propre à faire faire de grandes réflexions. J'ai parlé plus d'une
fois de ce prélat, de ma liaison étroite avec lui, de ses causes et de
ses suites, quoique lui et moi pensassions bien différemment sur
l'affaire de la constitution\,; du peu de vocation à son état, de son
ambition et de sa passion démesurée pour le cardinalat dès ses premiers
commencements\,; de ses démarches hardies et continuelles pour y
parvenir\,; de sa haine jusqu'à la fureur pour le cardinal de Noailles,
et de ses faibles et injustes causes\,; de son déchaînement forcené pour
la constitution, par toutes ces raisons, et uniquement de son aveu à moi
par ces raisons, jusqu'à m'avoir dit, dans ses plus grands emportements
sur cette affaire, que, si le cardinal de Noailles avait été pour la
constitution, lui Mailly aurait été contre avec la même rage qu'il était
pour cette bulle. Un léger abrégé suffira donc sur ce qui le regarde,
puisqu'on a vu en son lieu comment d'aumônier du roi, et vieux pour cet
emploi, avec une abbaye fort mince, il devint tout d'un coup archevêque
d'Arles, puis de Reims, par quels étranges chemins cardinal, puis
reconnu tel en France, enfin abbé de Saint-Étienne de Caen. Il eut Arles
en 1697, Reims en 1710\,; le chapeau, 19 novembre 1719, reconnu cardinal
plusieurs mois après par le régent et le roi avec grand'peine. Quoique
d'une santé ferme et que je n'ai vue altérée en rien jusqu'à l'événement
dont je vais parler, il vivait depuis qu'il fut cardinal dans le plus
exact régime, et sur ses heures, et sur le choix et la mesure de son
manger et sur mille sortes de bagatelles, tant il désirait jouir
longtemps de sa fortune. Il voyait le sacre instant et un conclave peu
éloigné. Ces cérémonies et la figure qu'il y allait faire le
transportaient. Il ne songea qu'à partir brusquement dès qu'on eut la
nouvelle de la mort du pape\,; mais il eut l'avisement de profiter de la
circonstance. En prenant congé du régent, il lui représenta que le sacre
était fort proche, qu'il aurait l'honneur de le faire, et de conférer le
lendemain l'ordre du Saint-Esprit au roi qui ne l'avait pas encore
reçu\,; que le roi choisissait toujours un seigneur pour porter ce
jour-là, et le lendemain qu'il faisait des chevaliers, la queue de son
grand manteau de l'ordre, ce qui lui donnait droit, quelque âge qu'il
eût, d'être compris dans la promotion suivante, comme il était arrivé de
M. de Nevers en 1661, à la première fleur de son âge, et là-dessus
demanda et obtint que son neveu le marquis de Nesle fût choisi pour
cette fonction. La promesse en fut si publique que, quoique le cardinal
de Mailly fût mort lorsque le roi fut sacré, la parole fut tenue, et le
marquis de Nesle fut chevalier de l'ordre de la promotion de 1724, si
nombreuse et si peu choisie, quelques années avant l'âge.

Je passai avec le cardinal de Mailly toute la soirée de la veille qu'il
devait partir pour Rome\,; je ne vis jamais un homme si content. Je le
quittai tard, se portant très bien. Le lendemain sur le midi, je fus
bien étonné d'apprendre par un homme qu'il m'envoya qu'il s'était trouvé
si mal la nuit, que dès le grand matin, il avait envoyé chercher du
secours, lequel lui avait trouvé la fistule, et si pressée à y
travailler que sans autre préparation l'opération lui avait été faite
fort heureusement, qu'il était aussi bien qu'il était possible, et qu'il
me priait de l'aller voir. Je le trouvai en effet fort bien pour son
état, mais bien touché de n'aller point à Rome. Le sacre prochain le
consolait et l'espérance de voir un autre conclave. Je ne m'étais jamais
aperçu qu'il fût attaqué d'aucun mal, et lui-même n'en avait jamais
parlé\,; il croyait de temps en temps avoir des hémorroïdes à ce qu'il
dit depuis, et n'en faisait point de cas. Je ne sais comment cette
opération fut faite\,; mais on apprit depuis sa mort qu'il lui était
demeuré un écoulement qu'on lui avait bien recommandé d'entretenir. Il
vit bientôt le monde, tant sa guérison s'avança sans aucun accident, et
en peu de temps reprit sa vie accoutumée. Cinq mois se passèrent de la
sorte. Il s'en alla à Reims où il n'était pas à son aise, et qu'il avait
accablé de lettres de cachet. Il se retira bientôt après à Saint-Thierry
qui n'en est qu'à quelques lieues, qui lui servait de maison de
campagne, ne respirant que feu et sang contre les opposants à la
constitution, et sa vengeance particulière de ceux qui osaient encore
lui résister, lorsque tout à coup cet écoulement s'arrêta, et fit une
révolution à la tête, où il sentit des douleurs à crier les hauts cris.
À peine ce tourment eut-il duré quatorze ou quinze heures, malgré les
saignées et tout ce qu'on put employer, qu'il perdit la connaissance et
la parole, et mourut dix ou douze heures après, sans avoir eu un moment
à penser à sa conscience. Quelle fin de vie dans un prêtre et dans un
évêque, toute d'ambition et persécuteur effréné par ambition et par
haine\,! Il passionna les honneurs, il goûta seulement des plus grands
comme pour s'y attacher davantage. Ce qu'ils avaient pour lui de plus
flatteur lui fut montré et porté, pour ainsi dire, jusqu'au bord de ses
lèvres. La coupe lui en fut subitement retirée sans qu'il y pût toucher
au moment d'y mettre la bouche et d'en boire à longs traits. Livré à des
douleurs cruelles, puis à un état de mort, et paraître devant Dieu tout
vivant de la vie du monde, sans avoir eu un moment à penser qu'il
l'allait quitter et paraître devant son juge voilà le monde, son
tourbillon, ses faveurs, sa tromperie et sa fin\,!

Fréjus tout appliqué au futur, mais au futur de ce monde, ne songeait
qu'à s'attacher le roi et y faisait les plus grands progrès et les plus
visibles. Quoique au fond très contraire au régent, il se conduisait à
son égard avec une grande circonspection\,; et en cultivant le parti
opposé, il le faisait avec une grande mesure. Le maréchal de Villeroy en
était le coryphée. Il était l'objet de la plus jalouse attention de
Fréjus\,; il ne voulait pas sa grandeur, qu'il regardait comme ruineuse
à ses projets de s'emparer du roi avec une autorité sans partage\,; il
sentait toute la disproportion et le poids du maréchal d'avec lui, et
personnellement empêtré de tout ce qu'il lui devait d'attachement et de
reconnaissance, parce que personne n'en ignorait les raisons. Il n'était
pas temps de sortir de ces liens, mais il n'avait garde de travailler à
les augmenter, en servant et encourageant contre le gouvernement et la
personne de M. le duc d'Orléans, un parti timide au fond, et mal
organisé pour les exécutions, abattu de celles qu'il avait essuyées,
mais plein de la plus ardente volonté, et qui, pouvant compter sur le
roi par Fréjus, aurait bientôt repris forces et courage, mais dont le
fruit principal serait recueilli par le maréchal de Villeroy, et par sa
place auprès du roi, et parce qu'il était à la tête de ce parti, ce qui
était fort éloigné de l'intérêt et de la volonté de Fréjus, qui
travaillait de loin à se rendre le maître, et qui se serait vu asservi
sous le maréchal, dont il regardait la ruine dans l'esprit du roi comme
essentielle à la grandeur qu'il méditait dès lors pour soi-même.

Ses progrès auprès du roi étaient si visibles qu'ils commençaient à
faire de lui un personnage que chacun voulait ménager de loin. S'il
sentait toute la supériorité d'état que le maréchal de Villeroy avait
sur lui, à plus forte raison sentait-il celle de M. le duc d'Orléans, le
poids de sa naissance, de sa place, de ses talents, de son âge, qui
devaient naturellement perpétuer son autorité encore plus de trente ans
après la fin de sa régence, et qui, ayant ôté le duc du Maine d'auprès
du roi, pouvait quand il voudrait l'en chasser lui-même, sans craindre
d'exciter aucun mouvement dans l'État, comme il y avait eu lieu de
l'appréhender sur M. du Maine, et de renverser par là ses espérances et
ses projets pour toujours. C'est ce qui le contenait à l'égard du régent
dans de si exactes mesures\,; c'est ce qui l'engageait à me cultiver
avec tant de soin et tant d'écorce de confiance, parce que j'étais le
seul dans l'intime confiance du régent que pût fréquenter sur le pied
d'amitié particulière un évêque qui voulait se parer des vertus et d'une
conduite de son état, et en tirer un grand parti dans la suite. C'est
aussi ce qui redoublait son application et son activité pour s'attacher
le roi de plus en plus et parvenir, s'il le pouvait, au point de se
faire un bouclier assuré de l'affection du roi pour lui en cas qu'il
prit envie au régent de le chasser.

Je voyais clairement tout ce manège de cour, et j'en instruisais les
négligences de M. le duc d'Orléans. Il lui importait de ménager le seul
homme pour qui l'amitié et la confiance du roi se déclarait de plus en
plus, et qui intérieurement était plus que détaché du maréchal de
Villeroy. Je le savais par les choses qu'il m'en disait souvent, et je
n'en pouvais douter par mille traits journaliers de bagatelles
intérieures, qui nous revenaient par les valets du dedans, qui étaient à
M. le duc d'Orléans, parce qu'il les traitait fort bien, et qu'outre les
miches\footnote{Les grâces.} qu'il leur élargissait volontiers, ils
sentaient, avec toute la disproportion des personnes, toute la
différence de la hauteur du maréchal de Villeroy avec eux, et de la
douceur, pour ne dire pas la politesse et la facilité qu'ils éprouvaient
dans l'accès de M. le duc d'Orléans. Je conseillai à ce prince de donner
à Fréjus l'archevêché de Reims, pour faire une chose agréable au roi,
pour s'attacher Fréjus par un présent si disproportionné de lui, au
moins pour lui montrer amitié et bonne volonté et le tenir par là hors
de mesure de lui être contraire, sans que cette grandeur lui pût donner
rien de réel qui ajoutât rien à l'amitié et à la confiance du roi, qui,
avec ou sans Reims, était la seule chose qui pût le rendre considérable
présentement et plus encore à mesure que le roi avancerait en âge, et
par son âge deviendrait le maître. Le régent me crut, alla trouver le
roi, et le lui proposa pour que lui-même eût le plaisir de le donner et
de l'apprendre à M. de Fréjus. Il l'envoya quérir sur-le-champ dans son
cabinet, où en présence de M. le duc d'Orléans et du maréchal de
Villeroy, il le lui dit. Fréjus témoigna sa gratitude, sa disproportion
d'un siège si relevé, l'incompatibilité des fonctions épiscopales avec
les siennes auprès du roi, et refusa avec fermeté, appuyant de plus sur
son âge, qui ne lui permettait plus le travail du gouvernement d'un
nouveau diocèse. Le roi parut mortifié. M. le duc d'Orléans insista
qu'on ne prétendait pas que Reims l'éloignât du roi\,; qu'il aurait des
grands vicaires qui lui rendraient compte de tout et gouverneraient par
ses ordres et un évêque in partibus, qu'on pourvoirait d'abbayes, qui
ferait sur les lieux les ordinations et les autres fonctions réservées
aux évêques\,; que plusieurs prélats avaient des évêques in partibus
pour faire ces fonctions pour eux dans leurs diocèses\,; que cela était
en usage de tout temps pour ceux qui croyaient en avoir besoin\,;
qu'entre ces besoins, il n'y en avait pas un plus légitime que ses
fonctions auprès du roi, et qu'il n'en devait faire aucune difficulté.
Fréjus se confondit en remerciements\,; mais toujours ferme au refus,
répondit qu'il était plus court et plus dans l'ordre de ne point
acquérir de pareils besoins que de s'en servir, et qu'il ne se tiendrait
point en sûreté de conscience d'accepter un évêché dans l'intention de
le laisser gouverner par d'autres, et de n'y point faire de résidence.
Le bon prélat n'avait pas pensé, et n'en avait pas usé ainsi pour
Fréjus, où il ne résida comme point, et n'osant être à Paris, courait
sans cesse le Languedoc et la Provence. Quoi que le roi, le régent et le
maréchal de Villeroy pussent dire et faire, ils ne purent ébranler
Fréjus, tellement que M. le duc d'Orléans finit ce long débat par lui
dire que le roi ne recevait point son refus\,; qu'il voulait au moins
qu'il y pensât et se consultât à loisir, et qu'il prit pour cela tout le
temps qu'il voudrait.

Au sortir de là, je fus instruit par M. le duc d'Orléans de ce qui
s'était passé, et quoique je n'en fusse pas surpris par quelques mots
qui s'en étaient auparavant jetés entré Fréjus et moi, mais en courant,
parce que tout se fit sur-le-champ, j'en fus très fâché. Je fis sentir à
ce prince combien Fréjus estimait plus le futur que le présent,
puisqu'il n'était pas ébloui d'une telle place ni entraîné par les
instances du roi et par les siennes\,; que cela méritait une grande
réflexion sur les projets de cet évêque à conscience devenue si
délicate, qu'il était clair qu'il ne voulait pas accepter, pour éviter
tout prétexte de quitter le roi de vue et un moyen si facile et si
naturel de l'en séparer, le temps de l'éducation fini, en l'envoyant
dans son diocèse, ce que sans cela la moindre bienséance exigerait de
lui, et l'y retenant après, ce qui le bornerait à cette fortune qu'il
aurait faite et lui ferait perdre terre, moyens et toute espérance de
celle qu'il se préparait par l'amitié et la confiance du roi, et qu'il
ne se pouvait bâtir que par la continuation et l'augmentation de cette
même confiance, qu'il ne se pouvait entretenir que par une présence et
une habitude continuelle, après le temps de l'éducation fini, et qui se
détruirait sans ressource par l'absence\,; enfin que cela même était la
plus forte de toutes les raisons, qui devait presser M. le duc d'Orléans
de ne rien oublier pour forcer Fréjus à l'acceptation, et s'ouvrir par
là, en le comblant et en ravissant le roi, s'ouvrir, dis-je, une porte
légitime et simple d'éloigner du roi cet évêque, sans que ni l'un ni
l'autre s'en pussent plaindre d'abord, et en le tenant dans son diocèse
laisser détruire au temps et à l'absence ce que les soins et l'assiduité
auraient édifié, et que la continuation de la présence aurait pu
achever, et donner trop d'ombrages à Son Altesse Royale, trop faible
peut-être alors contre un homme si adroit qui se trouverait en pleine
possession du roi et sans partage.

Ces raisons frappèrent M. le duc d'Orléans et le résolurent à faire tout
ce qui lui serait possible pour engager Fréjus à daigner être archevêque
de Reims\,; de mon côté je ne m'y oubliai pas\,; j'avais pour cela des
raisons particulières, outre les générales que je viens d'expliquer\,;
je les rapporterai ici naturellement avec la vérité qui fait l'âme de
ces Mémoires. À la conduite et aux progrès de Fréjus que je viens de
représenter, le moins à quoi il pouvait tendre en attendant mieux, si
les conjectures s'en offraient, étaient le chapeau et une place dans le
conseil à la majorité, et quelque prodigieux que cela fût pour un homme
de sa sorte, il avait déjà su se mettre avec le roi de façon que cette
énorme fortune en devenait une suite toute naturelle, à quoi M. le duc
d'Orléans ne pourrait s'opposer, surtout après ce qu'il avait fait de
tout semblable, et bien plus encore pour Dubois, son précepteur, plus
bas encore de naissance que Fréjus, et dont le personnel indigne ne
pouvait se comparer en rien au personnel de Fréjus. La calotte rouge, en
arrivant à ce dernier, s'amalgamait à celle de Dubois.

Je ne désespérais pas que le temps, les incartades, le poids de son
autorité sur la faiblesse de M. le duc d'Orléans, quelques manèges même
auprès du roi majeur qui avait un éloignement pour Dubois\,; que celui
de Fréjus qui enviait, haïssait et méprisait Dubois, le renvoyassent à
Cambrai, soit par le dégoût, peut-être même la jalousie que M. le duc
d'Orléans en pourrait enfin prendre, soit parce que, n'étant plus
régent, il n'oserait soutenir un homme si infime et si reconnu pour tout
ce qu'il était d'ailleurs, contre le dégoût du roi poussé par Fréjus,
qui en enhardirait d'autres, et qui rendraient le cri public plus fort.
Défait ainsi de lui, je ne sortais point d'embarras, Fréjus ayant la
pourpre. Mais il tombait entièrement s'il était archevêque de Reims, et
je pouvais dignement, moi et tout autre duc, me trouver avec lui au
conseil et partout, parce que je cédais non au cardinal mais à la
dignité de son siège qui nous précède tous sans difficulté, ainsi que
les cinq autres sièges dont les évêques sont pairs bien plus anciens que
nous. J'avais déjà gagné que Dubois depuis sa promotion n'entrait plus
au conseil de régence\,; je comptais bien en faire une planche pour le
conseil à la majorité, mais j'en espérais faiblement si Fréjus cardinal,
ou assuré de l'être bientôt, appuyait la pourpre de Dubois en
considération de la sienne, et qu'il ne serait pas facile d'exclure du
conseil pour la difficulté du rang, avec le roi en croupe, au lieu que
toute difficulté cessant par Reims et n'ayant plus affaire qu'à Dubois,
Fréjus hors de cause contribuerait de tout son pouvoir à l'exclure pour
son intérêt particulier. Plein donc de tant de motifs généraux et
particuliers, j'attaquai Fréjus de toutes mes forces pendant plusieurs
jours, et voyant bien à quoi il tenait le plus, qui était de n'avoir
point de diocèse où la bienséance l'obligeât d'aller et de faire de
hasardeuses absences, et qui pis encore pouvait devenir une occasion
toute naturelle de l'y envoyer et de l'y retenir, je lui proposai
d'accepter Reims, de le garder un an ou dix-huit mois, puis de le
remettre, dont il aurait mille bonnes raisons à alléguer l'avoir pris
par n'avoir pu résister au roi et au régent, le rendre après avoir, par
l'acceptation, marqué son respect, sa déférence, son obéissance\,; par
ne pouvoir se résoudre, dans un âge avancé, de se charger du
gouvernement d'un grand diocèse, moins encore de le faire gouverner par
autrui\,; que par cet expédient si simple et si plausible, il évitait
tout ce qui l'empêchait d'accepter, et conservait un rang qui le mettait
à la tête des pairs, et qui, le chapeau lui venant, l'affranchissait de
toutes sortes d'embarras et de difficultés.

J'eus beau étaler tout le bien-dire que je pus, tâcher à l'ébranler, par
la crainte que le refus si opiniâtre d'une place si unique ne persuadât
au régent qu'il ne voulait rien tenir de lui, et les conséquences et les
suites qui en résultaient, tout fut inutile. Il se tint ferme au refus
entier, et me dit dévotement que sa conscience ne lui pouvait permettre
d'accepter Reims, dans le dessein de le rendre, de n'y aller jamais, et
de se revêtir seulement du rang de ce grand siège, qu'il n'aurait
accepté que dans cette vue d'orgueil et de vanité, et non d'y servir
l'Église dans la conduite effective et sérieuse de cette portion du
troupeau, qui était la seule voie canonique dans laquelle on dût marcher
lorsqu'on acceptait un évêché. L'hypocrite me paya de cette monnaie\,;
c'est qu'il voulait demeurer libre à l'égard de M. le duc d'Orléans, et
qu'à l'égard de la préséance il méprisait Reims, parce qu'à la manière
dont il avait vu les ducs se conduire, et être traités dans toute cette
régence, il les regardait comme nuls\,; que tôt ou tard ils seraient
crossés par Dubois, et céderaient à sa pourpre, au pis aller à la sienne
à lui dès que le roi serait le maître, dont M. le duc d'Orléans, quelque
crédit qu'il conservât, lui ferait litière à sou accoutumée. Ce combat
qui dura plus de quinze jours avant que M. le duc d'Orléans, à bout de
voies, eût enfin admis son refus, fit l'entretien de tout le monde. Un
matin que j'en parlais avec regret à M\textsuperscript{me} de
Saint-Simon, comme elle se coiffait, car rien n'était alors si public,
une femme de chambre qui s'appelait Beaulieu, familière parce qu'elle
était à elle depuis notre mariage, et qui avait de l'esprit et du sens,
prit tout d'un coup la parole. «\,Je ne m'en étonne pas, dit-elle, il ne
veut point de Reims, il ne veut qu'être roi de France, et il le sera.\,»
Quoique j'en pensasse bien quelque chose, le propos de cette fille nous
surprit et s'est enfin trouvé une prophétie.

Une résistance si invincible nous fit aisément comprendre que Fréjus ne
voulait rien de la main de M. le duc d'Orléans. Il le sentit comme moi,
quoique Fréjus eût aussi d'autres raisons plus fortes. Je crus qu'il le
fallait pousser à bout là-dessus et lui donner la riche abbaye de
Saint-Étienne de Caen, que la mort du cardinal de Mailly laissait aussi
vacante, et qui n'avait point la raison de refus d'un diocèse à
conduire, ni la bienséance d'y aller, ni la crainte d'y pouvoir être
envoyé et retenu sous le spécieux prétexte du devoir épiscopal. M. le
duc d'Orléans goûta tout aussitôt ce que je lui en représentai et alla
chez le roi, qui comme l'autre fois envoya chercher Fréjus. Le roi lui
annonça l'abbaye, et M. le duc d'Orléans ajouta que, n'y ayant là ni
gouvernement d'âmes ni personne à conduire et point de résidence, il ne
croyait pas qu'il pût ni voulût refuser. Ce n'était pas le compte de
Fréjus, il voulut l'honneur du refus. Quoiqu'il n'eût que très -peu de
bénéfices, il protesta qu'il en avait assez et se fit battre plusieurs
jours, soit qu'en effet il ne voulût rien de M. le duc d'Orléans, bien
sûr qu'après la régence il recevrait du roi tout ce qu'il voudrait, soit
que résolu de ne pas laisser échapper ce gros morceau, il voulût se
faire honneur de cette momerie. Je me mis après lui comme j'avais fait
pour Reims, non dans le même désir, parce qu'il n'y avait plus d'intérêt
général ni particulier à l'égard de cette abbaye, mais pour la curiosité
de ce qu'il en arriverait\,; enfin, après avoir bien fait le béat et le
réservé sur les biens d'Église, il eut la complaisance de se laisser
forcer et même de laisser employer le nom du roi à Rome pour le gratis
entier qu'il obtint aussitôt. Il faut pourtant avouer qu'il ne fut
jamais intéressé. Depuis il a été longtemps à même de toutes choses, il
n'a jamais pris aucun bénéfice, et il n'a pas paru qu'il se fût beaucoup
récompensé d'ailleurs. Aussi dans le plus haut point de la
toute-puissance, avec le cardinalat, son domestique, son équipage, sa
table, ses meubles furent toujours au-dessous même de ceux d'un prélat
médiocre.

Achevons de suite ce qui regarde l'archevêché de Reims. J'étais fort des
amis de Castries, et l'abbé son frère, l'un chevalier d'honneur de
M\textsuperscript{me} la duchesse d'Orléans, l'autre qui avait été
premier aumônier de M\textsuperscript{me} la duchesse de Berry, que
j'avais fait mettre dans le conseil de conscience, qui avait été sacré
archevêque de Tours, par le cardinal de Noailles, et qui, sans y être
allé, passa tout aussitôt à Albi, comme l'abbé d'Auvergne, qui eut Tours
après lui, passa incontinent après à Vienne. Les Castries avec raison
désiraient passionnément Reims. Outre le rang et la décoration,
l'extrême éloignement d'Albi et la proximité de Reims était un grand
motif pour deux frères toujours infiniment unis, qui avaient passé toute
leur vie ensemble, et qui se voyaient séparés dans un temps où l'âge et
les infirmités de l'aîné et sa solitude domestique, ayant perdu sa femme
et son fils unique, lui rendaient la présence de son frère plus
nécessaire. Fréjus dès lors avait saisi assez de part dans la
distribution des grands bénéfices.

La constitution, la faiblesse, l'incurie de M. le duc d'Orléans, lui en
avaient frayé le chemin, de sorte que pour Reims il fallut compter avec
l'un et l'autre. On a vu ici ailleurs, par occasions, qui était Fréjus,
et qu'il devait tout au cardinal Bonzi, qui était frère de la mère des
Castries, et qui les avait toujours aimés et traités comme ses enfants.
Fréjus en avait été témoin, leur avait fait sa cour, en avait été
recueilli, en avait reçu des services importants et qui l'avaient sauvé
de sa perte. Il avait passé sa vie avec eux, souvent logé et défrayé
chez eux, dans une intimité parfaite avec mêmes amis et même société à
la cour. Il était donc bien naturel qu'il les servît en chose pour eux
de tous points si désirable. Je me chargeai de M. le duc d'Orléans, ils
furent surpris de trouver en cette occasion leur ami un ministre
prématuré qui se montra fort peu porté à les servir. J'y trouvai aussi
M. le duc d'Orléans fort peu disposé. Il n'y avait rien à dire sur la
conduite des Castries\,; d'ailleurs le régent n'y était ni difficile ni
scrupuleux. Il m'alla chercher des difficultés sur la naissance, pour
une place telle que Reims, et la proximité encore du sacre du roi. J'y
répondis par le collier de l'ordre de leur père, par sa charge de
lieutenant général de Languedoc et de gouverneur de Montpellier, par
l'alliance de Mortemart. Le débat fut souvent réitéré, et je dis à M. le
duc d'Orléans que je m'étonnais fort qu'il fût plus délicat que moi pour
Reims, lui qui l'était si peu pour ces sortes de choix\,; et je tâchai
de lui faire honte de tant faire le difficile pour le frère d'un homme
en charge principale chez M\textsuperscript{me} la duchesse d'Orléans
depuis si longtemps, dont il avait toujours été content, qui avait
épousé sa cousine germaine, si longtemps et morte sa dame d'atours et
cousine germaine, fille du frère de M\textsuperscript{me} de Montespan,
dont avec tant de raison elle se faisait honneur. J'en dis tant que je
vainquis la répugnance de M. le duc d'Orléans, qui me dit qu'il fallait
gagner Fréjus, qui y était fort opposé. Je tâchai de lui faire honte de
prendre une telle dépendance, et lui demandai s'il voulait morceler sa
régence et en abandonner une portion aussi considérable, aussi agréable,
aussi importante que l'est la nomination des bénéfices. Peu à peu, je
vins encore à bout de cette difficulté à toute reste, mais en me
recommandant toujours de tacher de gagner Fréjus. Ce prélat, qui devait
par ce qui a été dit être le grand arc-boutant des Castries en cette
occasion, se montra si contraire que ni les Castries, ni moi qui lui en
parlai souvent et fortement, n'en pûmes jamais tirer une seule bonne
parole, tellement que je me résolus à l'emporter de force, et malgré
lui, de M. le duc d'Orléans\,; je mis l'affaire au point où je la
pouvais désirer.

Mais mon départ s'approchait, et les Castries, que j'avertissais à
mesure que j'avançais, me dirent que sans mon départ ils tiendraient la
chose faite, mais que ce départ la ferait manquer. Elle se fût faite en
effet au point où je la laissai, si j'avais pu demeurer davantage, et
avoir le loisir d'achever de forcer M. le duc d'Orléans. Mais il fallut
partir et laisser le champ libre à Fréjus, qui dans sa rage de
constitution, écartait Albi, ami du cardinal de Noailles, et voulait
s'attacher le cardinal de Rohan, pour le chapeau, auquel il pensait déjà
beaucoup, et qui était à Rome, et au cardinal Dubois, à qui les
Castries, droits et fort honnêtes gens, n'avaient point fait leur cour,
lequel, pour entretenir les Rohan dans l'erreur de faire premier
ministre le cardinal de Rohan à son retour de Rome, voulait, de concert
avec Fréjus, mettre l'abbé de Guéméné à Reims, comme ils firent bientôt
après que je fus parti.

Poursuivons le peu qui reste à dire de cette année pour ne point
interrompre ce qui regarde mon ambassade. Il a été quelquefois mention
ici du duc de Brancas, et de la façon dont il était avec M. le duc
d'Orléans, qui s'amusait fort de ses saillies, et qui l'avait presque
toujours à ses soupers.

C'était un homme d'une imagination vive, singulière, plaisante, plein de
traits auxquels on ne pouvait s'attendre, qui avait sacrifié sa fortune
à ses plaisirs, et à une vie obscure, pauvre d'ailleurs, et fort
intéressé, tout à fait incapable de rien de sérieux, en quoi il se
faisait justice lui-même, et n'était pas sans esprit. Au travers de ses
débauches, il avait eu de fois à autres de faibles retours qui n'avaient
eu aucune suite. Enfin Dieu le toucha. Il s'adressa fort secrètement au
P. de La Tour, général de l'Oratoire, grand et sage directeur, dont il a
été parlé ici quelquefois, qui jugea qu'il avait besoin d'une forte
pénitence et d'une entière séparation du monde. Il l'y résolut, et se
chargea de lui choisir et de lui préparer une retraite. Pendant tout le
temps de ce commerce secret, le duc de Brancas avait quitté ses
débauches, mais conservé tout l'extérieur de sa vie, et soupait tous les
soirs avec M. le duc d'Orléans et ses roués, avec sa gaieté ordinaire.
Au commencement d'octobre, il disparut tout d'un coup, ayant soupé la
veille avec M. le duc d'Orléans, sans qu'il eût paru en lui aucun
changement\,; et on sut quelques jours après qu'il était allé se retirer
dans l'abbaye du Bec en Normandie, où sont les bénédictins de la
congrégation de Saint-Maur. M. le duc d'Orléans, également surpris et
fâché de sa retraite, espéra en sa légèreté, et lui écrivit une lettre
tendre et pressante pour le faire revenir. Le duc de Brancas lui fit une
réponse d'abord plaisante, puis sérieuse, sage et ferme, édifiante et
belle, qui ôta toute espérance de retour. Il y passa fort saintement
plusieurs années\,; plût à Dieu qu'il eût persévéré jusqu'à la fin\,!

Il y eut plusieurs morts\,: l'abbé de Camps, qui fit une fortune
singulière, et qui fut quelque peu de temps une sorte de personnage. Il
était d'Amiens, fils d'un quincaillier et cabaretier, fut amené à Paris
fort jeune, et mis à servir les messes aux jacobins du faubourg
Saint-Germain. Le P. Serroni du même ordre, qui avait gagné l'évêché
d'Orange à être le conducteur du P. Mazarin, archevêque d'Aix, cardinal
et frère fort imbécile du fameux cardinal Mazarin, se trouva à Paris
logé dans ce couvent. Devenu évêque de Mende, il prit ce petit garçon,
qui lui avait plu, le tint quelque temps clerc chez un notaire, en fit
après un sous-secrétaire, et enfin son secrétaire. Il s'en servit en
beaucoup d'affaires avec succès. Il lui donna et lui fit donner des
bénéfices, le fit députer à une assemblée du clergé où il montra
beaucoup d'esprit et de capacité. Serroni, toujours en crédit et en
considération, et pour lequel Albi qu'on lui avait donné fut érigé en
archevêché, le fit coadjuteur de Glandèves, et bientôt après nommer à
l'évêché de Pamiers. C'était au temps de l'affaire de la régale en
faveur de la quelle de Camps écrivit fortement, et s'y intrigua
tellement que, lorsque cette affaire fut terminée, Rome ne put jamais se
résoudre à lui donner les bulles de Pamiers, et que le roi eut la
complaisance de retirer sa nomination, et d'en faire une autre. Il l'en
dédommagea par l'abbaye de Signi, en Champagne, de plus de quarante
mille livres de rente, outre les bénéfices qu'il avait. Il s'acquit une
grande connaissance des médailles et de l'histoire, et a beaucoup écrit
sur celle de France qu'il a fort éclaircie\footnote{Les manuscrits de
  l'abbé de Camps sont conservés à la Bibl. Imp.}. Il ne fut pas content
avec raison de celle que le P. Daniel, jésuite, publia vers la fin du
dernier règne, et de laquelle j'ai parlé ici en son temps. Le P. Daniel
le trouva mauvais\,; ils écrivirent l'un contre l'autre, et l'auteur
mercenaire et menteur fut battu par l'abbé qui aimait la vérité. Il
savait en effet beaucoup, avec de l'esprit et du jugement, de la
vivacité et quelquefois de l'âcreté. Il passa sa longue vie de
quatre-vingt-deux ans à Paris, la plupart du temps dans sa belle
bibliothèque, à travailler et à étudier\,; voyait bonne compagnie, force
savants aussi, et se faisait honneur de son bien, mais avec mesure et
sagesse, estimé et considéré, bien reçu partout. Il allait assez souvent
faire sa cour au feu roi, et il n'y allait presque jamais sans que le
roi lui parlât, et lui témoignât bienveillance. Il passa toute sa vie
jusqu'au bout dans une santé parfaite de corps et d'esprit.

L'évêque-duc de Laon dans son diocèse, médiocrement vieux\,: il était
Clermont-Chattes, fort du monde, et toutefois bon évêque, assez résidant
et appliqué au gouvernement de son diocèse. Il était frère du chevalier
de Clermont, perdu pour l'affaire de M\textsuperscript{me} la princesse
de Conti et de M\textsuperscript{lle} Choin, dont il {[}a{]} été ici
amplement parlé en son temps, et qui, après un long exil en Dauphiné,
obtint de l'être à Laon, d'où M. le duc d'Orléans le tira à la mort du
roi, et lui donna depuis ses Cent-Suisses. C'était un très honnête homme
et galant homme. Il a été suffisamment parlé de cet évêque de Laon, en
différents endroits. Il s'était dignement et sagement signalé au
commencement de l'affaire de la constitution\,; mais le pauvre homme
n'eut pas le courage d'essuyer la pauvreté dont il fut menacé.
D'ailleurs bon homme et honnête homme, et fort estimé jusqu'à cette
chute, lui-même en fut si honteux qu'il ne reparut presque plus depuis,
et demeura presque toujours dans son diocèse, où il fut fort regretté.
Il eut pour successeur l'opprobre non seulement de l'épiscopat, mais de
la nature humaine, et pleinement connu pour tel quand il fut nommé. Il
continua et augmenta dans l'épiscopat les horreurs de sa vie, qui,
quoique assez courte, ne fut que trop longue. Je n'en dirai pas
davantage sur un si infâme sujet. Toutefois il faut observer qu'il ne
fut pas successeur immédiat. Il avait acheté à deniers comptants un
autre évêché d'un évêque qui se démit, et il passa tôt après à Laon, que
M. le duc d'Orléans avait donné, après M. de Clermont-Chattes, à un
bâtard fort bien fait, et qui en a fait depuis grand usage, qu'il avait
eu de la comédienne Florence, et qu'il n'a jamais reconnu, que les
jésuites élevèrent et gouvernèrent, et n'en firent pourtant qu'un
parfait ignorant. Il fit au sacre les fonctions de son siège\,; mais
quand il voulut se faire recevoir au parlement, il fut arrêté tout court
sur ce qu'il n'avait point de nom, et ne pouvait montrer ni père ni
mère. Cet embarras le fit passer à l'archevêché de Cambrai, à la mort du
cardinal Dubois, avec un brevet de continuation de rang et d'honneurs
d'évêque-duc de Laon, et ce monstre dont je viens de parler lui succéda
à Laon.

Trois jours après M. de Laon, Clermont-Chattes, mourut à Gaillon
l'archevêque de Rouen, frère du maréchal de Besons, qui avait été évêque
d'Aire, puis archevêque de Bordeaux et adoré dans tous ses diocèses\,:
il a été souvent parlé de lui ici en plusieurs occasions. C'était
l'homme du clergé qui en savait mieux les affaires, et il entendait très
bien à en manier d'autres. Sous une écorce rustre il n'en avait rien\,;
il était doux, poli, respectueux, point enflé de sa fortune, de son
esprit, de sa capacité, et il en avait beaucoup\,; bon, doux, obligeant,
sage et gai, de fort bonne compagnie, mesuré partout, bon évêque, et
entendant mieux qu'aucun le gouvernement d'un diocèse. Il fut toujours
estimé et considéré, aussi ne voulait-il déplaire à personne, et son
défaut était un peu de patelinage et grand'peur de se mettre mal avec
les gens en place et de crédit. M. le duc d'Orléans, qui aimait les deux
frères, dont l'union était intime, l'avait fait passer dans le conseil
de régence, comme on a vu à la chute de celui de conscience dont il
était. Son âge n'était pas extrêmement avancé, Tressan, évêque de
Nantes, qui avait sacré Dubois, fut son successeur.

Le maréchal de Berwick perdit en même temps son fils, le duc de
Fitz-James, à dix-neuf ans, qu'il avait marié à la fille aînée du duc de
Duras. Elle n'en eut point d'enfants et se remaria depuis au duc
d'Aumont.

M\textsuperscript{lle} de La Rochefoucauld à quatre-vingt-quatre ans\,:
elle était soeur du duc de La Rochefoucauld, qui toute sa vie avait eu
tant de part à la faveur du feu roi. Elle avait passé toute sa vie fille
dans l'hôtel de La Rochefoucauld, fut, considérée dans le monde et dans
sa famille, toujours très vertueuse et très peu de bien. Du côté de
l'esprit, elle tenait tout de son père.

La vicomtesse de Polignac, qui était soeur du feu comte du Roure. Son
mari et son frère étaient chevaliers de l'ordre, et elle était mère du
cardinal de Polignac\,: c'était une grande femme, qui avait été belle et
bien faite, sentant fort sa grande dame qu'elle était fort dans le grand
monde dans son temps. Beaucoup d'esprit, encore plus d'intrigue, fort
mêlée avec la comtesse de Soissons et M\textsuperscript{me} de Bouillon
dans l'affaire de la Voysin, dont elle eut grand'peine à se tirer, et en
fut exilée au Puy et en Languedoc, d'où elle ne revint qu'après la mort
du roi. Elle avait quatre-vingts ans.

Prior mourut en même temps à Londres, en disgrâce et en obscurité, après
avoir échappé pis\,; si connu pour avoir apporté à Paris les
préliminaires de la paix d'Utrecht, longtemps chargé des affaires
d'Angleterre à Paris, et dans l'intime secret des ministres qui
gouvernaient sous la reine Anne, qui furent recherchés après sa mort
avec tant de fureur, et que Prior, arrêté et menacé des supplices,
trahit complètement pour se sauver\,; il ne mena depuis qu'une vie
misérable, obscure, méprisée de tous les partis. C'était un homme
extrêmement capable, savant d'ailleurs, d'infiniment d'esprit, de bonne
chère et de fort bonne compagnie.

\hypertarget{chapitre-xi.}{%
\chapter{CHAPITRE XI.}\label{chapitre-xi.}}

1721

~

{\textsc{Raisons qui terminent les longs troubles du Nord.}} {\textsc{-
Paix de Nystadt entre la Russie et la Suède.}} {\textsc{- Réflexions.}}
{\textsc{- Mesures pour apprendre au roi son mariage et le déclarer.}}
{\textsc{- Le régent, en cinquième seulement dans le cabinet du roi, lui
apprend son mariage, et le déclare en sa présence au conseil de
régence.}} {\textsc{- Détail plus étendu de la scène du cabinet du roi
sur son mariage.}} {\textsc{- Déclaration du mariage du prince des
Asturies avec une fille de M. le duc d'Orléans.}} {\textsc{-
Réflexions.}} {\textsc{- Abattement et rage de la cabale opposée au
régent.}} {\textsc{- Ses discours\,; son projet.}} {\textsc{- Frauduleux
procédé du cardinal Dubois avec moi, qui veut me ruiner et me faire
échouer.}} {\textsc{- Mon ambassade déclarée.}} {\textsc{- Ma suite
principale.}} {\textsc{- Sartine\,; quel.}} {\textsc{- Je consulte
utilement Amelot et les ducs de Berwick et de Saint-Aignan.}} {\textsc{-
Utilité que je tire des ducs de Liria et de Veragua.}} {\textsc{- Leur
caractère.}} {\textsc{- Mon instruction.}} {\textsc{- Remarques sur
icelle.}} {\textsc{- Valouse\,; son caractère et sa fortune.}}
{\textsc{- La Roche\,; sa fortune\,; son caractère.}} {\textsc{-
Estampille\,; ce que c'est.}} {\textsc{- Laullez\,; sa fortune\,; son
caractère.}} {\textsc{- Mon utile liaison avec lui.}} {\textsc{-
Scélératesse du cardinal Dubois et faiblesse inconcevable de M. le duc
d'Orléans, dans les ordres nouveaux et verbaux que j'en reçois sur
préséance et visites.}} {\textsc{- Duc d'Ossone\,; quel.}} {\textsc{-
Nommé ambassadeur d'Espagne pour le mariage du prince des Asturies.}}
{\textsc{- On lui destine le cordon bleu.}} {\textsc{- Je ne veux point
profiter de la nouveauté de cet exemple.}} {\textsc{- Continuation de
l'étrange procédé du cardinal Dubois à mon égard, qui fait hasarder à M.
le duc d'Orléans une entreprise d'égalité avec le prince des Asturies.}}
{\textsc{- La Fare envoyé en Espagne de la part de M. le duc
d'Orléans.}} {\textsc{- Son caractère.}} {\textsc{- Malice grossière à
mon égard du cardinal Dubois, suivie de la plus étrange impudence.}}
{\textsc{- Il prend à Torcy la charge des postes.}} {\textsc{- Bon
traitement fait à Torcy.}} {\textsc{- La duchesse de Ventadour, et
M\textsuperscript{me} de Soubise en survivance, gouvernantes de
l'infante, et le prince de Rohan chargé de l'échange des princesses.}}

~

Il y avait longtemps que les alliés du nord, las de cette longue guerre,
et jaloux respectivement, se démanchaient les uns après les autres\,; et
chacun, dans la crainte de l'augmentation de la puissance déjà trop
formidable de la Russie prête d'envahir la Suède, s'était contenté de ce
qu'il en avait pu tirer, et avait cessé la diversion. Le czar avait des
raisons domestiques de finir cette guerre\,; et s'y portait d'autant
plus volontiers qu'il la pouvait terminer à son mot et donner la loi à
la Suède. Les plénipotentiaires russiens et suédois, assemblés à Nystadt
en Finlande, y conclurent la paix telle que la Suède la put obtenir dans
l'état de ruine et de dernier abattement où le règne de son dernier roi
l'avait mise, et que la continuation de la guerre contre tant d'ennemis
acharnés à profiter de ses dépouilles avait consommé. C'est cette paix
qui a si tristement mis la Suède dans l'état stable où elle est demeurée
depuis, et duquel il n'y a pas d'apparence qu'elle se puisse relever
sans des révolutions qu'on ne saurait attendre. C'est aussi ce qui
m'engage à la donner ici. La mort de Charles XII avait rendu l'autorité
première aux états et au sénat, et la couronne élective, et totalement
énervé l'autorité de leurs rois, dont les deux derniers avaient fait un
si funeste usage, et réglé le dedans de manière à ne plus retomber dans
ces malheurs. Voici comment la paix de Nystadt en régla le dehors déjà
si affaibli par la perte des duchés de Brême et de Verden, envahi sans
retour par la maison de Hanovre, et par le peu que le Danemark et le
Brandebourg en avaient su tirer. Je ne parlerai ici que des articles
principaux de cette paix entre la Russie et la Suède, qui termina
entièrement cette longue et cruelle guerre du nord.

La Suède céda à la Russie la Livonie, l'Estonie, l'Ingrie\footnote{L'Ingrie
  est maintenant comprise dans la province de Saint-Pétersbourg.}, une
partie de la Carélie et le district de Wiborg, les îles d'Oesel, Dagoë,
de Moen, et quelques autres. Le czar rendit la Finlande, excepté une
petite partie fixée et dénommée, et s'obligea de payer à la Suède dans
les termes convenus deux millions de rixdales\footnote{Le mot rixdale ou
  risdale vient de Reichsthaler (thaler ou écu de l'empire). C'est une
  monnaie d'argent, dont la valeur se rapproche de notre pièce de 5 fr.
  Elle vaut maintenant, en Suède, 5 fr. 75 c.}, d'évacuer la Finlande un
mois après l'échange des ratifications, de permettre aux Suédois
d'acheter tous les ans pour cinquante mille roubles de grains dans les
ports de Riga, Revel et Wiborg, excepté dans les années de disette, ou
lorsqu'il y aura des raisons importantes d'empêcher le transport des
grains, et de ne payer aucun droit de sortie de ces grains\,; le renvoi
de part et d'autre des prisonniers sans rançon, mais qui seront tenus de
payer les dettes qu'ils auront faites\,; que les habitants de la
Livonie, de l'Estonie et de l'île d'Oesel jouiront de tous les
privilèges qu'ils avaient sous la Suède\,; que l'exercice de la religion
y sera libre, mais que la grecque y sera tolérée\,; que les fonds de
terre y demeureront à ceux qui en prouveront la possession légitime\,;
que les biens confisqués pendant la guerre seront rendus à leurs
propriétaires, mais sans restitution de fruits et de revenus\,; que les
gentilshommes et autres habitants dés provinces cédées pourront prêter
serment de fidélité au czar sans que cela les empêche de servir
ailleurs\,; que ceux qui refuseront de le prêter auront trois ans pour
vendre leurs biens en remboursant les hypothèques dont ils se trouveront
chargés\,; que les contributions de la Finlande cesseront du jour de la
signature du traité, mais que la province fournira des vivres aux
troupes du czar jusqu'à ce qu'elles soient sur la frontière, et les
chevaux nécessaires pour emmener tout le canon\,; que les prisonniers
seront libres de demeurer au service du prince dans les États duquel ils
seront détenus. Le czar promet de ne se mêler en aucune manière des
affaires domestiques de la Suède (cet article déroge formellement au
précédent traité d'Abo, où le czar se fit garant qu'il ne pourrait être
rien changé en Suède à ce qui y fut établi pour la forme du gouvernement
après la mort de Charles XII)\,; que dans le règlement des différends
qui pourraient arriver dans la suite, il ne sera dérogé en rien au
présent traité\,; enfin, que les ambassadeurs de part et d'autre et les
autres ministres sous quelque nom que ce soit, ne seront plus défrayés
comme ils l'étaient auparavant dans la cour où ils résideront. Le roi de
Pologne fut compris dans le traité, et le czar engagé de procurer aux
Suédois d'être traités en Pologne pour le commerce comme la nation la
plus favorisée\,; liberté au czar et au roi de Suède de nommer dans
trois mois après les ratifications ceux qu'ils voudront comprendre dans
cette paix.

On voit aisément que cette paix si démesurément avantageuse à la Russie
fut la loi du vainqueur au vaincu, et que, outre tant d'États vastes et
riches dont la Suède se dépouillait pour obtenir cette paix, elle
demeurait encore ouverte et à découvert en bien des endroits. De plus
rien de plus clair et de plus nettement exprimé que toutes les cessions
de la Suède, rien de moins que les détails qui lui sont favorables, et
sur lesquels elle essuya bien des chicanes et des injustices, et ses
sujets, dans l'exécution. Aussi le czar dans l'excès de sa joie
voulut-il des fêtes et des réjouissances publiques dans toute la Russie,
et il en fit lui-même d'extraordinaires. Pour la Suède si près de sa
dernière ruine, elle se crut heureuse encore de s'en rédimer par de si
immenses pertes, qui, en la jetant dans le dernier affaiblissement et la
dernière pauvreté, lui ôtaient toute considération effective dans
l'Europe, reléguée qu'elle demeurait au delà de la mer Baltique, après
avoir vu ses rois, même un moment le dernier, en être les dictateurs, et
si puissants en Allemagne. Que de choses politiques à dire et à prévoir
là-dessus qui ne sont pas matière de ces Mémoires\,; mais le funeste
fruit de l'intérêt personnel de Dubois qui avait enchaîné la France à
l'Angleterre, et qui malgré tout ce que je pus représenter bien des fois
au régent, et que le régent sentit lui-même, ne voulut jamais lui
permettre {[}de profiter{]} du désir passionné que le czar eut de s'unir
étroitement avec la France, et que l'avarice et les ténèbres du cardinal
Fleury achevèrent de livrer la Russie à l'empereur et à l'Angleterre.

Il est enfin temps de venir à ce qui regarde mon ambassade, pour la
continuer de suite, comme je me le suis proposé en racontant, comme je
viens de faire, plusieurs choses postérieures à ce qui s'est passé
là-dessus entre M. le duc d'Orléans, le cardinal Dubois et moi, et à la
déclaration des mariages. Je commencerai par celle-ci, pour n'en pas
interrompre ce qui me regarde en particulier jusqu'à mon départ. Il
commençait à être temps de déclarer le mariage du roi, et M. le duc
d'Orléans ne laissait pas d'être en peine comment il serait reçu de ce
prince, que les surprises effarouchaient, et du public, à cause de l'âge
de l'infante encore dans la première enfance. Le régent résolut enfin de
prendre un jour de conseil de régence, et le moment avant de le tenir,
pour apprendre au roi son mariage et le déclarer sans intervalle au
conseil de régence, pour que tout de suite ce fût une affaire passée et
consommée.

Il arriva par hasard que ce même conseil de régence, où la déclaration
du mariage ne se pouvait plus différer par rapport à l'Espagne, se
trouvait destiné à une proposition d'affaire de papier que j'avais fort
combattue dans le cabinet de M. le duc d'Orléans, avec lequel j'étais
enfin convenu que je m'abstiendrais ce jour-là du conseil, comme on a vu
ici que cela arrivait quelquefois. Mais les lettres d'Espagne, qui
arrivèrent entre cette convention et la tenue du conseil, ayant obligé
M. le duc d'Orléans à y déclarer le mariage, et l'affaire du papier ne
se pouvant différer, il voulut que je me trouvasse au conseil. Je m'en
défendis, mais il craignait quelque mouvement de ceux du conseil qu'on
appelait de la vieille cour, qui était la cabale opposée à M. le duc
d'Orléans, et ce fut cette raison qui l'empêcha de déclarer les deux
mariages en même temps. Nous disputâmes donc tous deux sur la manière
dont j'opinerais sur l'affaire du papier, et après avoir bien tourné et
retourné, et cédé à la volonté absolue de M. le duc d'Orléans, qui
voulut que j'y assistasse à cause de la déclaration du mariage du roi,
je compris que, quoi que j'y pusse dire contre l'affaire du papier, elle
n'en passerait pas moins, et que, dans la nécessité où je me trouvais de
ne m'absenter pas de ce conseil et d'y opiner, je pouvais, pour cette
fois, m'abstenir de m'étendre et de disputer, et me contenter d'opiner
contre brièvement. M. le duc d'Orléans s'en contenta, mais je le
suppliai de se persuader que je me rendais à cette complaisance que pour
cette seule fois, à cause de la déclaration du mariage du roi, où il
exigeait si absolument que je me trouvasse, dans ce conseil, et de
continuer à trouver bon ou que je m'opposasse de toutes mes raisons aux
choses qu'il y voudrait faire passer dont je ne croirais pas en honneur
et en conscience pouvoir être d'avis, ou de m'ordonner de m'abstenir du
conseil où il les voudrait proposer, comme il lui était arrivé plusieurs
fois de me le défendre, à quoi j'avais obéi sans qu'on se fût aperçu de
la vraie raison de mon absence, comme je le ferais toujours quand le cas
en arriverait. Cette convention entre lui et moi fut donc renouvelée de
la sorte, et je me trouvai à cet important conseil duquel je craignis
moins que lui, sans toutefois que je le pusse bien rassurer.

L'embarras, à mon avis, fut plus grand du côté du roi, qui comme je l'ai
dit s'effarouchait des surprises. Quelque coup d'oeil ou quelque geste
du maréchal de Villeroy pouvait le jeter dans le trouble, et ce trouble
l'empêcher de dire un seul mot. Il fallait pourtant un oui et un
consentement exprimé de sa part, et s'il s'opiniâtrait à se taire, que
devenir pour le conseil de régence\,? Et si par dépit d'être pressé il
allait dire non, que faire et par où en sortir\,? Cet embarras possible
nous tint M. le duc d'Orléans, le cardinal Dubois et moi, en
consultations redoublées. Enfin il fut conclu que, dans la fin de la
matinée du jour du conseil de régence, qui ne serait tenu que
l'après-dînée, M. le duc d'Orléans manderait séparément M. le Duc et M.
de Fréjus M. le Duc, dont il n'y avait rien à craindre, et à qui ce
secret ne pouvait être, à ce qu'il était, caché plus longtemps, qui même
pouvait se blesser d'une si tardive confidence\,; Fréjus pour le
caresser par cette distinction sur le maréchal de Villeroy, l'avoir
présent lorsque M. le duc d'Orléans apprendrait au roi son mariage, et
qu'il fût là tout prêt à servir le régent de tout ce qu'il pouvait sur
le roi. M. le Duc fut surpris, mais ne se fâcha point, et fit très bien
auprès du roi. Fréjus fut froid, il parut sentir que le besoin lui
valait la confidence, loua l'alliance, par manière d'acquit, que M. le
Duc avait fort approuvée, trouva l'infante bien enfant, ce qui n'avait
fait aucune difficulté à M. le Duc, dit néanmoins qu'il ne croyait pas
que le roi résistât, ni qu'il en fût ni aise ni fâché, promit de se
trouver auprès de lui quand la nouvelle lui serait apprise, et fut
modeste sur le reste. Le secret sans réserve, et nommément pour le
maréchal de Villeroy, leur fut fort recommandé à tous deux. Je doute par
ce qu'on va voir que Fréjus y ait été fidèle, et qu'il n'en ait pas fait
sur-le-champ sa cour au maréchal, qu'il avait soigneusement l'air de
cultiver en choses qui n'intéressaient point ses vues.

Le moment venu nous arrivâmes tous aux Tuileries, où M. le duc
d'Orléans, qui, pour laisser assembler tout le monde, était arrivé le
dernier, me conta dans un coin avant d'entrer chez le roi ce qui s'était
passé quelques heures auparavant entre lui, M. le Duc et Fréjus, l'un
après l'autre. Il pirouetta un peu dans le cabinet du conseil, en homme
qui n'est pas bien brave et qui va monter à l'assaut. Je ne le perdais
point de vue, et à le voir de la sorte, j'étais inquiet\,; enfin il
entra chez le roi, je le suivis\,; il demanda qui était dans le cabinet
avec le roi, et sur ce qu'on ne lui nomma point Fréjus, il l'envoya
chercher. Il s'amusa là comme il put, peu de temps, puis il entra dans
le cabinet où était M. le Duc, qui y était entré en même temps que M. le
duc d'Orléans s'était arrêté dans la chambre, le maréchal de Villeroy et
quelques gens intérieurs, comme sous-gouverneur, etc. Je restai dans la
chambre où je pétillais de la lenteur de Fréjus, qui ne me paraissait
pas de bon augure. Enfin il arriva, l'air empressé comme un homme mandé
et qui a fait attendre. Fort peu après qu'il fut entré dans le cabinet,
j'en vis sortir le peuple, c'est-à-dire qu'il n'y demeura que M. le duc
d'Orléans, le cardinal Dubois, qui était entré dans le cabinet avec lui,
M. le Duc, le maréchal de Villeroy et Fréjus. Alors, me trouvant seul de
ma sorte et du conseil de régence dans cette chambre, et ma curiosité
satisfaite de les savoir aux mains, je rentrai dans le cabinet du
conseil, sans toutefois m'éloigner de la porte par où je venais d'y
rentrer.

Peu après, les maréchaux de Villars, d'Estrées et d'Huxelles, vinrent
l'un après l'autre à moi, surpris de cette conférence secrète qui se
tenait dans le cabinet du roi. Ils me demandèrent si je ne savoir point
ce que c'était. Je leur répondis que j'en étais dans la même surprise
qu'eux et dans la même ignorance. Ils demeurèrent tous trois à causer
avec moi, pendant un bon quart d'heure, ce me semble, car le temps me
parut fort long, et cette longueur me faisait craindre quelque chose de
fort fâcheux et de fort embarrassant. À la fin le maréchal de Villars
dit\,: «\,Entrons là dedans en attendant\,; nous y serons aussi bien
qu'ici\,;» et là-dessus nous entrâmes jusque dans la chambre du roi, où
il n'y avait que de ses gens et les sous-gouverneurs.

Très peu de temps après que nous y fûmes, la porte du cabinet
s'entrouvrit, je ne sais ni pourquoi ni comment, car je causais le dos
tourné à la porte avec le maréchal d'Estrées\,; un peu de bruit me fit
tourner, et je vis le maréchal d'Huxelles qui entrait dans le cabinet. À
l'instant le maréchal de Villars qui était avec lui nous dit\,: «\,Il
entre, pourquoi n'entrerions-nous pas\,?» et nous entrâmes tous trois.
Le dos du roi était vers la porte par où nous entrions\,; M. le duc
d'Orléans en face, plus rouge qu'à son ordinaire\,; M. le Duc auprès de
lui, tous deux la mine allongée\,; le cardinal Dubois et le maréchal de
Villeroy en biais\,; et M. de Fréjus tout près du roi, un peu de côté,
en sorte que je le voyais de profil d'un air qui me parut embarrassé.
Nous demeurâmes comme nous étions entrés derrière le roi, moi tout à
fait derrière. Je m'avançai la tête un instant pour tâcher de le voir de
côté, et je la retirai bien vite, parce que je le vis rouge, et les
yeux, au moins celui que je pus voir, pleins de larmes. Aucun de ce qui
était avant nous ne branla pour notre arrivée ni ne nous parla. Le
cardinal Dubois me parut moins empêtré, quoique fort sérieux, le
maréchal de Villeroy secouant sa perruque tout à son ordinaire, au moins
c'est ce qui me frappa au premier coup d'oeil en entrant. «\,Allons, mon
maître, disait-il, il faut faire la chose de bonne grâce.\,» Fréjus se
baissait et parlait au roi à demi bas, et l'exhortait, ce me sembla,
sans entendre ce qu'il lui disait. Les autres étaient en silence très
morne, et nous derniers entrés fort étonnés du spectacle, moi surtout
qui savais de quoi il s'agissait. À la fin je démêlai que le roi ne
voulait point aller au conseil de régence, et qu'on le pressait
là-dessus, je n'osai jamais faire aucun signe à M. le duc d'Orléans ni
au cardinal Dubois, pour tâcher d'en découvrir davantage. Tout ce manège
dura presque un quart d'heure. Enfin M. de Fréjus ayant encore parlé bas
au roi, il dit à M. le duc d'Orléans que le roi irait au conseil, mais
qu'il lui fallait quelques moments pour le remettre.

Cette parole remit quelque sérénité sur les visages. M. le duc d'Orléans
répondit que rien ne pressait, que tout le monde était fait pour
attendre ses moments\,; puis s'approchant entre le roi et Fréjus, tout
contre, il parla bas au roi, puis dit tout haut\,: «\,Le roi va venir,
je crois que nous ferons bien de le laisser\,;» sortit et nous tous,
tellement qu'il ne demeura avec le roi que M. le Duc, le maréchal de
Villeroy et l'évêque de Fréjus. En chemin pour aller dans le cabinet du
conseil, je m'approchai de M. le duc d'Orléans qui me pris sous le bras
et se jeta dans mon oreille, s'arrêta dans un détroit de porte, et me
dit que le roi, à la mention de son mariage, s'était mis à pleurer,
qu'ils avaient eu toutes les peines du monde, M. le Duc, Fréjus et lui,
d'en tirer un oui, et après cela qu'ils avaient trouvé la même
répugnance à aller au conseil de régence, dont nous avions vu la fin. Il
n'eut pas loisir de m'en dire là davantage, et nous rentrâmes dans le
cabinet du conseil avec lui. Or, il était essentiel que le roi y
déclarât, ou du moins y fût présent à la déclaration de son mariage, qui
était chose si personnelle qu'elle n'y pouvait passer sans lui. Ceux qui
le composaient et qui étaient demeurés dans le cabinet du conseil,
surpris de cette longue et inusitée conférence dans le cabinet du roi,
nous voyant rentrer, s'approchèrent avec curiosité, sans toutefois oser
demander ce que c'était\,; tous avaient l'air occupé. M. le duc
d'Orléans s'amusa comme il put avec les uns et les autres, disant que le
roi allait venir. Les trois maréchaux et moi qui rentrions avec M. le
duc d'Orléans, nous séparâmes sans nous trop mêler avec personne. Cela
fut court. Le roi entra avec M. le Duc et le maréchal de Villeroy, et
tout aussitôt on se mit en place. Le cardinal Dubois, qui n'entrait plus
au conseil de régence depuis qu'il portait la calotte rouge s'en était
allé tout de suite au sortir du cabinet du roi.

Assis tous en place, tous les yeux se portèrent sur le roi, qui avait
les yeux rouges et gros, et avait l'air fort sérieux. Il y eut quelques
moments de silence pendant lesquels M. le duc d'Orléans passa les yeux
sur toute la compagnie qui paraissait en grande expectation\,; puis les
arrêtant sur le roi, il lui demanda s'il trouvait bon qu'il fît part au
conseil de son mariage. Le roi répondit un oui sec, en assez basse note,
mais qui fut entendu des quatre ou cinq plus proches de chaque côté, et
aussitôt M. le duc d'Orléans déclara le mariage et la prochaine venue de
l'infante, ajoutant tout de suite la convenance et l'importance de
l'alliance, et de resserrer par elle l'union si nécessaire des deux
branches royales si proches, après les fâcheuses conjonctures qui les
avaient refroidies. Il fut court, mais nerveux, car il parlait à
merveilles et demanda les avis\,; on peut bien juger quels ils furent.
Presque aucun n'étendit le sien, sinon les maréchaux de Besons et
d'Huxelles un peu\,; l'évêque de Troyes, le maréchal d'Estrées un peu
davantage. Le maréchal de Villeroy n'approuva qu'en deux mots, ajoutant
d'un air chagrin qu'il était bien fâcheux que l'infante fût si jeune. Je
m'étendis plus qu'aucun, mais toutefois sobrement. Le comte de Toulouse
approuva en deux mots de fort bonne grâce, M. le Duc aussi\,; puis M. le
duc d'Orléans parla encore un peu sur l'unanimité des suffrages à
laquelle il s'était bien attendu sur un mariage si convenable, sur quoi
il s'étendit encore un peu. Puis se tournant vers le roi, il s'inclina,
et d'un air souriant, comme pour l'inviter à prendre le même, il lui
dit\,: «\,Voilà donc, sire, votre mariage approuvé et passé, et une
grande et heureuse affaire faite.\,» Puis tout aussitôt, il ordonna le
rapport de l'affaire du papier, qui passa avec un grand air de regret de
toute la compagnie, et dans laquelle j'opinai négativement en deux mots,
comme j'en étais convenu avec M. le duc d'Orléans.

Le conseil levé, chacun se retira sans trop se joindre les uns les
autres. Je démêlai sans peine que le gros approuvait la réunion avec
l'Espagne, mais était peiné de l'enfance de l'infante, qui retardait si
fort l'espérance d'en voir des enfants au delà du temps où le roi
pouvait devenir père, et j'en remarquai d'autres à qui rien n'en
plaisait, tels que les maréchaux de Villeroy, Villars, Huxelles et
sournaisement Tallard.

Je laissai rentrer M. le duc d'Orléans au Palais-Royal, puis j'allai l'y
trouver, curieux de savoir plus en détail ce qu'il n'avait pu me dire
qu'en gros à l'oreille entre ces deux portes. Il ne fit en effet
qu'étendre ce qu'il m'avait dit, parce que tout s'était passé avec peu
de paroles. Il me dit qu'après avoir dit au roi la convention de son
mariage sous son bon plaisir, il ne doutait pas qu'il n'y voulût rien
consentir, et qu'il ne l'approuvât\,; sur quoi voyant ses yeux rougir et
s'humecter en silence, il n'avait pas fait semblant de s'en apercevoir,
et s'était mis à expliquer à la compagnie la nécessité et les avantages
de ce mariage, tels qu'il avait estimé devoir passer par-dessus
l'inconvénient de l'âge de l'infante\,; que le Duc, après ce court
discours, l'avait repris et approuvé fort bien en deux mots\,; que le
cardinal Dubois avait étendu les raisons, et atténué l'inconvénient de
l'âge, par l'avantage d'élever ici l'infante aux manières françaises, et
d'accoutumer ensuite le roi et elle réciproquement, tout cela néanmoins
en assez peu de mots, tandis que les larmes tombaient des yeux du roi
assez dru, et que de fois à autre Fréjus lui parlait bas, sans en tirer
aucune réponse\,; que le maréchal de Villeroy, avec force gestes et
quelques phrases, avait dit qu'on ne pouvait s'empêcher de reconnaître
l'utilité de la réunion des deux branches, ni aussi l'importance que le
roi eût des enfants dès qu'il en pourrait avoir et que, dans une affaire
aussi désirable, il était malheureux qu'il n'y eût point en Espagne de
princesse d'un âge plus avancé\,; que néanmoins il ne doutait point que
le roi n'y donnât son consentement avec joie, et tout de suite lui en
dit quelques paroles d'exhortation. M. le duc d'Orléans reprit là-dessus
la parole sur les avantages et la nécessité incomparablement plus
considérables que l'inconvénient de l'âge, mais en deux mots. Le
cardinal Dubois ne parla plus et ils attendirent en grandes angoisses ce
que l'affaire deviendrait entre les mains de Fréjus, qui était leur
seule espérance. Ce prélat parla peu sur la chose. Il dit en s'adressant
au roi qu'il devait marquer sa confiance aux lumières de M. le duc
d'Orléans, sur un mariage qui le réunissait si heureusement avec le roi
son oncle, comme il la lui donnait sur le gouvernement de son royaume,
puis parlait bas au roi à reprises, et par-ci, par-là quelques paroles
d'exhortation sèches et tout haut du maréchal de Villeroy, jusqu'à ce
que enfin le roi eût prononcé qu'il y consentait. Tout cela s'était
passé avant que les trois maréchaux et moi entrassions dans le cabinet.
On en était alors à exhorter le roi d'aller au conseil de régence, où
aussitôt après qu'il eut donné son consentement, M. le duc d'Orléans lui
avait dit que sa présence était nécessaire pour un consentement public,
et pour que le mariage fût passé au conseil de régence, sur quoi le roi
larmoyait toujours et ne répondait point. Le reste dont nous fûmes
témoins, je l'ai expliqué.

Le cardinal Dubois arriva en tiers comme M. le duc d'Orléans raisonnait
avec moi sur tout ce détail qu'il venait de me raconter, et tous deux
convinrent que, sans l'évêque de Fréjus qui encore s'était fait attendre
et n'avait pas montré agir de trop bon coeur, ils ne savaient ce qui en
serait arrivé. L'angoisse en avait été si forte, qu'ils s'en sentaient
encore tous deux. Aussitôt on dépêcha un courrier en Espagne et un autre
au roi de Sardaigne, grand-père du roi. La nouvelle courut Paris dès que
ceux du conseil de régence en furent sortis\,; les Tuileries et le
Palais-Royal furent bientôt remplis de tout ce qui venait se présenter
devant le roi et faire des compliments au régent de la conclusion de ce
grand mariage, ce qui continua les jours suivants. Le roi eut peine à
reprendre quelque gaieté tout le reste du jour, mais le lendemain il fut
moins sombre, et peu à peu il n'y parut plus.

Rien ne fut plus marqué que le changement subit de cette cabale si
opposée au régent, qui tenait si fortement au duc du Maine et qu'on
appelait de la vieille cour, dont il a été parlé ici tant de fois. Elle
avait été jusqu'alors tout espagnole, et l'avait bien montré dans ses
liaisons avec le prince de Cellamare et dans son union avec lui dans
tous ses projets. L'Espagne, alors dominée par Albéroni, ne respirait
que la chute du régent, et de gouverner la France par un vice-régent
qu'elle nommerait et qui devait être le duc du Maine. Ainsi tant que
l'Espagne fut contraire au régent, cette cabale ne prêchait que
l'Espagne et professait un attachement public pour le roi d'Espagne. Sur
quoi elle eut beau jeu par rapport à l'incroyable ensorcellement
d'Angleterre, dû tout entier à l'intérêt personnel de l'abbé Dubois qui
en devint cardinal, avec une pension d'Angleterre immense. Dès que la
cabale vit le mariage d'Espagne fait par le régent, elle en fut outrée
et ne le put cacher. Ce fut bien pis dix ou douze jours après.

M. le duc d'Orléans, comme on l'a vu, jugea fort prudemment qu'il ne
devait pas déclarer les deux mariages à la fois, et l'expérience qu'il
eut de la déclaration de celui du roi, lui donna sujet de s'applaudir
beaucoup d'avoir pris un conseil si sage. Il crut même avec raison
devoir mettre cet intervalle avant de déclarer le second, pour laisser
raccoiser\footnote{Vieux mot qui signifie calmer, apaiser\,; il vient de
  coi (calme, tranquille).} les humeurs, et refroidir les esprits, mais
il fallait enfin finir cette seconde affaire\,; ainsi dix ou douze jours
après celle qui vient d'être rapportée, il alla chez le roi, après
l'avoir dite à M. le Duc, et à M. de Fréjus. Il les trouva dans le
cabinet du roi, il en fit sortir tous les autres, et entrer le cardinal
Dubois, et là il dit au roi l'honneur que le roi d'Espagne lui voulait
faire, et lui demanda la permission de l'accepter. Cela se passa tout
uniment, sans la moindre difficulté, mais le maréchal de Villeroy ne put
s'empêcher, dans le compliment qu'il fit sur-le-champ à M, le duc
d'Orléans, de témoigner son étonnement, qui sentit fort le dépit. Le
lendemain M. le duc d'Orléans en fit la déclaration au conseil de
régence, le roi présent, qui y assistait presque toujours, où les avis
et les courts compliments de chacun au régent ne furent qu'une même
chose. Les maréchaux de Villeroy, Villars et d'Huxelles y parurent le
visage enflammé, car le mariage de la fille de M. le duc d'Orléans avec
le prince des Asturies fut public dès qu'il eut été annoncé au roi, et
ne purent cacher leur dépit, pour ne pas dire leur désespoir. Le
maréchal de Tallard et quelques autres n'en étaient pas plus contents\,;
mais à travers un embarras qu'ils ne purent cacher, ils se
contraignirent davantage. Le lendemain le roi alla au Palais-Royal, puis
à Saint-Cloud, faire compliment sur ce grand et incroyable mariage à M.
{[}le duc{]} et à M\textsuperscript{me} la duchesse d'Orléans, à
M\textsuperscript{lle} de Montpensier et à Madame, où toute la cour,
tous les ministres étrangers et tout ce qu'il y eut de considérable à
Paris accourut en foule.

Il faut avouer ici qu'il n'y eut rien en soi de si surprenant que le
mariage du prince des Asturies avec une fille de M. le duc d'Orléans,
après tout ce qui s'était passé de personnel entre ce prince et le roi
d'Espagne, tant pendant les dernières années du dernier règne, où il ne
s'était agi de rien moins que de couper la tête à M. le duc d'Orléans,
par les menées de la princesse des Ursins, du duc du Maine, de
M\textsuperscript{me} de Maintenon, de la cabale de Meudon, comme on l'a
vu en son temps\,; de le chasser depuis de la régence et de le perdre
par les intrigues du duc du Maine qui voulait régner en sa place,
d'Albéroni et de l'ambassadeur Cellamare\,; enfin par tout ce qui
s'était passé d'inique contre l'Espagne pour favoriser l'Angleterre même
aux dépens de la France, par un aveuglement forcené pour l'intérêt
unique et personnel de Dubois\,; et que ce même Dubois\footnote{Passage
  omis dans les anciennes éditions depuis par les menées.}, qui devait
être si odieux à l'Espagne, ait osé concevoir le dessein d'y réconcilier
son maître, encore plus odieux, comme en ayant été si cruellement
offensé, et comme en ayant bien su depuis rendre l'offense\,; que
Dubois, dis-je, non seulement en soit venu à bout, mais encore de porter
une fille de M. le duc d'Orléans sur le trône d'Espagne, il faut
convenir que c'est un chef-d'oeuvre de l'audace et d'un bonheur sans
pareil\,! Le détail de la négociation n'est jamais venu à ma
connaissance.

M. le duc d'Orléans était tenu de trop court depuis longtemps par
Dubois, pour m'en faire part, et le secret du traité du double mariage
ne m'aurait jamais été confié quand il fut conclu, sans ce reste
d'amitié, de confiance, d'habitude, qui fut plus fort dans M. le duc
d'Orléans que le poids de Dubois sur sa faiblesse, fatiguée de m'avoir
caché le projet, tant qu'il ne fut pas arrêté et convenu. Je ne puis
donc dire rien de toute cette négociation, dont M. le duc d'Orléans m'a
laissé ignorer le détail après comme devant, et à qui aussi je n'en ai
point fait de question, sinon qu'il me dit que le mariage de sa fille
avait été la condition absolue de celui du roi, et que le roi d'Espagne
était si intimement et si parfaitement François, qu'il n'avait fait de
difficulté à rien moyennant le mariage de sa fille\,; de là je juge que,
s'il y eut de l'effronterie à tenter ce traité, il fut conclu tout de
suite par le bonheur sans pareil de l'inclination de Philippe V, si
passionnément française, qu'elle surnagea à tout pour mettre sa fille
sur le trône de ses pères. Fortuna e dormire, dit l'Italien, ou pour
mieux dire, la Providence qui règle tout et qui produit tout par des
ressorts profondément cachés aux hommes. Car il faut dire que, quoi
qu'il soit arrivé de ces mariages, par la mort de M. le duc d'Orléans
uniquement, il en a bien profité pendant le court reste de sa vie, et
lui et la France bien plus grandement, s'il avait vécu les années
ordinaires des hommes, auquel cas l'infante eût bien sûrement régné en
France.

Si la nouvelle de la déclaration du mariage du roi avait bien étourdi et
affligé la cabale opposée à M. le duc d'Orléans, celle de la déclaration
de celui d'une des princesses ses filles avec le prince des Asturies
l'atterra. Ce fut un accablement si marqué dans toute leur contenance,
qu'il les distinguait aux yeux les moins perçants, et les tint plusieurs
jours dans un morne silence. Aucun de ce qui la composait ne s'était
défié que le roi d'Espagne pût être réconcilié à M. le duc d'Orléans\,;
combien moins qu'il pût être capable d'accepter une de ses filles pour
lui faire porter sa couronne après lui\,! Dans la pleine confiance de
cette impossibilité en effet si parfaitement apparente, ils avaient sans
cesse les yeux et le coeur tournés sur le roi d'Espagne comme étant
également le fils de la maison et le plus irréconciliable ennemi de M.
le duc d'Orléans. Ils n'avaient donc aussi que l'Espagne dans la bouche,
qui était l'ancre de leurs espérances, la protection de leurs
mouvements, le seul moyen de l'accomplissement de leurs désirs, et par
tout ce que Dubois n'avait cessé de faire contre elle en faveur de
l'Angleterre, l'occasion continuelle et sans indécence de fronder et
décrier le régent et son gouvernement qui, d'ailleurs, leur avait donné
beau jeu du côté des finances et de celui de sa vie domestique. Toutes
ces choses si flatteuses qui, malgré le peu de succès de leur malignité,
de leur haine, de leurs efforts, faisaient toutefois encore la
nourriture de leur esprit, de leur volonté, de leurs vues, non seulement
tombaient et disparaissaient par ce double mariage, mais se tournaient
contre eux, et les laissaient, dans le moment même, en proie au vide, à
la nudité, au désespoir, sans nul point d'appui, sans bouclier, sans
ressources. L'horreur qu'ils conçurent aussi d'un revers si subit et si
complètement inattendu, fut plus visible que facile à représenter, et
plus forte qu'eux et que leurs plus politiques. J'avoue que c'était un
plaisir pour moi d'en rencontrer hommes, femmes, gens de tous états. Je
l'ai déjà dit, cette cabale s'était reconcertée depuis le rétablissement
du duc du Maine et les nouvelles entreprises du parlement, depuis le lit
de justice des Tuileries\,; mais ce dernier coup l'écrasa. Néanmoins,
ayant un peu repris ses esprits au bout de quelques jours, elle se mit à
détester l'Espagne et à la même mesure qu'elle s'y était attachée, et ce
contraste fut si subit, si entier, si peu mesuré, qu'il ne fallait que
le voir et l'entendre pour en sentir la cause, même dans ceux dont le
bas aloi avait détourné tous soupçons.

Le premier président et sa cabale des gens du parlement frémissaient
ouvertement, ainsi que beaucoup de gens de cette prétendue noblesse,
dont le duc et la duchesse du Maine s'étaient si heureusement servis par
leurs prestiges, comme on l'a vu ici en son temps, et dont l'imbécile
aveuglement subsistait encore pour eux. Force grands seigneurs, même du
conseil de régence, même des mieux traités d'ailleurs, ne pouvaient
cacher leur contrainte, en sorte que par le subit effet de la nouvelle
de ces mariages, dont ils ne se purent défendre dans le premier
étourdissement, qui fut même assez long, on en découvrit plus qu'on
n'avait fait par les perquisitions estropiées de l'affaire de Cellamare
et du duc et de la `duchesse du plaine, quoique dès lors on en eût plus
trouvé, même parmi les grands et les considérables, qu'on n'aurait
voulu, et qu'on crut devoir étouffer, comme il a été dit dans le temps.
Aux cris contre l'Espagne, ils en joignirent contre M. le duc d'Orléans
qui, disaient-ils, sacrifiait le roi à un enfant sorti à peine du
maillot, pour marier si grandement sa fille, et pour la criminelle
espérance qu'en retardant sa postérité, il pût manquer, avant l'âge de
l'infante, et M. le duc d'Orléans régner sur lui et la sienne en sa
place, après s'être fait un appui de l'Espagne si justement et si
longuement son ennemie personnelle. Ainsi, de rage, ils criaient à
l'habileté pour en donner l'impression la plus sinistre\,; mais la
douleur vive excite les cris. On les méprisa et on ne songea plus qu'à
exécuter promptement tout ce qui pouvait l'être de ce traité de double
mariage, et à jouir et profiter de ses fruits. On eut raison alors,
après l'imprudence d'une déclaration si étrangement précoce et si propre
à rallumer tous les mouvements du dehors et du dedans. On ne sera pas
longtemps sans voir combien il était devenu instant d'achever ce qu'on
avait déclaré. La cabale, tout accablée qu'elle fût pendant les premiers
jours, reprit encore quelque courage, et se mit à travailler à éloigner
les mariages pour se donner le temps de les pouvoir rompre tout à fait.
Ce fut aussi le coup de partie de ne lui en pas laisser le loisir.

J'étais, pendant toutes ces démarches si différentes, aux mains avec le
cardinal Dubois. Il était enragé de mon ambassade, et comme tout me le
montra manifestement dans tout son préparatif et sa durée, il avait
résolu, en gardant tous les dehors, de me ruiner et de me perdre. Je
m'en défiais bien, et j'eus lieu tout aussitôt de n'en point douter. De
lui à moi d'abord, profusions d'amitié, d'attachement, de chose à moi
due que cette ambassade et ses suites pour mes enfants, de tout ce que
M. le duc d'Orléans me devait de reconnaissance et d'amitié, et lui-même
de mes anciennes bontés pour lui de tous les temps. Avec ces propos et
des généralités sur la chose, il évita tant qu'il put d'entrer en
matière pour avoir lieu de tout précipiter et de ne me donner le loisir
de rien discuter avec lui, pour me faire tomber dans tous les panneaux
qu'il me tendrait, et d'ailleurs dans tous les inconvénients possibles.
Ce fut une anguille qui glissa sans cesse entre mes mains tant qu'il
sentit quelque distance jusqu'à mon départ. Comme il le vit s'approcher,
il se mit à me prêcher la magnificence et à vouloir entrer dans le
détail de mon train. Je le lui expliquai, et tout autre l'eût trouvé
plus que convenable\,; mais comme son dessein était de me ruiner, il
s'écria donc et l'augmenta d'un tiers. Je lui représentai l'excès de
cette dépense, l'état des finances, le déchet prodigieux du change\,;
j'en eus pour toute réponse que cela devait être ainsi pour la dignité
du roi dans une ambassade de cet éclat, et que c'était à Sa Majesté à en
porter toute la dépense. J'en parlai à M. le duc d'Orléans, qui me donna
plus de loisir à mes représentations\,; mais qui, persuadé par le
cardinal, me tint le même langage.

Cet article passé, ce dernier voulut savoir le nombre d'habits que
j'aurais et que je donnerais à mes enfants, et quels ils seraient\,; en
un mot, il n'est détail de table et d'écurie où il n'entrât et qu'il
n'augmentât du double. Embarrassé de ma résistance et de mes raisons, il
me détachait tantôt Belle-Ile, tantôt Le Blanc, qui, comme d'eux-mêmes
et comme mes amis, m'exhortaient à ne pas m'opiniâtrer contre un homme
si impétueux, si dangereux, si fort en totale possession de la facilité
et de la faiblesse de M. le duc d'Orléans, qui, moi parti, demeurait
sans contre-poids et aurait beau jeu à profiter de mon absence, tandis
que j'aurais à passer indispensablement par lui dans tout le cours de
mon ambassade. Tout cela n'était que trop vrai. Il fallut donc céder,
quoique je sentisse bien qu'une fois embarqué ils ménageraient la bourse
du roi aux dépens de la mienne.

Dès que les mariages furent déclarés, je pressai pour l'être, afin de
pouvoir faire travailler à mes équipages. Cela m'avait été très
expressément défendu jusque-là, et avec raison pour ne donner d'éveil à
personne, mais la raison, cessant avec la déclaration des mariages, et
d'ailleurs le temps pressant, je ne crus pas que cela pût recevoir
aucune difficulté. Je m'y trompai. Les défenses subsistèrent quoi que je
pusse alléguer. C'est que le cardinal voulait qu'il m'en coûtât le
double par la précipitation, ainsi qu'il arriva, et me mettre de plus
dans l'impossibilité d'avoir tout, faute de temps, et cette faute me
l'imputer tant auprès de M. le duc d'Orléans qu'il avait entièrement
prévenu, qu'en Espagne, et faire de plus crier les envieux après moi.
Néanmoins je ne cessais de presser là-dessus, et en même temps d'entamer
les instructions qui m'étaient nécessaires, et qui, se passant du
cardinal et de M. le duc d'Orléans à moi, n'affichaient rien au public
comme la préparation des équipages. Ce fut encore ce que je ne pus
obtenir\,; ils me répondaient lestement qu'en une ou deux conversations
la matière serait épuisée. C'est que le cardinal voulait que je ne fusse
instruit qu'en l'air, m'ôter le loisir des réflexions, des questions,
des éclaircissements, et me jeter dans les embarras et les occasions de
faire des sottises qu'il comptait bien de relever fortement. Enfin,
lassé de tant et de si dangereuses remises, et comprenant bien que ma
déclaration ne se différait que pour les faire durer jusqu'à
l'extrémité, j'allai le mardi 23 septembre trouver M. le duc d'Orléans,
et pris exprès mon temps qu'il était dans son appartement des
Tuileries\,; là, je lui parlai si bien, qu'il me dit qu'il n'y avait
qu'à monter chez le roi. Il m'y mena, et dans le cabinet du roi où il
était avec ses sous-gouverneurs et peu de monde qu'on n'en fit point
sortir, je fus déclaré. Au sortir du cabinet, M. le duc d'Orléans me fit
monter dans son carrosse qui l'attendait, et me mena au Palais-Royal où
nous commençâmes à parler sérieusement d'affaires sur mon ambassade.

Je crois que le cardinal Dubois fut bien fâché de la déclaration qu'il
voulait encore différer, et qu'elle se fût faite de la sorte. Mais après
cela, il n'y eut plus moyen de reculer. Dès le lendemain on se mit à
travailler à mes équipages, sur lesquels le cardinal montra autant
d'empressement et d'impatience qu'il avait auparavant affecté de lenteur
et de délais. Il envoyait presser les ouvriers, voulut voir un habit de
chaque sorte de domestiques, livrées et autres, en augmenta encore la
magnificence, et se fit apporter tous les habits faits pour moi et pour
mes enfants. Enfin la presse de me faire partir dès que je fus déclaré
fut si grande, qu'il fit transporter tout ce qui put l'être sur des
haquets en poste jusqu'à Bayonne, ce qui ne fut pas à bon marché pour
moi. Il voulut savoir qui je mènerais, en m'exhortant à une grande
suite. Je lui nommai le comte de Lorges, le comte de Céreste, mes deux
fils, l'abbé de Saint-Simon, son frère, le major de son régiment, qui
avait servi en Espagne, était fort entendu, officier de grande
distinction, et qui me fut infiniment utile\,; je le fis depuis
lieutenant de roi de Blaye\,; un mestre de camp réformé dans le régiment
de mon second fils, l'abbé de Mathan, ami de l'abbé de Saint-Simon, qui
est toujours depuis demeuré des miens. On a vu ailleurs que je l'étais
fort de M. de Brancas. Céreste, son frère de père et de mère, mais de
vingt-cinq ans plus jeune, était aussi ami de mes enfants. Il eut envie
de faire ce voyage\,; son frère aussi désira qu'il y vînt, et je le tins
à honneur. Nous fîmes lui et moi grande connaissance dans ce voyage. Je
trouvai en ce jeune homme un homme tout fait et fait également pour
l'agréable et le solide. L'estime forma l'amitié qui a depuis subsisté
intime.

Le cardinal approuva fort toute cette compagnie\,; mais je fus bien
surpris lorsqu'il m'envoya Belle-Ile et Le Blanc me dire qu'il fallait
que je menasse une quarantaine d'officiers des régiments de cavalerie de
mes enfants et de celui d'infanterie du marquis de Saint-Simon, à quoi
ils suppléeraient si ces corps ne m'en pouvaient fournir ce nombre. Je
m'écriai à la folie et à la dépense. Je représentai au régent et au
cardinal l'inutilité d'un accompagnement si nombreux, si coûteux, si
embarrassant\,; qu'on n'avait jamais fait d'accompagnement militaire à
aucun ambassadeur, excepté le marquis de Lavardin, parce qu'il allait à
Rome, malgré le pape Innocent XI, soutenir à vive force les franchises
des ambassadeurs que le pape avait supprimées, et à quoi les autres
puissances avaient consenti\,; qu'on savait que le pape, tout
autrichien, serait soutenu par les forces que feraient couler dans Rome
le vice-roi de Naples et le gouverneur de Milan, ce qui avait obligé
d'envoyer force gardes-marine et officiers à Rome, pour soutenir M. de
Lavardin\,; que moi, au contraire, j'allais exercer une ambassade de
paix, d'union, de ralliement intime, qui n'avait aucun besoin d'escorte
qu'outre l'inutilité et la dépense extrême de mener et défrayer quarante
officiers des troupes du roi, ces officiers ne pourraient être que de
jeunes gens dont la tête, la galanterie indiscrète et française, les
aventures me donneraient plus d'affaires que toutes celles de
l'ambassade. Rien de plus évidemment vrai et raisonnable que ces
représentations\,; rien de plus inutile et de plus mal reçu.

Le cardinal avait entrepris de me ruiner et de me susciter tout ce qu'il
pourrait d'embarras, d'affaires et de tracasseries eu Espagne. Il crut
avec raison que rien n'était plus propre à l'y faire réussir que de me
charger de quarante officiers. Faute d'en trouver, je n'en menai que
vingt-neuf, et si le cardinal réussit du côté de ma bourse, je fus si
heureux, et ces messieurs si sages, qu'il n'en tira rien de ce qu'il
s'en était proposé. Il manda à Sartine de faire en Espagne tout ce qui
ne se pouvait faire que là, pour mes équipages, mules, carrosses,
domestiques espagnols, provisions, outre celles que je tirerais de
France, lequel s'en acquitta à souhait.

Sartine était de Lyon, où il s'était mêlé de banque, et avait eu la
direction générale des vivres des armées d'Espagne\,; il s'y était
stabilié, il y avait eu force hauts et bas de la fortune. C'était un
homme de figure agréable, d'esprit et de beaucoup d'entendement,
d'intelligence, d'expédients, et beaucoup de facilité, d'agrément et
d'expédition dans le travail. Il était souvent consulté sur les
résolutions à prendre, personnellement bien avec le roi d'Espagne, et
avec la plupart des ministres et des grands, sur un pied d'honnête homme
et de considération. Je n'en ai jamais vu rien que de bon ni ouï dire
aucun mal tant soit peu fondé. Des amis si considérables et les marques
fréquentes de la confiance du roi, lui firent des ennemis. Il fut poussé
à l'intendance générale de la marine par son ami Tinnaguas, qui en était
secrétaire d'État, et eut aussi une place dans une junte formée pour le
commerce. Albéroni, dès ses premiers commencements, perdit Tinnaguas, et
Sartine remit son intendance qu'il sentit bien qu'on lui ôterait\,; mais
Albéroni le poussa sur des comptes quoique apurés, et lui retint en même
temps ses papiers. Il lui fit de plus un crime de ses liaisons avec le
duc de Saint-Aignan, et quand il força cet ambassadeur à se retirer en
France, de la façon qui a été racontée en son temps, il fit arrêter
Sartine, lui fit très inutilement subir divers interrogatoires, et
Sartine ne sortit de prison que lorsque Albéroni sortit lui-même
d'Espagne. Ce n'était pas un homme sans ambition, mais sage et sans se
méconnaître, laborieux, actif, pénétrant, extrêmement au fait de la
marine et du commerce d'Espagne et des Indes, d'ailleurs serviable et
bon ami, doux, aimable dans le commerce, fort Français sans s'en cacher,
et néanmoins généralement aimé des Espagnols dans tous les temps. Il
épousa une camariste\footnote{Femme de chambre.} de la reine, qui était
fort bien avec elle. Peu après mon départ, il fut intendant de
Barcelone, l'a été longtemps, et est mort dans cet emploi. Je me suis
étendu sur lui, parce qu'il m'a été très utile en Espagne, et pour mes
affaires, et pour mille choses de la cour et du gouvernement, en sorte
que j'étais demeuré en liaison avec lui.

Mon premier soin, sitôt que ma déclaration me mit en liberté, fut
d'écrire au duc de Berwick qui commandait en Guyenne, et se tenait pour
lors à Montauban, et de voir Amelot et le duc de Saint-Aignan, pour
tirer d'eux toutes les lumières et les instructions que je pourrais sur
l'Espagne où ils avaient tous trois été longtemps. J'en tirai de solides
d'Amelot, et du duc de Saint-Aignan un portrait des gens principaux en
crédit, ou par leur état, ou par leur intrigue, très bien écrit, et que
j'ai reconnu parfaitement véritable\,; du duc de Berwick, quelque chose
de semblable, mais fort en raccourci et avec plus de mesure\,; mais ce
qui me fut infiniment utile, c'est ce qu'il fit de lui-même qui fut de
mander au duc de Liria, son fils, établi, comme on l'a vu ici en son
temps, en Espagne, de me servir en toutes choses\,; il le fit au point
de ne dédaigner pas d'aider si bien Sartine sur ce qui regardait mes
équipages, que je dois avouer que, dans un temps si court pour la
paresse et la lenteur espagnole, je n'aurais, sans lui, trouvé rien de
prêt en arrivant.

Mais en quoi il me servit le plus utilement, ce fut à me faire connaître
les personnages, les liaisons, les éloignements, les degrés de crédit et
des caractères et mille sortes de choses qui éclairent et conduisent
dans l'usage, et conduisent adroitement les pas. Il me valut de plus la
familiarité du duc de Veragua, frère de sa femme, qui, bien que jeune,
avait passé par les plus grands emplois, avec grand sens et beaucoup
d'esprit, qu'il avait extrêmement orné et savait infiniment, tant sur
les personnages divers et les intrigues, que sur la naissance, les
dignités, et toute espèce de curiosités savantes de cette nature qui
m'en ont extrêmement instruit. Il était, comme d'avance on l'a vu ici,
en traitant des grands d'Espagne (voy. tome III, p.~224 et suiv.), il
était, dis-je, masculinement et légitimement d'une branche de la maison
de Portugal, et descendait, par sa grand'mère, du fameux Christophe
Colomb. Une maîtresse obscure, avec qui il ne se ruinait pas, car il
était avare, et la lecture partageaient son temps et sa paresse, fort
bien toutefois avec tout le monde, et considéré de la cour autant
qu'elle en était capable. Vilain de sa figure, sale et malpropre à
l'excès, avec des yeux pleins d'esprit, aussi en avait-il beaucoup, et
délié sous une apparence grossière, de bonne compagnie et quelquefois
fort plaisant sans y songer, d'ailleurs doux, de bon commerce, entendant
raillerie jusque-là que ses amis l'appelaient familièrement don
Puerco\footnote{Don Pourceau.}, et que dînant une fois chez le duc de
Liria, à Madrid, nous lui proposâmes de manger au buffet, parce qu'il
était trop sale pour être admis à table. Tout cela se passait en
plaisanteries qu'il recevait le mieux du monde. La duchesse de Liria, sa
soeur, et lui s'aimaient extrêmement\,; ils n'avaient point d'autre
frère ni soeur et avaient perdu père et mère, de sorte qu'étant mort
longtemps après sans s'être marié, ses grands biens passèrent à la
duchesse de Liria et à ses enfants. Le duc de Liria avait de l'esprit et
des vues\,; il était agissant et courtisan, connaissait très bien le
terrain et les personnages, était autant du grand monde que cela se
pouvait en Espagne, bien avec tous, lié avec plusieurs, mais désolé de
se trouver établi en Espagne, à la tristesse de laquelle il ne
s'accoutumait point\,; il n'aspirait qu'à s'en tirer par des ambassades,
comme il fit à la fin, et il aimait si passionnément le plaisir, qu'il
en mourut longtemps après à Naples. Après être revenu de son ambassade
d'Allemagne et de Moscovie, il passa, au retour, par la France, et me
donna par écrit des choses fort curieuses sur la cour de Russie.

Ce ne fut pas sans peine et sans tous les délais que le cardinal Dubois
y put apporter, que je tirai enfin de lui une instruction\,: j'y vis ce
que je n'ignorais pas sur la position présente de l'Espagne. Après qu'on
eut enfin arraché son accession aux traités de Londres, elle avait signé
une alliance défensive avec la France et l'Angleterre sur le fondement
des traités d'Utrecht, de la triple alliance de la Haye, et des traités
de Londres, laquelle alliance défensive contenait une garantie
réciproque des États dont la France, l'Espagne et l'Angleterre
jouissaient, et tacitement confirmait très fortement les renonciations
réciproques qui était le grand point de M. le duc d'Orléans, et la
succession protestante de l'Angleterre dans la maison d'Hanovre, qui
était le grand point du cardinal Dubois, et pas un des deux, celui
personnel du roi et de la reine d'Espagne qui eurent toujours le plus
vif esprit de retour. Par ce même traité d'alliance défensive, la France
et l'Angleterre promirent leurs bons offices à l'Espagne, pour régler au
congrès de Cambrai, oui il ne se fit rien du tout, les différends qui
restaient à ajuster entre l'empereur et le roi d'Espagne. Ce n'est pas
qu'il y eût rien à négocier là-dessus à Madrid, mais j'ai cru à propos
d'exposer la situation de l'Espagne, lorsque j'y allai, avec l'empereur,
la France, l'Angleterre et la Hollande, pour ne pas la laisser
oublier\,; avec cela le cardinal Dubois était fort en peine d'une
nouvelle promotion de grands d'Espagne que l'empereur venait de faire
contre ses propres engagements, et chargea mon instruction de ce qu'il
put, pour faire avaler cette continuation d'entreprise le plus doucement
qu'il se pourrait à la cour d'Espagne. La chose finit, parce que le roi
d'Angleterre obtint une déclaration de Vienne, que l'empereur n'avait
point entendu et ne prétendait point faire des grands d'Espagne, que
cette qualité ne se trouvait point dans les lettres patentes qu'il avait
accordées à quelques seigneurs, mais seulement des distinctions et des
honneurs, qu'il était maître de donner à qui lui plaisait dans sa cour.

Cette instruction, après avoir relevé avec beaucoup d'affectation
l'utilité pour l'Espagne de l'alliance d'Angleterre et les soins du
régent pour y parvenir, qui toutefois fut au mot de l'Angleterre et au
détriment de l'Espagne et même du commerce de France, pour favoriser en
tout celui d'Angleterre, comme il a été expliqué ici ailleurs et fort
insisté sur la passion du régent de servir en tout l'Espagne, a grand
soin de me recommander de prendre bien garde qu'il ne prit envie au roi
d'Espagne de porter de nouveau la guerre en Italie, comptant sur la
France et l'Angleterre, et à ce propos donne faussement pour motif à
l'invasion de la Sardaigne et à la guerre de Sicile l'emprisonnement de
Molinez. On a vu ici, d'après M. de Torcy, combien peu de cas, et
longtemps, Albéroni en fit, et qu'il {[}ne{]} réchauffa cette affaire
que quand il eut résolu de porter la guerre en Italie, pour des raisons
personnelles uniquement à lui. C'est ce que M. le duc d'Orléans avait
tant vu par les lettres de la poste qu'il était impossible que le
cardinal Dubois le pût ignorer.

De son extrême attention à me munir de tout ce qu'il put pour faire bien
valoir l'alliance d'Angleterre à l'Espagne, résultait une injonction
pathétique de vivre dans un commerce étroit à Madrid avec le colonel
Stanhope, ambassadeur d'Angleterre, et de lui confier tout ce qui
pourrait être relatif aux intérêts des trois couronnes\,; en même temps
de n'en avoir aucun sous tel prétexte que ce pût être avec les personnes
attachées au prétendant, surtout à l'égard des desseins ou projets que
ce prince ou ses serviteurs pourraient former de troubler le
gouvernement présent d'Angleterre\,; en particulier, d'éviter le duc
d'Ormond, toutefois sans incivilité marquée.

Après ce que M. le duc d'Orléans m'avait si précisément dit que c'était
l'Espagne qui lui avait forcé la main pour la déclaration actuelle des
mariages et l'échange des princesses, je fus très surpris de trouver le
contraire dans le narré de mon instruction. J'y trouvai aussi une
grossière ignorance qui regardait la façon de me faire dispenser d'une
entrée. Les ambassadeurs de l'empereur n'en faisaient point à Madrid
sous les rois d'Espagne de la maison d'Autriche comme ambassadeur de
famille. Sur cet exemple, aucun ambassadeur de France vers Philippe V
n'y en a fait, et je n'ai pas compris comment un fait si public, et si
fréquemment réitéré par le changement de nos ambassadeurs, a pu échapper
au cardinal Dubois et même à ses bureaux.

L'instruction me défendait de recevoir chez moi Magny et les Bretons
réfugiés en Espagne, et Marsillac\,; de n'avoir pas la même incivilité
pour ce dernier en lieux tiers que pour les autres, et de voir avec une
civilité simplement extérieure le prince de Cellamare, qui portait alors
le nom de duc de Giovenazzo, et les parents et amis de la princesse des
Ursins comme les autres.

Enfin, pour ne m'attacher qu'aux choses principales de l'instruction,
elle ne me prescrivit rien en particulier sur les visites et le
cérémonial, mais d'en user comme avait fait le duc de Saint-Aignan, et
le cardinal Dubois y joignit un extrait du cérémonial pratiqué par nos
ambassadeurs en Espagne et à leur égard, depuis M. de La Feuillade,
archevêque d'Embrun, mort évêque de Metz.

Je ne pouvais douter que je n'eusse affaire à un ennemi, et maître,
après mon départ, de l'esprit de M. le duc d'Orléans. Je voulus donc
avoir ma leçon faite jusque sur les plus petites choses, pour ne laisser
à sa malignité que ce qu'il serait impossible d'y dérober\,; ainsi je
lui fis à mi-marge plusieurs observations et questions, tant sur des
choses portées par l'instruction que sur d'autres qui ne s'y trouvaient
pas. Il répondit à côté assez bien et assez nettement. On verra bientôt
où il m'attendait.

Le cardinal Dubois n'oublia pas le P. Daubenton. L'instruction me
prescrivit des compliments, des témoignages de reconnaissance du régent,
de ses désirs empressés de la lui témoigner\,; de lui dire que rien ne
m'était plus recommandé que de prendre en lui une entière confiance.
Cela fort étendu était accompagné d'un fort grand éloge. C'étaient deux
fripons des plus insignes, dignes de se louer l'un l'autre et d'être
abhorrés de tout le reste des hommes, surtout des gens de bien et
d'honneur\,; l'instruction ne fit mention que de lui de toute la cour
d'Espagne, de Valouse et de La Roche, pour lesquels elle me prescrivit
de l'honnêteté, mais de les regarder comme des gens timides, inutiles,
dont on n'avait jamais tiré secours ni la moindre connaissance. Valouse,
du nom de Boutin, était un gentilhomme du Comtat, élevé page de la
petite écurie, très médiocrement bien fait, d'esprit court, mais sage,
appliqué, allant à son but et ne s'en écartant point, honnête homme et
droit, mais qui craignait tout. Du Mont, de qui il a été parlé plus
d'une fois dans ces Mémoires, le proposa, sur son esprit sage, doux et
timide, au duc de Beauvilliers pour écuyer de M. le duc d'Anjou, qu'il
suivit depuis en Espagne, et qui le fit quelque temps après majordome,
qui fut un grand pas. Au bout de plusieurs années, il l'avança bien
davantage, car ayant fait don Lorenzo Manriquez grand écuyer, duc del
Arco et grand d'Espagne, de premier écuyer qu'il était, il fit Valouse
premier écuyer. Cette promotion était récente à mon arrivée en Espagne.
Valouse fut premier écuyer jusqu'à sa mort, qui n'arriva que bien des
années après, toujours très bien avec le roi et la reine d'Espagne\,;
aussi bien avec le duc del Arco, toujours ne se mêlant que de sa charge
et d'aucune autre chose, toujours cultivant les gens en place, et
honnêtement avec M\textsuperscript{me} des Ursins, Albéroni, et ceux qui
ont succédé, parce qu'ils sentirent tous qu'ils n'en avaient rien à
craindre\,; enfin sur les dernières années de Valouse, le roi d'Espagne
lui donna la Toison d'or. Il avait depuis longtemps une clef de
gentilhomme de la chambre sans exercice. Cette Toison, ainsi que bien
d'autres, parut un peu sauvage.

La Roche n'était ni moins borné, ni moins timide, ni moins en garde de
se mêler de quoi que ce fût, que l'était Valouse, doux, poli et honnête
homme comme lui, mais aussi parfaitement inutile. Sa mère veuve était au
vieux Bontems ce que M\textsuperscript{me} de Maintenon était au roi,
mais plus à découvert, tenant son ménage, et maîtresse de tout chez lui.
Le plaisant est qu'on la courtisait pour plaire à Bontems, et que, quand
elle mourut, il fut au désespoir et que le roi prit soin de le consoler.
Il avait fait le fils de cette femme, tout jeune encore, valet de
garde-robe du roi, et au départ du roi d'Espagne, il le fit être son
premier valet de garde-robe. Sa sagesse, sa retenue, son air de respect
pour les Espagnols leur plut, et lui et Valouse furent par là toujours
bien avec eux. L'estampille est une manière de sceau sans armes, où la
signature du roi est gravée dans la plus parfaite imitation de son
écriture\,; ce sceau s'applique sur tout ce que le roi devrait signer et
lui en ôte la peine. Il semblerait qu'un sceau de cette importance ne
devrait être confié qu'à des personnes principales\,; mais l'usage
d'Espagne, depuis qu'il a été inventé, est qu'il ne soit remis qu'à des
subalternes de confiance. La Roche en fut chargé peu après qu'il fut en
Espagne, où il avait suivi Philippe V\,; il s'en acquitta très
fidèlement et poliment au gré de tout le monde, et s'y maintint toute sa
vie dans une sorte de confiance du roi d'Espagne, sous tous les divers
ministères, parce que tous sentirent bien qu'ils n'avaient rien à
craindre de lui. Il tenait, pour son état, une maison honorable où
allait bonne compagnie, et toujours plusieurs personnes à manger, ce que
ne faisait pas Valouse qui ne dépensait rien. À l'égard du P. Daubenton,
je me réserve d'en parler ailleurs.

Laullez était alors à Paris de la part de l'Espagne, et l'abbé Landi de
la part du duc de Parme. Le premier était un Irlandais, grand homme,
très bien fait et de bonne mine, qui avait été à l'abbé d'Estrées. Il le
donna au roi d'Espagne, à la formation de ses gardes du corps sur le
pied et le modèle de ceux du roi, comme un garçon brave et intelligent,
fort honnête homme, avec de l'esprit et de la sagesse. Laullez était tel
en effet, et par les détails de ces compagnies de gardes du corps, il
entra dans la familiarité du roi, de la reine sa première femme, de la
princesse des Ursins, et bientôt dans leur confiance\,; en quoi, pour
cette dernière qui lui valut celle des maîtres, sa nation, étrangère à
l'Espagne et à la France, lui servit beaucoup\,; il fut souvent chargé
de commissions secrètes et délicates qu'il exécuta toutes fort
heureusement. Il devint ainsi major des gardes du corps et lieutenant
général\,; c'est en cet état qu'il vint en France, où il reçut le
caractère d'ambassadeur au même temps que Maulevrier le reçut à Madrid.
Les vues qui m'avaient fait souhaiter d'aller en Espagne me firent aussi
désirer liaison avec ces deux envoyés. Louville se trouva en avoir
beaucoup avec l'abbé Landi\,; et le duc de Lauzun, qui attirait fort les
étrangers chez lui, et qui y voyait Laullez, me facilita ce que je
désirais auprès de lui. La connaissance fut bientôt faite\,: je voulais
plaire au ministre d'Espagne, et lui ne le désirait pas moins à un
serviteur intime de M. le duc d'Orléans\,; les choses se passèrent
tellement entre nous que l'amitié s'y mit, qui a duré au delà de sa vie.
Je reçus de lui mille bons avis et toutes sortes de bons offices et de
services en Espagne. Je le retrouvai à mon retour, et encore depuis la
mort de M. le duc d'Orléans, et je fis inutilement l'impossible pour lui
procurer l'ordre du Saint-Esprit. Enfin il retourna en Espagne avec
l'infante, d'où il fut envoyé à Majorque, gouverneur de l'île et
capitaine général, où il est mort très longtemps après sans avoir été
marié. Il y laissa deux soeurs filles qui y sont demeurées, qui
s'adressèrent bien des années après à moi pour être payées d'avances
faites par leur frère, et que j'ai servies de tout ce que j'ai pu dans
cette affaire par mes amis. Par l'abbé Landi je voulais me concilier la
petite cour de Parme qui avait en beaucoup de choses du crédit sur la
reine d'Espagne\,; je trouvai un homme poli, assez agréable dans le
commerce, qui fut court par mon départ, mais je n'en tirai rien à Paris
ni en Espagne\,; il n'était plus à Paris quand j'y revins.

J'ai rapporté ce qu'il y eut de plus important ou de plus remarquable de
l'instruction en forme qui me fut donnée. Quelle qu'elle fût, elle
satisfaisait à tout avec le cérémonial de tous nos ambassadeurs en
Espagne, depuis M. de La Feuillade, alors archevêque d'Embrun. J'eus
plusieurs entretiens sur l'Espagne avec M. le duc d'Orléans et le
cardinal Dubois ensemble ou séparément, et je n'imaginais pas qu'il se
pût rien ajouter de nouveau, lorsque le cardinal Dubois me dit chez lui
qu'il m'avertissait de prendre la première place à la signature du
contrat de mariage du roi, et à la chapelle aux deux cérémonies du
mariage du prince des Asturies, et de ne la laisser prendre sans
exception à qui que ce fût. Je lui représentai que cela ne se pouvait
entendre du nonce, à qui les ambassadeurs de France cédaient partout,
même celui de l'empereur qui, sans difficulté, précédait ceux du roi. Il
répondit que cela était vrai et bon partout, excepté dans ce cas
singulier et comme momentané, et que cela ne se pouvait autrement. Ma
surprise fut grande d'un ordre si étrange. J'essayai de le ramener peu à
peu en le touchant par son orgueil, en lui demandant comment j'en
userais avec les cardinaux, s'il s'en trouvait quelqu'un en ces
fonctions, et avec le majordome-major, qui répond, mais fort
supérieurement, à notre grand maître de France. Il se mit en colère, me
déclara qu'il fallait que j'y précédasse le majordome-major sans
difficulté, et glissant sur celle des cardinaux, m'assura qu'il ne s'y
en trouverait point. Je haussai les épaules, et lui dis que je le priais
d'y penser. Au lieu de me répondre, il me dit qu'il avait oublié une
chose essentielle, qui était de prendre bien garde à ne rendre la
première visite à qui que ce fût sans exception. Je répondis que
l'article des visites n'était point oublié dans mon instruction\,;
qu'elle portait que j'en userais à cet égard comme avait fait le duc de
Saint-Aignan, et que l'usage, lequel il avait suivi, était de rendre la
première visite au ministre chargé des affaires étrangères et aux
conseillers d'État quand il y en avait, qui est ce que nous connaissons
ici sous le nom de ministres. Là-dessus il s'emporta, bavarda, brava sur
la dignité du roi, et ne me laissa plus loisir de rien dire. J'abrégeai
donc la visite et m'en allai.

Quelque étranges que me semblassent ces ordres si nouveaux et verbaux,
je voulus en parler au duc de Saint-Aignan, surtout à Amelot, qui en
furent fort étonnés, et qui tous deux, ainsi que les précédents
ambassadeurs, avaient fait tout le contraire, et trouvèrent extravagante
la préséance sur le nonce en quelque occasion que ce fût. Amelot me dit
de plus que je jouerais à essuyer tous les dégoûts possibles et à ne
réussir à rien si je refusais la première visite au ministre des
affaires étrangères, car pour les conseillers d'État ce n'était plus
qu'un nom, et la chose tombée en désuétude\,; mais que je devais aussi
la première visite aux trois charges\footnote{C'est-à-dire aux trois
  grands officiers, au majordome-major du roi, au sommelier du corps et
  au grand écuyer.}, qui seraient très justement offensés et très piqués
si je leur refusais ce que tous ceux qui m'avaient précédé leur avaient
rendu, et que je me gardasse bien de le faire si je ne voulais pas
demeurer seul dans mon logis, et me faire tourner le dos au palais par
tout ce que j'y trouverais de grands. J'expliquerai ailleurs ce que
c'est que ces trois charges.

De cet avis d'Amelot, je compris aisément la raison de ces ordres
nouveaux et verbaux. Le cardinal me voulait faire échouer en Espagne et
me perdre ici\,: en Espagne, en débutant par offenser tout ce qui était
de plus grand, et le ministre par lequel seul j'aurais à passer pour
tout ce qui regardait mon ambassade\,; en attirer les plaintes ici, sûr
de n'avoir rien écrit de ces ordres, nier me les avoir donnés, me
désavouer, et en tirer contre moi tout le parti possible avec un prince
qui n'aurait osé lui imposer, et soutenir que ces ordres m'avaient été
donnés\,; que si, au contraire, je ne les exécutais pas, car il m'avoir
bien prescrit de rendre compte de leur exécution, il se donnerait beau
jeu à m'accuser d'avoir sacrifié l'honneur du roi et la dignité de sa
couronne à l'intérêt de plaire en Espagne pour en obtenir grandesse et
Toison, et me faire défendre de les accepter pour mes enfants. C'eût été
moins de vacarme sur le nonce\,; mais si j'avais pris place au-dessus de
lui, il s'attendait bien que la cour de Rome en demanderait justice, et
que cette justice entre ses mains serait un rappel honteux.

Ce détroit me parut si difficile que je résolus de ne rien omettre pour
faire changer ces ordres, et je ne crus pas que M. le duc d'Orléans pût
résister à l'évidence de ce qui les combattait, et à l'exemple constant
de tous ceux qui m'avaient précédé dans le même emploi. Je me trompai\,:
j'eus beau en parler à M. le duc d'Orléans, je ne trouvai que faiblesse
sous le joug d'un maître, d'où je jugeai ce que je pouvais espérer
pendant mon éloignement. J'insistai à plusieurs reprises, toujours
inutilement, et tous deux se tinrent fermés\footnote{Il y a dans le
  manuscrit fermés, et non fermes\,; fermés est pris dans le sens de
  résolus, affermis.} à me dire que, si les précédents ambassadeurs
avaient fait les premières visites, ce n'était pas un exemple pour moi
dans une ambassade aussi solennelle et aussi distinguée que celle que
j'allais exercer\,; et qu'à l'égard du nonce et du grand maître,
l'exemple de précéder quiconque était formel au mariage de la reine
Marie-Louise, fille de Monsieur, avec Charles II. Je représentai sur les
visites que quelque solennelle et quelque distinguée que fût l'ambassade
dont j'étais honoré, elle ne donnait point de rang supérieur à celui des
ambassadeurs extraordinaires\,; que je l'étais, et que je ne pouvais
prétendre rien de plus qu'eux, quelque différence qu'il y eût pour
l'agrément entre l'affaire dont j'étais chargé et les autres sortes
d'affaires. Sur l'exemple du mariage de Charles II avec la fille de
Monsieur, que j'avais dans le cérémonial qui m'avait été remis de tous
les ambassadeurs depuis M. de La Feuillade, archevêque d'Embrun, j'y
trouvai que le mariage s'était fait comme à la dérobée, dans un village,
pour fuir la difficulté entre le prince d'Harcourt et le père du
maréchal de Villars, ambassadeurs de France tous deux, d'une part, et
les grands d'Espagne, de l'autre\,; que les ambassadeurs s'étoient
rendus à l'église de ce village\,; qu'y ayant trouvé plusieurs grands
arrivés avant eux, saisis des premières places, ils s'en étaient plaints
sur-le-champ au roi qui leur fit céder les deux premières places par les
grands\,; que le nonce n'y était point, et nulle mention du
majordome-major. À cela point de réponse, mais l'opiniâtreté prévalut,
et je vis en plein l'extrême malignité du valet et l'indicible faiblesse
du maître. Ce fut donc à moi à bien prendre mes mesures là-dessus.

Le duc d'Ossone fut nommé par l'Espagne pour venir ici faire, pour le
mariage du prince des Asturies, avec le même caractère, les mêmes
fonctions que j'allais faire en Espagne pour le mariage du roi. Il était
frère du duc d'Ossone qui avait été ambassadeur d'Espagne au traité
d'Utrecht, et qui mourut peu après sans enfants. Celui-{[}ci{]} portait
le nom de comte de Pinto du vivant de son frère. Leur père avait été
gouverneur du Milanais, conseiller d'État et grand écuyer de la reine
d'Espagne\,: il mourut d'apoplexie étant en conférence avec le roi
d'Espagne, en 1694. C'était le sixième duc d'Ossone, grand de première
classe. Ils portaient le nom de Giron et de Tellez par une héritière
entrée dans leur maison\,; mais ils étaient Acuña y Pacheco, une des
premières d'Espagne en tout genre, et des plus nombreuses par ses
diverses branches, qui, par des héritières, portent divers noms, entre
autres, alors, le marquis de Villena, duc d'Escalona, majordome-major,
et le comte de San Estevan de Gormaz, son fils, premier capitaine des
gardes du corps, chef de toute cette grande maison\,; le duc d'Uzeda, le
marquis de Mancera, le comte de Montijo, tous grands d'Espagne. Ce duc
d'Ossone, ambassadeur ici, était donc un fort grand seigneur qui s'y
montra très magnifique et très poli, mais il n'était que cela\,: on sut
que M. le duc d'Orléans avait résolu de lui donner le cordon bleu. Je
m'exprime de la sorte parce que le roi, n'étant pas encore chevalier de
son ordre, et ne faisant que le porter jusqu'à ce qu'il reçût le collier
le lendemain de son sacre, il ne pouvait faire de chevalier de l'ordre.
Le duc d'Ossone ne pouvait donc qu'avoir parole de l'être quand le roi
en ferait, à quoi on voulut ajouter une chose, jusqu'alors sans exemple,
dans le cas où était le roi, qui fut de lui faire porter l'ordre en
attendant qu'il pût être nommé\,; on crut et il était vrai que M. le duc
d'Orléans étant régent et maître des grâces, il devait marquer par toute
la singularité de celle-ci combien il était touché de l'honneur du
mariage de sa fille.

Sur ce premier exemple, le duc de Lauzun me pressa fort de demander
aussi le cordon bleu comme une décoration convenable à porter en
Espagne, et qui, étant grâce d'ici, ne pourrait préjudicier à celles que
je pouvais attendre d'Espagne pour mes enfants\,; mais je n'en voulus
rien faire. Cette impatience de porter l'ordre, qui, dans la suite, ne
pouvait me manquer, me répugna. Je n'avais désiré cette ambassade que
pour faire mon second fils grand d'Espagne, et, si l'occasion s'en
offrait, de faire donner la Toison à l'aîné. Y réussissant et ayant en
même temps pris le cordon bleu, cela me parut un entassement trop
avide\,; d'ailleurs on ne pouvait faire en France d'autre grâce au duc
d'Ossone que celle-là, et moi j'en espérais une d'Espagne bien autrement
considérable\,; ainsi je ne fus pas tenté un moment du cordon bleu. Qui
m'eût dit alors que je ne serais pas de la première promotion qui s'en
ferait m'aurait bien surpris\,; qui y eût ajouté que je serais de la
suivante, où nous ne serions que huit avec Cellamare, les deux fils du
duc du Maine et le duc de Richelieu, m'aurait bien étonné davantage.

Le cardinal Dubois pressait ardemment mon départ et, en effet, il n'y
avait plus de temps à perdre. Il envoyait sans cesse hâter les ouvriers
qui travaillaient à tout ce qui m'était nécessaire, fâché peut-être
qu'il y en eût un si prodigieux nombre, qu'il ne pût trouver à les
augmenter. Il ne s'agissait plus de sa part qu'à me remettre les lettres
dont je devais être chargé\,; il attendit à la dernière extrémité du
départ pour le faire\,; c'est-à-dire à la veille même que je partis\,:
on en verra bientôt la raison. Elles étaient pour Leurs Majestés
Catholiques, pour la reine douairière, à Bayonne, et pour le prince des
Asturies, tant du roi que de M. le duc d'Orléans. Mais bien avant de me
les remettre, M. le duc d'Orléans me dit qu'il en écrirait deux
pareilles au prince des Asturies avec cette seule différence qu'il le
traiterait de neveu dans l'une, et dans l'autre de frère et de neveu, et
que je tâchasse de faire passer la dernière, ce qu'il souhaitait
passionnément\,; mais que, si après tout, j'y trouvais trop de
difficulté, que je ne m'y opiniâtrasse point, et que je donnasse la
première au prince des Asturies.

J'eus lieu de croire que ce fut encore un trait du cardinal Dubois pour
me jeter dans quelque chose de personnellement désagréable à M. le duc
d'Orléans et en faire usage. M. le duc d'Orléans était l'homme du monde
qui avait le moins de dignité et d'attachement à ces sortes de choses.
Ce traitement de frère était un traitement d'égal, que le feu roi
n'avait relâché, que depuis peu, de donner aux électeurs princes, car M.
de Savoie avait depuis longtemps le rang de tête couronnée pour ses
ambassadeurs\,; à prendre comme étranger il n'y avait pas de proportion
entre le fils aîné, héritier présomptif de la couronne d'Espagne, et un
petit-fils de France, car la régence n'ajoutait rien à son rang ni {[}à
ses{]} traitements. À prendre comme famille, ils étaient l'un et l'autre
petits-fils de France\,; mais, outre que le prince des Asturies avait
l'aînesse, il était fils de roi et héritier de la couronne, et, par là,
si bien devenu du rang de fils de France, qu'ils étaient réputés tels en
France, et que le feu roi avait toujours envoyé le cordon bleu à tous
les fils du roi d'Espagne aussitôt qu'ils étaient nés, ce qui ne se fait
qu'aux seuls fils de France. De quelque côté qu'on le regarde, M. le duc
d'Orléans était extrêmement inférieur au prince des Asturies, et c'était
une véritable entreprise et parfaitement nouvelle que de prétendre
l'égalité du style et du traitement. Ce fut pourtant ce dont je fus
chargé, et je crois, dans la ferme espérance du cardinal Dubois, que je
n'y réussirais pas, et de profiter d'un début fort désagréable.

J'étais près d'oublier que Belle-Ile me vint dire qu'il savait que M. le
duc d'Orléans devait envoyer un de ses premiers officiers en Espagne,
pour remercier, de sa part, en particulier, de l'honneur du mariage de
sa fille\,; que le choix de cet officier principal n'était pas fait, et
me demanda s'il n'y en avait point parmi eux que je voulusse plutôt que
les autres. Sur ce que je répondis, que je n'étais en liaison, ni même
en commerce, avec pas un, excepté Biron qui l'était devenu et à qui ce
voyage ne convenait pas, et que le choix m'était indifférent, il me pria
de demander La Fare, son ami, qui était capitaine des gardes de M. le
duc d'Orléans. Je le lui promis et je l'obtins\,: ce fut son premier pas
de fortune. C'est un fort aimable homme, de bonne compagnie, qui m'en a
toujours su gré depuis. Sans blesser l'honneur et avec un esprit
gaillard mais fort médiocre, il a su être bien et très utilement avec
tous les gens en place et en première place, se faire beaucoup d'amis,
et faire ainsi peu à peu une très grande fortune qui a dû surprendre,
comme elle a fait, mais qui n'a fâché personne.

Enfin la veille de mon départ on m'apporta le matin toutes les pièces
dont je devais être chargé, dont je ne ferai point le détail. Mais parmi
les lettres il n'y en avait point du roi pour l'infante. Je crus que
c'était oubli de l'avoir mise avec les autres. Je le dis à celui qui
m'apportait ces pièces. Je fus surpris de ce qu'il me répondit qu'elle
n'était pas faite, mais que je l'aurais dans la journée. Cela me parut
si étrange que j'en pris du soupçon. J'en parlai au cardinal et à M. le
duc d'Orléans, qui m'assurèrent que je l'aurais le soir. Il était minuit
que je ne l'avais pas encore. J'écrivis au cardinal. Bref, je partis
sans elle. Il me manda que je la recevrais avant que d'arriver à
Bayonne\,; mais rien moins. Je pressai de nouveau. Il m'écrivit que je
l'aurais avant que j'arrivasse à Madrid. Une lettre du roi à l'infante
n'était pas difficile à faire\,: je ne pus donc douter qu'il n'y eût du
dessein dans ce retardement. Quel il pût être, je ne pus le comprendre,
si ce n'est d'en envoyer une après coup et pour me faire passer pour un
étourdi qui avait perdu la première.

Il me fit un autre trait de la dernière impudence sept ou huit jours
avant mon départ. Il me fit dire de sa part, par Le Blanc et par
Belle-Ile, que l'emploi où il était des affaires étrangères exigeait
qu'il eût les postes, dont il né voulait et ne pouvait se passer plus
longtemps\,; qu'il savait que j'étais ami intime de Torcy, qui les
avait, dont il désirait la démission\,; qu'il me priait de lui en écrire
à Sablé, où il était allé faire un tour, et ce par un courrier exprès\,;
qu'il verrait, par l'office que je lui rendrais en cette occasion et par
son succès, de quelle façon il pouvait compter sur moi, et se conduirait
en conséquence\,; à quoi ses deux esclaves joignirent du leur, mais avec
très apparente mission, tout ce qui me pouvait persuader qu'il romprait
mon départ et mon ambassade, si je ne lui donnais pas contentement
là-dessus. Je ne doutai pas un moment, après ce que j'avais vu de
l'inconcevable faiblesse de M. le duc d'Orléans pour ses plus folles
volontés, telles que les premières visites et la préséance à prendre sur
le nonce, et bien d'autres que je supprime, qu'il ne fût en pouvoir de
me causer cet affront. En même temps je résolus d'en essuyer le hasard
plutôt que de me prêter à la violence à l'égard d'un ami sûr, sage,
vertueux, et qui avait servi avec tant de réputation et si bien mérité
de l'État.

Je répondis donc à ces messieurs que je trouvais la commission fort
étrange, et beaucoup plus son assaisonnement\,; que Torcy n'était pas un
homme à qui on pût ôter un emploi de cette confiance, et qu'il exerçait
depuis la mort de son beau-père si dignement, à moins qu'il ne le voulût
bien lui-même\,; que tout ce que je pouvais faire était de le savoir de
lui, et, au cas qu'il y voulût entendre, à quelles conditions\,; que
pour l'y exhorter, encore moins aller au delà avec lui, je priais le
cardinal de n'y pas compter, encore que je n'ignorasse pas ce qu'il
pouvait à l'égard de mon ambassade, et que quoi que ce pût être ne me
ferait passer d'une seule ligne ce que je leur répondais. Ils eurent
beau haranguer, ils ne remportèrent que cette très ferme résolution.

Castries et son frère l'archevêque étaient de tous les temps intimes de
Torcy et fort aussi de mes amis. Je les envoyai prier de venir chez moi
dans ce tumulte de départ où je me trouvais. Ils vinrent sur-le-champ.
Je leur racontai ce qui venait de m'arriver. Ils furent plus indignés de
la façon et du moment que de la chose, dont Torcy comptait bien que le
cardinal le dépouillerait tôt ou tard pour s'en revêtir. Ils louèrent
extrêmement ma réponse, m'exhortèrent à l'exécuter promptement pour
hâter le retour de Torcy, qui était même ou parti ou sur le point de
partir de Sablé, et qui ferait lui-même son marché avec M. le duc
d'Orléans bien plus avantageusement qu'absent. Je leur fis lire la
lettre que j'écrivis à Torcy en les attendant, qu'ils approuvèrent
beaucoup, et par leurs avis réitérés je la fis partir sur-le-champ.

Torcy avait naturellement avancé son retour. Mon courrier le trouva avec
sa femme dans le pare de Versailles, ayant passé par la route de
Chartres. Il lut ma lettre, chargea le courrier de mille compliments
pour moi, sa femme aussi, et de me dire qu'il me verrait le lendemain.
J'avertis les Castries de son armée. Nous nous vîmes tous quatre le
lendemain. Torcy sentit vivement mon procédé, et jusqu'à sa mort nous
avons toujours vécu dans la plus grande intimité, comme on le peut voir
par la communication qu'il me donna de ses Mémoires qu'il ne fil que
bien longtemps après la mort de M. le duc d'Orléans, et dont j'ai
enrichi les miens. Il me parut ne tenir point du tout aux postes,
moyennant un traitement honorable.

Je mandai alors son retour au cardinal Dubois, par lequel ce serait à
lui et à M. le duc d'Orléans à voir avec Torcy ce qu'ils voudraient
faire pour lui, et je m'en retirai de la sorte. Dubois, content de voir
par là que Torcy consentirait à se démettre des postes, ne se soucia
point du comment, tellement que celui-ci obtint de M. le duc d'Orléans
tout ce qu'il lui proposa pour s'en défaire\,: tout se passa de bonne
grâce des deux côtés. Torcy eut quelque argent et soixante mille livres
de pension sa vie durant, assignée sur le produit des postes, dont vingt
mille livres pour sa femme après lui. Cela fut arrêté avant mon départ
et fort bien exécuté depuis.

Peu après la déclaration des mariages, la duchesse de Ventadour et
M\textsuperscript{me} de Soubise, sa petite-fille, avaient été nommées,
l'une gouvernante de l'infante, l'autre en survivance, et toutes deux
pour aller la prendre à la frontière et l'amener à Paris, au Louvre, où
elle devait être logée, et pou après la déclaration de mon ambassade, le
prince de Rohan, son gendre, fut nommé pour aller faire l'échange des
princesses sur la frontière avec celui que le roi d'Espagne y enverrait
de sa part pour la même fonction. Je n'avais jamais eu de commerce avec
eux, sans être mal ensemble. Toutes ces commissions espagnoles firent
que nous nous visitâmes avec la politesse convenable. J'ai oublié de
l'écrire plus tôt et plus en sa place.

\hypertarget{chapitre-xii.}{%
\chapter{CHAPITRE XII.}\label{chapitre-xii.}}

1721

~

{\textsc{Mon départ de Paris pour Madrid.}} {\textsc{- Je rencontre et
confère en chemin avec le duc d'Ossone.}} {\textsc{- Je passe et
séjourne à Ruffec, à Blaye et à Bordeaux, et y fais politesse aux
jurats.}} {\textsc{- Arrivée à Bayonne.}} {\textsc{- Adoncourt et
Dreuillet, commandant et évêque de Bayonne\,; quels.}} {\textsc{-
Pecquet père et fils\,; quels.}} {\textsc{- Impatience de Leurs Majestés
Catholiques de mon arrivée, qui la pressent par divers courriers.}}
{\textsc{- Audience de la reine douairière d'Espagne.}} {\textsc{- Son
logement.}} {\textsc{- Elle me fait traiter à dîner.}} {\textsc{- Son
triste état.}} {\textsc{- Adoncourt fort informé.}} {\textsc{- Passage
des Pyrénées.}} {\textsc{- Je vais voir Loyola.}} {\textsc{- Arrivée à
Vittoria.}} {\textsc{- Présent et députation de la province.}}
{\textsc{- Trois courriers l'un sur l'autre pour presser mon voyage.}}
{\textsc{- Je laisse mon fils aîné fort malade à Burgos, et poursuis ma
route sans m'arrêter.}} {\textsc{- Cause de l'impatience de Leurs
Majestés Catholiques.}} {\textsc{- Basse et impertinente jalousie de
Maulevrier.}} {\textsc{- Arrivée à Madrid, où je suis incontinent visité
des plus grands, sans exception de ceux à qui je devais la première
visite.}} {\textsc{- Je fais ma première révérence à Leurs Majestés
Catholiques et à leur famille.}} {\textsc{- Conduite très singulière et
tout opposée des ducs de Giovenazzo et de Popoli avec moi.}} {\textsc{-
Visite à Grimaldo, particulièrement chargé des affaires étrangères.}}
{\textsc{- Succès de cette visite.}} {\textsc{- Il connaît parfaitement
le cardinal Dubois.}} {\textsc{- Esquisse du roi d'Espagne\,; de la
reine d'Espagne\,; du marquis de Grimaldo.}} {\textsc{- Le roi et la
reine d'Espagne consentent, contre tout usage, de signer eux-mêmes le
contrat du futur mariage du roi et de l'infante.}} {\textsc{- Ils y
veulent des témoins, que je conteste et que je consens enfin.}}
{\textsc{- Signature des articles.}} {\textsc{- Office à Laullez.}}

~

Enfin je partis en poste le 23 octobre, ayant avec moi le comte de
Lorges, mes enfants, l'abbé de Saint-Simon et son frère, et quelque peu
d'autres. Le reste de la compagnie me joignit à Blaye, comme l'abbé de
Mathan, et à Bayonne avec M. de Céreste. Nous couchâmes à Orléans, à
Montrichard et à Poitiers. Allant de Poitiers coucher à Ruffec, je
rencontrai le duc d'Ossone à Vivonne. Je m'arrêtai pour le voir, et
sachant qu'il était à la messe, j'allai l'attendre à la porte de
l'église. Comme il sortit, ce fut des compliments, des accueils et des
embrassades\,; puis nous allâmes ensemble à la poste, où lui et moi
avions mis pied à terre, car il venait en poste aussi. Force compliments
aux portes, où je voulus, comme de raison, lui faire les honneurs de la
France. Nous montâmes dans une chambre où on nous laissa seuls et où
nous nous entretînmes une heure et demie. Il parlait mal français, mais
plus que suffisamment pour la conversation.

Après un renouvellement de compliments sur les mariages et le
renouvellement si étroit de l'union des deux couronnes, et les
politesses personnelles sur nos deux emplois, il entra le premier en
matière sur la joie des véritables Français et Espagnols, et le dépit
amer des mauvais. Je fus surpris de le trouver si bien informé de nos
cabales et de ce qu'on appelait la vieille cour. Sans avoir voulu nommer
personne, il m'en désigna plusieurs, et rien ne pouvait être plus clair
que ses plaintes contre des gens entièrement attachés au roi d'Espagne
jusqu'aux mariages, et qui, depuis ce moment, se déchaînaient et contre
les mariages et contre l'Espagne. Il me dit que M. le duc d'Orléans
avait plus d'ennemis de sa personne et de son gouvernement qu'il ne
pensait\,; que je l'avertisse d'y prendre garde, et il ajouta que, dans
l'état où en étaient les choses, on ne pouvait trop se hâter de part et
d'autre de les finir. Il me parla, mais sans désigner personne, de force
mouvements dans notre cour et à Paris pour retarder, dans le dessein de
gagner du temps, pour se donner celui de faire tout rompre, et qu'en
Espagne on sentait le même esprit et de l'intelligence\,; en même temps
me protesta qu'il n'y avait personne qui osât en parler au roi ni à la
reine d'Espagne d'une manière directe\,; que tous efforts, quand même il
en paraîtrait à Madrid, seraient inutiles\,; de la joie et de
l'empressement de Leurs Majestés Catholiques\,; des avantages
réciproques de cette réunion. Ce que j'exprime ici en peu de paroles en
produisit beaucoup parce qu'il fut d'abord énigmatique et fort réservé,
et que l'ouverture ne vint qu'à peine sur tout ce que je lui dis pour le
déboutonner. Hors ce qui, de ma part, me sembla nécessaire pour y
parvenir, et sans descendre en aucun particulier, on peut juger que
j'eus les oreilles plus ouvertes que la bouche. Seulement je l'exhortai
à s'ouvrir franchement et nominalement avec M. le duc d'Orléans, et je
tâchai de lui persuader qu'il ne pouvait rendre un plus grand service,
non seulement à ce prince, et dont il lui sût plus de gré, mais à Leurs
Majestés Catholiques, à qui désormais ses intérêts étaient unis, et par
amitié et pour la grandeur des deux couronnes. Il m'assura qu'il
s'expliquerait avec M. le duc d'Orléans comme il faisait avec moi\,;
mais quoique j'insistasse pour qu'il lui nommât et que je lui répondais
du secret, je n'en pus tirer parole. Aussi ne m'en donna-t-il pas de
négative\,; mais je sentis bien à ses discours là-dessus que la
politesse pour moi y avait plus de part que la volonté d'une entière
confidence sur un article si important mais si délicat. Nous nous
séparâmes de la sorte, avec force compliments, accolades et
protestations. Je ne pus, quoi que je pusse faire, l'empêcher de
descendre\,; mais, à mon tour, il ne put m'obliger de monter dans ma
berline, qu'il ne se fût retiré. Il était assez peu accompagné.

Ma berline cassa en arrivant à Couhé, terre appartenant à M. de Vérac\,;
il fallut y mettre un autre essieu. J'y fus donc plus de trois heures,
que j'employai à écrire à M. le duc d'Orléans et au cardinal Dubois le
récit de cette conférence et aller voir le château et le parc un moment.
Ces retardements me firent arriver sur le minuit à Ruffec, où j'étais
attendu de bonne heure par force noblesse de la terre et du pays, à qui
je donnai à dîner et à souper les deux jours que j'y séjournai. J'eus un
vrai plaisir d'y embrasser Puy-Robert qui était lieutenant-colonel du
régiment Royal-Roussillon du temps que j'y avais été capitaine. De
Ruffec, j'allai en deux jours à la Cassine, petite maison à quatre
lieues de Blaye, que mon père avait bâtie au bord de ses marais de Blaye
que je pris grand plaisir à visiter\,; j'y passai la veille et le jour
de la Toussaint, et le lendemain je me rendis de fort bonne heure à
Blaye, où je séjournai deux jours. J'y trouvai plusieurs personnes de
qualité, force noblesse du pays et des provinces voisines, et Boucher,
intendant de Bordeaux, beau-frère de Le Blanc, qui m'y attendaient,
auxquels je fis grande chère soir et matin pendant ce court séjour. Je
l'employai bien à visiter la place dedans et dehors, le fort de l'île et
celui de Médoc vis-à-vis Blaye, où je passai par un très fâcheux temps.
Mais je les voulais voir, et j'y menai mon fils qui avait la survivance
de mon gouvernement. Nous passâmes à Bordeaux par un si mauvais temps,
que tout le monde me pressait de différer, mais on ne m'avait permis que
ce peu de séjour, que je ne voulus pas outrepasser. Boucher avait amené
son brigantin magnifiquement équipé, et tout ce qu'il fallait de barques
pour le passage de tout ce qui m'accompagnait, et de tout ce qui était
venu me voir à Blaye dont la plupart passèrent à Bordeaux avec nous. La
vue du port et de la ville me surprit avec plus de trois cents bâtiments
de toutes nations rangés sur deux lignes sur mon passage, avec toute
leur parure et grand bruit de leur canon et de celui du château
Trompette.

On connaît trop Bordeaux pour que je m'arrête à décrire ce spectacle\,;
je dirai seulement qu'après le port de Constantinople, la vue de
celui-ci est en ce genre ce qu'on peut admirer de plus beau. Nous
trouvâmes force compliments et force carrosses au débarquement, qui nous
conduisirent chez l'intendant, où les jurats\footnote{C'est-à-dire les
  magistrats municipaux qui portaient à Bordeaux le nom de jurats, comme
  ailleurs ceux d'échevins, de capitouls, etc.} de Bordeaux vinrent me
complimenter en habit de cérémonie. Comme ces messieurs sont les uns de
qualité et les autres considérables, et que cette jurade est extrêmement
différente en tout des autres corps de ville, je me tournai vers
l'intendant après leur avoir répondu, et je le priai de trouver bon que
je les conviasse de souper avec nous\,; ils me parurent sensibles à
cette politesse à laquelle ils ne s'attendaient pas\,; allèrent quitter
leurs habits, et revinrent souper. Il n'est pas possible de faire une
plus magnifique chère, ni plus délicate que celle que l'intendant nous
fit soir et matin, ni faire mieux les honneurs de la ville et de leur
logis que nous les firent l'intendant et sa femme les trois jours que
j'y séjournai, n'ayant pu y être moins pour l'arrangement du voyage.
L'archevêque et le premier président n'y étaient point\,; le parlement
était en vacance. Néanmoins je vis le palais et ce qu'il y avait à voir
dans la ville. Quoiqu'on me dégoûtât de voir l'hôtel de ville qui est
vilain, je persistai à y aller\,; je voulais faire une autre civilité
aux jurats, sans conséquence\,; ils s'y trouvèrent\,; je leur dis que
c'était beaucoup moins la curiosité qui m'amenait dans un lieu où on
m'avait averti que je ne trouverais rien qui méritât d'être vu, que le
désir d'une occasion de leur rendre à tous une visite, ce qui me parut
leur avoir plu extrêmement.

Enfin, après avoir bien remercié M. et M\textsuperscript{me} Boucher,
nous partîmes, traversâmes les grandes landes, et arrivâmes à Bayonne,
où nous mîmes pied à terre chez d'Adoncourt qui y commandait très
dignement, et y était adoré en servant parfaitement le roi. Mes enfants
et moi logeâmes chez lui, et tout mon monde dans le voisinage. Le
changement de voitures pour nous et pour le bagage nous y retint quatre
jours, pendant lesquels rien ne se peut ajouter aux soins d'Adoncourt, à
sa politesse aisée et sans compliments, et à sa chère soir et matin,
propre, grande, excellente. Il était venu accompagné d'officiers une
lieue au-devant de nous. J'étais dès lors monté à cheval. L'artillerie,
les compliments, il fallut essuyer cela comme à Bordeaux, et, pour ne le
pas répéter, ce fut la même chose au retour, excepté à Blaye où je le
défendis. Dreuillet, évêque de Bayonne, me vint voir, puis dîner avec
nous et ce qu'il y avait de plus principal dans la ville, mais en fort
petit nombre. Je fus le lendemain chez ce prélat qui était pieux,
savant, et toutefois de bonne compagnie, et parfaitement aimé dans son
diocèse et dans tout le pays. J'allai voir la citadelle, les forts, et
tout ce qu'il y avait qui méritât quelque curiosité.

Pecquet, qui avait été longtemps premier commis de M. de Torcy, et qui,
pour dire le vrai, avait fait toutes les affaires étrangères tant que le
maréchal d'Huxelles les avait eues, m'avait prié que son fils vint en
Espagne et fût chez moi, et il avait pris les devants quelques jours
auparavant. Je trouvai un courrier de Sartine arrivé à Bayonne une heure
avant moi. Sartine me mandait du 5, à onze heures du soir, que le roi
d'Espagne, ayant appris que Pecquet était arrivé la veille, était très
fâché de mon retardement, d'où résultait celui de l'échange des
princesses qui essuieraient le plus mauvais temps de l'hiver. Que Leurs
Majestés Catholiques n'attendaient que mon arrivée pour se mettre en
chemin pour Burgos, jusqu'où elles avaient résolu de conduire l'infante,
et qu'elles désiraient extrêmement que je pressasse ma marche. Sartine
tacha inutilement de les détourner de ce voyage. Il ajouta de lui-même
que Leurs Majestés Catholiques seraient sensiblement mortifiées, si le
départ de M\textsuperscript{lle} de Montpensier se retardait d'un moment
du jour fixé, et que le marquis de Grimaldo lui envoyait à l'heure qu'il
m'écrivait un courrier par ordre du roi d'Espagne pour me le dépêcher et
apporter ma réponse.

Je répandis à Sartine que je le priais de représenter à Leurs Majestés
Catholiques que de ma part je n'avais rien oublié ni n'oublierais pour
hâter mon voyage. Que les circonstances des précautions à l'égard de la
peste avaient empêché mes équipages de passer ni rien pu faire préparer
sur la route pour la diligenter, parce que les passeports d'Espagne
n'étaient arrivés que le 29 du mois dernier, et que ces passeports étant
pour le chemin qui passe à Vittoria, plus long que celui de Pampelune,
que je voulais prendre, me retardaient encore\,; qu'au surplus mon
arrivée à Madrid plus ou moins avancée ne pouvait rien influer sur le
départ de M\textsuperscript{lle} de Montpensier fixé au 15 de ce mois\,;
que tout le désir du roi et de M. le duc d'Orléans de l'avancer était
inutile, par l'impossibilité que les préparatifs pussent être prêts plus
tôt. Que de Paris à la frontière elle mettrait cinquante jours par la
difficulté des chemins et la quantité d'équipages, d'où il résultait que
de Madrid à la frontière, le chemin étant plus court d'un tiers,
l'infante ne pouvait être pressée de partir pour arriver juste au lieu
de l'échange, et que, par conséquent, j'aurais tout le temps nécessaire
pour m'acquitter de toutes les fonctions préalables à son départ, qui
n'en pourra être retardé d'un seul moment.

Le 9, lendemain de mon arrivée à Bayonne, j'envoyai faire compliment à
la duchesse de Liñarez, camarera-mayor de la reine douairière d'Espagne,
et la prier de lui demander audience, pour moi, pour l'après-dînée. Je
reçus en réponse un compliment de la reine. Ses carrosses vinrent me
prendre et me conduisirent chez elle\,: véritablement je fus étonné en y
arrivant. Elle s'était retirée depuis assez longtemps dans une maison de
campagne fort proche de la ville qui n'avait que deux fenêtres de face
sur une petite cour et guère plus de profondeur. De la cour, je
traversai un petit passage et j'entrai dans une pièce plus longue que
large, très communément meublée, qui avait vue sur un beau et grand
jardin. Je trouvai la reine qui m'attendait, accompagnée de la duchesse
de Liñarez et de très peu de personnes. Je lui fis le compliment du roi
et lui présentai sa lettre\,: on ne peut répondre plus poliment qu'elle
fit à l'égard du roi, ni avec plus de bonté pour moi. La conversation
fut sur la joie des mariages, le temps de l'échange et sur mon voyage.
Elle était, debout, sans siège derrière elle\,; je ne me couvris point,
et n'en fis pas même le semblant. La duchesse de Liñarez et d'Adoncourt
entrèrent seuls un peu dans la conversation. Je lui présentai mes
enfants et ces messieurs qui étaient avec moi à qui elle dit quelque
chose, cherchant à leur parler à tous avec un air d'attention et de
bonté et en fort bon français. Elle était fort grande, droite, très bien
faite, de grand air, de bonne mine, qui laissait voir qu'elle avait eu
de la beauté. Elle me demanda beaucoup des nouvelles de Madame. Tout son
habillement était noir et sa coiffure avec un voile, mais qui montrait
des cheveux, et sa taille paraissait aussi. Ce vêtement n'était ni
français ni espagnol, avec une longue queue dont la duchesse de Liñarez
tenait le bout, mais fort lâche. C'était un habit de veuve, mais mitigé
avec une longue et large attache devant le haut du corps, de très beaux
diamants. Pour la duchesse de Liñarez, son habit m'effraya\,: il était
tout à fait de veuve et ressemblait en tout à celui d'une religieuse. Je
ne dois pas oublier que je présentai aussi à la reine les compliments et
une lettre de M. le duc d'Orléans, à quoi elle répondit avec une grande
politesse.

Au sortir de l'audience, elle me fit inviter à dîner, pour le lendemain,
dans une maison de Bayonne où le gros de ses officiers demeurait et où
elle a aussi logé. J'y allai, sur l'exemple du comte de San Estevan del
Puerto, allant au congrès de Cambrai, et tout à l'heure, du duc d'Ossone
venant en France. Le sieur de Bruges, qui était chef de la maison de la
reine douairière, fit les honneurs du festin très bon et très
magnifique, où se trouva l'évêque de Bayonne, d'Adoncourt, et tout ce
qui m'accompagnait de principal. J'eus une seconde audience de la reine
pour la remercier du repas et prendre congé d'elle. La conversation fut
plus longue et plus familière que la première fois\,; elle finit par
m'exposer le très triste état où elle se trouvait, faute de tout
payement d'Espagne depuis des années, et me prier d'en parler à Leurs
Majestés Catholiques et de lui procurer quelque secours sur ce qui lui
était si considérablement dû.

J'appris d'Adoncourt plusieurs petits détails touchant les efforts
tentés à Paris et à la cour pour faire différer les mariages dans la vue
de profiter de ce délai pour tâcher de les rompre, mais qui ne me
donnèrent pas grande lumière là-dessus. Ce que je démêlai seulement fut
qu'Adoncourt, qui avait de grands commerces en Espagne pour tenir la
cour bien avertie de tout, et qui y était même en liaison avec plusieurs
seigneurs, avait eu plus de part que moi en la confidence du duc
d'Ossone qui lui avait nommé des personnages de cette intrigue, tant de
notre cour que de celle d'Espagne. Je l'exhortai à en instruire le
cardinal Dubois auquel je le mandai.

Passant les Pyrénées, je quittai, avec la France, les pluies et le
mauvais temps qui ne m'avaient pas quitté jusque-là, et trouvai un ciel
pur et une température charmante, avec des échappées de vues et des
perspectives qui changeaient à tout moment, qui ne l'étaient pas moins.
Nous étions tous montés sur des mules dont le pas est grand et doux. Je
me détournai en chemin à travers de hautes montagnes pour aller voir
Loyola, lieu fameux par la naissance de saint Ignace, situé tout seul
près d'un ruisseau assez gros, dans une vallée fort étroite, dont les
montagnes de roche qui la serrent des deux côtés doivent faire une
glacière quand elles sont couvertes de neige et une tourtière en été.
Nous trouvâmes là quatre ou cinq jésuites, fort polis et fort entendus,
qui prenaient soin du bâtiment prodigieux qui y était entrepris pour
plus de cent jésuites et une infinité d'écoliers, dans le dessein de
faire de cette maison un noviciat, un collège, une maison professe
{[}pour{]} qu'elle servit à tous les usages auxquels sont destinés leurs
différentes maisons et {[}fût{]} le chef-lieu de leur compagnie.

Ils nous firent voir le petit logis primitif du père de saint Ignace,
qui est une maison de cinq ou six fenêtres, qui n'a qu'un
rez-de-chaussée pour le ménage, un étage au-dessus et plus haut un
grenier. Ce serait tout au plus le louis d'un curé, et {[}cela{]} ne
ressembla jamais en rien à un château. Nous vîmes la chambre où saint
Ignace, blessé à la guerre, fut longtemps couché, et eut sa fameuse
révélation touchant la compagnie dont il devait être l'instituteur\,; et
l'écurie où sa mère voulut aller accoucher de lui, qui est au-dessous,
par dévotion pour l'étable de Bethléem. Rien de plus bas, de plus
étroit, de plus écrasé que ces deux pièces\,; rien aussi de si
éblouissant d'or qui y brille partout. Il y a un autel dans chacune des
deux où le saint sacrement repose, et ces deux autels sont de la
dernière magnificence.

La maison des jésuites qu'ils allaient détruire pour leur immense
bâtiment était fort peu de chose et pour loger au plus une douzaine de
jésuites. L'église nouvelle était presque achevée, en rotonde, d'une
grandeur et d'une hauteur qui surprend, avec des autels pareils entre
eux, tout autour en symétrie. L'or, la peinture, la sculpture, les
ornements de toutes les sortes et les plus riches, répandus partout avec
un art prodigue, mais sage\,; une architecture correcte et admirable,
les marbres les plus exquis, le jaspe, le porphyre, le lapis, les
colonnes unies, torses, cannelées, avec leurs chapiteaux et leurs
ornements de bronze doré, un rang de balcons, entre chaque autel, et de
petits degrés de marbre pour y monter et les cages incrustées, les
autels et ce qui les accompagne admirables. En un mot, un des plus
superbes édifices de l'Europe, le mieux entendu et le plus
magnifiquement orné. Nous y primes le meilleur chocolat dont j'aie
jamais goûté, et après quelques heures de curiosité et d'admiration,
nous regagnâmes notre route et notre gîte, fort tard et avec beaucoup de
peine.

Nous arrivâmes le 15 à Vittoria où je trouvai la députation de la
province qui m'attendait avec un grand présent d'excellent vin rancio\,;
c'étaient quatre gentilshommes considérables qui étaient à la tête des
affaires du pays. Je les conviai à souper, et le lendemain à déjeuner
avec nous\,: ils parlaient français, et je fus surpris de voir des
Espagnols si gais et de si bonne compagnie à table. La joie du sujet de
mon voyage éclata partout où je passai en France et en Espagne et me fit
bien recevoir. On se mettait aux fenêtres et on bénissait mon voyage. À
Salinas, entre autres, où je passais sans m'arrêter, des dames qui, à
voir leur maison et elles-mêmes aux fenêtres, me parurent de qualité, me
demandèrent de si bonne grâce de voir un moment celui qui allait
conclure le bonheur de l'Espagne, que je crus qu'il était de la
galanterie de monter chez elles\,; elles m'en parurent ravies, et j'eus
toutes les peines du monde à m'en débarrasser pour continuer mon chemin.

Je trouvai à Vittoria un courrier de Sartine pour me presser d'arriver,
mais dont la date était antérieure au retour de son courrier de
Bayonne\,; mais, étant le 17, à cinq heures du matin, prêt à partir de
Miranda d'Ebro, arriva un autre courrier de Sartine, qui me mandait que
les raisons, quoique sans réplique, que je lui avais écrites de Bayonne,
n'avaient point ralenti l'extrême empressement de Leurs Majestés
Catholiques, sur quoi je le priai de me faire tenir des relais le plus
qu'il pourrait, à quelque prix que ce fût, pour presser mon voyage tant
qu'il me serait possible.

J'arrivai le 18 à Burgos, où je comptais séjourner, pour voir au moins
un jour ce que deviendrait une fièvre assez forte qui avait pris à mon
fils aîné, qui m'inquiétait beaucoup, en attendant que mes relais
pussent se préparer\,; mais Pecquet arriva pour presser de nouveau ma
marche, et si vivement qu'il fallut abandonner mon fils et presque tout
mon monde. L'abbé de Mathan voulut bien demeurer avec lui pour en
prendre soin et ne le point quitter.

J'appris par Pecquet la cause d'une si excessive impatience. C'est que
la reine, qui n'aimait point le séjour de Madrid, pétillait d'en sortir
pour aller à Lerma, où on l'avait assurée qu'elle trouverait une chasse
fort abondante. Pecquet me dit que M. de Grimaldo et Sartine n'avaient
rien oublié pour rompre, au moins différer ce voyage, mais que
l'impatience avait été nourrie et augmentée par Maulevrier, enragé de
voir arriver un ambassadeur de naissance et de dignité personnelle, et
qui n'avait pu s'empêcher de dire qu'il l'aurait plus patiemment
souffert si c'eût été le duc de Villeroy, La Feuillade ou le prince de
Rohan. Ce seigneur Andrault, si délicat pour soi, ne cherchait pas les
amis de M. le duc d'Orléans par le désir de ces messieurs\,; et, outre
qu'il s'oubliait bien lui-même, il perdait promptement la mémoire qu'il
avait été laissé à mon choix de lui donner ou non le caractère
d'ambassadeur, que par conséquent il me devait, et qui en cette occasion
surtout l'honorait fort au delà de ses espérances. Toutefois je résolus
de n'en faire aucun semblant et de vivre avec lui comme si j'eusse
ignoré ce que je venais d'apprendre\,; mais je le mandai au cardinal
Dubois.

Je partis donc de Burgos le 19 avec mon second fils, le comte de Lorges,
M. de Céreste (ces deux derniers ne vinrent qu'un peu après ensemble),
l'abbé de Saint-Simon, son frère, le major de son régiment et très peu
de domestiques. Nous trouvâmes peu de relais et mal établis\,; marchâmes
jour et nuit, sans nous coucher, jusqu'à Madrid, nous servant des
voitures des corrégidors\footnote{Ce nom, qui signifie correcteur,
  désignait les magistrats dans les villes, où il n'y avait ni
  gouverneur, ni tribunal royal. Les corrégidors étaient à la fois
  juges, chefs du corps municipal et chargés de l'administration
  financière et de la police.}, où nous pûmes, tellement que je fus
obligé de faire les dernières douze lieues à cheval en poste, qui en
valent le double d'ici. Nous arrivâmes de la sorte à Madrid le vendredi
21, à onze heures du soir. Nous trouvâmes à l'entrée de la ville, qui
n'a ni murailles, ni portes, ni barrières, ni faubourgs, des gens en
garde qui demandèrent qui nous étions et d'où nous venions, et qu'on y
avait mis exprès pour être avertis du moment de mon arrivée. Comme
j'étais fort fatigué d'avoir toujours marché sans arrêter depuis Burgos,
et qu'il était fort tard, je répondis que nous étions des gens de
l'ambassadeur de France, qui arriverait le lendemain. Je sus après que,
par le calcul de Sartine, de Grimaldo, et de Pecquet arrivé devant moi,
ils avaient tous compté que je ne serais à Madrid que le 22.

Dès que je fus arrivé chez moi, j'envoyai chercher Sartine pour prendre
langue avec lui, fermai bien ma porte, et donnai ordre de dire à
quiconque pourrait venir qu'on ne m'attendait que le lendemain. Je sus
par Sartine que, grâces à ses précautions et aux peines que le duc de
Liria en avait bien voulu prendre, j'aurais le surlendemain de quoi me
mettre en public, et que huit jours après je serais en état d'avoir tous
mes équipages et de prendre mon audience solennelle. Cependant tout ce
qui n'était point destiné à demeurer à Burgos avec mon fils aîné arriva
en poste à la file, en sorte que personne et que rien ne me manqua. Le
lendemain matin samedi 22, de bonne heure, Sartine accompagna mon
secrétaire chez le marquis de Grimaldo, tandis que j'envoyai faire les
messages accoutumés quand on arrive aux ministres des cours étrangères.
Grimaldo, surpris et fort aise de mon arrivée qu'il n'attendait que le
soir de ce jour, fut au palais le dire à Leurs Majestés Catholiques,
qui, dans leur impatience de partir, furent ravies. Du palais, Grimaldo
vint chez moi au lieu d'attendre ma première visite\,: il me trouva avec
Maulevrier, le duc de Liria et quelques autres.

Ce fut apparemment sur l'exemple de Grimaldo que les trois charges
vinrent aussi chez moi\,; le marquis de Santa Cruz, majordome-major de
la reine, et très -bien avec elle\,; le duc d'Arcos\,; le marquis de
Bedmar, président du conseil de guerre et de celui des ordres, et
chevalier de celui du Saint-Esprit\,; le duc de Veragua président du
conseil des Indes, tous grands d'Espagne\,; l'archevêque de Tolède, le
grand inquisiteur, évêque de Barcelone, presque tous ayant le vain titre
de conseillers d'État. La plupart vinrent le matin, les autres
l'après-dînée, et les jours suivants tout ce qu'il y eut à Madrid de
grands, de seigneurs et de ministres étrangers. Le gouverneur du conseil
de Castille, qui ne visite jamais personne ni n'envoie, si ce n'est pour
affaire, envoya me complimenter, quoique je n'eusse point envoyé chez
lui, par la raison que je dirai lorsque je parlerai de cette première
charge d'Espagne. Castellar, secrétaire d'État pour la guerre, vint
aussi chez moi ce même jour. Le duc de Liria se disposait à venir une
lieue au-devant de moi avec Valouse et Sartine, et de son côté
Maulevrier avec Robin.

Grimaldo me témoigna la joie de Leurs Majestés Catholiques de mon
arrivée, et après m'avoir fait les plus gracieux compliments pour
lui-même, me donna le choix de leur part de les aller saluer ce même
matin ou dans l'après-dînée. Je crus l'empressement mieux séant, et j'y
allai avec lui sur-le-champ dans le carrosse de Maulevrier qui y vint
aussi. De cette sorte fut levée toute difficulté sur la première visite,
à l'égard de tous ceux à qui elle était due de ma part, et de ceux qui
la pouvaient prétendre, dont j'eus le sang bien rafraîchi.

Nous arrivâmes au palais comme le roi était sur le point de revenir de
la messe, et nous l'attendîmes dans le petit salon qui est entre le
salon des Grands et celui des Miroirs, dans lequel personne n'entre que
mandé. Peu de moments après, le roi vint par le salon des Grands.
Grimaldo l'avertit comme il entrait dans le petit salon\,: il vint à moi
aussitôt, précédé et suivi d'assez de courtisans, mais qui ne
ressemblaient pas à la foule des nôtres. Je lui fis ma profonde
révérence\,; il me témoigna sa joie de mon arrivée, demanda des
nouvelles du roi, de M. le duc d'Orléans, de mon voyage, et des
nouvelles de mon fils aîné qu'il avait su être demeuré malade à Burgos,
puis entra seul dans le cabinet des Miroirs. À l'instant je fus
environné de toute la cour, avec des compliments et des témoignages de
joie des mariages et de l'union des deux couronnes. Grimaldo et le duc
de Liria me nommaient les seigneurs, qui presque tous parlaient
français, aux civilités infinies desquels je tâchai de répondre par les
miennes.

Un demi-quart d'heure après que le roi fut rentré, il m'envoya appeler.
J'entrai seul dans le salon des Miroirs, qui est fort vaste, bien moins
large que long. Le roi, et la reine à sa gauche, étaient presque au fond
du salon, debout, et tout joignant l'un l'autre. J'approchai avec trois
profondes révérences, et je remarquerai une fois pour toutes que le roi
ne se couvre jamais qu'aux audiences publiques, et quand il va et vient
de la messe en chapelle, terme que j'expliquerai en son lieu. L'audience
dura demi-heure (car c'est toujours eux qui congédient) à témoigner leur
joie, leurs désirs, leur impatience, avec un épanchement infini, très
bien aussi sur M. le duc d'Orléans et sur le désir de rendre
M\textsuperscript{lle} de Montpensier heureuse sur un portrait d'elle et
un autre du roi qu'ils me montrèrent à la fin de la conversation, où la
reine parla bien plus que le roi dont néanmoins la joie éclatait avec
ravissement, ils me firent l'honneur de me dire qu'ils me voulaient
faire voir les infants, et me commandèrent de les suivre. Je traversai
seul à leur suite la chambre et le cabinet de la reine, une galerie
intérieure, où il se trouva deux dames de service et deux ou trois
seigneurs en charge, qui, apparemment, avaient été avertis, comme je
l'expliquerai ailleurs, et passai avec cette petite suite toute cette
galerie, au bout de laquelle était l'appartement des infants. Je n'ai
point vu de plus jolis enfants, ni mieux faits que don Ferdinand et don
Carlos, ni un plus beau maillot que don Philippe. Le roi et la reine
prirent plaisir à me les faire regarder, et à les faire tourner et
marcher devant moi de fort bonne grâce. Ils entrèrent après chez
l'infante, où je tâchai d'étaler le plus de galanterie que je pus. En
effet, elle était charmante, avec un petit air raisonnable et point
embarrassé. La reine me dit que l'infante commençait à apprendre assez
bien le français\,; et le roi, qu'elle oublierait bientôt l'Espagne.
«\,Oh\,! s'écria la reine, non seulement l'Espagne, mais le roi et moi,
pour ne s'attacher qu'au roi son mari\,;» sur quoi je tâchai de ne pas
demeurer muet. Je sortis de là à la suite de Leurs Majestés Catholiques,
que je suivis à travers cette petite galerie et leur appartement. Elles
me congédièrent aussitôt avec beaucoup de témoignages de bonté\,; et,
rentré dans le salon avec tout le monde, j'y fus environné de nouveau,
avec force compliments.

Peu de moments après, le roi me fit rappeler pour voir le prince des
Asturies, qui était avec Leurs Majestés dans ce même salon des Miroirs.
Je le trouvai grand, et véritablement fait à peindre\,; blond et de
beaux cheveux, le teint blanc avec de la couleur, le visage long, mais
agréable, les yeux beaux, mais trop près du nez\,: je lui trouvai
beaucoup de grâce et de politesse. Il me demanda fort des nouvelles du
roi, puis de M. le duc d'Orléans et de M\textsuperscript{lle} de
Montpensier, et du temps de son arrivée.

Leurs Majestés Catholiques me témoignèrent beaucoup de satisfaction de
ma diligence, me dirent qu'ils avaient retardé leur voyage pour me
donner le temps de me mettre en état de prendre mes audiences\,; qu'une
seule suffirait pour faire la demande de l'infante et l'accorder\,; que
les articles pourraient être signés la veille de cette audience, et
l'après-dînée de ce jour de l'audience signer le contrat. Ensuite ils me
demandèrent quand tout serait prêt\,; je leur dis que ce serait le jour
qu'il leur plairait, parce que tout ce que je faisais préparer n'étant
que pour leur en faire ma cour, je croirais y mieux réussir avec moins
pour ne pas retarder leur départ, que de différer pour étaler tout ce à
quoi on travaillait encore. Il me parut que cette réponse leur plut
fort, mais elles ne voulurent jamais déterminer le jour, sur quoi enfin
je leur proposai le mardi suivant. La joie de cette promptitude parut
sur leur visage, et {[}ils{]} me témoignèrent m'en savoir beaucoup de
gré. Là-dessus, le roi se recula un peu, parla bas à la reine, et elle à
lui, puis se rapprochèrent du prince des Asturies et de moi, et fixèrent
leur départ au jeudi suivant, 27 du mois. Tout de suite ils me permirent
non seulement de les y suivre, mais m'ordonnèrent de les suivre de près,
parce que l'incommodité des logements ne permettait qu'à peine aux
officiers de service les plus nécessaires de les accompagner dans la
route. Ce fut la fin de toute cette audience.

Maulevrier seul me ramena chez moi, où je trouvai don Gaspard Giron,
l'ancien des quatre majordomes, qui s'était emparé de ma maison avec les
officiers du roi, qui me traita magnifiquement, avec beaucoup de
seigneurs qu'il avait invités, et fit toujours les honneurs\,; ce qui,
quoi que je pusse faire, dura jusqu'au mercredi suivant inclus, avec un
carrosse du roi toujours à ma porte pour me servir\,; mais à ce dernier
égard, j'obtins enfin que cela ne durerait que trois jours, pendant
lesquels il fallut toujours m'en servir\,; il était à quatre mules, avec
un cocher du roi et quelques-uns de ses valets de pied en livrée. Ce
traitement de table et de carrosse est une coutume à l'égard des
ambassadeurs extraordinaires. Si je m'étends sur les honneurs que j'ai
reçus, c'est un récit que je dois à l'instruction et à la curiosité,
plus encore à la joie extrême du sujet de cette ambassade qui fit passer
par-dessus toutes règles, comme pour les premières visites, et en bien
d'autres choses, ainsi qu'aux accueils et aux empressements que je reçus
de tout le monde, et qui furent toujours les mêmes tant que je demeurai
en Espagne.

La conduite de deux seigneurs principaux me surprit également par leur
opposition à mon égard. Cellamare, qui avait pris le nom de duc de
Giovenazzo depuis la mort de son père, et qui était grand écuyer de la
reine, surpassa toute cette cour en empressements pour moi et chez moi,
et au palais, en protestations de joie de l'union et des mariages,
d'attachement et de reconnaissance des bons traitements qu'il avait
reçus en France, me conjura que le roi et M. le duc d'Orléans en fussent
informés, et se répandit assez inconsidérément en tendresse pour le
maréchal de Villeroy, auquel il me dit qu'il voulait écrire, ainsi qu'au
roi et à M. le duc d'Orléans. Je reçus toutes ces rares effusions aussi
poliment que me le permit la plus extrême surprise, après tout ce qu'il
avait brassé à Paris et ce qui en était suivi pour lui-même. Ces mêmes
empressements continuèrent tant que je fus en Espagne, mais il ne mangea
pas une seule fois chez moi. Aussi, ne l'en priai-je qu'une de devoir,
le jour de la couverture\footnote{Voy. sur cette cérémonie, le t. III,
  p.~261 et suiv.} de mon fils.

Son contradictoire fut le duc de Popoli, capitaine général, grand maître
de l'artillerie, chevalier du Saint-Esprit et gouverneur du prince des
Asturies, dont je reçus force compliments au palais où je ne le
rencontrais guère, et qui ne vint et n'envoya chez moi qu'une fois. On
verra aussi comment j'en usai avec lui.

Ce même jour, j'allai voir le marquis de Grimaldo, particulièrement
chargé des affaires étrangères. Il entendait parfaitement le français,
mais il ne le voulait pas parler. Orondaya, son principal commis, nous
servit toujours d'interprète\,; on ne peut en recevoir plus de
politesses\,; je fus étonné au dernier point qu'il me rapportât tous les
efforts que j'avais faits auprès de M. le duc d'Orléans pour le
détourner de la guerre qu'il fit à l'Espagne en faveur des Anglais, et
je n'imagine pas comment Laullez l'avait su, qui l'avait mandé fort tôt
après qu'il fut arrivé à Paris. Je présentai à Grimaldo les copies des
lettres que je devais rendre. Ce fut un long combat de civilité entre
nous, lui de ne les vouloir pas prendre, moi d'insister\,; mais je m'y
opiniâtrai tellement qu'enfin il les reçut. J'eus pour cela mes raisons,
je voulais faire passer la lettre de M. le duc d'Orléans au prince des
Asturies, avec le traitement de frère\,; je ne voulais pas m'y exposer
témérairement\,; il fallait donc, pour ne rien hasarder, que Grimaldo en
eût la copie et point de celle où le traitement de frère était omis,
qu'il n'était temps de produire qu'au cas que Grimaldo ne voulût point
passer l'autre\,; c'est ce qui me fit tant insister\,; heureusement je
n'en entendis plus parler, et sur cette confiance, je rendis celle où
était le traitement de frère le lendemain au prince des Asturies. Elle
passa doux comme lait, et j'eus le plaisir de renvoyer aussitôt après à
M. le duc d'Orléans celle où le traitement de frère n'était pas employé.

Restait l'embarras de n'avoir point de lettre pour l'infante. J'en fis
la confidence à Grimaldo, qui se mit à rire et me dit qu'il m'en
tirerait et ferait que, lorsque le lendemain j'irais à l'audience de
l'infante, la gouvernante me viendrait dire dans l'antichambre qu'elle
dormait et m'offrirait de la réveiller, ce que je refuserais, après quoi
je n'irais plus chez elle, que la lettre du roi pour elle ne me fût
arrivée, et que j'irais lui remettre alors sans façon et sans audience.
Cela commença à nous ouvrir un peu l'un avec l'autre sur le cardinal
Dubois, et je vis dans la suite qu'il le connaissait tel qu'il était,
aussi parfaitement que nous. La journée finit fort tard par la
communication que je donnai à Maulevrier de tout ce qui m'avait été
remis touchant l'ambassade, et je lui remis aussi les pleins pouvoirs
qui lui donnaient le caractère d'ambassadeur.

Lui et moi avions, dès auparavant, agité ensemble la difficulté qui se
rencontrait dans le préambule du contrat de mariage du roi, qui
s'expliquait de manière que ce n'était point le roi et la reine
d'Espagne qui contractaient, mais des commissaires, nommés par eux, qui
stipulaient en leur nom, tant pour Leurs Majestés Catholiques, que pour
l'infante, ce qui nous aurait mis dans la nécessité de nommer aussi des
commissaires dont nous n'avions pas pouvoir. J'avais donc prié
Maulevrier de me venir trouver chez Grimaldo pour nous en expliquer avec
lui. Il nous représenta que telle était la coutume en Espagne\,; que nos
deux dernières reines avaient été mariées de cette façon, et qu'encore
qu'au dernier de ces deux mariages, le roi et le roi d'Espagne Philippe
IV fussent en personne sur la frontière, le roi Philippe IV n'en avait
pourtant pas signé lui-même le contrat, à quoi Grimaldo nous pressa fort
de nous conformer et de donner des commissaires\,; nous insistâmes sur
notre défaut de pouvoir, sur la longueur où jetterait la nécessité de
dépêcher un courrier et d'en attendre le retour, enfin sur ce que le roi
comptait si fort sur la signature de Leurs Majestés Catholiques, que
cela même était porté précisément dans nos instructions. Cette
discussion fut beaucoup moins une dispute qu'une conversation fort
polie, à la fin de laquelle Grimaldo, qui m'adressa toujours la parole,
me dit que le roi d'Espagne avait tant de désir de complaire au roi et
de voir la fin d'une affaire si désirée, qu'il espérait qu'il voudrait
bien passer pardessus la coutume d'Espagne et signer lui-même avec la
reine\,; qu'il allait leur en rendre compte tout sur-le-champ et nous
informerait le lendemain dimanche 23, de la réponse, jour auquel je
devais avoir le matin ma première audience particulière et rendre les
lettres dont j'étais chargé. Mais avant de passer outre, je crois
nécessaire de dire quelque chose du roi et de la reine d'Espagne et du
marquis de Grimaldo.

Le premier coup d'oeil, lorsque je fis ma première révérence au roi
d'Espagne en arrivant, m'étonna si fort, que j'eus besoin de rappeler
tous mes sens pour m'en remettre. Je n'aperçus nul vestige élu duc
d'Anjou, qu'il me fallut chercher dans son visage fort allongé, changé,
et qui disait encore beaucoup moins que lorsqu'il était parti de France.
Il était fort courbé, rapetissé, le menton en avant, fort éloigné de sa
poitrine, les pieds tout droits, qui se touchaient, et se coupaient en
marchant, quoiqu'il marchât vite et les genoux à plus d'un pied l'un de
l'autre. Ce qu'il me fit l'honneur de me dire était bien dit, mais si
l'un après l'autre, les paroles si traînées, l'air si niais, que j'en
fus confondu. Un justaucorps, sans aucune sorte de dorure, d'une manière
de bure brune, à cause de la chasse où il devait aller, ne relevait pas
sa mine ni son maintien. Il portait une perruque nouée, jetée par
derrière, et le cordon bleu par-dessus son justaucorps, toujours et en
tout temps, et de façon qu'on ne distinguait pas sa Toison qu'il portait
au cou avec un cordon rouge, que sa cravate et son cordon bleu cachaient
presque toujours. Je m'étendrai ailleurs sur ce monarque.

La reine, que je vis un quart d'heure après, ainsi qu'il a été rapporté
plus haut, m'effraya par son visage marqué, couturé, défiguré à l'excès
par la petite vérole\,; le vêtement espagnol d'alors pour les dames,
entièrement différent de l'ancien, et de l'invention de la princesse des
Ursins, est aussi favorable aux dames jeunes et bien faites, qu'il est
fâcheux pour les autres dont l'âge et la taille laissent voir tous les
défauts. La reine était faite au tour, maigre alors, mais la gorge et
les épaules belles, bien taillée, assez pleine et fort blanche, ainsi
que les bras et les mains\,; la taille dégagée, bien prise, les côtés
longs, extrêmement fine et menue par le bras, un peu plus élevée que la
médiocre\,; avec un léger accent italien, {[}elle{]} parlait très bien
français, en bons termes, choisis, et sans chercher, la voix et la
prononciation fort agréables. Une grâce charmante, continuelle,
naturelle, sans la plus légère façon, accompagnait ses discours et sa
contenance, et variait suivant qu'ils variaient. Elle joignait un air de
bonté, même de politesse, avec justesse et mesure, souvent d'une aimable
familiarité, à un air de grandeur et à une majesté qui ne la quittaient
point. De ce mélange, il résultait que, lorsqu'on avait l'honneur de la
voir avec quelque privance, mais toujours en présence du roi, comme je
le dirai ailleurs, on se trouvait à son aise avec elle, sans pouvoir
oublier ce qu'elle était, et qu'on s'accoutumait promptement à son
visage. En effet, après l'avoir un peu vue, on démêlait aisément qu'elle
avait eu de la beauté et de l'agrément dont une petite vérole si cruelle
n'avait pu effacer l'idée. La parenthèse, au courant vif de ce
commencement de fonctions d'ambassadeur, serait trop longue si j'en
disais ici davantage\,; mais il est nécessaire d'y remarquer en un mot,
qui sera plus étendu ailleurs, que jour et nuit, travail, audiences,
amusements, dévotions, le roi et elle ne se quittaient jamais, pas même
pour un instant, excepté les audiences solennelles qu'ils donnaient l'un
et l'autre séparément, l'audience du roi publique et celle du conseil de
Castille et les chapelles publiques. Toutes ces choses seront expliquées
en leur lieu.

Grimaldo, naturel Espagnol, ressemblait à un Flamand. Il était fort
blond, petit, gros, pansu, le visage rouge, les yeux bleus, vifs, la
physionomie spirituelle et fine, avec cela de la bonté. Quoique aussi
ouvert et aussi franc que sa place le pouvait permettre, complimenteur à
l'excès, poli, obligeant, mais au fond glorieux comme nos secrétaires
d'État, avec ses deux petites mains collées sur son gros ventre, qui,
sans presque s'en décoller ni se joindre, accompagnaient ses propos de
leur jeu\,: tout cela faisait un extérieur dont on avait à se défendre.
Il était capable, beaucoup d'esprit et d'expérience, homme d'honneur et
vrai, solidement attaché au roi et au bien de ses affaires, grand
courtisan toutefois, et dont les maximes furent en tous les temps
l'union étroite avec la France. En voilà ici assez sur ce ministre, dont
je sus gagner l'amitié et la confiance, qui me furent très utiles et qui
ont duré entre lui et moi jusqu'à sa mort, comme je le dirai ailleurs,
qui n'arriva qu'après sa chute et bien des années. Retournons maintenant
à notre ambassade.

Le dimanche 23 j'eus ma première audience particulière, le matin, du roi
et de la reine ensemble, dans le salon des Miroirs, qui est le lieu où
ils la donnent toujours. J'étais accompagné de Maulevrier. Je présentai
à Leurs Majestés Catholiques les lettres du roi et de M. le duc
d'Orléans. Les propos furent les mêmes sur la famille royale, la joie,
l'union, le désir de rendre la future princesse des Asturies heureuse. À
la fin de l'audience, je présentai à Leurs Majestés Catholiques le comte
de Lorges, le comte de Céreste, mon second fils, l'abbé de Saint-Simon,
et son frère. Je reçus force marques de bonté du roi et de la reine dans
cette audience, qui me parut fort sèche pour Maulevrier. Ils me
demandèrent fort des nouvelles de mon fils aîné, et dirent quelques mots
de bonté à ceux que je venais de leur présenter. Nous fûmes de là chez
l'infante, où je fus reçu comme Grimaldo et moi en étions convenus. Nous
descendîmes ensuite chez le prince des Asturies, à qui je présentai les
lettres du roi et de M. le duc d'Orléans, puis à la fin les mêmes
personnes que j'avais présentées au roi et à la reine. Les propos furent
à peu près les mêmes, et avec beaucoup de grâce et de politesse. Je me
conformai à l'usage et le traitai toujours de Monseigneur et de Votre
Altesse, sans y rien ajouter. J'en usai de même avec les infants.

Au sortir de là nous passâmes dans la cavachuela\footnote{Saint-Simon
  écrit toujours cavachuela, au lieu de covachuela, mot qui signifie
  bureau d'un ministère. On appelle en espagnol covachuelista un employé
  des ministères.} du marquis de Grimaldo. J'expliquerai ailleurs ce que
c'est. Il nous dit que le roi d'Espagne avait consenti à signer lui-même
le contrat et la reine\,; mais don Joseph Rodrigo qui, comme secrétaire
d'État intérieur, devait l'expédier, et qui ne parlait et n'entendait
pas un mot de français, ni à ce qu'il me parut d'affaires, proposa qu'il
y eût des témoins, et je compris que Grimaldo, qui s'attendait à notre
visite pour la réponse à la difficulté sur la signature, l'avait aposté
là exprès pour se décharger sur lui de la proposition de cette nouvelle
difficulté. Je répondis que nous n'avions point d'ordre là-dessus\,;
qu'on ne connaissait point cette formalité en France, et que tout
récemment le roi et tous ceux du sang avaient signé le contrat de la
duchesse de Modène d'une part, et d'autre part le seul plénipotentiaire
de Modène sans aucun témoin, et qu'il n'y en avait point eu non plus au
mariage de nos deux dernières reines. Ces messieurs ne se contentèrent
point de ces raisons. Rodrigo se débattit et baragouina fort. Grimaldo
nous dit avec beaucoup de douceur et de politesse qu'il fallait suivre
les coutumes des lieux où on était pour la validité et la sûreté des
actes qu'on y passait\,; que les contrats se passaient en Espagne par un
seul notaire, avec la nécessité de la présence de témoins, qui était une
formalité essentielle qu'ils ne pouvaient omettre. Nous nous défendîmes
sur ce qu'elle nous était inconnue et qu'il n'y en avait rien dans nos
instructions. Grimaldo allégua la complaisance du roi et de la reine
d'Espagne de signer eux-mêmes contre la coutume, sur ce que nous avions
représenté que cette signature était expressément dans nos instructions,
et que nous n'avions point de pouvoir pour nommer des commissaires qui
signassent avec les leurs\,; qu'ici il n'y avait ni pour ni contre dans
nos instructions, loin d'y avoir rien de contraire à la formalité des
témoins, et qu'il ne nous fallait point de pouvoir pour en nommer,
puisque rien ne s'y opposait dans nos instructions\,; enfin que nous ne
pouvions refuser, avec des raisons valables, de nous rendre à un usage
constant du pays qui, sans préjudice aucun ni à la chose ni à nos
ordres, n'allait qu'à la plus grande validité, que les parties
désiraient et voulaient également, et dont le refus jetterait dans un
grand embarras et une grande longueur. Je répondis que nos instructions
ne pouvaient rien contenir sur une formalité inconnue et jamais usitée
en France, à laquelle, par conséquent, on n'avait pu penser, mais que je
croyais qu'il suffisait qu'il n'y eût rien dedans ni pour ni contre pour
nous renfermer dans ce qu'elles contenaient, c'est-à-dire pour
n'admettre point de témoins. J'ajoutai que nous ne ferions aucune
difficulté qu'il y en eût de la part de l'Espagne, pourvu qu'il n'y en
eût point de la nôtre, comme je n'en ferais pas non plus qu'il y eût des
commissaires d'Espagne au cas {[}que{]} ces messieurs trouvassent qu'il
y en pût avoir, sans empêcher que Leurs Majestés Catholiques signassent
elles-mêmes le contrat. Que je les suppliais de considérer que Leurs
Majestés Catholiques pouvaient agir en souverains chez elles sans que
nous y pussions trouver à redire, mais que pour nous, nous étions bornés
aux ordres que nous avions reçus et aux termes de notre instruction sans
pouvoir les outrepasser. Grimaldo et Rodrigo insistèrent sur l'exemple
de la condescendance de Leurs Majestés Catholiques de signer elles-mêmes
contre la coutume, sur la nécessité des témoins pour la validité de
l'acte par la coutume d'Espagne, sur ce que des témoins n'avaient aucun
besoin de pouvoir, sur ce qu'il n'y avait rien dans nos instructions de
porté au contraire, sur ce que, par conséquent, admettre des témoins
n'était pas les outrepasser. Je continuai à me défendre par mes raisons
précédentes. Nous ne convînmes point et tout se passa doucement et très
poliment de part et d'autre. Maulevrier me laissa froidement faire et ne
dit que quelques mots à mesure que je l'interpellai.

Grimaldo nous proposa ensuite la signature des articles pour le
lendemain 24, l'après-dînée, avec le marquis de Bedmar et lui, nommés
commissaires du roi d'Espagne pour cela. Je m'expliquai que je
prétendais que cette signature se fît chez moi, à moins que le roi
d'Espagne n'aimât mieux qu'elle se fît dans son appartement, ce que
j'estimais encore plus convenable à la dignité de cette fonction et une
facilité qui pouvait être agréable à Sa Majesté Catholique. Cela fut
accepté sur-le-champ par Grimaldo, et l'heure convenue pour le lendemain
cinq heures après midi, au palais. Nous eûmes après quelque peu de
conversation de civilité, et nous prîmes congé.

Comme il achevait de nous conduire, il rappela Maulevrier à qui il
demanda les noms des personnes principales qui m'accompagnaient, et le
pria de lui envoyer ces noms dans le soir de ce même jour. Comme il fut
tard, Maulevrier m'envoya dire par son secrétaire que Grimaldo voulait
absolument avoir ces noms avant de se coucher, tellement que je les fis
écrire, et remettre à ce secrétaire.

Le lendemain matin, lundi 24, je reçus un paquet du marquis de Grimaldo
contenant une lettre pour moi, et cinq autres pour les comtes de Lorges
et de Céreste, l'abbé de Saint-Simon, et les marquis de Saint-Simon et
de Ruffec. Je récrivis sur-le-champ à Grimaldo, qui insistait toujours
par sa lettre sur les témoins, pour lui demander un entretien dans la
fin de la matinée, et pour le faire souvenir que les ambassadeurs de
famille ne faisaient point d'entrée. Sur la fin de la matinée, j'allai à
la cavachuela de Grimaldo pour m'expliquer avec lui sur ce qu'il
entendait par ces cinq lettres, et j'y allai seul, parce que Maulevrier,
à qui j'avais envoyé communiquer tout ce paquet de Grimaldo, voulut
demeurer à faire ses dépêches.

Grimaldo me dit nettement que le roi d'Espagne dans l'empressement de
finir une affaire si désirée, ayant condescendu de si bonne grâce à
signer lui-même avec la reine le contrat de mariage contre l'usage des
rois ses prédécesseurs, il était juste aussi que je condescendisse, non
par une simple complaisance, mais à un point nécessaire à la validité de
l'acte, qui est celui des témoins\,; que depuis notre conférence de la
veille, le roi d'Espagne avait cherché les moyens de concilier là-dessus
sa délicatesse avec nos difficultés, et qu'il avait cru prendre
l'expédient le plus convenable, même le plus honorable pour moi, de
nommer lui-même les cinq personnes les plus distinguées de tout ce que
j'avais amené pour être témoins afin de lever la difficulté que nous
faisions d'en nommer\,; que cette sûreté nécessaire dans l'occurrence
présente ne pouvait être refusée, puisque, outre qu'elle n'était pas de
mon choix, le roi d'Espagne ayant nommé à mon insu les cinq témoins
français, je ne pouvais alléguer que mes instructions portassent rien
qui y fût contraire.

Je répondis à cet honneur inattendu et rien moins que désiré de la
nomination du roi d'Espagne des témoins français, avec tout le respect
possible, sans toutefois m'engager à rien que je n'eusse vu jusques où
il voulait porter l'usage de ces témoins, et s'il avait dessein de leur
faire signer le contrat de mariage\,; mais il convint avec moi qu'ils
n'auraient pas cet honneur\,; que le roi d'Espagne se contenterait
qu'ils fussent présents à la signature de notre part, comme de la leur y
assisteraient aussi comme témoins les trois charges, qui sont le
majordome-major du roi, le sommelier du corps et le grand écuyer, avec
le majordome-major et le grand écuyer de la reine, qui étaient lors le
marquis de Villena ou duc d'Escalona, le marquis de Montalègre et le duc
del Arco\,; le marquis de Santa Cruz et Cellamare, ou le duc de
Giovenazzo\,; mais le premier et le dernier ne portaient que le nom de
marquis de Villena et de duc de Giovenazzo\,; que cette fonction des dix
témoins serait exprimée par un acte séparé qui serait seulement signé du
même secrétaire d'État tout seul, qui recevrait le contrat de mariage en
qualité de notaire du roi d'Espagne, lequel était don Joseph Rodrigo.

Cette assurance que la fonction des témoins ne paraîtrait que dans un
acte séparé, lequel même ils ne signeraient point, et qui ne le serait
que par un seul secrétaire d'État, me dérida beaucoup. Je considérai
qu'avec cette forme il ne se faisait rien contre la lettre ni contre
l'esprit de mon instruction, ni d'aucun ordre que j'eusse reçu\,; {[}je
considérai{]} leur opiniâtre attachement à une formalité espagnole
nécessaire dans tous les actes qui se passent en Espagne, et qui, bien
que omise aux mariages de nos deux dernières reines, leur paraissait
nécessaire et essentielle dans une circonstance aussi singulière que la
rendait l'âge de l'infante, où ils voulaient accumuler tout ce qu'ils
pouvaient de sûretés. Je m'aperçus aussi qu'ils n'avaient si facilement
accordé la signature du roi et de la reine au contrat de mariage, contre
tout usage et tout exemple, que pour obtenir une formalité aussi hors de
nos usages, mais à leur sens si fortement confirmative de la validité et
sûreté de l'engagement du roi pour le mariage. J'en fus d'autant plus
persuadé, et de l'opinion qu'ils avaient prise de l'importance de cette
formalité pour la sûreté du futur mariage que les cinq grands d'Espagne
qu'ils choisirent pour témoins étaient ce qu'il y avait de plus relevé
en Espagne en âge, en dignité, en charges et tous en naissance, excepté
Giovenazzo, mais si grandement décoré d'ailleurs\,; enfin {[}je
considérai{]} l'amère impatience de Leurs Majestés Catholiques, car elle
l'était devenue, de l'arrivée des dispenses de Rome et du départ de
M\textsuperscript{lle} de Montpensier, qui deviendrait bien autre, si
par une fermeté sans aucun véritable fondement je les jetais dans les
longueurs d'attendre le retour du courrier qu'il me faudrait dépêcher
sur cette difficulté des témoins. Je pris donc mon parti. Je me fis
répéter et confirmer par le marquis de Grimaldo que la fonction des
témoins ne paraîtrait que par l'acte séparé que même ils ne signeraient
point, et qui ne le serait que par Rodrigo tout seul, et je cédai enfin
avec tout l'assaisonnement de respects et du désir de complaire à Leurs
Majestés Catholiques et des compliments personnels à Grimaldo, qui prit,
à ce consentement, un air épanoui, et me proposa la signature du contrat
de mariage du roi avec l'infante pour le lendemain, après dîner, chez le
roi.

Quelques heures après être sorti d'avec lui, il m'envoya un paquet dans
lequel il n'y avait point de lettre pour moi, mais cinq autres pour les
cinq témoins français, dans lesquelles cette qualité était énoncée, au
lieu qu'elle ne l'était pas dans les premières qui ne portaient que le
choix du roi d'Espagne pour assister à la signature du contrat, parce
qu'alors ils n'osèrent aller plus loin sur la difficulté où nous en
étions demeurés à cet égard. Il paraît qu'il eut peur que, même après
avoir mon consentement, je ne m'opposasse à cette qualité nette de
témoins qui leur était si chère, parce qu'il ne me parla point d'envoyer
d'autres lettres, et qu'elles me surprirent quand je les reçus. Je les
remis aux cinq à qui elles étaient adressées et n'en parlai point à
Grimaldo, parce qu'elles n'innovaient et n'ajoutaient rien à ce à quoi
j'avais cru devoir consentir, d'autant qu'au terme de témoin près, elles
n'étaient que la copie exacte des premières.

Le même jour, lundi 24 novembre, je me rendis au palais avec Maulevrier
sur les cinq heures du soir. Le marquis de Bedmar et Grimaldo nous y
attendaient. Ils nous conduisirent, à travers le salon des Grands, au
coin du bout de ce salon, dans un cabinet petit et fort orné, dont les
tapis qui couvraient le plancher étaient d'une richesse et d'une beauté
si singulière, que j'avais de la peine à me résoudre à marcher dessus.
Cette pièce, ainsi que le salon des Grands, le petit salon où la cour
s'assemble pour attendre, et le salon des Miroirs, donnent sur le
Mançanarez et la campagne au delà\,; dans ce cabinet, nous trouvâmes une
table, une écritoire et quatre tabourets. Les deux commissaires
espagnols nous firent les honneurs et nous prîmes la droite. Tout était
convenu et écrit longtemps avant mon arrivée, en sorte que nous n'eûmes
qu'à collationner exactement les deux instruments que nous devions
signer avec la copie des mêmes articles que nous avions apportée, après
quoi nous signâmes en la manière accoutumée, et avec les compliments,
les protestations et les effusions de joie qu'on peut s'imaginer. Je fus
assis vis à vis du marquis de Bedmar, et Maulevrier vis à vis de
Grimaldo.

Je m'étais fait charger de témoigner à Grimaldo que le roi d'Espagne
avait fait un vrai plaisir à M. le duc d'Orléans et au cardinal Dubois
de donner à Laullez le caractère d'ambassadeur, comme le roi le venait
de donner ici à Maulevrier, et leur en ferait un autre très sensible de
lui marquer de plus par quelque autre grâce que Sa Majesté Catholique
était contente de lui. J'avais pris mon temps pour faire cet office
aussitôt que j'eus consenti aux témoins. J'avais à coeur de servir
Laullez, parce que je reconnaissais à tout moment qu'il n'avait rien
oublié pour me rendre agréable. Je vis, à la façon dont cela fut reçu,
qu'on était content de lui à la cour d'Espagne. J'en rafraîchis la
mémoire à Grimaldo en sortant du cabinet de la signature. En effet, il
écrivit de la part et par ordre du roi d'Espagne, à Laullez, avec
assurance des premières grâces qu'il serait possible de lui faire, et
Grimaldo me promit de fort bonne grâce d'y tenir très soigneusement la
main.

\hypertarget{chapitre-xiii.}{%
\chapter{CHAPITRE XIII.}\label{chapitre-xiii.}}

1721

~

{\textsc{Audience solennelle pour la demande de l'infante en mariage
futur pour le roi.}} {\textsc{- Audience de la reine d'Espagne.}}
{\textsc{- Audience du prince des Asturies et des infants.}} {\textsc{-
Bêtise de Maulevrier, qui ne se couvrit point.}} {\textsc{- Conduite
énorme de Maulevrier avec moi, bien pourpensée et bien exécutée jusqu'au
bout, pour me jeter dans le plus Fâcheux embarras sur les raisonnements
du contrat de mariage, de guet-apens, en pleine cérémonie de la
signature.}} {\textsc{- Ma conduite pour y précéder, comme je fis, le
nonce et le majordome-major du roi, sans les blesser.}} {\textsc{-
Signature solennelle du contrat du futur mariage du roi et de
l'infante.}} {\textsc{- Le prince des Asturies cède partout à l'infante
depuis la déclaration de son futur mariage avec le roi.}} {\textsc{- Je
me maintiens adroitement en la place que j'avais prise.}} {\textsc{-
Difficulté poliment agitée sur la nécessité ou non d'un instrument en
français.}} {\textsc{- Maulevrier forcé de laisser voir toute sa
scélératesse, de laquelle je me tire avec tout avantage, sans montrer la
sentir.}} {\textsc{- Autre honte à Maulevrier chez Grimaldo.}}
{\textsc{- Politesse de ce ministre.}} {\textsc{- Facilité pleine de
bonté du roi d'Espagne.}} {\textsc{- Ma conduite égale avec Maulevrier,
et mes raisons pour cette conduite.}} {\textsc{- Bonté de Leurs Majestés
Catholiques.}} {\textsc{- Conclusion de mon désistement d'un instrument
en français.}}

~

Le mardi 25 novembre, j'eus mon audience solennelle. Maulevrier, qui,
pour son caractère d'ambassadeur, ne s'était mis en aucune sorte de
dépense, vint de bonne heure chez moi le matin, où quelque temps après
arriva don Gaspard Giron et un carrosse magnifique du roi, à huit
chevaux gris pommelés admirables, dans lequel, à l'heure marquée, nous
montâmes tous trois. Deux garçons d'attelage tenaient chaque quatrième
cheval à gauche par une longe. Il n'y avait point de postillon, et le
cocher du roi nous mena son chapeau sous le bras. Cinq carrosses à moi,
remplis de tout ce que j'avais amené, suivaient, et une vingtaine
d'autres de seigneurs de la cour, qu'ils avaient envoyés pour me faire
honneur par les soins du duc de Liria et de Sartine, avec des
gentilshommes à eux dedans. Le carrosse du roi était environné de ma
nombreuse livrée à pied et des officiers de ma maison, c'est-à-dire
valets de chambre, sommeliers, etc. Les gentilshommes et les secrétaires
étaient dans mes derniers carrosses. Ceux de Maulevrier (et il n'en
avait que deux), remplis de Robin et de son secrétaire, suivaient le
dernier des miens. Arrivant à la place du palais, je me crus aux
Tuileries. Les régiments des gardes espagnoles, vêtus, officiers et
soldats, comme le régiment des gardes françaises, et le régiment des
gardes wallonnes, vêtus, officiers et soldats, comme le régiment des
gardes suisses, étaient sous les armes, les drapeaux voltigeants, les
tambours rappelant et les officiers saluant de l'esponton\footnote{L'esponton
  était une espèce de demi-pique que portaient les officiers
  d'infanterie et de dragons sous les règnes de Louis XIV et de Louis
  XV. La longueur de cette arme fut fixée à sept pieds et demi par une
  ordonnance du 20 mai 1690. Le salut de l'esponton demandait une
  certaine adresse dans le maniement de cette arme.
  M\textsuperscript{me} de Sévigné, parlant d'une revue de la maison du
  roi, à laquelle elle avait assisté, dit\,: «\,Nous avons eu le salut
  de l'esponton.\,»}. En chemin les rues étaient pleines de peuple, les
boutiques de marchands et d'artisans, toutes les fenêtres parées et
remplies de monde. La joie éclatait sur tous les visages, et nous
n'entendions que bénédictions.

Sortant de carrosse, nous trouvâmes le duc de Liria, le prince de
Chalais, grands d'Espagne, et Valouse, premier écuyer, qui nous dirent
qu'ils venaient nous rendre ce devoir comme Français. Caylus eût bien pu
y faire le quatrième. L'escalier était garni des hallebardiers avec
leurs officiers, vêtus comme nos Cent-Suisses, mais en livrée, la
hallebarde à la main, et leurs fonctions sont les mêmes. Entrant dans la
salle des gardes, nous les trouvâmes en haie sous les armes, et nous
traversâmes jusque dans la pièce contiguë à celle de l'audience, dont la
porte était fermée. Là étaient tous les grands et une infinité de
personnes de qualité, en sorte qu'il n'y avait guère moins de foule
qu'en notre cour, mais plus de discrétion. L'introducteur des
ambassadeurs a peu de fonctions. Il est fort effacé par celles du
majordome. Ce fut là un renouvellement de compliments et de joie, où
presque chacun me voulut particulièrement témoigner la sienne, et cela
dura près d'un quart d'heure que la porte s'ouvrit et que les grands
entrèrent\,; puis elle se referma.

Je demeurai encore un peu avec cette foule de gens de qualité, pendant
quoi le roi vint de son appartement, et entra dans la pièce de
l'audience par la porte opposée à celle par où les grands étaient
entrés, qui l'y attendaient, et par laquelle tout ce que nous étions à
attendre allions entrer. J'avouerai franchement ici que la vue du roi
d'Espagne m'avait si peu imposé la première fois, si peu encore les
autres fois que j'avais eu l'honneur d'approcher de lui, qu'au moment où
j'étais lors, je n'avais pas songé encore à ce que je devais lui dire.

Je fus appelé, et tous ces seigneurs entrèrent en foule avant moi, qui
me laissai conduire par don Gaspard Giron, qui prit ma droite, et
l'introducteur la gauche de Maulevrier, qui était à côté de moi. Comme
j'approchais de la porte, La Roche me vint dire de la part du roi, entre
haut et bas, que Sa Majesté Catholique m'avertissait et me priait de
n'être point surpris s'il ne se découvrait qu'à ma première et dernière
révérence, et point à la seconde\,; qu'il voudrait plus faire pour un
ambassadeur de France que pour aucun autre\,; mais que c'était un usage
de tout temps qu'il ne pouvait enfreindre. Je priai La Roche de
témoigner au roi ma très respectueuse et très sensible reconnaissance
d'une attention si pleine de bonté, et j'entrai dans la porte. Ce défilé
mit Maulevrier et les deux autres qui nous côtoyaient derrière, et
l'attention à ce que j'allais dire et au spectacle fort imposant
m'empêcha de plus songer à ce qu'ils devenaient.

Au milieu de cette vaste pièce et du côté que j'avais en face en
entrant, était un dais à queue sans estrade, sous lequel le roi était
debout, et à quelque distance, précisément derrière lui, le grand
d'Espagne capitaine des gardes en quartier, qui était le duc de
Bournonville\,; du même côté, presque au bout, le majordome major du
roi, appuyé à la muraille, seul\,; en retour, le long de la muraille qui
par un coin joignait l'autre muraille dont je viens de parler, étaient
les grands appuyés contre, et aussi contre la muraille en retour
vis-à-vis du roi jusqu'à la cheminée, grande comme autrefois et qui
était assez près de la porte par où je venais d'entrer et point tout à
fait au milieu de cette muraille\,; les quatre majordomes étaient le dos
à la cheminée. De la cheminée à la porte par où j'étais entré, et en
retour le long de la muraille et des fenêtres jusqu'au coin de la porte
par où le roi était entré, étaient en foule les gens de qualité les uns
devant les autres\,; dans la porte par où le roi était entré étaient
quelques seigneurs familiers par leurs emplois, qui regardaient comme à
la dérobée, mais dont aucun n'était grand, et derrière eux quelques
domestiques intérieurs distingués, qui voyaient à travers. Le roi et
tous les grands étaient couverts, et nuls autres\,; il n'y avait aucun
ambassadeur.

Je m'arrêtai un instant au-dedans de la porte à considérer ce spectacle
extrêmement majestueux, où qui que ce soit ne branlait et où le silence
régnait profondément. Je m'avançai lentement quelques pas et fis au roi
une profonde révérence, qui à l'instant se découvrit, son chapeau à la
hauteur de sa hanche\,; au milieu de la pièce je fis ma seconde
révérence, et en me baissant je me tournai un peu vers ma droite,
passant les yeux sur les grands, qui tous se découvrirent, mais non tant
qu'à la première révérence, où ils avaient imité le roi, qui à cette
seconde ne branla pas, comme il m'en avait fait avertir. J'avançai après
avec la même lenteur jusques assez près du roi, où je fis ma troisième
révérence, qui se découvrit comme il avait fait à la première, et se
couvrit aussitôt, en quoi tous les grands l'imitèrent. Alors je
commençai mon discours et me couvris au bout des cinq ou six premières
paroles sans que le roi me le dit.

Il roula sur les compliments du roi, l'union de la maison royale, celle
de leurs couronnes, la joie et l'affection des deux nations, celle que
j'avais trouvée répandue partout sur ma route en France et en Espagne,
l'attachement personnel du roi pour le roi son oncle, et son désir de
lui complaire et de contribuer à tout ce qui pourrait être de sa
grandeur, de ses intérêts, de ses affections, avec autant de passion que
pour les siens propres\,; enfin la demande de l'infante pour étreindre
encore plus intimement entre eux les liens déjà si forts du sang et des
intérêts de leurs couronnes, et lui témoigner sa tendresse par toute
celle qu'il aurait pour l'infante, ses soins, ses égards et l'attention
continuelle de la rendre parfaitement heureuse. Je passai de là au
remercîment du roi et à celui de M. le duc d'Orléans de l'honneur de son
choix de M\textsuperscript{lle} de Montpensier pour M. le prince des
Asturies\,; j'ajoutai que, quelque grand que Son Altesse Royale le
sentit, il était encore plus touché de recevoir une aussi grande marque
de ses bontés pour lui, et de l'acceptation, de son plus profond respect
et de ses protestations les plus sincères de sa passion de lui plaire et
de ne rien oublier pour resserrer de plus en plus une si heureuse union
des deux royales branches de leur maison, en contribuant de ses conseils
et de tous les moyens qu'il pourrait tirer de sa qualité de régent de
France pour servir et porter les intérêts et la grandeur de Sa Majesté
Catholique avec autant de zèle et d'attachement que ceux mêmes de la
France, et la persuader de plus, ce qu'il souhaitait avec le plus de
passion, de son infinie reconnaissance, de son attachement, de son
profond respect et de sa vénération parfaite pour sa personne. Je finis
mon discours par témoigner combien je ressentais de joie et combien je
me trouvais honoré d'avoir le bonheur de paraître devant Sa Majesté
Catholique, chargé par le roi de contribuer de sa part à mettre la
dernière main à un ouvrage si désirable\,; ce qui me comblait en mon
particulier de la plus sensible satisfaction, outre celle de toute la
France et de l'Espagne, parce que je n'avais jamais pu oublier d'où Sa
Majesté Catholique était issue, et toujours nourri et témoigné en tous
les temps mon très profond respect et l'attachement le plus vrai et le
plus naturel pour elle.

Si j'avais été si surpris de la première vue du roi d'Espagne à mon
arrivée, et si les audiences que j'en avais eues jusqu'à celle-ci
m'avaient si peu frappé, il faut dire ici avec la plus exacte et la plus
littérale vérité que l'étonnement où me jetèrent ses réponses me mit
presque hors de moi-même. Il répondit à chaque point de mon discours
dans le même ordre, avec une dignité, une grâce, souvent une majesté,
surtout avec un choix si étonnant d'expressions et de paroles par leur
justesse et un compassement si judicieusement mesuré, que je crus
entendre le feu roi, si grand maître et si versé en ces sortes de
réponses.

Philippe V sut joindre l'égalité des personnes avec un certain air de
plus que la déférence pour le roi son neveu, chef de sa maison, et
laisser voir une tendresse innée pour ce fils d'un frère qu'il avait
passionnément aimé et qu'il regrettait toujours. Il laissa étinceler un
coeur François sans cesser de se montrer en même temps le monarque des
Espagnes. Il fit sentir que sa joie sortait d'une source plus pure que
l'intérêt de sa couronne, je veux dire de l'intime réunion du même
sang\,; et à l'égard du mariage du prince des Asturies, il sembla
remonter quelques degrés de son trône, s'expliquer avec une sérieuse
bonté, sentir moins l'honneur qu'il faisait à M. le duc d'Orléans en
faveur du même sang, que la grâce signalée, et je ne dis point trop et
je n'ajoute rien, qu'il lui faisait d'avoir bien voulu ne point penser
qu'à le combler par une marque si certaine de sa bonne volonté pour lui.
Cet endroit surtout me charma par la délicatesse avec laquelle, sans
rien exprimer, il laissa sentir sa supériorité tout entière, la grâce si
peu méritée de l'oubli des choses passées, et le sceau si fort
inespérable que sa bonté daignait y apposer. Tout fut dit avec tant
d'art et de finesse, et coula toutefois si naturellement, sans
s'arrêter, sans bégayer, sans chercher, qu'il fit sentir tout ce qu'il
était, tout ce qu'il pardonnait, tout en même temps à quoi il se
portait, sans qu'il lui échappât un seul mot ni une seule expression qui
pût blesser le moins du monde, et presque toutes au contraire
obligeantes. Ce que j'admirai encore fut l'effectif, mais toutefois
assez peu perceptible changement de ton et de contenance en répondant
sur les deux mariages. Son amour tendre pour la personne du roi, son
affection hors des fers pour la France, la joie d'en voir le trône
s'assurer à sa fille, se peindre sur son visage et dans toute sa
personne à mesure qu'il en parlait\,; et lorsqu'il répondit sur l'autre
mariage, la même expression s'y peignit aussi, mais de majesté, de
dignité, de prince qui sait se vaincre, qui le sent, qui le fait, et qui
connaît dans toute son étendue le poids et le prix de tout ce qu'il veut
bien accorder. Je regretterai à jamais de n'avoir pu écrire sur-le-champ
des réponses si singulières et de n'en pouvoir donner ici qu'une idée si
dissemblable à une si surprenante perfection.

Quand il eut fini je crus lui devoir un mot de louange sur ce dernier
article, et un nouveau remercîment de M. le duc d'Orléans, comme son
serviteur particulier. Au lieu de m'y répondre, le roi d'Espagne me fit
l'honneur de me dire des choses obligeantes et du plaisir qu'il avait
que j'eusse été choisi pour faire auprès de lui des fonctions qui lui
étaient si agréables. Ensuite m'étant découvert, je lui présentai les
officiers des troupes du roi qui m'accompagnaient, et le roi d'Espagne
se retira en m'honorant encore de quelques mots de bonté.

Je fus environné de nouveau par tout ce qui était là de plus
considérable, avec force civilités\,; après quoi la plupart des grands
et des gens de qualité allèrent chez la reine, tandis que quelques-uns
d'eux tous demeurèrent à m'entretenir pour laisser écouler tout ce qui
sortait, et se placer chez la reine, où au bout de fort peu de temps
nous y fûmes aussi conduits comme nous l'avions été chez le roi. Arrivés
dans la pièce joignant celle où l'audience se devait donner, on nous fit
attendre que tout y fût préparé.

Avant d'aller plus loin il faut expliquer que don Gaspard Giron ne me
conduisit, allant chez la reine, que jusqu'au bout de l'appartement du
roi, et qu'à l'entrée de celui de la reine il se retira et laissa sa
fonction à un majordome de la reine. J'avais su que Magny, qui {[}en{]}
était un, se trouvait justement en semaine, par conséquent que c'était à
lui à m'introduire. J'en avais parlé à Grimaldo et demandé qu'on en
chargeât un autre. Non seulement je l'obtins, mais Magny, qui avait été
nommé pour le voyage de Lerma, en fut rayé, et un autre majordome de la
reine mis de ce voyage au lieu de lui, mais il reçut défense expresse de
se trouver en aucun lieu où je serais, même au palais\,; Grimaldo me le
dit lui-même. Soit que cette défense eût été étendue aux autres François
réfugiés pour l'affaire de Cellamare et de Bretagne, ou qu'ils l'aient
cru sur l'exemple de Magny, ils évitèrent tous et toujours ma rencontre,
et presque toujours celle de tout ce qui était venu avec moi en Espagne.

Tout étant prêt, la porte s'ouvrit et nous fûmes appelés la pièce de
l'audience était le double de la petite galerie intérieure par laquelle
on a vu que le jour de ma première révérence j'avais suivi Leurs
Majestés Catholiques chez les infants. Ce double était moins long mais
aussi large que la galerie à laquelle elle était unie par de grandes
arcades ouvertes, desquelles seules cette pièce tirait son jour. Nous
arrivâmes par le côté de l'appartement des infants, et la reine et sa
suite était entrée par le sien au bout opposé.

Le bas de cette pièce que nous trouvâmes d'abord en y entrant était
obscur et plein de monde qui était arrêté par une barrière à sept ou
huit pas en avant où l'obscurité s'éclaircissait. La porte de la pièce
et celle de la barrière qui ne se tira que lorsque j'en fus tout près,
fit un défilé qui me laissa passer seul, en sorte que je ne pus voir
ensuite derrière moi. Au fond de cette pièce qui était fort longue, la
reine était assise sur une espèce de trône, c'est-à-dire un fauteuil
fort large, fort évasé, et fort orné\,; les pieds sur un carreau
magnifique, d'une largeur et d'une hauteur extraordinaire, qui cachait,
comme je le vis quand la reine en sortit, quelques marches assez basses.
Le long de la muraille étaient les grands, rangés, appuyés et couverts.
Vis-à-vis le long des arcades, des carreaux carrés, longs plus que
larges, et médiocrement épais, de velours et de satin rouge ou de damas,
tous également galonnés d'or tout autour, de la largeur de la main au
plus, avec de grosses houppes d'or aux coins. Sur les carreaux de
velours étaient les femmes des grands d'Espagne, et les femmes de leurs
fils aînés sur ceux de satin ou de damas, toutes également assises sur
leurs jambes et sur les talons. Cette file de grands à la muraille, et
de dames sur ces carreaux, vis-à-vis d'eux, tenait toute la longueur de
la pièce, laissant un peu de distance en approchant de la reine, et une
autre en approchant de la barrière par où j'entrais.

Je m'arrêtai quelques moments dans la porte de cette barrière à
considérer un spectacle si imposant, tandis que, par derrière moi, les
ducs de Veragua et de Liria, le prince de Masseran et quelques autres
grands qui avaient voulu me faire l'honneur de m'accompagner depuis
l'appartement du roi, se glissèrent à la muraille, à la suite des
derniers placés. Le majordome-major du roi ne se trouva point à cette
audience parce que, ayant de droit la première place partout, il ne la
veut pas céder au majordome-major de la reine qui, chez elle, prétend
l'avoir et en est en possession. Aussi était-il à la tête des grands à
la muraille, y ayant une place vide entre lui et le grand d'Espagne qui
était le plus près de lui, comme vis-à-vis de lui, entre le carreau de
la camarera-mayor de la reine et le carreau le plus près d'elle. Le
majordome-major de la reine était placé là parce que la reine tenait
tout le fond de cette pièce, ayant deux officiers des gardes du corps un
peu en arrière à côté de son fauteuil. Les dames de qualité étaient en
grand nombre debout derrière les carreaux des dames assises, et
remplissaient le vide de chaque arcade. Quelques gens de qualité
s'étaient mis derrière elles, mais le gros de ceux-là se tint contre les
barrières, en dedans qui put, et en dehors en foule.

Après avoir arrêté mes yeux quelques moments sur ce beau spectacle fort
paré, je m'avançai lentement jusqu'au second carreau d'en bas, marchant
au milieu de la largeur de la pièce, et là, je fis une profonde
révérence. Je continuai à m'avancer de même jusqu'au milieu de la
longueur qui restait, où je fis la seconde révérence, me tournant un peu
vers les carreaux en me baissant, passant les yeux dessus ce qui en
était à portée, et j'en fis de même en me relevant vers les grands qui
se découvrirent, comme les dames m'avaient fait une légère inclination
du corps de dessus leurs carreaux. J'avançai ensuite jusqu'au pied du
carreau de la reine où je fis ma troisième révérence, à laquelle seule
la reine répondit par une inclination de corps fort marquée. Un instant
après je dis\,: «\,Madame,\,» et ce mot achevé je me couvris, et tout de
suite me découvris sans avoir ôté ma main de mon chapeau et ne me
couvris plus. Les grands, depuis ma seconde révérence, étaient demeurés
découverts et ne se couvrirent plus.

Mon discours roula sur les mêmes choses qu'avait fait celui que je
venais de faire au roi, retranchant et ajustant à ce qui lui convenait,
également ou différemment du roi d'Espagne. Elle était parée
modestement, mais brillante d'admirables pierreries et avait une grâce
et une majesté qui sentaient bien une grande reine. Elle fut surprise
d'un si grand transport de joie qu'elle s'en laissa apercevoir
embarrassée, et elle prit plaisir depuis à m'avouer son embarras\,; elle
ne laissa pas de me répondre en très bons termes sur sa joie du mariage
de l'infante, sur son estime et son affection pour le roi et sa passion
même pour lui, sur son amitié pour M. le duc d'Orléans, et son désir de
voir sa fille heureuse en Espagne, surtout sur son désir et sa joie
extrême de l'union des couronnes, des personnes royales de la même
maison, de leur commune grandeur et de leurs intérêts qui ne pouvaient
jamais être que les mêmes, puis des marques de bonté pour moi.

Si cette audience eût été la première, sa réponse m'aurait charmé tant
elle était bien faite et accompagnée de toutes les grâces possibles et
de majesté. Mais il faut avouer qu'avec beaucoup d'esprit, de tour
naturel et de facilité de s'énoncer, elle ne put s'élever jusqu'à la
justesse et la précision du roi, si diversement modulées, sur chaque
point, beaucoup moins jusqu'à ce ton suprême qui sentait la descendance
directe d'un si grand nombre de rois, qui se proportionnait avec tant de
naturelle majesté aux choses et aux personnes dont il fit plus entendre
qu'il n'en dit dans, sa réponse.

Quand elle eut achevé, je lui fis une profonde révérence et je me
retirai le plus diligemment que la décence me le permit pour gagner le
dernier carreau de velours d'en bas et les parcourir promptement tout en
ployant un peu le genou devant chacun et disant à la dame assise
dessus\,: «\,A los pies à Vuestra Excellentia,\,» ce qui suppose\,:
«\,Je me mets aux pieds de Votre Excellence,\,» à quoi chacune sourit et
répondit par une inclination de corps\,; il faut être preste à cette
espèce de course qui se fait, tandis que la reine se débarrasse de ce
gros carreau qu'elle a sous les pieds, qu'elle se lève, qu'elle descend
les marches de cette espèce de trône et qu'elle retourne dans son
appartement par la porte de la galerie qui y donne, et qui n'est presque
éloignée de ce trône que de la demi-largeur de la pièce où il est, et de
la largeur entière de la galerie, qui sont très médiocres, et il faut
avoir achevé le dernier carreau près de celui de la camarera-mayor, qui
se lève en même temps que la reine pour la suivre, à temps de trouver la
reine à la porte de son appartement, mettre un genou à terre devant
elle, lui baiser la main qu'elle vous tend et la remercier en cinq ou
six paroles, à quoi elle répond de même.

Je ne pus avoir sitôt expédié les carreaux, que je vis la reine dans la
porte de son appartement\,; elle m'avait déjà traité avec tant de bonté
et de familiarité que je crus pouvoir user de quelque sorte de liberté
dans ces moments d'une si grande joie, tellement que je courus vers elle
et lui criai que Sa Majesté se retirait bien vite, et, comme je la vis
s'arrêter et se retourner, je lui dis que je ne voulais pas perdre un
moment et un honneur si précieux, elle se mit à rire, et moi, un genou à
terre à lui baiser la main qu'elle me tendit dégantée et me parla fort
obligeamment\,; mon remercîment suivit et cela fit un entretien de
quelques moments dans cette porte, ses dames en cercle autour qui
arrivaient cependant.

La reine et quelques-unes de ses dames rentrées, je lis plus posément,
et avec plus de loisir, des compliments à celles qui, par leurs charges,
allaient aussi rentrer chez la reine, qui étaient demeurées pour m'en
faire\,; puis j'allai remplir le même devoir de galanterie auprès des
principales des autres que je trouvai le plus sous ma main, puis à
beaucoup de seigneurs qui m'environnèrent. J'oubliais mal à propos qu'à
la fin de l'audience je présentai à la reine tous les officiers des
troupes du roi qui m'avaient suivi en Espagne.

Débarrassé peu à peu de tant de monde, et toujours avec les mêmes
seigneurs susnommés, qui m'avaient fait l'honneur de vouloir
m'accompagner de chez le roi chez la reine et qui, quoi que je pusse
faire, voulurent absolument aller partout avec moi, nous allâmes chez le
prince des Asturies, où tout se passa sans aucune cérémonie\,: je fis
une seule révérence au prince qui était découvert et qui ne se couvrit
point du tout. Ce fut moins une audience qu'une conversation dans
laquelle le prince n'oublia rien de tout ce qui convenait de dire, et
sans aucun embarras.

Le duc de Popoli, qui, comme à ma première audience, m'était venu
recevoir et conduire à l'entrée de l'appartement, fut plus embarrassé
que lui. Il m'accabla de ses sentiments de joie sur les mariages, et
d'attachement pour le roi et pour M. le duc d'Orléans, et de compliments
pour moi, avec force excuses sur ce que son esclavage chez le prince, ce
fut le terme dont il se servit, ne lui avait pas encore pu permettre de
venir me rendre ses devoirs. Je lui répondis avec toute sorte de
politesse, mais avec peine, tant son affluence de protestations était
continuelle, et me divertissant à part moi de son embarras.

L'introducteur des ambassadeurs nous conduisit après chez l'infante et
chez les infants. Le dernier dormait, et, suivant ce que Grimaldo
m'avait promis, l'infante dormait aussi. Je sortis du palais avec les
mêmes honneurs que j'y avais été reçu, les bataillons étant demeurés
pour cela dans la place\,; et je trouvai chez moi don Gaspard Giron qui
m'attendait en grande et illustre compagnie, et un magnifique repas. Il
s'en alla chez lui\,; on en verra bientôt la raison.

En arrivant chez moi, je fus averti que Maulevrier ne s'était point
couvert aux audiences que nous venions d'avoir du roi et de la reine,
{[}ce{]} dont je n'avais pu m'apercevoir parce qu'il s'était tenu, à
toutes les deux, fort en arrière de moi. Il m'avait auparavant fait la
question s'il ferait aussi la demande de l'infante, et comme je lui
répondis que l'usage n'était pas que deux ambassadeurs fissent cette
demande l'un après l'autre, je ne sais ce qu'il en conclut. Je trouvai
la chose si étrange, que je m'en voulus assurer tant par les principaux
de ceux qui m'y avaient suivi, que par les ducs de Veragua et de Liria,
le prince de Masseran et quelques autres de ceux qui se trouvèrent chez
moi pour dîner, avec qui déjà j'avais contracté le plus de familiarité,
qui, tous, m'assurèrent l'avoir très bien vu et remarqué, et que la
surprise en avait été générale\,; ils ajoutèrent même qu'il n'avait pas
fait le plus léger semblant de se couvrir. Je lui en parlai dans la
suite, n'ayant pu le faire alors, et le plus poliment qu'il me fut
possible\,; il me répondit froidement et tout court qu'il en était
fâché, qu'il n'avait pas cru devoir se couvrir, qu'il se trouverait
d'autres occasions de réparer ce manquement. Mettant pied à terre chez
moi, il ne voulut pas monter dans mon appartement, où toute la grande
compagnie m'attendait, et quoi que je pusse faire, je ne pus jamais
l'engager à dîner avec nous. Il me dit qu'il avait affaire chez lui, et
qu'il serait exacte à l'heure de revenir chez moi pour aller ensemble à
la signature du contrat. Ce fut une bêtise, mais voici une perfidie, et
bien pourpensée et bien exécutée de guet-apens dans toutes ses
circonstances.

L'instrument des articles avait été signé double\,; un en espagnol,
l'autre en français. Cela m'avait persuadé qu'il en serait de même de
l'instrument du contrat de mariage. Il n'y avait rien ni pour ni contre
dans mon instruction, comme il n'y en avait rien non plus sur
l'instruction des articles, et le cardinal Dubois ne m'avait rien dit
là-dessus, ni moi pensé à lui en faire question. J'en parlai dès les
premiers jours à Maulevrier, qui ne douta pas un moment des deux
instruments\,; ce qui me confirma encore dans cette persuasion. Je ne
savais pas un mot d'espagnol\,; Maulevrier et Robin, son mentor, dont je
dirai un mot dans la suite, lé savaient fort bien. Maulevrier s'était
donc chargé du changement à faire dans la préface du contrat de mariage,
lorsque j'eus obtenu qu'il n'y aurait point de commissaires, et que le
roi et la reine d'Espagne le signeraient eux-mêmes. Maulevrier avait
fait ce changement, il l'avait montré à Grimaldo, tous deux me dirent
qu'il était bien, ce n'était qu'une affaire de style\,: dès lors que
j'étais assuré que Leurs Majestés Catholiques signeraient elles-mêmes,
je m'en reposai sur ce qu'ils m'en dirent, et en effet il était bien.
Ils m'en promirent une copie en français. Je convins avec Maulevrier
qu'il porterait à la signature du contrat de mariage les deux copies de
ce même contrat, l'une espagnole qu'il lirait tout bas à mesure que le
contrat en espagnol serait lu tout haut pour le collationner ainsi
lui-même, et que j'en ferais autant de la copie française à mesure que
le contrat en français serait lu tout haut pour être ensuite signés l'un
et l'autre également.

Dès avant d'aller le matin à l'audience, je lui parlai de ces copies\,;
il me dit qu'elles n'étaient pas encore faites, mais qu'elles le
seraient avant le dîner. Comme il s'opiniâtra à s'en aller dîner chez
lui, je le priai de m'envoyer la copie française\,; il me le promit et
s'en alla. Pendant le dîner, qui fut long chez moi, j'envoyai deux fois
chercher ces copies\,; il me manda la dernière qu'il les apporterait\,:
prêt à partir, et l'heure pressant, j'envoyai un homme à cheval chez
lui\,; il me fit dire par lui que j'allasse toujours, et qu'il se
trouverait au palais. Cette réponse me parut singulière pour une
cérémonie aussi solennelle\,: véritablement ses deux seuls carrosses et
sa médiocre livrée de cinq ou six personnes ne pouvaient donner ni ôter
grand lustre à mon cortège, mais ce procédé me surprit fort sans en rien
témoigner.

Dans l'embarras où la méchanceté du cardinal Dubois m'avait mis sur le
nonce et le majordome-major, tel qu'on l'a vu ci-dessus en son lieu,
j'avais affecté de rendre infiniment à l'un et à l'autre, toutes les
fois que je les avais rencontrés et visités, pour leur ôter toute sorte
d'idée que j'imaginasse de les précéder, quand je les précéderais
effectivement\,; je pensai que les précéder effectivement et nettement
l'un ou l'autre serait une entreprise que je ne pourrais soutenir. La
place du grand maître, à cette signature, était derrière le fauteuil du
roi, un peu à la droite, pour laisser place au capitaine des gardes en
quartier\,; m'y mettant, c'était prendre sa place, y intéresser le
capitaine des gardes, jeté plus loin, et conséquemment ce qui devait
être de suite. Celle du nonce était à côté du roi, le ventre au bras
droit de son fauteuil\,; la prendre, c'était le repousser hors du bras
du fauteuil, contre le bout de la table, et sûrement il ne l'aurait pas
souffert non plus que le majordome-major pour la sienne. Je résolus donc
de hasarder un milieu\,; de tâcher de me fourrer au haut du bras droit
du fauteuil, un peu en travers, pour ne prendre nettement la place ni de
l'un ni de l'autre, mais de les écorner toutes les deux pour m'en faire
une, et de couvrir cela d'un air d'ignorance et de simplicité d'une
part, et de l'autre, d'empressement, de joie, de curiosité, d'engouement
de courtisan qui veut parler au roi et l'entretenir tant qu'il sera
possible\,: ce fut aussi ce que j'exécutai en apparence niaisement, et
en effet très heureusement. L'inconvénient était de Maulevrier, qui
devait être naturellement à côté de moi. Je ne crus pas lui devoir la
confidence de ce que je me proposais, et je résolus, pour confirmer mon
ignorance, de le laisser tirer d'affaires comme il pourrait sans y
prendre part, pourvu que je m'en tirasse moi-même dans un pas si
délicat, où cet honnête homme de Dubois avait bien compté me perdre
d'une façon ou d'une autre.

Dans cette inquiétude de place et d'instruments, je partis, conduit par
don Gaspard Giron, dans le carrosse du roi, et le même cortège que
j'avais eu le matin pour mon audience solennelle, moi seul sur le
derrière, don Gaspard seul, vis-à-vis de moi, parmi les acclamations de
joie de la foule des rues et des fenêtres, remplies comme elles
l'avaient été le matin. Je trouvai le palais rempli de tout ce qui était
à Madrid de quelque considération. Tous les grands avaient été mandés,
le nonce, l'archevêque de Tolède, le grand inquisiteur, et les
secrétaires d'État et le P. Daubenton. Le salon entre celui des Miroirs
et celui des Grands, où la cérémonie s'allait faire, était rempli à ne
pouvoir s'y tourner. Dans mon dessein, je me coulai peu à peu parlant
aux uns et aux autres tout auprès de la porte du salon des Miroirs, et
je m'y tins causant avec ce qui s'y trouva à portée\,; l'attente dura
bien trois quarts d'heure et m'ennuya fort dans cette foule avec ma
double inquiétude. Enfin la porte s'ouvrit, et le roi parut avec la
reine, et derrière eux l'infante et les infants.

Dès la porte, je me mis à parler au roi, marchant à côté de lui. Je le
conduisis de la sorte jusqu'à sa place dans le salon des Grands où je
pris tout de suite celle que j'avais projetée. Voici comment ce salon se
trouva disposé, et ceux qui assistèrent à cette signature. Une longue
table était placée en travers, ayant un bout vers les fenêtres, l'autre
vers la porte par où on y était entré, et cette table couverte d'un
tapis avec une écritoire dessus. Six fauteuils rangés le long de la
table, le dos à la muraille mitoyenne de ce salon et de celui où on
avait attendu le roi, mais laissant un large espace entre la muraille et
le dos des fauteuils dont les bras se joignaient. Les infants ont un
fauteuil devant le roi d'Espagne\,; j'en dirai la raison dans la suite,
mais j'ignore celle de leur arrangement, tout différent de celui des
autres pays. Le roi se mit au premier fauteuil tout à la droite, la
reine au second, l'infante au troisième, le prince des Asturies, qui lui
céda toujours partout depuis la déclaration du mariage futur du roi avec
elle, au quatrième\,; don Ferdinand au cinquième, et don Carlos au
sixième. La gouvernante de l'infante demeura derrière son fauteuil à
cause de l'enfance de la princesse, sans aucune autre femme, pas même la
camarera-mayor. Cette forme de séance à la file se garde la même au bal,
à la comédie, etc.

J'ai dit d'avance qui était derrière le roi. Le marquis de Santa Cruz,
majordome-major de la reine, était derrière elle, et le duc de Popoli
derrière le prince des Asturies, dont il était gouverneur. Les deux
infants n'avaient personne derrière eux. Les grands et les cinq témoins
français faisaient un demi-cercle devant toute la table. L'archevêque de
Tolède et le grand inquisiteur y étoient un peu à part d'eux, et
derrière eux les secrétaires d'État et le P. Daubenton qui s'y était
fourré. Près des fenêtres, assez loin de la table, était une petite
table avec un tapis et une écritoire, cachée par le cercle qui
environnait la grande table. Il n'entra qui que ce soit que tous les
grands, le nonce et ceux qui viennent d'être nommés, et aussitôt après
les portes furent fermées sans aucun domestique ni officier du roi
dedans. On a dit ailleurs, en parlant des grands d'Espagne, qu'ils
n'observent entre eux aucun rang d'ancienneté ni de classe\,; ainsi ils
se rangèrent les uns auprès des autres comme le hasard les fit
rencontrer. Le roi fut toujours découvert.

Le majordome-major et le nonce, qui suivaient le dernier infant, me
trouvant à ce coin de fauteuil où je m'étais placé, entrant à côté du
roi et lui parlant, parurent fort surpris. J'entendis répéter signore et
señor à droite et à gauche en me parlant, car tous deux s'exprimaient
difficilement {[}en{]} français, moi révérences de côté et d'autre, air
riant d'un homme tout occupé de la joie de la fonction, et qui
n'entendait rien à ce qu'ils me voulaient dire, reprenant la parole avec
le roi avec une sorte de liberté, d'enthousiasme, tellement que tous
deux se lassèrent d'interpeller un homme dont l'esprit transporté ne
comprenait rien à ce qu'ils lui voulaient dire ni à la place qu'il avait
prise. Ce ne fut que là où je revis Maulevrier depuis que nous nous
étions séparés en arrivant chez moi de l'audience. Il tâcha de se
fourrer entre le nonce et moi, mais le nonce tint ferme après une petite
révérence, et je n'osai essayer de lui faire place, ce qui d'ailleurs,
serré comme j'étais, m'eût été bien difficile, parce que l'aidant ainsi
à se mettre au-dessus du nonce, aurait montré trop à découvert que je
savais mieux où je m'étais mis que ces deux messieurs ne le pensaient,
et que le nonce voyant alors le dessein n'eût souffert au-dessus de lui
ni Maulevrier ni moi, tellement que je le laissai dans la presse, ce qui
servit à leur persuader que je ne pensais à rien. Maulevrier donc
demeura couvert par le nonce et par moi, en sorte que sa tête paraissait
seulement entre les nôtres en arrière.

Don Joseph Rodrigo, tout près de la table vis-à-vis de la reine, reçut
ordre de faire la lecture du contrat, sitôt que le premier brouhaha de
tout ce qui entrait et s'arrangeait fut passé, et un moment après, le
roi et tout ce qui devait remplir les six fauteuils s'assirent, tout le
reste demeurant debout\,; comme la lecture commençait, je me tournai à
l'oreille de Maulevrier, comme je pus, et lui demandai s'il avait sa
copie espagnole pour collationner, et la française pour me la donner. Il
me répondit qu'à son départ de chez lui elles n'étaient pas encore
achevées, mais qu'on allait les lui apporter. Il sera bien temps, lui
repartis-je en me retournant, et je me remis à entretenir le roi,
toujours dans la crainte de mes deux voisins, et pour leur persuader un
engouement qui, sans en sentir la conséquence, m'avait fait mettre et
demeurer dans la place où j'étais. La lecture fut extrêmement longue\,;
Rodrigo lut fort haut et fort distinctement le contrat de mariage futur
du roi et de l'infante\,; un double de ce contrat, aussi en espagnol,
l'acte séparé où il fut fait mention de la qualité des dix témoins et de
la présence distincte de tous les grands d'Espagne qui s'y trouvèrent.
Ne sachant plus sur la fin de quoi continuer d'entretenir le roi, je
m'avisai de lui demander audience pour le lendemain qu'il m'accorda
volontiers, ce qui fit durer un peu la conversation que je tâchais de
soutenir jusqu'à la fin de la lecture par tout ce dont je pus sagement
m'aviser par la raison que j'en ai dite.

Cette lecture ennuya assez la reine pour qu'elle demandât si elle
durerait encore longtemps. Elle s'attendait si bien qu'il y aurait un
instrument en français à lire, que j'en pris occasion de lui dire qu'on
se pourrait passer d'en lire le préambule qui ne contenait rien
d'essentiel. C'est que je voulais cacher que cette préface nous
manquait, Maulevrier n'en ayant point de copie sur lui, lui qui l'avait
refaite comme il a été dit avec Grimaldo, pour en ôter ce qui regardait
les commissaires, et moi ne l'ayant point en français, parce que je
n'avais que la copie du contrat de mariage telle que le cardinal Dubois
me l'avait donnée.

Toutes les lectures espagnoles étant achevées, don Joseph Rodrigo
s'approcha du bout de la table pour présenter la plume au roi d'Espagne,
lequel, au lieu de la prendre, proposa de faire toutes les lectures de
suite. Je dis aussitôt, d'un ton modeste et demi-bas, que je croyais
qu'il y avait un instrument en français. Don Rodrigo, à qui le roi le
rendit en espagnol, répondit qu'il ne le croyait pas, qu'en tout cas, il
n'en avait point apporté. Sur quoi Maulevrier, qui jusqu'à ce moment
avait gardé un parfait silence, dit qu'il l'allait envoyer chercher, et
sans une parole de plus sortit de sa place pour le faire. Dans cet
intervalle, le roi d'Espagne me dit qu'apparemment il n'en fallait
point, puisqu'on n'en avait point apporté. Pour toute réponse, je lui
proposai de faire appeler Grimaldo qui était derrière le cercle des
grands. Le roi lui manda aussitôt de lui venir parler\,; il vint et
s'approcha du fauteuil entre le majordome-major et moi qui lui fîmes le
peu de place que nous pûmes.

Sur la question que le roi lui fit, il répondit qu'il ne fallait point
d'instrument français. J'objectai ce qui s'était passé pour les articles
que nous avions signés avec le marquis de Bedmar et lui sur deux
instruments, l'un espagnol, l'autre français. Grimaldo répliqua que ce
n'était pas la même chose. Je n'en entendis que cela, parce que le roi
d'Espagne, qui prenait la peine de nous servir d'interprète, ne m'en
expliqua pas davantage. Je répliquai modestement qu'il semblait que la
dignité des deux couronnes demandait que chacune eût un instrument signé
en sa langue, et en ce moment Maulevrier revint auprès de moi au même
lieu où il était avant de sortir. Grimaldo me répondit avec beaucoup de
politesse qu'il ne croyait pas que cela pût faire difficulté, d'autant
qu'il avait vu une lettre du cardinal Dubois à Maulevrier, qui le
portait expressément. Je regardai Maulevrier me tournant vers lui avec
l'étonnement qu'il est aisé de se représenter. Il me dit avec un air
fort embarrassé qu'il y avait quelque chose de cela dans une lettre que
le cardinal Dubois lui avait écrite. Cela me fit prendre mon parti
sur-le-champ. Je dis au roi et à la reine que je ferais aveuglément tout
ce qu'il leur plairait me commander, ce que j'assaisonnai de tout ce que
le respect, la confiance, l'union, la joie de ce grand jour, me purent
fournir en peu de paroles, et que j'espérais que, s'il se trouvait qu'il
fallût un instrument en français, Leurs Majestés Catholiques voudraient
bien ne pas faire de difficulté de le signer après coup en particulier.
En même temps je me mis comme en devoir d'approcher du roi le contrat
qui était sur la table, pour lui marquer mon empressement, mais sans y
toucher toutefois, parce que c'était la fonction du secrétaire d'État
Rodrigo. Il parut à quelques discours et à l'air du roi et de la reine
d'Espagne, que cette démonstration leur fut extrêmement agréable.

À l'instant Rodrigo s'approcha du nonce, qu'il couvrit un peu, et de là
présenta le contrat et la plume au roi d'Espagne, et aussitôt se retira
au-devant de la table, qu'il suivit, amenant l'écritoire dessus à mesure
qu'on signait tout de suite. Le roi, ayant signé, poussa le contrat
devant la reine, et lui présenta la plume. Elle signa, puis ajusta le
contrat devant l'infante, lui donna la plume et lui tint un peu la main
pour signer, ce qu'elle fit le plus joliment du monde. La reine après,
lui reprit la plume, la donna par devant l'infante au prince des
Asturies, et lui poussa le contrat. Il signa donc et les deux princes
ses frères, en se donnant de même la plume et se poussant le contrat. La
dernière signature achevée, don Joseph Rodrigo reprit la plume des mains
de l'infant don Carlos et le contrat de dessus la table. La joie qui
accompagna ces signatures ne se peut exprimer.

Un moment après qu'elles furent achevées, le roi et la reine se
levèrent, et aussitôt don Rodrigo vint à moi et me conduisit avec
Maulevrier à la petite table près des fenêtres, dont j'ai fait mention.
Le roi et la reine s'y trouvèrent aussitôt que nous, et nous
commandèrent de signer en leur présence. On jugera bien, sans qu'on le
dise, qu'il n'y avait point de sièges, et que nous signâmes debout.
Comme je me mis en devoir de signer à côté du dernier infant, don
Joseph, qui était à côté de moi, m'arrêta et me montra à côté du
pénultième. J'en fis quelques petites difficultés, sur quoi il me fit
expliquer qu'il fallait que cela fût ainsi pour laisser place à la
signature de Maulevrier à côté de celle du dernier infant. Alors je
signai à côté de celle de l'infant don Ferdinand, et, après avoir dit
quelques mots de respect et de joie au roi et à la reine d'Espagne, qui
étaient tout près de moi, et s'étaient baissés sur la table pour me voir
mieux signer, je donnai la plume à Maulevrier, qui, après avoir signé,
la laissa sur la table. Comme cette manière de signer nous était plus
honorable que celle que j'étais près de garder, et que ce fut le
secrétaire d'État qui me la fit changer, je ne crus pas devoir résister
davantage. Je lis à Leurs Majestés Catholiques des remercîments de
l'honneur que leur joie et leur bonté nous venait de procurer de signer
en leur présence. Ce fut un redoublement de joie et de compliments à
Leurs Majestés Catholiques de ce qui se trouva là de plus près et de
plus familier avec elles. Les louanges de la contenance de l'infante
pendant un si longtemps en place devant tant de monde, et de sa
signature, ne furent pas oubliées. J'accompagnai le roi et la reine
jusqu'à la porte du salon des Miroirs, ayant soin alors, autant que cela
se put, de montrer toute déférence au majordome-major et au nonce, et
que je lui cédais pour leur ôter toute impression de dessein dans la
place que j'avais prise et maintenue.

Dès que Leurs Majestés Catholiques et les princes leurs enfants furent
rentrés, et aussitôt la porte du salon des Miroirs fermée sur eux, je
fus environné et, pour ainsi dire, presque étouffé de tout ce qui était
là, les uns après les autres à l'envi, avec les plus grandes
démonstrations de joie et mille compliments. La foule distinguée qui
sortit du salon des grands était grossie, dans le salon qui le sépare de
celui des Miroirs, de l'autre foule de gens de qualité, qui y avaient
attend la fin de la cérémonie pour voir repasser le roi et la reine, et
les plus considérables de ceux-là pour leur témoigner leur joie en
passant, à quoi, dans les deux salons, Leurs Majestés Catholiques se
montrèrent très affables par leur air et leurs réponses.

Pour achever ce qui regarde l'instrument français, je menai Maulevrier à
la cavachuela de Grimaldo. Je m'étais plaint cependant à Maulevrier sans
aigreur et avec beaucoup de mesure de ne m'avoir pas informé de la
lettre du cardinal Dubois. Il ne me répondit autre chose sinon, et très
froidement, qu'il me la ferait chercher. Arrivés ensemble chez le
marquis de Grimaldo, ce ministre soutint, mais avec beaucoup de
politesse, ce qu'il avait dit de cette lettre à la signature. Il ajouta
qu'il n'y avait qu'à se conformer à ce qui se passerait à Paris au
contrat de mariage du prince des Asturies, et qu'encore qu'il arrivât
qu'il n'y en fût pas signé d'instrument en espagnol, le roi d'Espagne
venait de le charger de m'assurer qu'il ne ferait aucune difficulté de
signer un instrument en français du contrat de mariage du roi, si je
persévérais ce nonobstant à le désirer. J'en remerciai extrêmement ce
ministre, auquel et encore moins au roi d'Espagne je ne voulus pas
témoigner la moindre chose sur Maulevrier dont le froid, l'embarras et
le silence portaient sa condamnation sur le front. Je ne voulus mander
cette altercation qu'au cardinal Dubois, et rien de cela à M. le duc
d'Orléans, ni dans la dépêche du roi qui se lisait au conseil de
régence, et encore ne m'en pris-je dans ma lettre au cardinal qu'à un
oubli ou à un défaut de mémoire de Maulevrier, avec lequel je continuai
de vivre comme auparavant, avec la politesse et les égards dus au
caractère que je lui avais apporté, et conférant avec lui de tout ce qui
regardait l'ambassade, tellement qu'il vint continuellement dîner chez
moi, souvent familièrement sans que je l'en priasse, et qu'il ne parut à
qui que ce fût que j'en fusse mécontent.

Ce n'était pas que je ne sentisse toute la conduite si pourpensée et si
parfaitement exécutée d'une noirceur si peu méritée, dont la perfidie me
commit d'une manière si publique en présence du roi et de la reine
d'Espagne, et de tout ce que leur cour avait de plus grand\,; mais la
façon dont j'en sortis, pleine des bontés du roi d'Espagne aussi
publiques\,; l'affront tacite que Maulevrier reçut dans une si auguste
assemblée de {[}m'avoir{]} laissé ou plutôt induit à m'embarquer en cet
instrument français, en ayant la négative en main de celle du cardinal
Dubois, d'en être convaincu par le ministre espagnol, à qui il l'avait
montrée, et par son propre aveu de me l'avoir cachée\,; l'indécence de
me brouiller et de vivre mal en pays étranger avec un collègue si
disproportionné et avec qui je ne pouvais éviter des rapports
nécessaires\,; et, s'il faut tout dire, le mépris extrême que j'en
conçus de lui\,; enfin le doute, si la scélératesse était de son cru ou
concertée et commandée par le cardinal Dubois, toutes ces raisons me
résolurent au parti que je pris là-dessus, jusqu'à glisser légèrement ou
éviter de répondre à beaucoup de seigneurs, qui m'en parlèrent sans
ménagement pour lui, parce qu'il était fort haï de toute la cour
d'Espagne, et jusque de la ville de Madrid et même du bas peuple, comme
j'aurai lieu de le répéter ailleurs\,; mais, tout en politesse et en
conduite ordinaire avec lui, je m'en gardai comme d'un très impudent
fripon, et je ne fus pas fâché de l'en laisser souvent apercevoir, sans
toutefois lui laisser la plus légère occasion de plainte.

Le lendemain du départ du roi, 20 novembre, pour achever cette matière,
Maulevrier vint le matin chez moi avec Robin, et m'apporta la lettre du
cardinal Dubois, par laquelle il lui mandait nettement qu'il ne doit y
avoir qu'un instrument du contrat de mariage, signé en la langue du pays
de la princesse où on contracte, et qu'il suffit d'en faire expédier une
copie traduite en l'autre langue, certifiée par le même secrétaire
d'État qui a reçu le contrat. C'était précisément ce que Grimaldo nous
avait dit chez lui et ce qui me fit demeurer d'accord avec lui de
différer jusqu'à Lerma à voir de quoi je me pourrais contenter.

Il venait de m'arriver un courrier de Burgos avec de meilleures
nouvelles de mon fils aîné. Ce courrier avait rencontré le roi et la
reine d'Espagne, qui l'avaient fait approcher de leur portière à la vue
de ma livrée. Ils s'étaient informés des nouvelles de mon fils et
{[}avaient{]} chargé le courrier de me dire de leur part la joie qu'ils
avaient de l'apparence de la guérison. J'avais donc à écrire au marquis
de Grimaldo pour remercier par lui Leurs Majestés Catholiques de ces
marques de bonté. J'y ajoutai ce que Maulevrier venait de me montrer de
la lettre du cardinal Dubois, dont je viens de parler, au moyen de quoi
je demeurais parfaitement content de ce qui s'était fait et n'en
demandais pas davantage. J'avais raison moyennant cette lettre d'être
content, puisqu'elle, ne demandait qu'une copie collationnée du contrat
en français, certifiée du secrétaire d'État, au lieu de quoi j'envoyais
au roi un instrument original du contrat de mariage en espagnol, signé
de la main du roi et de la reine d'Espagne, etc., tout tel et tout
pareil que celui qui demeurait à Leurs Majestés Catholiques, signé
d'elles, etc.\,; et qu'à l'égard des témoins on m'avait tenu exactement
parole, en sorte qu'ils n'avaient rien signé et n'avaient paru que dans
l'acte séparé, signé du seul secrétaire d'État uniquement, qui avait
passé le contrat, c'est-à-dire par don Joseph Rodrigo, Retournons
maintenant à ce qui se passa après la signature.

\hypertarget{chapitre-xiv.}{%
\chapter{CHAPITRE XIV.}\label{chapitre-xiv.}}

1721

~

{\textsc{Forme de demander les audiences particulières du roi
d'Espagne.}} {\textsc{- Jalousie de la reine pour y être toujours
présente.}} {\textsc{- Trait important d'amitié pour moi de Grimaldo.}}
{\textsc{- Illumination de la place Major, admirable et surprenante.}}
{\textsc{- Bal superbe chez le roi d'Espagne.}} {\textsc{- Leurs
Majestés Catholiques y dansent et m'y font danser.}} {\textsc{- Échappé
avec tout avantage de tous les pièges du cardinal Dubois, j'en aperçois
son dépit à travers ses louanges.}} {\textsc{- Audience particulière que
j'eus seul le lendemain de la signature.}} {\textsc{- Manége de la
reine.}} {\textsc{- Service de Grimaldo.}} {\textsc{- Office à don
Patricio Laullez.}} {\textsc{- Attachement du roi d'Espagne aux
jésuites, peu conforme au goût de la reine.}} {\textsc{- Bontés ou
compliments singuliers de la reine pour moi.}} {\textsc{- Audience
particulière du comte de Céreste.}} {\textsc{- Je consulte Grimaldo sur
les bontés ou les compliments de la reine.}} {\textsc{- J'en reçois un
bon conseil.}} {\textsc{- Confiance et amitié véritable entre ce
ministre et moi.}} {\textsc{- Pompe de Leurs Majestés Catholiques allant
à Notre-Dame d'Atocha.}} {\textsc{- Compétence}} \footnote{On a déjà vu
  ce mot employé par Saint-Simon dans le sens de discussion de
  préséance.} {\textsc{entre les deux majordomes-majors, uniquement aux
audiences publiques de la reine, qui en exclut celui du roi, et entre
les mêmes et les deux grands écuyers, uniquement dans les carrosses du
roi et de la reine, qui en exclut les deux majordomes-major.}}
{\textsc{- Départs (18 novembre) de M\textsuperscript{lle} de
Montpensier de Paris.}} {\textsc{- Leurs Majestés Catholiques donnent
une longue audience à Maulevrier et à moi seuls, étant au lit, contre
tout usage d'y être vus par qui que ce soit.}} {\textsc{- Maulevrier en
étrange habitude de montrer au ministre d'Espagne les dépêches qu'il
devais de sa cour.}} {\textsc{- Départ de Leurs Majestés Catholiques
pour Lerma.}} {\textsc{- Je présente enfin une lettre du roi à l'infante
au moment de son départ pour Lerma.}} {\textsc{- Je reçois chez moi les
compliments de la ville de Madrid.}} {\textsc{- Lettre curieuse du
cardinal Dubois à moi, sur l'emploi de l'échange des princesses.}}
{\textsc{- Santa-Cruz chargé par le roi d'Espagne de l'échange des
princesses.}} {\textsc{- Je prends avec lui d'utiles précautions à
l'égard du prince de Rohan, chargé par le roi du même échange.}}

~

Je retournai chez moi après la cérémonie qui, par la longueur des
lectures et cette difficulté sur un instrument en français, avait duré
fort longtemps. On se souviendra que, voulant toujours entretenir le roi
d'Espagne pendant cette lecture pour cacher par cet air de courtisan
empressé l'affectation de la place que j'avais prise et conservée, ne
sachant plus que dire au roi pour continuer à lui parler, je lui
demandai audience pour le lendemain, qu'il m'accorda volontiers. Or,
cette demande directe était contraire à l'usage de cette cour, où les
ambassadeurs, les autres ministres étrangers, et tous les sujets de
quelque rang ou état qu'ils soient, ne la demandent qu'en s'adressant à
celui qui est préposé pour en rendre compte au roi et leur dire le jour
et l'heure, quand le roi accorde l'audience, qu'il ne refuse jamais aux
ministres étrangers et rarement à ses sujets. Celui qui avait alors cet
emploi était le même La Roche dont j'ai parlé ci-devant, et qui avait
aussi l'estampille.

Grimaldo était allé travailler avec le roi en présence de la reine comme
cela se faisait toujours\,; peu après la fin de la cérémonie de la
signature. Je fus surpris, une heure et demie après être rentré chez
moi, de recevoir une lettre de ce ministre, qui me demandait si j'avais
à dire quelque chose de particulier au roi sans la reine, sur ce que
j'avais demandé moi-même audience au roi pendant la lecture du contrat,
et qu'il me priait de lui mander naturellement ce qui en était. Je lui
récrivis sur-le-champ qu'ayant trouvé cette commodité de demander
audience au roi je m'en étais servi tout simplement\,; que, si je n'y
avais pas fait mention de la reine, c'est que j'avais cru sa présence
aux audiences particulières tellement d'usage que je n'avais pas imaginé
qu'il fût besoin d'en faire mention\,; qu'au reste je n'avais que des
remercîments à faire au roi sur tout ce qui venait de se passer, quoi
que ce soit à lui dire que je n'eusse à dire de même à la reine, et que
je serais très fâché qu'elle ne se trouvât pas à cette audience
particulière le lendemain.

Comme j'écrivais cette réponse, don Gaspard Giron m'invita d'aller voir
l'illumination de la place Major. J'achevai ma lettre promptement\,;
nous montâmes en carrosse, et les principaux de ceux que j'avais amenés,
dans d'autres des miens. Nous fûmes conduits par des détours pour éviter
la vue de la lueur de l'illumination en approchant, et nous arrivâmes à
une belle maison qui donne sur le milieu de la place, qui est celle où
le roi et la reine vont pour voir les fêtes qui s'y font. Nous ne nous
aperçûmes d'aucune clarté en mettant pied à terre ni en montant
l'escalier\,; on avait bien tout fermé\,; mais en entrant dans la
chambre qui donnait sur la place, nous fûmes éblouis, et tout de suite
en entrant sur le balcon la parole me manqua de surprise plus de sept ou
huit minutes.

Cette place est en superficie beaucoup plus vaste qu'aucune que j'eusse
encore vue à Paris ni ailleurs, et plus longue que large. Les cinq
étages des maisons qui l'environnent sont du même niveau, chacune avec
des fenêtres égales en distance et en ouverture, qui ont chacune un
balcon dont la longueur et l'avance sont parfaitement pareilles, avec un
balustre de fer aussi de hauteur et d'ouvrage semblables entre eux, et
tout cela parfaitement pareil en tous les cinq étages. Sur chacun de
tous ces balcons on met deux gros flambeaux de cire blanche, un seul à
chaque bout de chaque balcon, simplement appuyés contre le milieu du
retour de la balustrade, tant soit peu penchés en dehors, sans être
attachés à rien. Il est incroyable la clarté que cela donne, la
splendeur en étonne et a je ne sais quelle majesté qui saisit. On y lit
sans peine les plus petits caractères dans le milieu et dans tous les
endroits de la place sans que le rez-de-chaussée soit illuminé.

Dès que je parus sur le balcon, tout ce qui était dans la place s'amassa
sous les fenêtres et se mit à crier\,: Señor, tauro\,! tauro\,! C'était
le peuple qui me demandait d'obtenir une fête de taureaux, qui est la
chose du monde pour laquelle il a le plus de passion, et que le roi ne
voulait plus permettre depuis plusieurs années par principe de
conscience. Aussi me contentai-je le lendemain de lui dire simplement
ces cris du peuple sans lui rien demander là-dessus, en lui témoignant
mon étonnement d'une illumination si surprenante et si admirable. Don
Gaspard Giron et des Espagnols qui se trouvèrent dans la maison d'où je
la vis, charmés de l'étonnement dont j'avais été frappé à la vue de ce
spectacle, le publièrent avec d'autant plus de complaisance, qu'ils
n'étaient pas accoutumés à l'admiration des Français, et beaucoup de
seigneurs m'en parlèrent avec grand plaisir. À peine eus-je loisir de
souper au retour de cette belle illumination, qu'il fallut retourner au
palais pour le bal que le roi avait fait préparer dans le salon des
Grands, et qui dura jusqu'après deux heures après minuit.

Ce salon, qui est également vaste et superbe en bronzes, en marbres, en
dorures, en tableaux, était magnifiquement éclairé\,; tout au bout
opposé à la porte d'entrée il y avait, comme à la signature, six
fauteuils de front, où le roi, la reine, etc., s'assirent dans le même
ordre. À côté du bras droit de celui du roi, sans distance aucune et
beaucoup moins qu'un demi-pied moins avancé, un siège ployant de velours
cramoisi de franges d'or et les bois dorés, pour le majordome-major du
roi, qui s'assit dessus en même temps que le roi se mit dans son
fauteuil. Au bras gauche du fauteuil du dernier infant était dans la
même disposition un carreau de velours noir, sans or, avec des houppes
noires aux coins, pour la camarera-mayor de la reine, vêtue en veuve un
peu mitigée, parce que la reine n'avait pu souffrir tout ce grand
attirail de religieuse, qui est l'habit des veuves tant qu'elles le
sont, que j'avais vu à Bayonne à la duchesse de Liñarez. Par la même
raison, le carreau était noir, qui sans cela aurait été de velours
cramoisi avec de l'or. Cette dame aurait pu avoir un ployant pareil à
celui de la droite, mais par habitude elle préférait le carreau, qui est
la même distinction. Derrière les fauteuils il y avait des tabourets de
velours rouge à franges d'or et à bois dorés, pour le capitaine des
gardes du roi en quartier, le sommelier du corps, le majordome-major de
la reine, la gouvernante de l'infante et le duc de Popoli, gouverneur du
prince des Asturies. Dans une fausse porte, tout en arrière des
fauteuils du côté de la camarera-mayor, mais non vis-à-vis de son dos,
étaient deux sièges ployants de velours cramoisi à franges d'or et à
bois dorés, où don Gaspard Giron nous conduisit, Maulevrier et moi, sans
jalousie devant nous, qui fut une faveur singulière, et qui que ce soit
devant nous, en sorte que nous vîmes toujours en plein tout ce beau
spectacle et les danses.

Un peu plus bas que la camarera-mayor, le long de la muraille, à quelque
distance jusque vers le bas bout, il y avait des tabourets comme les
nôtres entremêlés de carreaux pareils, et d'autres tabourets et carreaux
de damas et de satin rouge, pareillement dorés, pour les femmes des
grands d'Espagne et de leurs fils aînés qui, à leur choix, s'asseyaient
sur les tabourets ou sur les carreaux, mais les femmes des grands sur le
velours et les femmes des fils aînés sur le satin ou le damas. Ces
tabourets et ces carreaux allaient jusqu'à la moitié ou environ de la
longueur de ce côté long du salon, le reste était occupé par les dames
de qualité, femmes ou filles, assises par terre sur le vaste tapis qui
couvrait tout le salon, desquelles plusieurs se tenaient debout, ce qui
était à leur choix, et tout aux dernières places. Quelques jeunes
camaristes de la reine placées là pour danser. Vis-à-vis ce long rang de
dames de l'autre côté, toute la cour en hommes, grands et autres, tous
debout, le dos aux fenêtres à distance d'elles, laquelle distance était
remplie de moindres spectateurs, comme aussi était l'espace vis-à-vis,
entre la muraille et les dames. Au bas bout du côté des hommes étaient,
un peu en potence, les quatre majordomes du roi pour donner ordre à
tout. Vis-à-vis des fauteuils, au bas bout, étaient les danseurs debout,
grands et autres, les officiers venus en Espagne avec moi, et des
spectateurs de qualité\,; une barrière derrière eux traversait le salon,
derrière laquelle était la foule des voyeurs.

Dans une pièce à côté de l'entrée étaient toutes sortes de
rafraîchissements, de pâtisseries, de vins, avec profusion\,; mais grand
ordre, où, pendant la confusion des contredanses, allait qui voulait et
en apportait aux dames. La parure éclatait avec somptuosité\,: il faut
avouer que le coup d'oeil de nos plus beaux bals parés n'approche point
de celui-là.

Ce qui m'y parut de fort étrange furent trois évêques en camail vers le
haut bout du côté des hommes pendant tout le bal\,; c'étaient le duc
d'Abrantès, évêque de Cuença, deux évêques in partibus, suffragants à
Madrid de l'archevêque de Tolède\,; et l'accoutrement de la
camarera-mayor pour un bal, qui tenait un grand chapelet à découvert,
causant et devisant sur le bal et les danses, tout en marmottant ses
patenôtres qu'elle laissait tomber à mesure, tant que le bal dura. Ce
que je trouvai aussi de très fâcheux est que nul homme ne s'y assit
excepté les six charges que j'ai nommées, Maulevrier {[}et{]} moi, pas
même les danseurs, en sorte qu'il n'y avait pas un seul siège dans tout
ce salon, même derrière tout le monde, outre ceux que j'ai spécifiés.

La reine, qui ne peut danser de danse sérieuse qu'avec les infants,
ouvrit le bal avec le roi\,; la danse de ce prince qu'il aimait fort fut
pour moi un grand sujet de surprise\,; en dansant ce fut tout un autre
homme, redressé du dos et des genoux, de la justesse, en vérité de la
grâce. Pour la reine qui prit après le prince des Asturies, qui étaient
tous deux extrêmement bien faits, je n'ai vu qui que ce soit danser
mieux en France, en hommes ni en femmes, peu en approcher, moins encore
aussi bien, les deux autres infants fort joliment pour leur âge.

En Espagne, hommes et femmes portent toutes sortes de couleurs à tout
âge, et danse qui veut jusqu'à plus de soixante ans, sans le plus léger
ridicule, même sans que cela paroisse extraordinaire, et j'en vis
plusieurs exemples d'hommes et de femmes\,: le dernier infant prit la
princesse de Robecque qui ne s'éloignait pas de cinquante ans et qui les
paraissait bien.

Elle était Croï, fille du comte de Solre, et veuve du prince de
Robecque, que le roi d'Espagne avait fait par la princesse des Ursins
grand d'Espagne, chevalier de la Toison et depuis colonel du régiment
des gardes wallonnes. La comtesse de Solre qui était Bournonville,
cousine germaine de la maréchale de Noailles, étant assez mal avec son
mari avait mené sa fille se marier en Espagne et y était demeurée avec
elle. M\textsuperscript{me} de Robecque était dame du palais de la reine
et passait, ainsi que sa mère, pour être fort bien avec elle. Je les
avais fort connues avant qu'elles allassent en Espagne\,; et ce fut une
des premières visites que je fis\,; nous avions autrefois fort dansé
ensemble, apparemment qu'elle le dit à la reine.

Aussitôt après avoir dansé avec l'infant, car, étant étrangère, elle
n'était pas sujette aux règles espagnoles du veuvage, elle traversa
toute la longueur du salon, fit une belle révérence à Leurs Majestés
Catholiques et vint me dénicher dans ma reculade pour me prendre à
danser par une belle révérence en riant\,; je la lui rendis en lui
disant qu'elle se moquait de moi\,; dispute, galanteries, enfin elle fut
à la reine, qui m'appela et qui me dit que le roi et elle voulaient que
je dansasse. Je pris la liberté de lui représenter qu'elle voulait se
divertir\,; que cet ordre ne pouvait pas être sérieux\,; j'alléguai mon
âge, mon emploi, tant d'années que je n'avais dansé, en un mot tout ce
qui me fut possible. Tout frit inutile, le roi s'en mêla, tous deux me
prièrent, tachèrent de me persuader que je dansais fort bien, enfin
commandèrent et de façon qu'il fallut obéir\,; je m'en tirai donc comme
je pus.

La reine affecta de faire danser des premiers nos témoins français,
excepté l'abbé de Saint-Simon qui n'était pas de robe à cela, et dans la
suite du bal, deux ou trois officiers des plus distingués des troupes du
roi qui étaient venus avec moi.

Une heure après l'ouverture du bal on mena l'infante se coucher. Les
contredanses coupèrent souvent les menuets. Le prince des Asturies y
menait toujours la reine\,; rarement le roi les dansait, mais comme aux
contredanses on se mêle, et, suivant l'ordre de la contredanse, chacune
se trouve danser avec tout ce qui danse, l'un après l'autre, et se
retrouve au bout avec son meneur, la reine y dansait de même avec tout
le monde\,; j'en esquivai ce que je pus, quoique fort peu\,; on peut
juger que je n'en savais aucune.

Le bal fini, le marquis de Villagarcias, un des majordomes et un des
plus honnêtes et des plus gracieux hommes que j'aie vus, qui a été
depuis vice-roi du Pérou, ne voulut jamais me laisser sortir que je ne
me fusse reposé dans le lieu des rafraîchissements, où il me fit avaler
un verre d'excellent vin pur, parce que j'étais fort en sueur à force de
menuets et de contredanses, avec un habit très pesant. Le roi et la
reine d'Espagne et le prince des Asturies furent fort sur le bal et y
parurent prendre grand plaisir. Ce même soir et le lendemain je fis
illuminer toute ma maison, dedans et dehors, n'ayant pas eu un moment de
loisir d'y donner aucune fête, an milieu de tant de fonctions si
précipitées et si fort entassées les unes sur les autres.

Ce ne fut pas sans un grand plaisir que je fis, le mercredi 26 au matin,
lendemain de la signature, les dépêches que je devais envoyer après mon
audience de remercîment qui devait terminer cette même matinée, par
lesquelles je rendais compte de tout ce qui s'était passé par un
courrier qui ne put être dépêché que le {[}sur{]} lendemain 28 novembre.
J'étais aisément parvenu à éluder les commissaires et à faire signer par
Leurs Majestés Catholiques elles-mêmes, contre tout usage et exemple,
non seulement un instrument du contrat du futur mariage du roi et de
l'infante, mais deux instruments dont j'envoyai un au roi signé de leur
main par ce courrier, ce qui était bien plus qu'il ne m'avait été
demandé, puisque le cardinal Dubois se contentait d'une simple copie
signée du seul secrétaire d'État. J'avais fait passer l'entreprise de M.
le duc d'Orléans sur le prince des Asturies sans aucune difficulté et
lui avais renvoyé sa lettre à ce prince où la qualité de frère était
omise. Les témoins du mariage, je ne les admis qu'à condition qu'ils ne
paraîtraient tels que dans un acte séparé, signé du seul secrétaire
d'État, et qu'eux ne signeraient quoi que ce fût. J'étais sorti du piége
qui m'avait été si bien tendu sur l'instrument du contrat en français,
tellement à mon avantage, que l'infamie en sauta aux yeux de Leurs
Majestés Catholiques et de tout ce qu'il y avait de plus illustre en
Espagne rassemblé dans la cérémonie de la signature, et que Leurs
Majestés Catholiques voulurent bien me promettre de signer un instrument
en français si je persévérais à le désirer. Enfin, la joie du sujet de
mon ambassade qui m'attira en foule les premières visites dès le matin
du lendemain de mon arrivée, de tous ceux même qui étoient en droit et
en usage d'attendre auparavant la mienne, et si j'ose le dire, l'adresse
que je sus employer pour la place que je pris et que je conservai à la
signature, me tirèrent des étranges filets où le cardinal Dubois avait
bien compté de me prendre.

Le tour des louanges excessives qu'il me donna en réponse aux dépêches
de ce courrier, et dont il farcit celle du roi et celle de M. le duc
d'Orléans, et les bagatelles qu'il conta sans oser les désapprouver
ouvertement, comme la difficulté des témoins, celle de l'instrument en
français, qui du moins était la faute de son silence, celle de la petite
table pour signer, celle de n'avoir pas été à Notre-Dame d'Atocha,
toutes choses auxquelles je sus très bien lui répondre, me montrèrent le
dépit, caché sous tant de fleurs et de parfums, qu'il ressentait de me
voir échapper contre toute espérance à tant de sortes de parties qu'il
avait pris tant de soin à me dresser. Il loua surtout ma modération à
l'égard de Maulevrier en tombant sur lui, soit qu'il le blâmât en effet,
ou qu'il voulût me cacher par le mépris et le peu de confiance qu'il me
témoigna pour lui, qu'il eût part en sa noire et hardie friponnerie,
trop profonde et trop adroitement ourdie, et exécutée avec trop
d'effronterie pour la croire du seul cru de Maulevrier, dont la malice,
quelle qu'elle pût être, était trop dépourvue d'esprit pour pouvoir lui
en attribuer plus que la simple exécution. Je ne parle point ici de la
lettre du roi à l'infante qui était lors encore à venir. Ce ne fut
qu'une niche en comparaison des autres pièges et niche dont je me donnai
le plaisir de lui mander comment je m'en étais tiré par le secours du
marquis de Grimaldo\,; mais s'il eut le chagrin de me voir hors des
prises qu'il s'était si bien su préparer, pour ce qui regardait les
affaires et les fonctions de l'ambassade, on verra qu'il sut bien s'en
dédommager sur ma bourse, et que ce ne fut pas sa faute si je ne revins
pas sans avoir pu recueillir le fruit qui, uniquement, m'avait fait
désirer cette ambassade.

Tout à la fin de la matinée de ce même mercredi 26, je fus introduit
seul, car Maulevrier s'excusa d'y venir avec moi sur les dépêches qu'il
avait à faire, je fus, dis-je, à l'audience que j'avais moi-même
demandée au roi d'Espagne, la veille, pendant la lecture du contrat de
mariage, et qu'il m'avait accordée. Je vis, dès en approchant de Leurs
Majestés Catholiques, l'importance du service que {[}m'avait rendu{]} le
marquis de Grimaldo par la lettre qu'il m'écrivit le soir tout tard de
la veille, dont j'ai parlé ci-dessus, et de ma réponse\,; car la reine,
dès avant que je fusse proche du roi et d'elle, s'avança à moi, et me
dit d'un air fort libre\,: «\,Ho çà, monsieur, point de façons\,; vous
avez envie de dire au roi quelque chose en particulier, je m'en vais à
la fenêtre et vous laisser faire.\,» Je lui répondis la même chose que
ce que j'avais mandé en réponse à Grimaldo, à quoi j'ajoutai qu'il était
si vrai que je n'avais rien à dire au roi en particulier, que si j'avais
eu le déplaisir de ne la pas trouver auprès de lui, j'aurais été obligé
de lui demander à elle une audience pour lui faire les mêmes
remercîments qu'au roi de tout ce qui s'était passé la veille. «\, Non,
non, reprit-elle avec vivacité, je vous laisse avec le roi, et je me
rapprocherai quand vous aurez fait.\,» Et en disant cela, elle gagna la
fenêtre comme en deux sauts légers, car il y avait assez loin par la
grandeur de ce salon des Miroirs où j'étais seul avec Leurs Majestés
Catholiques, tellement que je me mis à la suivre, lui protestant que je
n'ouvrirais pas la bouche devant le roi qu'elle ne fût retournée près de
lui, qui, pendant tout cela, demeura immobile\,; enfin la reine se
laissa vaincre et revint près du roi où je la suivis. Elle aurait su
également par le roi ce que je lui aurais dit sans elle, et ne me
l'aurait jamais pardonné.

Je commençai alors par les remercîments de tout ce qui s'était passé la
veille, en attendant ceux dont je serais chargé par le roi dès qu'il
aurait reçu le compte que j'avais l'honneur de lui en rendre. On peut
juger que ce que je dis ici en deux mots se débita à Leurs Majestés
Catholiques d'autre sorte, et que les grâces de l'infante, à se tenir si
convenablement et si longtemps en place et à signer, ne furent pas
oubliées, non plus que la beauté si surprenante de l'illumination de la
place Major, la magnificence singulière du bal, et les grâces de Leurs
Majestés Catholiques et du prince des Asturies, et des jeunes infants à
danser, tous articles que j'étendis assez à mesure du plaisir que je
voyais qu'elles y prenaient, et sur quoi la reine se mit fort à louer le
roi d'Espagne., et à me faire admirer jusqu'à sa beauté, dont il ne fit
que sourire. Il me demanda si je n'enverrais pas un courrier\,; je
répondis que l'instrument signé de leurs mains, etc., était trop
précieux pour le confier à la voie ordinaire\,: il me parut qu'ils en
avaient fort envie, et que ma réponse leur plut.

Je passai de là à l'office en faveur de don Patricio Laullez, dont je
m'étais procuré l'ordre, et dont on a vu que j'avais parlé à Grimaldo
qui en avait prévenu Leurs Majestés. Je me mis donc, tant que je pus,
sur mon bien-dire par la passion que j'avais de rendre utilement à cet
ambassadeur les services que j'en avais premièrement reçus. Il me parut
que le roi d'Espagne m'écouta là-dessus avec satisfaction, mais beaucoup
plus la reine, qui en mêla quelques mots à mon discours en regardant le
roi avec un désir très marqué d'en attirer des grâces à Laullez.

Le roi d'Espagne interrompit ce propos pour me dire, sans occasion et
tout à coup qu'il désirait que l'infante fût mise sous la conduite d'un
jésuite, pour former sa conscience et lui apprendre la religion\,; qu'il
avait eu toute sa vie confiance aux pères de la compagnie, et qu'il me
priait de le demander de sa part à M. le duc d'Orléans. Je répondis que
j'exécuterais avec beaucoup d'exactitude et de respect le commandement
qu'il me faisait, et que je ne doutais point que M. le duc d'Orléans ne
cherchât à lui complaire dans toutes les choses qui n'avaient aucun
véritable inconvénient. Je remarquai qu'il prolongea cette proposition
qui pouvait être plus courte, et qu'il me regardait cependant fixement
comme cherchant à voir ce que j'en pensais moi-même. Ce désir me parut
en lui d'autant plus affectionné, que la reine, qui entrait toujours
dans tout ce qu'il disait, et qui l'appuyait, ne dit alors presque
rien\,; que le peu qu'elle dit fut très -faible, le roi poussant
toujours sa pointe.

Après quelques autres affaires de simple recommandation, l'audience se
tourna en conversation. Ils me menèrent aux fenêtres voir leur belle vue
sur le Mançanarez, la Casa del Campo presque vis-à-vis, et la campagne
au delà\,; on parla de plusieurs choses indifférentes qui conduisirent à
des choses de leur cour, et moi à leur témoigner la satisfaction que
j'avais d'avoir l'honneur de les approcher dans tous les moments où cela
était permis. Là-dessus la reine regarda le roi, puis me dit avec un air
de bonté qu'il ne fallait point qu'il y eût d'heure pour moi, ni
d'étiquette\,; que je pouvais les venir voir à toute heure, quand je
voudrais, sans audience et sans avoir rien à leur communiquer\,; que le
roi et elle seraient ravis de me voir ainsi familièrement, et que je
leur ferais plaisir d'user de cette liberté. Je ne manquai pas de
répondre à une grâce si peu attendue et si unique de la meilleure façon
que je pus\,; après quoi je leur dis que le marquis de Grimaldo devait
leur avoir rendu compte que le comte de Céreste, frère du marquis de
Brancas, désirait avoir l'honneur de présenter au roi une lettre de son
frère. Je fus congédié après un peu moins d'une heure d'audience ou de
conversation, en me disant que Céreste allait être appelé. Il le fut en
effet quelques moments après que je fus sorti. Le marquis de Brancas
avait eu permission d'écrire au roi d'Espagne, et il avait chargé son
frère d'y ajouter quelque chose de bouche en présentant sa lettre. Je
l'attendis\,; il me dit que cette audience s'était tout à fait passée à
sa satisfaction.

Quoique en me retirant d'auprès de Leurs Majestés Catholiques, la reine
m'eût encore répété de ne me point arrêter aux usages pour les voir à
toute heure quand je voudrais, et de ne pas craindre d'en abuser, et que
le ton et l'air du discours fût tout à fait naturel et avec beaucoup de
grâces, je crus devoir en faire la confidence à Grimaldo et le consulter
là-dessus. Je craignis que ce couvi redoublé de chose qui sans exception
n'était accordée à personne ne fût qu'un excès, si j'ose user du terme,
de politesse, où la joie et le désir de la marquer les jetait, dont
l'usage, quelque discret qu'il fût, pourrait les importuner. J'eus peur
aussi qu'en usant sans l'attache pour ainsi dire de Grimaldo, il n'en
conçut de la jalousie et de la froideur à mon égard, lui sans qui je ne
pouvais rien faire, quelque privance dont je jouisse, et je compris
qu'abandonnant là-dessus ma conduite à son jugement, je le gagnerais
véritablement, et que je ne pourrais mal faire.

Je descendis donc dans sa cavachuela au sortir de l'audience. Je lui
racontai tout ce qui s'était passé et lui dis que, pour l'usage ou
non-usage de cette liberté de voir à toute heure et sans audience Leurs
Majestés Catholiques quand je voudrais, je venais franchement à son
conseil, résolu de me conduire en cela uniquement par ce qu'il jugerait
à propos que je fisse, ce que j'assaisonnai de tout ce que je crus le
plus propre à le flatter et à l'ouvrir sincèrement. Après les préambules
de remercîments et de compliments sur ma confiance, il me dit que,
puisque je voulais qu'il me parlât franchement, il me conseillait de
regarder l'invitation de la reine comme une politesse, une honnêteté
singulière qu'elle avait voulu me faire, mais dont le roi et elle ne
seraient pas fort aises que j'en usasse, et qu'ils s'en trouveraient
bientôt importunés\,; que, de plus, je n'avancerais rien dans ces
particuliers, si j'y voulais mêler des affaires sur lesquelles ils ne me
répondraient point sans s'en être consultés, et que cela les
embarrasserait davantage\,; enfin qu'ils me verraient sûrement de
meilleur oeil dans les temps où il était permis à tout le monde de les
voir, et en audience quand j'aurais raison et occasion d'en demander, et
qu'il s'offrait à moi pour tous les offices et toutes les choses où je
voudrais l'employer auprès de Leurs Majestés, soit de ma part, soit
comme de lui-même. Je le remerciai fort de son conseil, que je l'assurai
que je suivrais, comme je fis en effet, et j'acceptai ses offres avec
tous les témoignages de confiance et de reconnaissance qu'ils
méritaient, et je me trouvai parfaitement de l'un et de l'autre\,; de
cette façon je fus avec ce ministre sur un pied d'amitié, de liberté, de
confiance, qui, outre les agréments, les facilités et la commodité qu'il
me procura, me fut aussi extrêmement utile.

L'après-dînée de ce jour, mercredi 26, le roi et la reine d'Espagne
allèrent en pompe à Notre-Dame d'Atocha, c'est la grande dévotion du
pays, qui est tout au bout et comme hors de la ville, joignant le parc
du Buen Retiro. L'église est grande, médiocrement belle pour l'Espagne,
desservie par une grande communauté de dominicains logés dans un vaste
et superbe monastère. Le roi, sans entrer dans le couvent, met pied à
terre à un petit corps de logis où on trouve d'abord un escalier de
quelques marches, deux assez grandes pièces de la dernière desquelles le
roi et la reine entrent dans une grande tribune, et leur suite dans une
autre fort longue à tenir vingt personnes tout du long.

Les descriptions des lieux ne sont point de mon sujet, mais je ne crois
pas devoir me dispenser de décrire comment le roi y va en cérémonie avec
la reine, comme il fit à cette fois, et comme il est d'usage que les
rois d'Espagne y aillent de la sorte toutes les fois qu'une calamité ou
une occasion de remercier Dieu publiquement oblige à des prières ou à
des actions de grâces publiques, et toutes les fois encore que les rois
partent pour un voyage long et éloigné et qu'ils en reviennent à Madrid.
Voici donc l'ordre de la marche\,: un carrosse du roi où sont ses quatre
majordomes\,; trois autres, mais du corps, pour les gentilshommes de la
chambre\,; un du corps plus beau rempli par le grand écuyer, le
sommelier du corps, le capitaine des gardes en quartier\,; un carrosse
du roi vide\,; le carrosse où le roi et la reine sont seuls\,; un
carrosse de la reine vide, un carrosse de la reine où sont son grand
écuyer et son majordome-major. Mais ce carrosse ne va plus, parce que le
majordome-major n'y veut pas céder la première place au grand écuyer qui
l'a de droit sur lui et sur tous, dans le carrosse seulement\,; ainsi le
grand écuyer de la reine se met dans le carrosse du roi, avec son grand
écuyer, et y a place immédiatement avant le capitaine des gardes du
corps en quartier. Ainsi, après le carrosse vide de la reine, marche le
carrosse propre de sa camarera-mayor, carrosse encore une fois non de la
reine, mais de la camarera-mayor, à quatre mules, à ses armes et à ses
livrées, entouré de toute sa livrée à pied, son écuyer à cheval, à sa
portière droite, et elle seule dans son carrosse\,; deux carrosses de la
reine remplis de ses dames du palais\,; deux autres carrosses de la
reine qui ne sont pas du corps et plus simples que les précédents,
remplis des señoras de honor\,; un carrosse de la reine, non du corps et
plus uni encore que les deux derniers précédents, dans lequel est
l'assafeta\footnote{Le sens de ce mot a été indiqué plus haut.
  L'assafeta était la première femme de chambre de la reine.} toute
seule, puis deux carrosses semblables à ce dernier remplis des
camaristes de la reine. Le carrosse à huit chevaux avec un postillon,
dans lequel sont le roi et la reine, est environné de valets de pied à
pied, de plusieurs officiers des gardes du corps à cheval, avec chacun
leur premier écuyer à leur portière, tous à cheval, et force gardes du
corps devant et derrière, avec les trompettes et les timbales sonnantes.
Les régiments des gardes espagnoles et wallonnes, partie en bataille
dans la place du Palais, partie en haie dans les rues, les officiers à
leur tête et les drapeaux déployés, saluant dans la place avec force
tambours battant au champ. La marche se fait au plus petit pas\,; les
cochers des carrosses du corps du roi et de la reine et de ceux réputés
tels, ainsi que le cocher de la camarera-mayor, sont chapeaux bas. Ceux
des carrosses des majordomes du roi, des señoras de honor, de l'assafeta
et des camaristes, ont leurs chapeaux sur leur tête.

Une des plus belles, des plus larges, des plus droites et des plus
longues rues de Madrid, fait le principal du chemin. Il y demeure un
grand nombre d'orfèvres. Toutes les boutiques sont ornées de gradins
chargés avec élégance de tout ce que ces orfèvres ont de plus riche\,;
les autres boutiques, à proportion par toutes les rues. Tous les
balcons, dont il y a quantité à Madrid, et les fenêtres de tous les
étages magnifiquement ornés de tapis pendants larges et bas, et de
coussins sur les fenêtres, remplies entièrement de spectateurs et de
dames parées, et tout cela admirablement illuminé au retour, ainsi que
la place Major par où le roi revint. Il faut convenir que ce spectacle
est admirable par son ordre, car les rues sont pleines de peuple sans en
être le moins du monde surchargées ni embarrassées, et qu'il est le plus
imposant que j'aie jamais vu par sa majesté et par la plus superbe
magnificence et la plus parfaitement ordonnée. Les grands étaient allés
attendre le roi à Notre-Dame d'Atocha, mais dans l'église, et le
majordome-major du roi aussi, parce qu'il ne va jamais dans le carrosse
où est le grand écuyer, qui est celui où il devrait aller, parce que, le
précédant partout, il n'a pourtant que la seconde place dans le
carrosse, où le grand écuyer est en droit et en usage de ne la céder à
lui ni à qui que ce soit. C'est encore par la même raison que le
majordome-major du roi ne se trouve jamais aux audiences publiques de la
reine, et n'y vint pas aussi à la mienne, parce {[}que{]}, précédant
partout le majordome-major de la reine, celui-ci est en droit et en
usage de la première place, et distinguée, en ces audiences de la reine,
et de ne la pas céder au majordome-major du roi.

Je crus que Maulevrier et moi devions nous trouver aussi à Notre-Dame
d'Atocha, étant si principaux acteurs dans l'affaire qui engageait Leurs
Majestés Catholiques à y aller rendre à Dieu leurs actions de grâces.
Maulevrier fut sagement, pour cette fois, fort d'avis de s'informer au
marquis de Montalègre, sommelier du corps, comme au plus expert aux
cérémonies et aux usages de la cour d'Espagne, pour savoir s'il n'y
aurait point d'inconvénient. Montalègre crut qu'il s'y en pourrait
rencontrer, et lui conseilla que nous nous abstinssions d'y aller. Sur
cet avis je crus, ainsi que Maulevrier, que nous ferions bien de le
suivre. Nous vîmes donc la marche du roi y allant, et pour son retour
nous allâmes le voir passer dans la place Major illuminée, dans la même
maison où j'avais déjà vu cet éclatant et si surprenant spectacle. Je ne
sus point la raison de l'avis du marquis de Montalègre. J'imaginai que
le roi d'Espagne étant en des tribunes et non dans l'église où étaient
les grands, il y aurait de la difficulté à nous placer, qui disparaît
quand le roi tient chapelle, où il est dans l'église et où la place des
ambassadeurs est établie. J'oublie, ce que j'aurais dû ajouter en sa
place, que le majordome-major de la reine se trouve sans difficulté aux
audiences publiques du roi d'Espagne, où il prend place parmi les
grands, quand il l'est, comme il l'est presque toujours, et sans aucune
prétention de distinction.

Le jeudi 27 novembre, jour du départ du roi et de la reine pour Lerma,
et lendemain de leurs pompeuses actions de grâces à Notre-Dame d'Atocha,
Maulevrier vint chez moi le matin de fort bonne heure avec les dépêches
qu'un courrier venait de lui apporter et leur duplicata pour moi. Le
cardinal Dubois avait calculé sur mes lettres de Bordeaux que je
n'arriverais que le 28, à Madrid, et avait chargé le courrier, qui vint
chez moi avec Maulevrier, de me remettre où il me rencontrerait le
paquet qui m'était adressé, qui contenait le duplicata de celui qui
était adressé à Maulevrier, et de continuer sa course ensuite pour le
lui porter. Ce courrier apportait l'avis du départ de Paris de
M\textsuperscript{lle} de Montpensier, le 18 novembre, de ses journées,
de ses séjours, de son accompagnement et de sa suite, du jour qu'elle
arriverait sur la frontière, et des personnes qui seraient chargées de
l'échange des deux princesses\,; en même temps du récit abrégé de tout
ce qui s'était passé à l'égard du duc d'Ossone et de la signature du
contrat de mariage du prince des Asturies. Outre ce duplicata, il y
avait une lettre à part du cardinal Dubois, dont je parlerai après, et
une à part à Maulevrier sur les grandesses d'Espagne données puis
désavouées par l'empereur, avec ordre de me la montrer dès que je serais
arrivé à Madrid. Ce courrier ne pouvait arriver plus à propos, puisque
la cour d'Espagne partait ce jour-là même, et nous fit un extrême
plaisir, par l'amertume que le roi et la reine d'Espagne commençaient à
mêler dans l'impatience qu'ils nous témoignaient des délais de ce départ
toutes les fois qu'ils nous voyaient, et que les raisons les plus
péremptoires et les plus répétées n'avaient pu diminuer.

Nous crûmes, Maulevrier et moi, qu'il n'y avait point de temps à perdre
pour porter cette nouvelle à Leurs Majestés Catholiques, qu'elles
attendaient si impatiemment, et nous nous en allâmes aussitôt au palais.
Je voulais commencer par Grimaldo, qui nous conduirait en cette
occasion, à cause de l'heure trop matinale, et à qui ce devoir était dû.
Maulevrier fut d'avis d'aller droit chez le roi pour flatter son
impatience\,; que Grimaldo n'en serait point blessé à cause de
l'occurrence\,; que, si le roi et la reine n'étaient pas encore
visibles, nous descendrions à la cavachuela en attendant, et que Leurs
Majestés Catholiques n'auraient point à trouver mauvais que nous
eussions différé à terminer leur impatience. Comme je savais à part moi
à quoi m'en tenir avec Grimaldo, et que de plus j'aurais à lui dire que,
contre mon avis de le voir d'abord, j'en avais cru Maulevrier qui devait
connaître le terrain mieux que moi, je me rendis à son avis, et nous
allâmes droit à la porte du salon des Miroirs.

Tout étant à cette heure-là désert dans le palais, nous grattâmes avec
bruit à cette porte pour nous faire entendre\,; un valet intérieur
français ouvrit, et nous dit que Leurs Majestés Catholiques étaient
encore au lit. Nous nous en doutions bien, et nous le priâmes de les
faire avertir sur-le-champ, que nous demandions à avoir l'honneur de
leur parler. Or, il est inouï que, sans charge fort intérieure et fort
rare, qui que ce soit les vît jamais au lit, encore n'y avait-il, par
usage, que le seul Grimaldo qui venait y travailler les matins, et nul
autre, ni grand officier ni ministre, comme je l'expliquerai ci-après.
Le valet intérieur ne fit qu'aller et venir, il nous dit que Leurs
Majestés nous mandaient, qu'encore qu'il fût contre toute règle et usage
qu'elles vissent qui que ce fût au lit, elles trouvaient bon que nous
entrassions.

Nous traversâmes donc le long et grand salon des Miroirs, tournâmes au
bout à gauche dans une grande et belle pièce, puis tout court, à gauche,
dans une très petite pièce en double d'une très petite partie de cette
grande, qui en tirait son jour par la porte et par deux petites fenêtres
percées tout au haut du plancher. Là, était un lit de quatre pieds et
demi tout au plus, de damas cramoisi, avec de petites crépines d'or, à
quatre quenouilles et bas, les rideaux du pied et de toute la ruelle du
roi ouverts. Le roi, presque tout couché sur des oreillers, avec un
petit manteau de lit de satin blanc\,; la reine à son séant, un morceau
d'ouvrage de tapisserie à la main, à la gauche du roi, des pelotons près
d'elle, des papiers épars sur le reste du lit et sur un fauteuil au
chevet, tout près du roi qui était en bonnet de nuit, la Leine aussi et
en manteau de lit, tous deux entre deux draps que rien ne cachait que
ces papiers fort imparfaitement.

Ils nous firent abréger nos révérences, et le roi avec impatience, se
soulevant un peu, demanda ce qu'il y avait. Nous entrâmes tous deux
seuls, le valet intérieur s'était retiré après nous avoir montré la
porte. «\,Bonne nouvelle\,! sire, lui répondis-je.
M\textsuperscript{lle} de Montpensier est partie le 18, le courrier
arrive dans l'instant, et aussitôt nous sommes venus nous présenter pour
l'apprendre à Vos Majestés.\,» La joie se peignit à l'instant sur leurs
visages, et tout aussitôt les questions sur le chemin, les séjours,
l'arrivée à la frontière, l'accompagnement, raisonnements là-dessus,
conversation. De là nous leur dîmes tout ce que nos dépêches nous
apprenaient des honneurs faits au duc d'Ossone et à
M\textsuperscript{lle} de Montpensier depuis la signature de son contrat
de mariage, que nous fîmes valoir, ce qui s'était passé à cette
signature, les réjouissances, le bal, en un mot tout ce qui put le mieux
marquer la joie publique, la part que le roi y prenait, le respect de M.
le duc d'Orléans et sa profonde reconnaissance de l'honneur que sa fille
recevait. On peut juger que le champ fut vaste et bien parcouru de notre
part, et par la curiosité de Leurs Majestés Catholiques, qui se
prenaient souvent la parole l'une à l'autre pour nous faire des
questions et en raisonner, en sorte que cela dura plus d'une heure. Ils
me parurent extrêmement sensibles à tous ces honneurs extraordinaires
que nous leur expliquions (je dis nous, quoique Maulevrier parlât peu,
qui n'en savait ni la force, ni les usages, ni les différences), et à la
joie publique de notre cour et de tout le royaume.

Sur la fin, Maulevrier dit au roi qu'il avait, par ce courrier, une
dépêche sur l'affaire des grands d'Espagne de l'empereur. À ce mot, le
roi d'Espagne s'altéra au point que je lui dis vitement qu'il serait
content de ce que portait la fin de la dépêche. Cela l'apaisa. Alors
Maulevrier tira la dépêche de sa poche, et, à mon extrême étonnement, se
mit à la leur lire d'un bout à l'autre. Elle ne contenait rien qui ne
pût être vu\,; mais qu'un ambassadeur montre ses dépêches au prince
auprès duquel il est ou à son ministre me parut la chose du monde la
plus dangereuse et un sacrilège d'État\,; je sus depuis que Maulevrier
était dans cette habitude. La dépêche portait que l'empereur avait fait
ces grands d'Espagne par le conseil de Rialp. À ce nom le roi me regarda
d'un air piqué et me dit\,: «\,C'est un Catalan.\,» Je répondis en
souriant un peu, et le regardant fixement\,: «\,Sire, il n'y a rien de
plus mauvais que les transfuges, ils sont pires que tous les autres.\,»
À cette réponse la reine se mit à rire en me regardant, et je connus
très bien qu'elle avait bien senti qu'elle portait à plomb sur les
Français de l'affaire de Bretagne et de Cellamare réfugiés en Espagne,
qui était aussi ce que j'avais voulu leur faire entendre. La fin de la
dépêche, qui contenait la déclaration de l'empereur dont j'ai parlé plus
haut d'avance, satisfit en effet beaucoup le roi d'Espagne, qui était
infiniment sensible là-dessus.

Enfin Leurs Majestés Catholiques nous congédièrent, après nous avoir
témoigné que nous leur avions fait grand plaisir de n'avoir pas perdu un
moment à leur apprendre le départ de M\textsuperscript{lle} de
Montpensier, surtout de ne nous être pas arrêtés par l'heure et parce
qu'elles étaient au lit.

Nous descendîmes aussitôt après à la cavachuela du marquis de Grimaldo,
à qui nous dîmes la nouvelle et ce que nous venions de faire\,; je
n'oubliai pas d'ajouter que ç'avait été sur l'avis de Maulevrier. Il
nous parut qu'il le trouva fort bon. Nous l'informâmes de tout ce qui
s'était passé à Paris, comme nous avions fait le roi et la reine, et,
comme à eux, Maulevrier lui lut sa dépêche sur les grands d'Espagne de
l'empereur. Les questions, les raisonnements, la conversation, où ce qui
regardait l'échange et les accompagnements ne fut pas oublié, durèrent
près de deux heures.

Nous vînmes dîner chez moi et retournâmes au palais pour voir partir le
roi et la reine d'Espagne. J'en reçus là encore mille marques de bonté.
Tous deux, surtout la reine insista à deux ou trois reprises à ce que je
{[}ne{]} différasse pas après eux à me rendre à Lerma, sur quoi je les
assurai que je m'y trouverais à leur arrivée et à la descente de leurs
carrosses.

Après leur départ j'allai chez moi ajouter à mes dépêches ce qui venait
de se passer depuis l'arrivée du courrier et de la nouvelle du départ de
M\textsuperscript{lle} de Montpensier, et expédier mon courrier, qui
portait aussi les précédentes dépêches et l'un des deux instruments du
contrat de mariage du roi, signé des mains du roi et de la reine
d'Espagne, de l'infante, des princes ses frères, de moi et de
Maulevrier. Je choisis pour cela un gentilhomme de bon lieu, peu à son
aise, lieutenant dans le régiment du marquis de Saint-Simon, bon et
brave officier, et jeune et dispos, pour lequel je demandai au cardinal
Dubois la commission de capitaine, la croix de Saint-Louis et une
pension. La façon dont on verra que ces trois choses furent accordées
mérite assurément de trouver place ici.

Ce même courrier, qui apporta la nouvelle du départ de
M\textsuperscript{lle} de Montpensier, m'apporta enfin la lettre du roi
pour l'infante, que je lui allai présenter au sortir de la cavachuela de
Grimaldo, avant d'aller dîner, qu'elle reçut de la meilleure grâce du
monde, comme elle allait partir ainsi que le prince des Asturies, à qui
je présentai aussi des lettres. Le roi d'Espagne, ayant appris, par le
récit que nous lui fîmes de ce qui s'était passé à Paris à l'égard du
duc d'Ossone, que la ville de Paris avait été par ordre du roi lui faire
compliment, voulut que je reçusse le même honneur, que la ville de
Madrid me vint rendre dès le lendemain. Venons maintenant à la lettre
particulière du cardinal Dubois à moi, que je n'ai fait qu'annoncer
ci-dessus, et que je reçus par le courrier qui apporta la nouvelle du
départ de M\textsuperscript{lle} de Montpensier.

J'étais si bien informé avant de partir de Paris que le prince de Rohan
était chargé de l'échange des princesses, que, quoique lui et moi
n'eussions jamais été en aucun commerce ensemble que celui des
compliments aux occasions, nous nous étions réciproquement visités, vus
et entretenus sur nos emplois réciproques. M. le duc d'Orléans et le
cardinal Dubois n'avaient pas ignoré ces visites, tous deux même m'en
avaient parlé après qu'elles furent faites, et de nos compliments et
visites réciproques de M\textsuperscript{me} de Ventadour et de moi,
avec satisfaction, laquelle je ne voyais pas plus familièrement que je
viens de dire, que je voyais le prince de Rohan son gendre. Je fus donc
étonné de recevoir la lettre dont je parle du cardinal Dubois, du 18
novembre, qui, après avoir commencé en deux mots par le départ de
M\textsuperscript{lle} de Montpensier, etc., m'apprenait, comme si je
l'avais ignoré, le choix fait du prince de Rohan pour l'échange des
princesses, avec toutes les raisons de ce choix qui sentaient l'embarras
et l'excuse. Il relevait tant qu'il pouvait la grande considération que
méritait la duchesse de Ventadour, qui était le motif de ce choix, et il
ajoutait qu'il convenait si fort qu'elle fût la maîtresse du voyage et
qu'elle eût le commandement sur tout ce qui en était, que le choix du
prince de Rohan avait été nécessaire, qui par sa fonction avait ce
commandement et la disposition de tout le voyage, mais qui pour le
laisser à sa belle-mère n'arriverait à la frontière que pour l'échange
et s'en reviendrait tout court à Paris dès qu'il servit fait, ménagement
qui n'aurait pu se demander à tout autre.

Ce précis était étendu et paraphrasé en homme qui sentait que j'aurais
dû être chargé de l'échange, mais qui, trop occupé de cette pensée,
oubliait l'inutilité de l'excuse et du prétexte, puisque, étant en
Espagne pour la demande et pour la signature du contrat, je n'aurais pu
marcher avec M\textsuperscript{lle} de Montpensier, et devant assister à
la célébration de son mariage je n'aurais pu accompagner l'infante en
France, par conséquent que je n'aurais pu ôter à la duchesse de
Ventadour le commandement du voyage ni en venant ni en retournant. Cette
lettre finissait par d'assez longs propos sur la grandesse que je
désirais et sa volonté de m'y servir efficacement.

Je ne dissimulerai pas que cette lettre me fit un peu rire. Je l'en
remerciai par ma réponse, en lui laissant toutefois très -poliment
apercevoir que j'y avais remarqué quelque embarras sur mon compte, et
cet embarras n'était pas mal fondé. Au demeurant le désir de former une
seconde branche était le seul motif qui m'avait conduit. Je ne pouvais
espérer d'y réussir que par l'ambassade, et jamais par l'échange, qui
n'était que la suite et l'effet de la demande de l'infante et de la
signature de son contrat de mariage avec le roi. Bien est vrai que
j'aurais pu être chargé aussi de l'échange\,; mais ce dernier emploi ne
me conduisait à rien, et il a été toujours d'usage de nommer deux
personnes, l'une pour l'ambassade, l'autre pour recevoir la princesse à
la frontière et la conduire à la cour. Ainsi le choix du prince de Rohan
ne me fit aucune peine, parce que j'avais l'emploi unique par lequel je
pouvais arriver à ce que je m'étais proposé.

Mais quoique je n'en eusse aucune jalousie, je crus devoir prendre à cet
égard les mêmes précautions que ma dignité de duc et pair de France
m'aurait inspirées indépendamment de tout autre caractère, si je m'en
étais trouvé à portée comme j'y étais en effet sur les lieux. Le marquis
de Santa Cruz, ancien grand d'Espagne de Philippe II et de grande
maison, majordome-major de la reine, fut chargé de l'échange des
princesses de la part du roi d'Espagne avec le prince de Rohan\,; l'acte
de l'échange devait être chargé de leurs noms, de leurs titres, de leurs
qualités. Je compris bien que le seigneur breton voudrait y faire le
prince, et qu'il fallait exciter sur cela et punto\footnote{Le point
  d'honneur.} du seigneur espagnol. Quoique celui-ci n'aimât point les
Français, je m'étais mis fort bien avec lui, et je m'étais attaché à y
réussir, parce que c'était l'homme de toute la cour, quoique Espagnol,
qui était le mieux et le plus familièrement avec la reine, dont sa
charge l'approchait le plus continuellement\,; il était de plus ami
intime du duc de Liria, avec qui j'étais intimement aussi et à qui
j'expliquai le fait. Il en sentit toute la conséquence pour la dignité
des grands, et se chargea de la bien faire entendre à Santa Cruz. Santa
Cruz était haut et sentait fort tout ce qu'il était. Je lui en parlai
aussi\,; il comprit qu'il ne fallait pas mollir dans une occasion
pareille, il me le promit bien positivement et il me tint parole très
fermement, comme on le verra quand il sera temps de parler de l'échange.

\hypertarget{chapitre-xv.}{%
\chapter{CHAPITRE XV.}\label{chapitre-xv.}}

1721

~

{\textsc{Arrivée, réception, traitement, audiences, magnificence du duc
d'Ossone.}} {\textsc{- Signature des articles du prince des Asturies et
de M\textsuperscript{lle} de Montpensier chez le chancelier de France.}}
{\textsc{- Signature du contrat de mariage du prince des Asturies et de
M\textsuperscript{lle} de Montpensier.}} {\textsc{- Elle est visitée par
le roi.}} {\textsc{- Fêtes.}} {\textsc{- Départ de
M\textsuperscript{lle} de Montpensier.}} {\textsc{- La ville de Paris
complimente le duc d'Ossone chez lui.}} {\textsc{- Mort du comte de
Roucy.}} {\textsc{- Mort de Surville.}} {\textsc{- Mort de Torcy, des
chevau-légers.}} {\textsc{- Arrivée de La Fare chargé des compliments de
M. le duc d'Orléans sur le mariage de M\textsuperscript{lle} sa fille.}}
{\textsc{- Vaines prétentions de La Fare, que son maître n'avait
point.}} {\textsc{- Conduite que je me suis proposé d'avoir en
Espagne.}} {\textsc{- Tentative du P. Daubenton auprès de moi pour faire
rendre aux jésuites le confessionnal du roi.}} {\textsc{- Droiture et
affection de Grimaldo pour moi.}} {\textsc{- L'empereur fait une
nombreuse promotion de l'ordre de la Toison d'or, dont il met le prince
héréditaire de Lorraine.}} {\textsc{- Omission de plusieurs affaires peu
importantes et des embarras étranges d'argent où la malice du cardinal
Dubois m'attendait et me jeta.}} {\textsc{- Courte description de Lerma
et de Villahalmanzo.}} {\textsc{- Grands mandés avec quelques autres
personnes distinguées pour assister au mariage du prince des Asturies.}}
{\textsc{- Pour quelles personnes ont été faites les érections des
duchés de Pastrane, Lerma et l'Infantade, et comment tombés au duc de
l'Infantade, de la maison de Silva.}} {\textsc{- Caractère et famille du
duc de l'Infantade, et leur conduite à l'égard de Philippe V.}}
{\textsc{- Richesses de ce duc.}} {\textsc{- Sa folie en leur emploi.}}
{\textsc{- Maisons du prince et de la princesse des Asturies.}}
{\textsc{- Je vais par l'Escurial joindre la cour à Lerma.}} {\textsc{-
Pouvoir du nonce.}} {\textsc{- Hiéronimites\,; leur grossièreté et leur
superstition.}} {\textsc{- Appartement où Philippe II est mort.}}
{\textsc{- Pourrissoir.}} {\textsc{- Sépultures royales.}} {\textsc{-
Petite scène entre un moine et moi sur la mort du malheureux don
Carlos.}} {\textsc{- Fanatisme sur Rome.}} {\textsc{- Panthéon.}}
{\textsc{- J'arrive à mon quartier près de Lerma, où je tombe malade
tout aussitôt de la petite vérole.}} {\textsc{- Indication pour se
remettre sous les yeux tout ce qui regarde les personnages, charges,
emplois, grandesses d'Espagne.}} {\textsc{- Précis sur les grandesses.}}

~

Disons maintenant deux mots de ce qui se passa à Paris à l'égard du duc
d'Ossone, de M\textsuperscript{lle} de Montpensier, et de ce qui arriva
d'ailleurs à Paris jusqu'à la fin de cette année.

La veille de mon départ de Paris, M\textsuperscript{lle} de Montpensier
reçut sans cérémonie celles du baptême dans la chapelle du Palais-Royal,
et fut nommée Louise par Madame et par M. le duc de Chartres. L'infante
reçut les mêmes cérémonies, le 9 novembre, par le nonce du pape, et eut
le prince des Asturies son frère pour parrain.

Le duc d'Ossone arriva le 29 octobre à Pars\,; il eut le 31 audience
particulière du roi\,; il fut logé et défrayé lui et toute sa nombreuse
suite à l'hôtel des ambassadeurs extraordinaires tout le temps qu'il
demeura à Paris, ce qui ne se fait jamais pour les ambassadeurs
extraordinaires d'aucun prince de l'Europe, et le fut magnifiquement. Il
y traita très souvent les principaux seigneurs et dames, dont les plus
distingués seigneurs lui donnèrent des repas qui pouvaient passer pour
des fêtes. Il donna aussi de belles illuminations et des feux d'artifice
dont la beauté, la nouveauté et la durée effaça de bien loin tous les
nôtres. Il traita et visita plusieurs fois M\textsuperscript{me} de
Saint-Simon, comme je rendis aussi de fréquents devoirs aux duchesses
d'Ossone sa femme et sa belle-soeur. Il visita à l'ordinaire les princes
et les princesses du sang et fut visité de ces princes, qu'après quelque
petite difficulté il traita d'Altesses, sur l'ancien exemple du marquis
de Los Balbazès, qui vint ambassadeur d'Espagne à Paris aussitôt après
le mariage du feu roi.

Le même jour 31, M\textsuperscript{lle} de Montpensier reçut au
Val-de-Grâce la confirmation que lui donna le cardinal de Noailles et
fit sa première communion. Le 13, le duc d'Ossone fut conduit à
l'audience publique du roi par le prince d'Elboeuf avec les honneurs et
les cérémonies accoutumées. Il y fit les compliments sur le futur
mariage de l'infante avec le roi, la demande de M\textsuperscript{lle}
de Montpensier pour le prince des Asturies, le remercîment de ce qu'elle
lui fut sur l'heure accordée\,; et l'après-dînée il fut avec son même
cortège au Palais-Royal. Plus délicat que moi il ne voulut pas être
accompagné de don Patricio Laullez, et prétendit qu'il ne devait entrer
en fonction d'ambassadeur qu'après qu'il aurait fait seul cette demande
solennelle.

Le 15, don Patricio Laullez commença d'entrer en fonction. Le duc
d'Ossone et lui, sans conducteurs, allèrent chez le chancelier où ils
trouvèrent le maréchal de Villeroy et La Houssaye, contrôleur général
des finances, nommés commissaires du roi pour signer les articles avec
les deux ambassadeurs, auxquels les trois commissaires du roi donnèrent
la droite, et ils signèrent les articles en la même façon que nous à
Madrid ceux du roi et de l'infante.

L'après-dînée du même jour, le duc d'Ossone, conduit par le prince
d'Elboeuf et le chevalier de Sainctot, introducteur des ambassadeurs,
dans un carrosse du roi, et don Patricio Laullez, conduit par le prince
Charles de Lorraine, grand écuyer de France, et par Rémond, introducteur
aussi des ambassadeurs, dans un autre pareil carrosse du roi, allèrent
et furent reçus aux Tuileries avec tous les honneurs accoutumés, ayant
de nombreux cortèges, et des carrosses très -magnifiques ainsi que leurs
livrées et tout ce qui les accompagnait. Ils trouvèrent le roi dans un
grand cabinet, debout sous un dais, ayant un fauteuil derrière lui et
découvert, une table et une écritoire devant lui, sur une estrade
couverte d'un tapis qui débordait fort l'estrade de tous côtés\,; ceux
des grands officiers qui devaient être derrière le roi en leurs places,
Madame et M. le duc d'Orléans à droite et à gauche aux deux bouts de la
table et la joignant, le cardinal Dubois un peu en arrière de M. le duc
d'Orléans vers le coin de la table hors de l'estrade, les princes et
princesses du sang en cercle vis-à-vis du roi et de la table sur le
tapis hors de l'estrade, derrière {[}eux{]} le chancelier et les
secrétaires d'État, et sur les ailes, derrière Madame et M. le duc
d'Orléans, quelques seigneurs principaux. Les ambassadeurs
s'approchèrent du roi à qui le duc d'Ossone fit un court compliment, et
se retirèrent aux places où ils furent conduits, au-dessous des princes
et princesses du sang, mais sur le tapis et sur la même ligne. Le
contrat lu par le cardinal Dubois fut signé par le roi et par tout ce
qui était là présent du sang, puis, sur une autre colonne, par les deux
ambassadeurs, sur la même table\,; en quoi-ils furent mieux traités que
nous, comme aussi nous fûmes mieux traités qu'eux pour la signature des
articles qui se fit, comme on l'a vu, chez le chancelier à Paris, et à
Madrid dans un cabinet de l'appartement du roi. Après la signature, le
duc d'Albe\footnote{On a reproduit le nom donné par le manuscrit de
  Saint-Simon\,; mais il faudrait lire le duc d'Ossone au lieu du duc
  d'Albe.} se rapprocha encore du roi avec Laullez, fit un court
compliment, et {[}ils{]} se retirèrent reconduits chez eux en la manière
accoutumée, d'où ils allèrent au Palais-Royal.

En peu après, le roi alla voir M\textsuperscript{lle} de Montpensier au
Palais-Royal, qu'il trouva auprès de Madame, puis dans la grande loge de
M. le duc d'Orléans, avec le tapis et les gardes du corps au bas de la
loge sur le théâtre, et répandus de tous côtés, où il vit pour la
première fois l'Opéra, qui fut celui de Phaéton, ayant Madame à sa
droite et M. le duc d'Orléans à sa gauche, et derrière lui ceux de ses
grands officiers qui y devaient être. Après l'opéra, où on avait eu soin
de bien placer les ambassadeurs et leur principale suite, et où se
trouva tout ce qu'il y avait de plus brillant à la cour, le roi retourna
souper aux Tuileries. Il revint après au Palais-Royal, où il trouva un
superbe bal paré qui l'attendait. Il l'ouvrit avec
M\textsuperscript{lle} de Montpensier, et y dansa ensuite plusieurs
fois. Au bout d'une heure et demie il s'en alla et il traversa huit
salles remplies de masques magnifiquement parés. Après son départ M. le
duc de Chartres emmena les deux ambassadeurs d'Espagne dans la galerie
de son appartement, avec les principaux de leur suite et beaucoup de
seigneurs distingués de la cour, où ils trouvèrent une grande table
splendidement servie. Tous les masques furent cependant admis dans le
bal, où on dansa dans toutes les pièces jusqu'à six heures du matin. On
y servit force rafraîchissements, et il y en avait de toutes sortes de
dressés dans les pièces voisines.

Enfin, le 18 au matin, le maréchal de Villeroy vint de la part du roi
complimenter M\textsuperscript{lle} de Montpensier, puis la ville de
Paris, après quoi elle monta dans un carrosse du roi avec M. le duc
d'Orléans sur le derrière, M. le duc de Chartres et la duchesse de
Ventadour sur le devant, et aux portières la princesse de Soubise et la
comtesse de Cheverny, gouvernante de la princesse. Elle était
accompagnée d'un détachement des gardes du corps jusqu'à la frontière,
et de force carrosses pour sa suite. M. le duc d'Orléans et M. le duc de
Chartres la conduisirent deux lieues, puis s'en revinrent à Paris. Peu
de jours après le duc d'Ossone fut, par ordre du roi, complimenté chez
lui par Châteauneuf, prévôt des marchands, à la tête des échevins et des
conseillers de ville, en habits de cérémonie, qui lui présentèrent les
présents de vin et de confitures de la ville de Paris. Ce fut encore un
honneur qui ne se rend point aux ambassadeurs extraordinaires d'aucun
prince. Le duc d'Ossone le reçut étant accompagné de don Patricia
Laullez, mais à qui la parole ne fut point du tout adressée.

Le comte de Roucy était mort à Paris, quinze jours auparavant, à
soixante-trois ans, lieutenant général et gouverneur de Bapaume. On a
vu, t. XIII, p.~272, le procédé étrange qu'il eut avec moi, qui nous
brouilla avec le plus grand éclat après une longue suite de liaison
étroite et de services de ma part. Plus religieux, quoique moins dévot
que sa femme, qui l'affichait, et lui le contraire, il envoya prier
M\textsuperscript{me} de Saint-Simon de vouloir bien l'aller voir. Elle
y fut, et en reçut toutes les marques du plus sensible regret de sa
conduite avec moi, et mourut deux jours après. J'ai eu si souvent
occasion de parler de lui que je n'y ajouterai rien, non plus qu'à
l'égard de Surville, qui mourut quinze jours après, duquel il a été
amplement parlé à l'occasion des disgrâces qu'il s'était attirées dans
le brillant d'un chemin de fortune très mal mérité.

Torcy, dont c'était le nom, et point parent des Colbert, mourut en même
temps à soixante-treize ans. Il avait été sous-lieutenant des
chevau-légers de la garde avec réputation de probité et de valeur, du
reste un fort pauvre homme. Il était riche et avait épousé en premières
noces la fille du duc de Vitry, et en secondes la fille de Gamaches. Il
ne laissa point d'enfants. Il était maréchal de camp.

La Fare arriva à Madrid le lendemain du départ de la cour et vint
descendre chez moi. Dès ce premier entretien il m'exposa des prétentions
sauvages c'était d'être reçu comme le sont les envoyés des souverains\,;
d'être conduit à l'audience dans la même forme, et d'être reçu et traité
comme eux. J'essayai de lui faire entendre que ceux que feu Monsieur
avait envoyés faire ses compliments dans les cours étrangères, à
Londres, même à Heidelberg, à l'occasion de ses mariages, à Madrid, à
l'occasion du mariage de la reine sa fille, et en d'autres occasions en
ces mêmes cours et en d'autres, n'avaient jamais prétendu ces
traitements, quoique venant de la part d'un fils de France, et que lui
pouvait encore moins prétendre venant de la part d'un petit-fils de
France. La Fare me répondit que ce petit-fils de France était régent\,;
que cette qualité changeait tout\,; que de plus la conjoncture était
heureuse et qu'il fallait en profiter.

Je répliquai que la qualité de régent ne changeait rien au rang et à
l'état personnel de petit-fils de France à l'égard de M. le duc
d'Orléans, qu'il le voyait tous les jours en France et en était témoin
qu'il en était de même dans les pays étrangers, de pas un desquels il
n'avait prétendu quoi que ce pût être de nouveau à titre de régent\,;
qu'à la vérité la conjoncture était heureuse, mais qu'il ne la fallait
pas forcer et s'attirer un refus qui changerait en dégoût et ensuite en
éloignement la réunion qui faisait la joie publique des deux nations et
la gloire personnelle de M. le duc d'Orléans, et sûrement la jalousie
des autres princes qui sauraient bien nourrir, se réjouir et profiter
d'un mécontentement de cérémonial\,; qu'il ne pouvait pas douter
qu'étant depuis toute ma vie ce que j'étais à M. le duc d'Orléans, et
lui devant l'ambassade où j'étais, je ne fusse ravi d'en profiter pour
lui procurer toute sorte de grandeur\,; mais que dans ce même emploi, où
je me trouvais par son choix, les désirs devaient, quant aux démarches,
être bornés par les règles, et que ce serait fort préjudicier à cette
même grandeur que de la commettre par des prétentions qui n'avaient pas
été conçues jusqu'à ce moment en aucun lieu, et s'exposer à un refus
qui, outre son extrême désagrément, changerait aisément en dégoût, en
froideurs, en éloignement le fruit d'une réunion qui se pouvait dire le
chef-d'oeuvre de l'adresse et de la capacité de la politique après les
choses passées\,; et le sceau le plus solide de la grandeur réelle de M.
le duc d'Orléans en tout genre, par le mariage de sa fille, avec le
prince des Asturies. J'ajoutai que M. le duc d'Orléans ni le cardinal
Dubois ne m'avaient jamais dit un mot de cette prétention, ni mis sur
son envoi quoi que ce fût dans mes instructions, et que c'était à lui à
me dire s'il en avait là-dessus, dont on ne m'avait rien dit ni écrit.
La Fare devint embarrassé\,; il n'en avait point, n'osait me le dire, ne
voulait pas aussi me tromper, et parce qu'il n'était pas capable de se
porter à ce mensonge, et parce qu'il sentait bien que je ne serais pas
longtemps, s'il m'eût avancé faux, d'être éclairci de la vérité.

Mais il ne se rendit point, et me pressa de telle sorte que j'entrai en
capitulation. Je fis une lettre pour Grimaldo, par laquelle, lui donnant
avis de l'arrivée de La Fare, je lui exposais la convenance de le
recevoir et de le traiter avec des distinctions particulières, mais sans
rien spécifier ni demander distinctement ni directement, me contentant
de m'étendre sur la faveur de la conjoncture, sur celle de La Fare
auprès de M. le duc d'Orléans, qui serait flatté pour soi et pour lui
des bontés et des distinctions que Sa Majesté Catholique voudrait bien
lui accorder. Je montrai ma lettre à La Fare\,; je l'envoyai à Grimaldo
et une copie au cardinal Dubois.

La Fare ne fut pas content d'une lettre qui n'exprimait point ses
prétentions, moins encore de l'envoi de sa copie au cardinal Dubois. Il
comptait d'emporter d'emblée -ce qu'il avait imaginé, et de s'en faire
grand honneur en Espagne et un grand mérite auprès de M. le duc
d'Orléans. Toutefois il aima mieux cela que rien. Grimaldo qui suivait
la cour avait eu avis de son passage par les chemins, et La Fare en
reçut ordre dès le lendemain d'aller incontinent joindre la cour. Il
partit donc peu satisfait de moi, et par ce qu'on va voir qui m'arriva,
nous fûmes près de deux mois sans nous rejoindre. Il reçut de la cour
d'Espagne tout l'accueil et les distinctions possibles, mais aucunes de
celles qu'il prétendait et qui fussent de caractère. Je fus approuvé
dans ce que j'avais fait là-dessus\,; et M. le duc d'Orléans était bien
éloigné d'avoir formé aucune prétention nouvelle.

Cela même me confirma dans la pensée que j'avais toujours eue que les
deux lettres de M. le duc d'Orléans, dont je fus chargé pour le prince
des Asturies, l'une dans le style ordinaire, l'autre avec l'innovation
du mot de frère, était une friponnerie du cardinal Dubois, qui espérait
bien que je ne ferais point passer cette dernière, et de s'en avantager
contre moi auprès de M. le duc d'Orléans, d'autant que ce prince, tout
en me marquant son désir là-dessus qui lui était enjoint, ne me
recommanda rien plus que de ne rien hasarder, de ne point insister à la
moindre difficulté que j'y rencontrerais, de la retirer et de présenter
l'autre, au lieu que le cardinal ne me recommanda rien davantage que de
la faire passer, jusqu'à me piquer d'honneur sur mon attachement pour M.
le duc d'Orléans, sur ce premier moyen de lui témoigner ma
reconnaissance dans cette ambassade, et de marquer mon adresse et mon
esprit par un si agréable début. On a vu que je n'eus besoin ni de l'un
ni de l'autre, et que cette lettre passa doux comme lait, sans même
qu'il en fût dit un seul mot. Si on l'avait refusée, ce petit dégoût se
serait passé dans l'intérieur et le secret, et c'est sûrement ce qui le
fit entreprendre au cardinal Dubois, au lieu que, s'il eût conçu les
chimères de La Fare, leur refus aurait été public, et c'est ce qui
empêcha le cardinal Dubois de les former et de m'en charger, quelque
joie qu'il eût eue de me les voir péter dans la main. Ce petit fait
méritait d'être expliqué, d'autant que dans la suite il se verra encore
une prétention fort singulière de La Fare, qui, comme celle-ci, périt
pour ainsi dire avant que de naître.

Quelque occupé que j'eusse été depuis mon arrivée, en affaires, en cour,
en cérémonial, en fonctions, en fêtes, en festins, je n'avais pas laissé
de faire plus de quatre-vingts visites avant le départ de la cour, après
lequel j'en fis encore et en reçus beaucoup jusqu'au mien départ quatre
jours après la cour\,: je m'étais particulièrement proposé de plaire,
non seulement à Leurs Majestés Catholiques, mais à leur cour, mais en
général aux Espagnols et jusqu'aux peuples, et j'ose dire que j'eus le
bonheur d'y réussir par l'application continuelle que j'eus à ne rien
oublier pour ce dessein, en évitant en même temps jusqu'à la plus légère
affectation, mais louant avec soin tout ce qui pouvait l'être, toutefois
en mesure des différents degrés, m'accommodant à leurs manières avec un
air d'aisance, n'en blâmant aucune, admirant avec satisfaction les
belles choses en tout genre qui s'y voient, évitant soigneusement toute
préférence et toute légèreté française, ajustant avec une attention
exacte, mais qui ne paraissait pas, la dignité du caractère avec tous
les divers genres de politesse que je pouvais rendre au rang, à la
considération, à l'âge, au mérite, à la réputation, aux emplois présents
et passés, à la naissance de toutes les personnes que je voyais,
politesse à tous, mais politesse mesurée à ces différences sans être
empesée ni embarrassée, qui, pour ainsi dire, distribuée sur cette
mesure avec connaissance et discernement, oblige infiniment, tandis
qu'une politesse générale et sans choix dégoûte toutes les personnes
qu'elle croit gagner et qu'elle ne se concilie point, parce qu'elle les
rend égales.

Je me fis, dès le jour que j'arrivai, une affaire principale d'acquérir,
à travers toutes mes occupations, cette connaissance de ces différentes
choses dans les personnes principales que j'eus à fréquenter, puis des
unes aux autres de parvenir à celle de tout ce qui se pouvait présenter
sous mes yeux. Ce fui en cela que Sartine, les ducs de Liria et de
Veragua, me furent tout d'abord d'une utilité extrême. Par eux, je fis
d'autres connaissances, je m'informai à plusieurs, je combinai et me mis
ainsi avec un peu de temps en état de discerner par moi-même sur les
lumières qu'on, m'avait données. Quand je devins un peu plus libre avec
tous ces seigneurs, ce qui arriva bientôt par les prévenances, les
politesses, et leurs retours que j'en reçus, je leur semai des
cajoleries que me fournissaient les connaissances de leurs maisons et de
ce qui s'y était passé de grand et d'illustre, de leurs emplois, de
leurs parentés, la valeur et la fidélité de la nation espagnole, enfin
tout ce qui les pouvait flatter en général et en particulier. Plaçant
les choses avec discernement et sobriété pour mieux faire goûter ce qui
ne se disait qu'avec une sorte de rareté, mais coulant toujours à propos
des choses dont on s'entretenait et les amenant tout naturellement. Rien
ne leur plut davantage que de me trouver instruit de leurs maisons, de
ce qu'elles ont produit d'illustre, de leurs alliances, de leurs
dignités, de leur rang, de leurs emplois, de leurs fonctions, de leurs
services. Ces connaissances les persuadaient de l'estime que j'en
faisais\,; cela les charmait, ils s'écriaient quelquefois que j'étais
plus Espagnol qu'eux, et qu'ils n'avaient jamais vu de Français qui me
ressemblât. Jusqu'à leur manger, je m'en accommodais\,; ils en étaient
surpris, et je voyais qu'ils m'en tenaient compte. Surtout ils étaient
charmés de la juste préférence que je donnais à leurs fêtes sur les
nôtres, parce qu'ils voyaient que je leur en disais les raisons et que
je le pensais véritablement. Tant que je fus en Espagne, je ne me lassai
pas un moment de cette conduite qui m'était agréable par le fruit
continuel et toujours nouveau que j'en retirais, et qui m'attira leur
amitié, leur estime et leur confiance, comme on en verra quelques traits
que je choisirai sur beaucoup d'autres, par lesquels je me trouvai
surabondamment récompensé de mon application à les capter.

Ce grand nombre de visites, que je trouvai moyen de rendre à travers
tant de sortes de fonctions, fut pour moi un début très heureux. L'usage
en Espagne est que tout ce qu'il y a de gens considérables visitent les
principaux ambassadeurs qui arrivent. J'appelle ainsi les nonces, les
impériaux, ceux de France et d'Angleterre. Ils sont flattés qu'ils les
leur rendent promptement\,; dans ce grand nombre, on choisit un petit
nombre des plus distingués chez qui on va à heure de les trouver\,; tout
le reste on prend le temps de leur méridienne. Ils ne le trouvent point
du tout mauvais, et de la sorte on en expédie un grand nombre\,; moi
surtout, qui pour ne manquer à personne, me mis sur le pied d'aller par
les rues au trot, au lieu d'aller au pas comme c'est l'usage\,: mais ils
m'en surent gré par la raison qui me le fit faire, et que je leur dis
franchement\,: mais quand ce n'était pas pour expédier ainsi des
visites, j'allais au pas suivant la coutume.

On peut juger que, parmi tant de visites, je n'oubliai pas le P.
Daubenton. Cela m'était singulièrement recommandé par le cardinal
Dubois, et je me recommandais bien à moi-même à cause de ce que je
pouvais tirer de lui auprès du roi d'Espagne, tant pour le peu
d'affaires que je pourrais avoir à traiter, que pour la personnelle qui
m'avait fait désirer l'ambassade. Cette dernière raison m'engagea à le
voir plusieurs fois dans ces premiers dix ou douze jours que je fus à
Madrid, parce qu'il eût été indécent de débuter promptement par là. Je
le trouvai très ouvert là-dessus et prodigue de désirs de m'y servir,
efficacement, de plaire à M. le duc d'Orléans et d'étreindre de tout son
pouvoir l'union par lui si désirée des deux couronnes et de ce prince
avec le roi d'Espagne.

Le bon père essaya aussitôt de profiter de l'occasion. Il se mit à me
vanter son attachement pour moi sans me connaître, par la bonté qu'il
savait que j'avais toujours eue pour les jésuites, me parla des
confesseurs que j'y avais eus si longtemps, de l'estime et de la
confiance du P. Tellier pour moi\,; car il était bien informé de tout et
savait en faire usage, me dit le dessein qu'avait le roi d'Espagne de
m'employer, comme il fit deux jours après, pour que l'infante fût mise
entre les mains d'un jésuite, sur quoi il me demanda ce que j'en
pensais. Sur ma réponse, qui fut telle qu'il la souhaitait il se mit à
me faire véritablement les yeux doux, à tenir des propos généraux sur sa
compagnie et son dévouement pour le roi, puis à balbutier, à commencer à
s'interrompre, à se reprendre, enfin il accoucha sans aucun secours de
ma part, qui vis d'abord où il en voulait venir, et il me dit enfin que
le roi d'Espagne mourait d'envie de me prier de demander au roi son
neveu de sa part, de prendre un jésuite pour son confesseur et d'en
prier en son nom M. le duc d'Orléans, et de lui faire ce plaisir en même
temps que j'écrirais sur celui de l'infante, parce que l'âge et les
infirmités de l'abbé Fleury pouvaient à tous moments l'engager à cesser
de confesser le roi.

Cette proposition se fit avec tout l'art et l'insinuation possible à
l'issue de toutes les offres de ses services pour faciliter la grandesse
que je souhaitais, et tout de suite me demanda ce que j'en pensais, mais
avec un air de confiance. Je le payai de la même monnaie qu'il m'avait
donnée sur mon amitié pour les jésuites, puis je lui dis que le
confessionnal du roi n'était pas la même chose que celui de l'infante\,;
qu'il était très naturel à la tendresse du roi d'Espagne pour sa fille
et à sa confiance aux jésuites de demander qu'elle fût instruite à son
âge par un jésuite, et que, lorsqu'elle serait en âge de se confesser,
ce fût à celui-là ou à un autre de la même compagnie\,; que cela n'avait
point d'inconvénient, et que je ne doutais pas du succès en cela du
désir du roi d'Espagne, par celui que je connaissais en M. le duc
d'Orléans de lui complaire en toutes les choses possibles\,; mais que le
roi d'Espagne allât jusqu'à se mêler de l'intérieur du roi son neveu, je
ne croyais pas que, malgré les circonstances, cela fût mieux reçu en
France qu'il le serait en Espagne de changer le confesseur du roi
d'Espagne ou quelqu'un de ses ministres à la prière de la France\,; que
je suppliais donc instamment Sa Révérence de faire en sorte que le roi
d'Espagne se contentât de me faire l'honneur de me charger de demander
de sa part un jésuite pour l'infante, sans toucher l'autre corde si
délicate dont il fallait laisser la disposition au temps, au roi son
neveu et à ceux qui dans sa cour et le gouvernement de ses affaires se
trouveraient avoir sa confiance, lorsque l'abbé Fleury cesserait d'être
son confesseur.

Quelque déplaisante que fût cette réponse, malgré tout le moins mauvais
assaisonnement que j'y pus mettre, le bon père n'insista pas, il parut
même trouver que ce que je lui dis avait sa raison. La sérénité, la
suavité de son visage ne s'en obscurcit point\,; je le promenai sur les
espérances des futurs contingents, que je ne croyais pas si proches et
sur les convenances que le confessionnal du roi leur fût rendu. Il
revint après à mon affaire personnelle, redoubla de protestations, et
nous nous séparâmes le mieux du monde. Je n'oubliai pas de rendre un
compte exact de cette conversation, de laquelle je fus fort approuvé.

J'avais déjà fait parler à Grimaldo par Sartine, et je lui avais parlé
moi-même\,; ce ministre était vrai et droit\,; j'eus tout lieu de
compter sur lui, et on verra bientôt que je ne me trompai pas.

L'empereur, apparemment fâché de la protestation que la France et
l'Angleterre avait enfin arrachée de lui sur ces grands d'Espagne qu'il
avait faits et qu'il s'était mis ainsi hors d'état d'en plus faire, s'en
voulut dépiquer par une nombreuse promotion de l'ordre de la Toison d'or
comme souverain des Pays-Bas, où cet ordre avait été institué. Le
cardinal Dubois voulait que le roi d'Espagne n'en fît que rire en
attendant que cette prétention fût réglée au congrès de Cambrai, à
l'avantage de Sa Majesté Catholique, mais en même temps il trouvait
mauvais que le fils aîné du duc de Lorraine fût de cette promotion, et
me chargea de faire auprès du roi d'Espagne qu'il lui en marquât son
ressentiment en refusant longtemps de consentir à l'accession du duc de
Lorraine à la paix, à laquelle il désirait passionnément d'être reçu.

J'omets à dessein plusieurs affaires peu embarrassées ou peu
importantes, dont le cardinal Dubois m'écrivit, d'autant que la maladie
où je tombai incontinent me mit hors de tout commerce jusqu'au jour du
mariage du prince des Asturies.

J'omets pareillement les extrémités d'embarras où le cardinal Dubois
m'attendait, et qu'il m'avait si hautement préparées en décuplant
forcément ma dépense. On a vu que je n'avais point voulu
d'appointements, mais qu'il m'avait été promis qu'on ne me laisserait
point manquer, et qu'on fournirait exactement à la dépense qu'on
exigeait de moi\,; mais rien moins. Dès ces commencements, le cardinal
Dubois sut y mettre bon ordre, mais toujours avec ses protestations
accoutumées\,; il se vengeait de l'ambassade emportée à son insu et
malgré lui en me ruinant\,; à la fin il en vint à bout\,; mais, au moins
à mon honneur et à celui de la France, il n'eut pas le plaisir de me
décrier en Espagne, d'où je partis à la fin de mon ambassade sans y
devoir un sou à qui que ce pût être, et sans avoir diminué rien de
l'état que j'avais commencé à y tenir, sinon qu'en allant à Lerma, je
renvoyai en France presque tous les officiers des troupes du roi que ce
bon prêtre m'avait forcé, comme on l'a vu, de mener en Espagne.

La cour d'Espagne, qui marchait avec la lenteur des tortues, devait
arriver, et arriva en effet à Lerma le 11 décembre. C'est un beau bourg
situé en amphithéâtre sur la petite rivière d'Arlanzon, qui forme une
petite vallée fort agréable à six lieues à côté de Burgos. Le château
bâti par le duc de Lerme, premier ministre de Philippe III, et mort
cardinal en 1625, est magnifique par toute sa structure, son
architecture, par son étendue, la beauté et la suite de ses vastes
appartements, la grandeur des pièces, le fer à cheval de son escalier.
Il tient au bourg par une belle cour fort ornée, et par une magnifique
avant-cour, mais fort en pente, qui le joint. Quoiqu'il soit bien plus
élevé que le haut de l'amphithéâtre du bourg, le derrière de ce château
l'est encore davantage, tellement que le premier étage est de plain-pied
à un vaste terrain qui, dans un pays où on connaîtrait le prix des
jardins, en ferait un très beau, très étendu, en aussi jolie vue que ce
paysage en peut donner sur la campagne et sur le vallon, avec un bois
tout joignant le château au même plain-pied, dans lesquels on entrerait
par les fenêtres ouvertes en portes. Ce bois est vaste, uni, mais clair,
rabougri, presque tout de chênes verts, comme ils sont tous dans les
Castilles. Il est du côté de la campagne, et le jardin serait en
terrasse naturelle, fort élevée sur le vallon et sur la campagne au
delà. Le peu de logement que Lerma pouvait fournir à la cour ne permit
d'y en marquer que pour le service et les charges nécessaires. On prit
les villages des environs pour le reste de la cour, pour les grands et
pour les ambassadeurs.

J'eus le choix de plusieurs, et je choisis celui de Villahalmanzo, sur
le récit qu'on m'en fit, à une petite demi-lieue de Lerma, et, tout
vis-à-vis et à vue, la petite vallée entredeux, qu'on passait sur une
chaussée et la petite rivière sur un pont de pierre. On y accommoda la
maison du curé, petite, aérée, jolie, pour moi seul, avec des cheminées
qu'on fit exprès, et toutes les autres maisons du village pour ceux qui
étaient avec moi et pour toute ma suite. Ce village assez étendu, bien
bâti, bien situé, sans voisinage, était très agréable, et il n'y avait
que nous, le curé et les habitants. Il n'y eut pas dans tout notre
séjour la plus légère difficulté avec eux\,; leurs maisons gagnèrent
beaucoup aux accommodements qu'on y fit, et ils furent si contents de
nous qu'ils s'étaient tous apprivoisés avec nos domestiques. On ne leur
fit pas le moindre tort en rien\,; ils eurent quelques présents en
partant, en sorte qu'ils s'étaient tous pris d'affection pour nous, et
qu'ils nous regrettèrent, quelques-uns même avec larmes. Ce voyage fut
pour moi une transplantation très ruineuse de mes tables et de toute ma
maison.

Le roi d'Espagne avait nommé la maison du prince et de la future
princesse des Asturies, et cette dernière pour servir l'infante jusqu'à
l'échange, et en amener et servir au retour la future princesse des
Asturies. Le roi, en partant de Madrid, avait fait dire à tous les
grands et à quelques autres gens distingués, qu'il désirait ne voir à
Lerma que ceux qui l'y accompagneraient jusqu'à l'échange fait, mais
qu'alors il serait bien aise que tous les grands, et ce peu d'autres
personnes distinguées, s'acheminassent à Lerma, où on leur ferait
trouver des logements, ou aux environs, pour assister au mariage du
prince des Asturies, et cela fut exécuté ainsi. Quant aux dames, il n'y
eut que celles du service.

Il faut ajouter, pour tout éclaircir, que Burgos, qui est sur le chemin
de Paris à Madrid, n'est guère plus éloigné de cette dernière ville que
Poitiers l'est de Paris, et que Lerma est à la même hauteur que Burgos,
ainsi à la même distance de Madrid. Lerma fut préféré à Burgos qui avait
été choisi d'abord à cause de la commodité des chasses. Ce comté fut
érigé par les rois catholiques, c'est-à-dire Ferdinand et Isabelle, pour
don Bernard de Sandoval y Roxas, second marquis de Denia, puis en duché
par Philippe III, en 1599, pour don Fr.~Gomez de Sandoval y Roxas,
cinquième marquis de Denia, son premier ministre, puis cardinal après la
mort de sa femme, fille du quatrième duc de Medina-Coeli. Don Diego
Gomez de Sandoval, cinquième duc de Lerma, mourut en 1668, sans enfants,
et le dernier mâle de la postérité du cardinal, duc de Lerma. Ce dernier
mâle avait deux soeurs, de l'aînée desquelles Lerma est tombé aux ducs
de l'Infantao que les François prononcent l'Infantade. Leur nom est
Silva.

Cette maison est très certainement reconnue descendre masculinement
jusqu'à aujourd'hui des anciens rois de Léon, par l'infant Aznar, fils
puîné du roi Fruela\footnote{On écrit ordinairement Froila. Froila II
  fut roi de Léon de 923 à 924.}. Don Ruy Gomez de Silva, si connu sous
le nom de prince d'Eboli, qu'il avait eu de sa femme Anne Mendoza y La
Cerda, maîtresse de Philippe II, acheta en 1572 Pastrane de don Gaston
Mendoza y La Cerda, que Philippe Il érigea pour lui en duché, et il
préféra d'en porter le nom à celui de duc d'Estremera, que le même roi
avait érigé pour lui depuis peu. Cette maison de Silva, de si haute
origine, s'est partagée en beaucoup de branches en Espagne, et jusqu'en
Portugal. Ce prince d'Eboli, premier duc de Pastrane, était de la
dernière de toutes ces branches connue sous le nom de Chamusca, dont il
fut le quatrième seigneur. Il eut plusieurs enfants, dont, outre les
ducs de Pastrane, sortirent aussi les ducs d'Hijar et trois autres
branches. Don Roderic de Silva d'aîné en aîné mâle de ce prince d'Eboli,
premier duc de Pastrane et duc de Pastrane aussi, épousa la sueur aînée
du susdit Diego Gomez de Sandoval, cinquième duc de Lerme, dernier mâle
de la postérité du cardinal duc de Lerme, et par elle devint duc de
Lerme et de l'Infantade en 1668, dont le fils Marie-Grégoire de Silva,
duc de l'Infantade, de Lerma, etc., mort en 1693, fut père du duc de
l'Infantade et de Lerma, vivant lorsque j'étais en Espagne, et longues
années depuis.

À l'égard de l'Infantade, c'est un État, comme ils parlent en Espagne,
composé de trois villes et de plusieurs bourgs qui en dépendent, situé
en Castille, qui, pour avoir été longtemps possédé par plusieurs
infants, fils de rois, fut insensiblement nommé Infantao\,; de ces
princes cet État passa dans différentes maisons par héritage, par
acquisition, par don des rois, qui le retirèrent plus d'une fois. Ce fut
de cette dernière sorte qu'il tomba en 1470 entre les mains d'Henri IV,
roi de Castille, qui en fit don à don Hurtado Mendoza, second marquis de
Santillana, en faveur duquel il fut érigé en duché en 1475 par les rois
catholiques, c'est-à-dire par Ferdinand et Isabelle.

Enfin Catherine Mendoza y Sandoval hérita de ses deux frères, l'un duc
de l'Infantade, l'autre duc de Lerma, et comme on l'a vu ci-dessus,
épousa don Roderic de Silva, duc de Pastrane. De ce mariage vint le père
du duc de l'Infantade, de Lerma et de Pastrana, etc., vivant lorsque
j'étais en Espagne, et connu comme son père sous le seul nom, de duc de
l'Infantade.

Il est né en 1672\,; il est frère du comte de Galve, de la comtesse de
Lemos, dont le mari est Portugal y Castro, et de la comtesse de Niebla,
dont le mari est Perez de Gusman.

Cette branche de Silva Infantade était fort autrichienne, et vit passer
la couronne d'Espagne dans, la maison de France avec tant de chagrin que
le comte de Galve se jeta dans le parti de l'archiduc, puis dans ses
troupes dès qu'elles parurent en Espagne. Le comte et la comtesse de
Lemos, entraînés, dans les mêmes intérêts, furent pris par un parti des
troupes du roi d'Espagne, comme ils allaient joindre celles de
l'archiduc, et le duc de l'Infantade, qui n'osa en faire autant, donna
jusqu'à la fin de la guerre toutes les marques qu'il put de son
attachement au parti de l'archiduc. On s'assura longtemps du comte et de
la comtesse de Lemos, qui donnèrent depuis toutes sortes de marques de
repentir. Le comte n'avoir que sa grande naissance, sans aucun talent ni
suite qui pût le faire craindre, et passait sa vie à fumer, chose fort
extraordinaire en Espagne, où on ne prend du tabac que par le nez. Il
n'en était pas de même de la comtesse, pleine d'esprit et de grâces, et
fort capable de nuire ou de servir. Mais cette ouverture d'esprit lui
fit voir de bonne heure qu'il ne fallait pas attendre, mais tâcher de se
raccommoder à temps, et elle y réussit, en sorte qu'elle regagna de la
considération, et s'est toujours depuis très bien conduite à l'égard de
la cour d'Espagne. Le comte de Calve ne put se détacher des
Autrichiens\,: il les servit jusqu'à la fin de la guerre, et se retira à
Vienne où il a vécu longues années, et y est mort assez obscurément sans
avoir voulu venir jouir en Espagne de l'amnistie accordée par le traité
de Vienne fait par Riperda, lors du renvoi de l'infante, comme firent
beaucoup d'autres, ravis de quitter Vienne et de revenir jouir de leurs
biens, de leurs proches et de leurs amis dans le sein de leur patrie.

Le duc de l'Infantade n'imita ni son frère ni sa soeur\,; il s'approcha
rarement de la cour, vit peu le roi et ses ministres, ne prit à rien, ne
demeura à Madrid qu'à courtes reprises, vécut en grand seigneur peu
content, qui n'a besoin de rien, se mit à prendre soin de ses affaires
et de ses grandes terres, vint à bout bientôt de payer toutes ses dettes
et de devenir le plus grand et le plus riche seigneur d'Espagne,
jouissant d'environ deux millions de revenu, quitte, et s'amusant à
l'occupation la plus triste, mais où il avait mis son punto\,: ce fut de
se bâtir une sépulture aux capucins de Guadalajara, petite ville près de
Madrid, sur le chemin de France, qui lui appartenait, et de le faire
exactement sur le modèle et avec la même magnificence de la sépulture
des rois à l'Escurial, excepté que le panthéon de Guadalajara est
beaucoup plus petit. Je les ai vus tous deux\,; ce dernier disposé de
même en tous points et aussi superbe, en marbres, en bronze, en lapis,
en autels, en niches et tiroirs\,; en un mot, à la grandeur près, forme
et parité entière. J'en admirai d'autant plus la folie que le duc de
l'Infantado n'avait que deux filles, et qu'il protestait par modestie
qu'il n'y voulait pas être enterré, mais y faire transporter les corps
de ses pères.

Ce fut donc dans son château de Lerma que le roi et la reine voulurent
aller chasser, attendre la future princesse des Asturies, et y célébrer
son mariage. Ils en firent avertir le duc de l'Infantade, parce qu'il
n'y allait presque jamais, et des moments, et que tout y était sans
aucun meuble et assez en désordre. Le duc reçut cet avis sans s'émouvoir
ni donner aucun ordre\,: on le sut et on redoubla l'avis\,; il fut aussi
inutile que le premier, tellement qu'on prit enfin le parti d'y envoyer
des meubles et des ouvriers de toutes les sortes. Ils y trouvèrent tant
de travail qu'il n'était pas achevé quand la cour en partit, laquelle
s'y trouva si mal à l'aise, qu'après le départ de l'infante elle alla
s'établir dans un petit château voisin plus clos et plus habitable,
laissant le gros de leur suite à Lerma où la cour ne revint que sur la
nouvelle de l'échange. Le roi et la reine furent vivement piqués de ce
procédé du duc de l'Infantade, ils s'en laissèrent même entendre, mais
ce fut tout. Ce duc ne vint point à la célébration du mariage, et ne
parut point à Madrid dans tout le temps que je fus en Espagne\,; de
sorte que je ne l'ai jamais vu. J'ai ouï dire qu'il avait de l'esprit,
et qu'il l'avait même assez orné, ce qui n'est pas fort commun en
Espagne. Le nom et le choix de Lerma et l'étrange singularité de la
conduite du seigneur de ce lieu à cette occasion, m'ont fait étendre sur
son sujet d'autant plus que se tenant, comme il faisait, à l'écart de la
cour et de Madrid, je n'aurais pas trouvé lieu d'expliquer ces petites
curiosités ailleurs.

Le roi d'Espagne avait fait les maisons du prince et de la princesse des
Asturies\,; celle du prince était composée des personnes suivantes\,: le
duc de Popoli, conservant les fonctions de gouverneur, mais n'en pouvant
plus garder le nom auprès d'un prince marié, fut majordome-major\,; le
comte d'Altamire, sommelier du corps\,; le comte de San Estevan del
Puerto, grand écuyer\,; il était lors au congrès de Cambrai de la part
du roi d'Espagne\,; le duc de Gandie et le marquis de Los Balbazès,
gentilshommes de la chambre. Ces cinq seigneurs étaient grands
d'Espagne\,; le marquis del Surao en eut aussi la clef, et fut premier
écuyer\,; il avait été sous-gouverneur du prince\,; les comtes Safaleli
et d'Anénales, majordomes. Pour la princesse des Asturies, la duchesse
de Monteillano, camarera-mayor\,; le marquis de Valero,
majordome-major\,; il était lors vice-roi du Mexique, et n'était pas
grand. Le roi, qui l'avait toujours aimé, se souvint de lui en son
absence et le fit grand à son retour. Le marquis de Castel Rodrigo, mais
plus connu sous le nom de prince Pio, qu'il portait, et grand d'Espagne,
{[}fut{]} grand écuyer\,; la duchesse de Liria, la marquise de Torrecusa
et la marquise d'Assentar, dames du palais\,; doña M. de Niéves,
gouvernante destinée de l'infante, pour aller et demeurer en France avec
elle jusqu'à un certain âge, et doña Is. Martin, señoras de honor\,; le
comte d'Anguisola, premier écuyer. Il était fils du comte de Saint-Jean,
premier écuyer de la reine, qui leur fit faire depuis une prodigieuse
fortune. Ce comte d'Anguisola fut aussi majordome avec don Juan Pizzarro
y Aragon. Le P. Laubrusselle, jésuite français, précepteur des enfants,
confesseur.

Je partis le 2 décembre de Madrid pour me rendre à la cour, et je fus
coucher à l'Escurial avec les comtes de Lorges et de Céreste, mon second
fils, l'abbé de Saint-Simon et son frère. Pecquet et deux principaux des
officiers des troupes du roi, qui demeurèrent avec moi tant que je fus
en Espagne. Outre les ordres du roi d'Espagne et les lettres du marquis
de Grimaldo, je fus aussi muni de celles du nonce pour le prieur de
l'Escurial, qui en est en même temps gouverneur, pour me faire voir les
merveilles de ce superbe et prodigieux monastère, et m'ouvrir tout ce
que je voudrais y visiter, car j'avais été bien averti que, sans la
recommandation du nonce, celles du roi et de son ministre ni mon
caractère ne m'y auraient pas beaucoup servi. Encore verra-t-on que je
ne laissai pas d'éprouver la rusticité et la superstition de ces
grossiers hiéronimites.

Ce sont des moines blancs et noirs, dont l'habit ressemble à celui des
célestins, fort oisifs, ignorants, sans aucune austérité, qui, pour le
nombre des monastères dont aucun n'est abbaye et pour les richesses,
sont à peu près en Espagne ce que sont les bénédictins en France, et
sont comme eux en congrégation. Ils élisent aussi comme eux leurs
supérieurs généraux et particuliers, excepté le prieur de l'Escurial qui
est à la nomination du roi, qui l'y laisse tant et si peu qu'il lui
plait, et qui est à proportion bien mieux logé à l'Escurial que Sa
Majesté Catholique. C'est un prodige de bâtiments de structure de toute
espèce de magnificence, que cette maison, et que l'amas immense de
richesses qu'elle renferme en tableaux, en ornements, en vases de toute
espèce, en pierreries semées partout, dont je n'entreprendrai pas la
description qui n'est point de mon sujet\,; il suffira de dire qu'un
curieux connaisseur en toutes ces différentes beautés s'y appliquerait
plus de trois mois sans relâche et n'aurait pas encore tout examiné. La
forme de gril a réglé toute l'ordonnance de ce somptueux édifice, en
l'honneur de saint Laurent et de la bataille de Saint-Quentin, gagnée la
veille par Philippe II, qui, voyant l'action de dessus une hauteur, voua
d'édifier ce monastère si ses troupes remportaient la victoire, et
demandait à ses courtisans si c'était là les plaisirs de l'empereur son
père qui, en effet, les y prenait bien de plus près. Il n'y a portes,
serrures, ustensiles de quelque sorte que ce soit, ni pièce de vaisselle
qui ne soit marquée d'un gril.

La distance de Madrid à l'Escurial approche fort de celle de Paris à
Fontainebleau. Le pays est uni et devient fort désert en approchant de
l'Escurial, qui prend son nom d'un gros village dont on passe fort près
à une lieue. L'Escurial est sur un haut où on monte imperceptiblement,
d'où l'on voit des déserts à perte de vue des trois côtés\,; mais il est
tourné et comme plaqué à la montagne de Guadarrama qui environne de tous
côtés Madrid à distance de plusieurs lieues plus ou moins près. Il n'y a
point de village à l'Escurial\,; le logement de Leurs Majestés
Catholiques fait la queue du gril, les principaux grands officiers et
les officiers les plus nécessaires sont logés, même les dames de la
reine, dans le monastère\,; tout le reste l'est fort mal sur le côté par
lequel on arrive, où tout est fort mal bâti pour la suite de la cour.

L'église, le grand escalier et le grand cloître me surprirent. J'admirai
l'élégance de l'apothicairerie et l'agrément des jardins, qui pourtant
ne sont qu'une large et longue terrasse. Le Panthéon m'effraya par une
sorte d'horreur et de majesté. Le grand autel et la sacristie épuisèrent
mes yeux par leurs immenses richesses. La bibliothèque ne me satisfit
point, et les bibliothécaires encore moins. Je fus reçu avec beaucoup de
civilité et de bonne chère à souper, quoique à l'espagnole, dont le
prieur et un autre gros moine me firent les honneurs. Passé ce premier
repas, mes gens me firent à manger\,; mais ce gros moine y fournit
toujours quelques pièces qu'il n'eût pas été honnête de refuser, et
mangea toujours avec nous, parce qu'il ne nous quittait point pour nous
mener partout. Un fort mauvais latin suppléait au français qu'il
n'entendait point, ni nous l'espagnol.

Dans le sanctuaire, au grand autel, il y a des fenêtres vitrées derrière
les sièges du prêtre célébrant la grand'messe et de ses assistants. Ces
fenêtres, qui sont presque de plain-pied à ce sanctuaire, qui est fort
élevé, sont de l'appartement que Philippe II s'était fait bâtir, et où
il mourut. Il entendait les offices par ces fenêtres. Je voulus voir cet
appartement où on entrait par derrière. Je fus refusé. J'eus beau
insister sur les ordres du roi et du nonce de me faire voir tout ce que
je voudrais, je disputai en vain. Ils me dirent que cet appartement
était fermé depuis la mort de Philippe II, sans que personne y fût entré
depuis. J'alléguai que je savais que le roi Philippe V l'avait vu avec
sa suite. Ils me l'avouèrent, mais ils me dirent en même temps qu'il y
était entré par force et en maître qui les avait menacés de faire briser
les portes, qu'il était le seul roi qui, depuis Philippe II, y fût entré
une seule fois, et qu'ils ne l'ouvraient et ne l'ouvriraient jamais à
personne. Je ne compris rien à cette espèce de superstition\,; mais il
fallut en demeurer là. Louville, qui y était entré avec le roi, m'avait
dit que le tout ne contenait que cinq ou six chambres obscures et
quelques petits trous, tout cela petit, de charpenterie bousillée, sans
tapisserie lorsqu'il le vit, ni aucune sorte de meubles\,: ainsi je ne
perdis pas grand'chose à n'y pas entrer.

En descendant au Panthéon, je vis une porte à gauche à la moitié de
l'escalier. Le gros moine qui nous accompagnait, nous dit que c'était le
Pourrissoir, et l'ouvrit. On monte cinq ou six marches dans l'épaisseur
du mur, et on entre dans une chambre étroite et longue. On n'y voit que
les murailles blanches, une grande fenêtre au bout près d'où on entre,
une porte assez petite vis-à-vis, pour tous meubles une longue table de
bois, qui tient tout le milieu de la pièce qui sert pour poser et
accommoder les corps. Pour chacun qu'on y dépose, on creuse une niche
dans la muraille, où on place le corps pour y pourrir. La niche se
referme dessus sans qu'il paroisse qu'on ait touché à la muraille, qui
est partout luisante et qui éblouit de blancheur, et le lieu est fort
clair. Le moine me montra l'endroit de la muraille qui couvrait le corps
de M. de Vendôme près de l'autre porte, lequel, à sa mine et à son
discours, n'est pas pour en sortir jamais. Ceux des rois, et des reines
lesquelles ont eu des enfants, en sont tirés au bout d'un certain temps
et portés sans cérémonie dans les tiroirs du Panthéon qui leur sont
destinés. Ceux des infants et des reines qui n'ont point eu d'enfants,
sont portés dans la pièce joignante dont je vais parler, et y sont pour
toujours.

Vis-à-vis de la fenêtre, à l'autre bout de la chambre, en est une autre
de forme semblable, et qui n'a rien de funèbre. Le bout opposé à la
porte et les deux côtés de cette pièce, qui n'a d'issue que la porte par
où on y entre, sont accommodés précisément en bibliothèque\,; mais, au
lieu que les tasseaux d'une bibliothèque sont accommodés à la proportion
des livres qu'on y destine, ceux-là le sont aux cercueils qui y sont
rangés les uns auprès des autres, la tête à la muraille, les pieds au
bord des tasseaux, qui portent l'inscription du nom de la personne qui
est dedans. Les cercueils sont revêtus, les uns de velours, les autres
de brocart, qui ne se voit guère qu'aux pieds, tant ils sont proches les
uns des autres, et les tasseaux bas dessus.

Quoique ce lieu soit si enfermé, on n'y sent aucune odeur. Nous lûmes
des inscriptions à notre portée, et le moine d'autres à mesure que nous
les lui demandions. Nous fîmes ainsi le tour, causant et raisonnant
là-dessus. Passant au fond de la pièce, le cercueil du malheureux don
Carlos s'offrit à notre vue. «\,Pour celui-là, dis-je, on sait bien
pourquoi et de quoi il est mort.\,» À cette parole, le gros moine
s'altéra, soutint qu'il était mort de mort naturelle, et se mit à
déclamer contre les contes qu'il dit qu'on avait répandus. Je souris en
disant que je convenais qu'il n'était pas vrai qu'on lui eût coupé les
veines. Ce mot acheva d'irriter le moine, qui se mit à bavarder avec une
sorte d'emportement. Je m'en divertis d'abord en silence. Puis je lui
dis que le roi, peu après être arrivé en Espagne, avait eu la curiosité
de faire ouvrir le cercueil de don Carlos, et que je savais d'un homme
qui y était présent (c'était Louville) qu'on y avait trouvé sa tête
entre ses jambes\,; que Philippe II, son père, lui avait fait couper
dans sa prison devant lui. «\,Hé bien\,! s'écria le moine tout en furie,
apparemment qu'il l'avait bien mérité\,; car Philippe II en eut la
permission du pape,\,» et de là crier de toute sa force merveilles de la
piété et de la justice de Philippe II, et de la puissance sans bornes du
pape, et à l'hérésie contre quiconque doutait qu'il ne pût pas ordonner,
décider et dispenser de tout. Tel est le fanatisme des pays
d'inquisition, où la science est un crime, l'ignorance et la stupidité
la première vertu. Quoique mon caractère m'en mît à couvert, je ne
voulus pas disputer et faire avec ce piffre de moine une scène ridicule.
Je me contentai de rire et de faire signe de se taire, comme je fis à
ceux qui étaient avec moi. Le moine dit donc tout ce qu'il voulut à son
aise, et assez longtemps sans pouvoir s'apaiser. Il s'apercevait
peut-être à nos mines que nous nous moquions de lui, quoique sans gestes
et sans parole. Enfin il nous montra le reste du tour de la chambre,
toujours fumant\,; puis nous descendîmes au Panthéon. On me fit la
singulière faveur d'allumer environ les deux tiers de l'immense et de
l'admirable chandelier qui pend du milieu de la voûte, dont la lumière
nous éblouit, et faisait distinguer dans toutes les parties du Panthéon,
non seulement les moindres traits de la plus petite écriture, mais ce
qui s'y trouvait de toutes parts de plus délié.

Je passai trois jours à l'Escurial, logé dans un grand et bel
appartement, et tout ce qui était avec moi fort bien logé aussi. Notre
moine qui avait toujours montré sa mauvaise humeur depuis le jour du
Pourrissoir, n'en reprit de belle qu'au déjeuner du départ. Nous le
quittâmes sans regret, mais non l'Escurial, qui donnerait de l'exercice
et du plaisir à un curieux connaisseur pour plus de trois mois de
séjour. Chemin faisant, nous rencontrâmes le marquis de Montalègre, et
arrivâmes en même temps que lui à la dînée. Il m'envoya aussi prier à
dîner avec ces messieurs qui étaient avec moi. Il était fort accompagné,
et nous fit très promptement fort grande chère et bonne à l'espagnole,
ce qui nous fit un peu regretter le dîner que mes gens avaient préparé
pour nous. J'aurai lieu de parler de ce seigneur.

Enfin nous arrivâmes le 9 à notre village de Villahalmanzo, où je me
trouvai le plus commodément du monde, ainsi que tout ce qui était avec
moi. J'y trouvai mon fils aîné encore bien convalescent avec l'abbé de
Mathan, qui venaient de Burgos. Nous soupâmes fort gaiement, et je
comptais de me bien promener le lendemain, et m'amuser à reconnaître le
village et les environs\,; mais la fièvre me prit la nuit, augmenta dans
la journée, devint violente la nuit suivante, tellement qu'il ne fut
plus question d'aller le 11, qui était ce jour-là, à la descente du
carrosse du roi et de la reine d'Espagne à Lerma. Le mal augmenta avec
une telle rapidité qu'on me trouva en grand danger, et incontinent après
à l'extrémité. Je fus saigné peu après, la petite vérole parut dont tout
le pays était rempli. Ce climat était tel cette année\,; qu'il y gelait
violemment douze ou quatorze heures tous les jours, tandis que depuis
onze heures du matin jusqu'à près de quatre, il faisait le plus beau
soleil du monde, et trop chaud sur le midi pour s'y promener\,; et où il
ne donnait point par quelque obstacle de muraille, il n'y dégelait pas
un moment. Ce froid était d'autant plus piquant, que l'air était plus
pur et plus vif, et le ciel de la sérénité la plus parfaite et la plus
continuelle.

Le roi d'Espagne, qui craignait extrêmement la petite vérole, et qui
n'avait confiance avec raison qu'en son premier médecin, me l'envoya dès
qu'il fut informé de ma maladie, avec ordre de ne me pas quitter d'un
moment jusqu'à ce que je fusse guéri. J'eus donc continuellement cinq ou
six personnes auprès de moi, outre ceux de mes domestiques qui me
servirent, un des plus sages et des meilleurs médecins de l'Europe, qui
de plus était de très bonne compagnie, qui ne me quittait ni jour ni
nuit, et trois fort bons chirurgiens dont La Fare m'en envoya un qu'il
avait amené. J'eus une grande abondance partout de petite vérole de bon
caractère, sans aucun accident dangereux depuis qu'elle eut paru, et on
sépara de table et de tout commerce maîtres et valets qui me voyaient,
même de cuisine, de ceux qui faisaient la mienne, et de ceux qui ne me
voyaient point. Le premier médecin se précautionnait presque tous les
jours de nouveaux remèdes en cas de besoin, et ne m'en fit aucun que de
me faire boire pour toute boisson de l'eau dans laquelle on jetait selon
sa quantité des oranges avec leur peau coupées en deux, qui frémissait
lentement devant mon feu, quelques rares cuillerées d'un cordial doux et
agréable dans le fort de la suppuration, et dans la suite un peu de vin
de Rota avec des bouillons où il entrait du boeuf et une perdrix. Rien
ne manqua donc aux soins de gens qui n'avaient que moi de malade, et
qu'ils avaient ordre de ne pas quitter, et rien ne manqua à mon
amusement quand je fus en état d'en prendre, par la bonne compagnie qui
était auprès de moi, et cela dans un temps où les convalescents de cette
maladie en éprouvent tout l'ennui et le délaissement. Tout à la fin du
mal je fus saigné et purgé une seule fois, après quoi je vécus à mon
ordinaire, mais dans cette espèce de solitude. J'aurai bientôt lieu de
parler de ce premier médecin.

Pendant le grand intervalle que cette maladie me tint hors de tout
commerce, l'abbé de Saint-Simon en entretint même d'affaires avec le
cardinal Dubois, avec Grimaldo, avec Sartine et avec quelques autres. Je
crois ne pouvoir mieux remplir ici ce vide forcé d'une oisiveté de six
semaines que par un léger tableau de la cour d'Espagne, telle qu'elle
était pendant le séjour de six mois que je demeurai en ce pays-là. Le
détail étendu qui se trouve t. III, p. 88 et suivantes, qui se voit sur
l'Espagne à l'occasion de l'avènement de Philippe {[}V{]} à cette
couronne, et un autre précédent à propos du testament de Charles II,
m'en épargnera beaucoup ici qui ne seraient que des redites.

On voit\footnote{Tout ce résumé depuis on voit dans le détail jusqu'à
  véritables noms et maisons (p.~361) est omis dans les anciennes
  éditions. Elles ont rejeté ici les passages que nous avons rétablis,
  au t. III, à la place que leur avait assignée Saint-Simon.} dans ce
détail, à propos du testament {[}de Charles II{]}, les emplois et les
caractères des personnages qui y eurent le plus de part, celui de la
reine épouse de Charles II, et des personnages autrichiens. Dans celui
qui est t. III, p.~88 et suivantes, on trouve celui de l'origine et des
progrès en Espagne des trois branches sorties de la maison de Portugal,
de celle de Cadaval de la même origine restée en Portugal, enfin de
celle d'Alencastro, portugaise aussi, et des ducs d'Aveiro, d'Abrantès
et Liñarez en Espagne, et des principaux personnages de ces maisons\,;
le fond et les fonctions des conseils de Castille et d'Aragon, de leurs
présidents et gouverneurs, de ce qu'étaient le conseil d'État et les
conseillers d'État, les maisons, noms, dignités, caractères de ceux qui
l'étaient alors\footnote{On voit que Saint Simon n'a voulu parler que de
  l'état de l'Espagne en 1700, dans le passage cité plus haut, et que
  les éditeurs ont eu tort de le reporter à l'année 1721.}\,; plusieurs
curiosités sur des façons de signer particulières à quelques grands, et
de ce qui s'appelle la saccade du vicaire pour des mariages. Enfin on y
trouve l'explication de l'être et des fonctions du secrétaire des
dépêches universelles, les changements produits par l'arrivée de
Philippe V dans la manière du gouvernement\footnote{Nouvelle preuve de
  la nécessité de conserver, comme nous l'avons fait, les passages
  relatifs à l'Espagne aux diverses années où l'auteur les avait placés.}
à l'égard des grandes charges de la cour, les majordomes-majors, grands
écuyers du roi et de la reine, sommelier du corps du roi, camarera-mayor
de la reine, ses dames du palais, ses señoras de honor et ses
caméristes, premiers écuyers du roi et de la reine, gentilshommes de la
chambre du roi, capitaine des hallebardiers, patriarche des Indes,
majordomes du roi et de la reine, estampilla.

Ce détail des charges, de leurs fonctions et des possesseurs s'y trouve
exactement, ainsi que le caractère et les fonctions du P. Daubenton,
confesseur du roi, et les voyages en France et en Flandre des ducs
d'Arcos et de Baños pour s'être seuls, entre tous les grands, opposés,
par un mémoire au roi d'Espagne, à l'égalité des rangs, honneurs et
distinctions, réciproquement convenue par, les deux rois, entre les ducs
de France et les grands d'Espagne dans les deux monarchies. Ce dernier
fait se trouve t. III, p.~224, et si on veut repasser de suite les pages
suivantes jusqu'à la page 327, on y verra une digression sur la dignité
de grand d'Espagne, et sa comparaison avec celle de nos ducs\,; ce que
c'étaient que les ricos-hombres\,; ce qu'ils sont devenus\,; comment la
dignité des grands d'Espagne leur a été substituée\,; l'origine des uns
et des autres, et leurs distinctions\,; quelle part {[}eurent{]} aux
affaires leur multiplication, {[}et{]} leur affaiblissement\,; comment
disparus et renés sous le nom nouveau de grands\,; l'adresse des rois et
jusqu'où portée par les sept différentes gradations, qui ont porté
autant de grands coups à la dignité des grands\,; et l'introduction des
trois classes, toutes choses si peu connues hors de l'Espagne, et qui
causent une grande surprise par le pouvoir que les rois s'y sont donné
de suspendre, de confirmer, d'ôter même la grandesse à volonté, et sans
forme ni crime, et d'en tirer des tributs annuels\,; la proscription de
tout rang étranger séculier et de toute prétention étrangère\,; le
mystère que font les grands de leurs classes et de leur ancienneté\,;
leur attachement à n'avoir égard ni aux unes ni à l'autre, et de marcher
et se placer partout entre eux comme le hasard les fait rencontrer\,; la
raison de cette conduite\,; ce que l'on sait à peu près des
ricos-hombres devenus grands\,; l'indifférence entière pour les grands
des titres de duc, prince, marquis, comte\,; la raison de cette
indifférence\,; les successions aux grandesses\,; leur difficile
extinction\,; leur fréquente accumulation sur la même tête\,; l'égalité
en tout entre ceux qui en ont plusieurs et ceux qui n'en ont qu'une\,;
ce que sont les majorasques\,; les démissions des grandesses
inconnues\,; mais le rang effectif de leurs héritiers présomptifs\,; le
chaos si difficile à percer de la confusion des noms et des armes, et sa
cause\,; le poids des successions\,; les avantages des bâtards et leurs
différences en Espagne\,; nulle marque de dignité aux armes, aux
carrosses, aux maisons que le dais\,; ce qui équivaut à ce qui est connu
en France sous le nom d'honneurs du Louvre\,; quelques distinctions
particulières au-dessus des grands\,; le plan figuré et l'explication de
la couverture d'un grand chez le roi et chez la reine, suivant les trois
différentes classes, et de l'assiette de la séance quand le roi tient
chapelle\,; les cérémonies de la Chandeleur et des Cendres\,; banquillo
du capitaine des gardes en quartier, et raison pour laquelle il faut que
les capitaines des gardes soient toujours grands\,; cortès ou états
généraux\,; rangs et distinctions des grands, de leurs femmes, des
héritiers présomptifs des grandesses en toutes cérémonies et fêtes
ecclésiastiques et séculières\,; traitement par écrit, dans les
églises\,; honneurs militaires\,; égalité chez tous souverains non
rois\,; honneurs à Rome\,; bâtards des rois\,; grands nuls en toutes
affaires\,; n'ont aucun habit de cérémonie, non plus que le roi\,; n'ont
nulle préférence de rang dans les ordres d'Espagne ni dans celui de la
Toison d'or\,; acceptent de fort petits emplois\,; leur dignité s'achète
du roi quelquefois\,; elle n'a point de serment\,; comparaison des deux
dignités des ducs de France et des grands d'Espagne, et de leur fond
dans tous leurs âges. La dignité de grand d'Espagne ne peut être
comparée à celle des ducs de France, beaucoup moins à celle des pairs.
Comparaison de l'extérieur des dignités de duc de France et de grand
d'Espagne\,; spécieux avantages des grands d'Espagne\,; un seul solides
désavantages effectifs et réels des grands d'Espagne\,; désavantage des
grands d'Espagne jusque dans le droit de se couvrir\,; abus des
grandesses françaises. Enfin on a tâché de n'oublier rien dans ces longs
détails de ce qui est des grands et des grandesses d'Espagne, et des
prérogatives et des fonctions des charges, après s'en être instruit à
fond en Espagne, et par des grands d'Espagne de
Charles-Quint\footnote{C'est-à-dire de ceux donna grandesse remontait à
  Charles-Quint.}, des plus instruits, ainsi que de leurs véritables
noms et maisons. Il ne reste donc ici que de donner la liste de ceux qui
étaient grands, quand j'ai quitté l'Espagne, et, à côté, {[}celle{]} de
leurs noms et maisons.

\hypertarget{chapitre-xvi.}{%
\chapter{CHAPITRE XVI.}\label{chapitre-xvi.}}

1721

~

{\textsc{Grands d'Espagne constamment de la première origine.}}
{\textsc{- Liste alphabétique de tous les grands d'Espagne existant
pendant que j'y étais, en 1722, où les maisons et les personnages sont
courtement expliqués.}} {\textsc{- Duc d'Alencastro.}} {\textsc{- Duc
d'Albe.}} {\textsc{- Duc d'Albuquerque.}} {\textsc{- Duc del Arco.}}
{\textsc{- Duc d'Arcos.}} {\textsc{- Duc d'Aremberg.}} {\textsc{- Duc
d'Arion.}} {\textsc{- Duc d'Atri.}} {\textsc{- Duc d'Atrisco.}}
{\textsc{- Duc de Baños.}} {\textsc{- Duc de Bejar.}} {\textsc{- Duc de
Berwick.}} {\textsc{- Duc de Bournonville.}} {\textsc{- Duc Doria.}}
{\textsc{- Duc d'Estrées, maréchal de France.}} {\textsc{- Duc de Frias,
connétable héréditaire de Castille.}} {\textsc{- Titres de connétable et
d'amirante de Castille supprimés par Philippe V.}} {\textsc{- Duc de
Gandie.}} {\textsc{- Duc de Giovenazzo.}} {\textsc{- Duc de Gravina.}}
{\textsc{- Duc d'Havré.}} {\textsc{- Duc d'Hijar.}} {\textsc{- Duc del
Infantado.}} {\textsc{- Duc de Licera.}} {\textsc{- Duc de Liñarez.}}
{\textsc{- Duc de Liria.}} {\textsc{- Ducs de Medina-Coeli.}} {\textsc{-
La Cerda, seigneurs de Medina-Coeli.}} {\textsc{- Dernier direct comte
de Foix, etc.}} {\textsc{- Succession de ses États après lui.}}
{\textsc{- Ses deux bâtards.}} {\textsc{- Fin malheureuse du cadet.}}
{\textsc{- Fortune énorme de l'aîné.}} {\textsc{- Bâtards de Foix,
comtes, puis ducs de Medina-Coeli.}} {\textsc{- Figuerroa, ducs de
Medina-Coeli.}} {\textsc{- Amirante de Castille.}} {\textsc{- Duc de
Medina-Sidonia.}} {\textsc{- Duc de Saint-Michel.}} {\textsc{- Duc de La
Mirandole.}} {\textsc{- Duc et duchesse de Monteillano.}} {\textsc{- Duc
de Monteléon.}} {\textsc{- Duc de Mortemart.}} {\textsc{- Duc de
Najera.}} {\textsc{- Duc de Nevers.}} {\textsc{- Duc de Noailles.}}
{\textsc{- Duc d'Ossone.}} {\textsc{- Duc et duchesse de Saint-PierreDuc
de Popoli\,; son caractère\,; son fils et sa belle-fille\,; leur
caractère.}} {\textsc{- Duc de Sessa\,; duc de Saint-Simon et son second
fils conjointement.}} {\textsc{- Duc de Solferino\,; sa fortune.}}
{\textsc{- Duc de Turcis.}} {\textsc{- Duc de Veragua.}} {\textsc{-
Maréchal-duc de Villars.}} {\textsc{- Duc d'Uzeda\,; sa défection.}}
{\textsc{- Prince de Bisignano.}} {\textsc{- Prince de Santo-Buono.}}
{\textsc{- Remède sûr et sans inconvénient pour la goutte, au Pérou.}}
{\textsc{- Prince de Butera.}} {\textsc{- Prince de Cariati.}}
{\textsc{- Prince de Chalais\,; sa fortune.}} {\textsc{- Prince de
Chimay.}} {\textsc{- Prince de Castiglione.}} {\textsc{- Connétable
Colonne.}} {\textsc{- Prince Doria.}} {\textsc{- Prince de Ligne.}}
{\textsc{- Prince de Masseran\,; son caractère\,; sa fortune.}}
{\textsc{- Prince de Melphe.}} {\textsc{- Prince de Palagonia.}}
{\textsc{- Prince de Robecque.}} {\textsc{- Prince de Sermonetta.}}
{\textsc{- Prince de Sulmone.}} {\textsc{- Prince de Surmia.}}
{\textsc{- Prince d'Ottaïano.}} {\textsc{- Marquis d'Arizza.}}
{\textsc{- Marquis d'Ayétone.}} {\textsc{- Marquis de Los Balbazès.}}
{\textsc{- Marquis de Bedmar.}} {\textsc{- Marquis de Camaraça.}}
{\textsc{- Marquis de Castel dos Rios.}} {\textsc{- Marquis de
Castel-Rodrigo.}} {\textsc{- Prince Pio.}} {\textsc{- Marquis de
Castromonte.}} {\textsc{- Marquis de Clarafuente.}} {\textsc{- Marquis
de Santa-Cruz\,; sa fortune.}} {\textsc{- Marquis de Laconi.}}
{\textsc{- Marquis de Lede.}} {\textsc{- Marquis de Mancera.}}
{\textsc{- Marquis de Mondejar.}} {\textsc{- Marquis de Montalègre.}}
{\textsc{- Marquis de Pescaire.}} {\textsc{- Marquis de Richebourg.}}
{\textsc{- Marquis de Ruffec.}} {\textsc{- Marquis de Torrecusa\,;
caractère de son épouse.}} {\textsc{- Marquis de Villena, duc
d'Escalona\,; sa naissance, ses actions, son éloge, sa famille.}}
{\textsc{- Marquis Visconti.}} {\textsc{- Comte d'Aguilar\,; ses
faits.}} {\textsc{- Grandeur de la maison d'Avellano.}} {\textsc{-
Grandeur de la maison de Manrique de Lara.}} {\textsc{- Comte
d'Altamire\,; sa famille, son caractère.}} {\textsc{- Comte d'Aranda.}}
{\textsc{- Comte de Los Arcos.}} {\textsc{- Comte d'Atarès.}} {\textsc{-
Comte de Baños.}} {\textsc{- Comte de Benavente\,; grandeur de la maison
de Pimentel\,; jésuites.}} {\textsc{- Comte de Castrillo.}} {\textsc{-
Comte d'Egmont.}} {\textsc{- Comte de San-Estevan de Gormaz.}}
{\textsc{- Comte de San-Estevan del Puerto.}} {\textsc{- Comte de
Fuensalida.}} {\textsc{- Comte de Lamonclava.}} {\textsc{- Comte de
Lemos\,; son caractère et celui de la comtesse sa femme.}} {\textsc{-
Comte de Maceda\,; son fils et sa belle-fille.}} {\textsc{- Comte de
Miranda.}} {\textsc{- Comte de Montijo.}} {\textsc{- Comte d'Oñate.}}
{\textsc{- Comte d'Oropesa.}} {\textsc{- Comte de Palma.}} {\textsc{-
Comte de Parcen.}} {\textsc{- Comte de Parédes.}} {\textsc{- Comte de
Peñeranda.}} {\textsc{- Comte de Peralada.}} {\textsc{- Comte de
Priego\,; son adresse à obtenir la grandesse\,; son caractère.}}
{\textsc{- Comte de Salvatierra.}} {\textsc{- Comte de Tessé.}}
{\textsc{- Comte Visconti.}} {\textsc{- Grands d'Espagne par charge ou
état, mais imperceptibles.}} {\textsc{- Oubli.}} {\textsc{- Marquis de
Tavara.}} {\textsc{- Marquis de Villafranca.}} {\textsc{- Mystère des
classes et des dates des grandesses.}} {\textsc{- Impossibilité sur les
classes.}} {\textsc{- Difficultés sur les dates.}} {\textsc{- Comment
reconnues pour la plupart.}} {\textsc{- État des grands, suivant
l'ancienneté entre eux qu'on a pu reconnaître, et par règnes de leurs
érections, et les maisons pour qui elles ont été faites, et les maisons
où elles se trouvent en 1722.}} {\textsc{- Medina-Coeli.}} {\textsc{-
Benavente.}} {\textsc{- Amirante de Castille.}} {\textsc{- Lemos.}}
{\textsc{- Medina-Sidonia.}} {\textsc{- Miranda.}} {\textsc{-
Albuquerque.}} {\textsc{- Villena et Escalona.}} {\textsc{- Origine de
dire les rois jusqu'à aujourd'hui, lorsqu'on a à dire le roi et la
reine.}} {\textsc{- Albe.}} {\textsc{- Oñate.}} {\textsc{- Infantado.}}
{\textsc{- Oropesa.}} {\textsc{- Najera.}} {\textsc{- Gandie.}}
{\textsc{- Sessa.}} {\textsc{- Bejar.}} {\textsc{- Frias.}} {\textsc{-
Villafranca.}} {\textsc{- Egmont.}} {\textsc{- Veragua.}} {\textsc{-
Pescaire.}} {\textsc{- Ayétone.}} {\textsc{- Ossone.}} {\textsc{-
Terranova et Monteléon.}} {\textsc{- Santa-Cruz\,; cause horrible de
cette érection.}} {\textsc{- Aranda.}} {\textsc{- Uzeda.}} {\textsc{-
Peñeranda.}} {\textsc{- Mondejar.}} {\textsc{- Hijar.}} {\textsc{-
Havré.}} {\textsc{- Sulmone.}} {\textsc{- Los Balbazès.}} {\textsc{-
Altamire.}} {\textsc{- Abrantès et Liñares.}} {\textsc{- Bisignano.}}
{\textsc{- Castel-Rodrigo.}} {\textsc{- Torrecusa.}} {\textsc{-
Colonne.}} {\textsc{- Camaraça.}} {\textsc{- Aguilar.}} {\textsc{-
Aremberg.}} {\textsc{- Ligne.}} {\textsc{- Fuensalida.}} {\textsc{-
Saint-Pierre.}} {\textsc{- Palma.}} {\textsc{- Nevers.}} {\textsc{-
Santo-Buono.}} {\textsc{- Surmia.}} {\textsc{- Liñares.}} {\textsc{-
Baños C. Parédes.}} {\textsc{- Lamonclava.}} {\textsc{- San-Estevan del
Puerto.}} {\textsc{- Montalègre.}} {\textsc{- Los Arcos.}} {\textsc{-
Montijo.}} {\textsc{- Baños D. Castromonte.}} {\textsc{- Castiglione.}}
{\textsc{- Ottaïano.}} {\textsc{- Castel dos Rios.}} {\textsc{-
Mortemart, éteint.}} {\textsc{- Estrées, éteint.}} {\textsc{- Liria.}}
{\textsc{- Gravina.}} {\textsc{- Bedmar.}} {\textsc{- Tessé.}}
{\textsc{- La Mirandole.}} {\textsc{- Atri.}} {\textsc{- Chimay.}}
{\textsc{- Monteillano.}} {\textsc{- Priego.}} {\textsc{- Noailles.}}
{\textsc{- Popoli.}} {\textsc{- Masseran.}} {\textsc{- Richebourg.}}
{\textsc{- Chalais.}} {\textsc{- Robecque.}} {\textsc{- Maceda.}}
{\textsc{- Solfarino.}} {\textsc{- San-Estevan de Gormaz.}} {\textsc{-
Bournonville.}} {\textsc{- Villars.}} {\textsc{- Lede.}} {\textsc{-
Saint-Michel.}} {\textsc{- Del Arco.}} {\textsc{- Ruffec.}} {\textsc{-
Arion.}} {\textsc{- Oubli sur Mancera avec quelque éclaircissement.}}

~

Le chiffre à côté des grands marquera le nombre de grandesses sur la
même tête, accumulées par héritages en ceux qui en ont plusieurs, qui
toutes ne se peuvent partager, mais tombent au même et seul héritier, et
ne donnent jamais en rien aucune distinction ni préférence au-dessus de
ceux qui n'en ont qu'une seule. Comme le secret qu'ils affectent de
leurs diverses classes et de leur ancienneté les oblige de marcher et de
se placer entre eux comme ils se rencontrent, et que les titres de duc,
prince, marquis et comte leur sont indifférents, jusque-là que le
marquis de Villena porte toujours ce titre de préférence à celui de duc
d'Escalona qu'il a aussi, parce qu'il se prétend le premier marquis de
Castille, quoique cette qualité ne lui donne quoi que ce soit, je n'ai
pu que choisir l'ordre alphabétique pour donner ici la liste des grands
d'Espagne, par laquelle on verra qu'il y en a bien plus que de ducs en
France, même sans y comprendre ceux qui ont été faits depuis mon retour
d'Espagne, ni ceux qui vivent et sont établis hors de l'Espagne. À
l'égard de leurs différentes nations, elles se reconnaîtront aisément
par les noms de leurs maisons\,; on remarquera seulement qu'aucun grand
espagnol n'a porté le titre de prince jusqu'à présent. Ajoutons
seulement ici que l'opinion commune en Espagne, et qui usurpe l'autorité
de la notoriété publique, admet un premier ordre de grands devenus
insensiblement tels de ricos-hombres qu'ils étaient lors de
l'établissement des grands par Charles-Quint, au lieu des ricos-hombres
qu'il abolit. Mais il faut remarquer en même temps que ce premier ordre
de grands d'Espagne, dont la liste va suivre, ne leur donne aucune sorte
de préférence ni de distinction sur pas un des autres grands les plus
modernement faits, ce qui me les fera insérer de nouveau dans la liste
générale qui suivra immédiatement celle-ci. Comme il y en a de cette
première liste plusieurs qui ont passé depuis en d'autres maisons, je me
contenterai, dans la liste générale, de marquer d'une croix à côté du
nom de maison des grands, celles qui dans cette liste-ci ont passé de
l'état de ricos-hombres à celui de grands d'Espagne.

LISTE EXPLIQUÉE À LA FIN DE LA PAGE PRÉCÉDENTE.

Castille.

Aragon.

\emph{Ducs de}

\emph{Marquis de}

\emph{Ducs de}

\emph{Marquis de}

Medina-Coeli.

Villena.

Ségorbe.

D'Ayétone.

Escalone.

Astorga.

Montalte.

L'Infantade.

Albuquerque.

\emph{Comtes de}.

Albe.

Benevente.

Bejar.

Lemos.

Arcos.

PLUSIEURS Y AJOUTENT\,:

\emph{Ducs de}

\emph{Marquis}

Medina-Sidonia.

D'Aguilar.

Ces cinq-ci à côté sont, à la vérité, si fort en tout des plus grands et
des plus distingués seigneurs, qu'on aurait peine à leur disputer la
même origine des précédents.

Najara.

Frias, connétable,

Medina di Rioseco, amirante, héréditaires.

GRANDS D'ESPAGNE EN ORDRE ALPHABÉTIQUE EXISTANT EN TOUS PAYS PENDANT QUE
J'ÉTAIS EN ESPAGNE, 1722.

Ducs de\,:

Abrantès est \emph{Alencastro}. Voy. p.~392.

9 ALBE\footnote{Il est certain que cette maison tire son nom de la ville
  archiépiscopale de Tolède, capitale de la Castille-Nouvelle, et qu'il
  y a des seigneurs de maison entièrement différente, qui portent ce
  même nom, pour distinction de quoi la maison d'Albe a pris le nom ou
  avant-nom d'Alvarez de Tolède. Pourquoi et comment ce nom de Tolède
  est devenu celui de ces Albe et de ces autres seigneurs différents,
  c'est ce qui est caché dans l'obscurité des temps, et qui ne peut être
  venu que d'exploits militaires faits à Tolède, dont le nom leur aura
  été approprié pour honorer l'exploit et en conserver la mémoire\,; car
  pas un d'eux n'a jamais rien possédé dans Tolède qui ait pu leur en
  faire prendre le nom. On en doit dire le même du nom de Cordoue, qui
  se trouvera dans cette liste, et que le fameux Gonzalve, si connu sous
  le nom tout court de grand capitaine, a comme consacré en le portant,
  et pareillement du nom de Léon de la maison Ponce de Léon, mais qui
  vient de descendance des rois de Léon. On verra ici que je ne m'étends
  guère que sur les grands espagnols. Il faut remarquer que \emph{j} se
  prononce \emph{c}, mais un peu de la gorge, comme dans le nom de Bejar
  et autres semblables, et que \emph{ñ} avec un tiret dessus se prononce
  en le mouillant comme dans le nom de Baños, qui se prononce Bagnos et
  autres pareils. (\emph{Note de Saint-Simon}).} est \emph{Tolède}. Jean
II, roi de Castille, fit don, en 1430, de la ville d'Albe en titre de
comté, dans le pays de Salamanque, à Guttiere Gomez de Tolède, évêque de
Palencia, puis archevêque de Séville, enfin de Tolède, qui le légua à
son neveu Ferd. Alvarez de Tolède, dont le fils, Garcia Alvarez de
Tolède, qui lui succéda, fut fait duc d'Albe, en 1469, par les rois
catholiques. On avertit, une fois pour toutes, que les rois catholiques,
dont il sera souvent parlé, sont les célèbres Ferdinand, roi d'Aragon,
et Isabelle, reine de Castille, dont le mariage réunit ces deux
couronnes et les conquêtes sur les Mores qu'ils repoussèrent en Afrique,
leur acquit toutes les Espagnes, excepté le Portugal. Ce premier duc
d'Albe fut de mâle en mâle bisaïeul du duc d'Albe, trop fameux par ses
cruautés aux Pays-Bas et par la facile conquête du Portugal, dont, peu
avant de mourir, il s'empara pour Philippe II, après la mort du cardinal
Henri, roi de Portugal. Son fils aîné, premier duc d'Huesca, mourut sans
enfants. Il avait un frère dont le fils lui succéda\,; il s'appelait
Antoine de Tolède-Beaumont, parce que sa mère, Briande de Beaumont,
était héritière du comté de Lerins et des offices de connétable et de
chancelier héréditaires de Navarre, où cette maison avait si longtemps
et si grandement figuré. De ce cinquième duc d'Albe est venu, de mâle en
mâle, le duc d'Albe mort à Paris ambassadeur d'Espagne, y ayant perdu
son fils unique\,; l'oncle paternel de ce neuvième duc d'Albe, lui
succéda. Il avait suivi l'archiduc et s'était retiré à Vienne, où le
comte de Galve, frère du duc de l'Infantade, épousa sa fille. Son
beau-père fit enfin sa paix, revint à Madrid, et s'y couvrit comme duc
d'Albe. Le duc del Arco, parrain de mon second fils pour sa couverture,
prit ce duc d'Albe pour lui aider à en faire les honneurs. Je l'ai fort
peu vu à Madrid où il menait une vie fort retirée. Il y passait pour un
bon et honnête homme. Il me parut fort poli et savoir l'être en grand
seigneur. Ces Tolède se distinguent d'autres Tolède par le prénom
d'Alvarez.

Albuquerque, \emph{Bertrand La Cueva}. Henri IV, roi de Castille, fit
don, en 1464, d'Albuquerque, dans l'Estremadure castillane, à Bertrand
de La Cueva et l'érigea en même temps en duché pour lui, alors comte de
Ledesma, dont la postérité masculine finit vers le XVe siècle. M. de La
Cueva, héritière, porta le duché d'Albuquerque en mariage à un François
nommé Hugues Bertrand, qui prit le nom seul et les armes de La Cueva,
duquel toute cette maison descend aujourd'hui. Ce duché y a toujours été
conservé par le soin qu'on a pris d'y marier toujours les filles
héritières. Cet heureux François ne pouvait pas être un homme du commun
pour trouver un tel établissement en Espagne. On ne peut néanmoins dire
qui il était\,; mais on connaît des Bertrand qui, dès avant 1040,
étoient barons de Briquebec en Normandie, qui ont grandement figuré de
père en fils, et immédiatement alliés aux maisons des comtes d'Aumale,
de Trie, de Tancarville, de Craon, de Nesle, d'Estouteville, de Coucy,
de Sully, cadets des comtes de Champagne, Paynel et Chabot. Robert
Bertrand, baron de Briquebec, vicomte de Roncheville, connétable de
Normandie, fit, comme seigneur de Honfleur, des dons à l'abbaye du Bec,
en 1240. Il fut grand-père de Robert VII Bertrand, lieutenant du roi en
Guyenne, Saintonge, Normandie et Flandre, maréchal de France en 1325. Il
fut présent à l'hommage qu'Édouard III, roi d'Angleterre, rendit, en
1329, à Amiens, à Philippe de Valois, eut divers autres grands emplois,
mourut en 1348 et ne laissa que des filles. Il eut un frère évêque,
comte de Beauvais, pair de France, et un autre frère, vicomte de
Roncheville, dont pourrait bien être sorti ce Hugues Bertrand si bien
établi en Espagne. Mais quelque favorable que puisse en être la
conjecture, elle est sans aucune sorte de preuves.

Le douzième duc d'Albuquerque, que j'ai vu en Espagne, était petit-fils
d'une duchesse d'Albuquerque, laquelle était aussi La Cueva, qui avait
beaucoup d'esprit et de lecture, et qui tenait presque tous les jours
chez elle une assemblée de savants et de personnes distinguées et de
bonne compagnie. Elle fut camarera-mayor de la reine Louise, fille de
Monsieur, lorsqu'elle obtint que la duchesse de Terranova, qui l'était,
fût renvoyée, ce qui était sans exemple en Espagne. Cette duchesse
d'Albuquerque la fut aussi de la palatine de Neubourg, seconde femme de
Charles II, dont elle obtint la vice-royauté du Mexique, vers la fin de
son règne, pour ce duc d'Albuquerque son petit-fils, où il était lors de
l'avènement de Philippe V à la couronne. Il se mit fort bien avec lui en
lui envoyant, aussitôt après qu'il en fut informé, un grand secours
d'argent, hors les temps accoutumés, qui arriva fort heureusement et
fort à propos. Il y perdit sa femme, et à ce qu'il me dit, son estomac,
tellement qu'il ne mangeait plus que des potages. Ce fut son excuse de
se trouver aux repas de cérémonie que je donnai. À la fin il me dit, sur
le dernier, dont par règle je le conviai pour la Toison de mon fils
aîné, qu'il ne pouvait plus me refuser toujours. Il y vint donc et me
parut surpris du service où il y avait quantité de potages\,; il mangea
de tous, mais il se contenta, pour tout le reste, de quelques petites
mies de pain qu'il trempa dans toutes les sauces, une seule fois par
plat, et témoignait les trouver fort bonnes.

La première fois que je le vis, ce fut dans une porte de l'appartement
de la reine, à mon audience de cérémonie. J'aperçus devant moi, tout
contre, un petit homme trapu, mal bâti, avec un habit grossier sang de
boeuf, les boutons du même drap, des cheveux verts et gras qui lui
battaient les épaules, de gros pieds plats et des bas gris de porteur de
chaise. Je ne le voyais que par derrière, et je ne doutai pas un moment
que ce ne fût le porteur de bois de cet appartement. Il vint à tourner
la tête et me montra un gros visage rouge, bourgeonné, à grosses lèvres
et à nez épaté\,; mais ses cheveux se dérangèrent par ce mouvement et me
laissèrent apercevoir un collier de la Toison. Cette vue me surprit à
tel point que je m'écriai tout haut\,: «\,Ah\,! mon Dieu, qu'est-ce que
cela\,?» Le duc de Liria, qui était derrière moi, jeta les mains à
l'instant sur mes épaules et me dit\,: «\,Taisez-vous, c'est mon
oncle.\,» Le duc de Veragua et lui me le nommèrent et me le présentèrent
aussitôt. Je l'ai fort vu depuis\,: c'était un homme d'esprit, très
instruit, fin et adroit courtisan, qui avait su tirer de la cour et s'y
maintenir bien et en considération dans le monde. Sa conversation était
agréable, polie, instructive. Il avait, vis-à-vis l'Incarnation, un des
plus beaux palais de Madrid et des plus vastes, magnifiquement meublé,
avec force argenterie, et jusqu'à beaucoup de bois de meubles qui, au
lieu d'être de bois, étaient d'argent. Il était fort riche et parlait
assez bien français. Il avait plusieurs fils\,: l'aîné, déjà âgé, dont
on disait beaucoup de bien et qui, avant mon départ, fut un des
gentilshommes de la chambre du prince des Asturies.

DelArco, \emph{Manrique de Lara}. Quoique grandement et prochainement
allié, il n'était pas reconnu unanimement pour être d'une si grande
origine, quoique ses pères en eussent toujours porté le nom. La fortune
du sien était médiocre, et lui crut en avoir fait une que d'être parvenu
à une des quatre places de majordome de Philippe V, tôt après son
arrivée en Espagne. C'est ce qui me fait différer à parler de cette
maison sous un autre titre. C'était un grand homme parfaitement bien
fait, blond, chose très rare dans un Espagnol, d'un visage agréable,
l'air noble et naturel, l'abord gracieux, poli et attentif pour tout le
monde, doux et néanmoins ferme et nullement ployant. Il fut tel toute sa
vie sans que la faveur y ait jamais rien altéré. Il était adroit en
toutes sortes d'exercices, grand toréador et fort brave. Il s'était fort
distingué à la suite du roi dans ses armées en Italie et en Espagne\,;
le roi prit du goût pour lui fort peu après qu'il fut majordome, et lui
d'un grand attachement pour le roi\,; cette amitié réciproque parut
bientôt en tout et n'a jamais souffert la moindre éclipse, tellement que
tout \emph{in minoribus} qu'il était encore, jamais le cardinal Albéroni
n'a pu ni le gagner ni l'entamer. Le roi le fit son premier écuyer, et
il était dans cette charge lors de deux actions qu'il fit qui
redoublèrent extrêmement l'estime et l'amitié du roi pour lui. La
première fut à une chasse où le roi blessa un sanglier qui vint sur lui
et qui l'eût tué, si dans l'instant don Alonzo Manrique ne se fût jeté
entre-deux et dessus, et ne l'eût tué. La seconde fut encore à une
chasse où le roi et la reine sa première femme étaient à cheval. Ils se
mirent à galoper ; la reine tomba le pied pris dans son étrier qui
l'entraînait. Don Alonzo eut l'adresse et la légèreté de se jeter à bas
de son cheval et de courir assez vite pour dégager le pied de la reine.
Aussitôt après il remonta à cheval et s'enfuit à toutes jambes jusqu'au
premier couvent qu'il put trouver. C'est qu'en Espagne toucher au pied
de la reine est un crime digne de mort. On peut juger que là rémission
lui fut bientôt accordée, avec de grands applaudissements.

Sa faveur croissant toujours, le roi fit en sorte que le duc de La
Mirandole voulut bien se démettre de la charge de grand écuyer qu'il
avait, dont les honneurs et les appointements lui furent conservés, et
la donna à don Alonzo Manrique, qu'il lit en même temps duc del Arco et
grand d'Espagne. Il était noble en toutes ses manières, et magnifique et
libéral en tout, avec cela extrêmement simple et modeste, et d'un esprit
sage, mais médiocre, et beaucoup d'équité et de ménagement. Il avait
l'air si parfaitement et si naturellement Français, qu'il aurait passé
dans Paris pour l'être, et que j'en fus surpris extrêmement. Avec sa
faveur, il ne se voulut jamais mêler de rien, ne demanda jamais rien
pour lui, et passa même toute circonspection dans son extrême retenue à
demander pour les autres. Par sa charge, il avait celle de toutes les
chasses, où il suivait toujours le roi, et était très charitable et très
judicieux à l'égard de ces milliers de paysans employés sans cesse aux
battues, dont je parlerai en leur lieu, et c'était encore lui qui, comme
grand écuyer, ouvrait et fermait la portière du carrosse du roi. De tous
les gentilshommes de la chambre, lui et le marquis de Santa Cruz étaient
seuls toute l'année en exercice\,; ainsi il fallait habiller et
déshabiller le roi tous les jours, et l'hiver porter une bougie dans un
flambeau devant lui, depuis son carrosse jusqu'à son cabinet. Tant de
fonctions et de détails de charges l'obligeaient à une incroyable
assiduité, qui m'empêcha de pouvoir être en commerce avec lui autant que
lui et moi l'aurions souhaité. Il portait derrière sa médaille de
chevalier de Saint-Jacques, un petit portrait du roi en miniature, qui
était très ressemblant. Il se retira avec lui à Saint-Ildefonse à son
abdication, et revint avec lui à la mort du roi Louis. Il eut la Toison
et le Saint-Esprit, et mourut longues années après, presque aveugle,
sans enfants\,; son frère, assez obscur, hérita de sa grandesse.

6 Arcos, \emph{Ponce de Léon}. Jacques II, roi de Castille, avait donné
le comté de Medellin à Pierre Ponce de Léon en récompense de ses
services contre les Mores. Il était lors cinquième seigneur de Marchea,
et le lui retira en 1440 en lui donnant en titre de comté Arcos en
Andalousie. Cette maison prétend sortir des anciens comtes de Toulouse.
Rodrigue, troisième comte d'Arcos, petit-fils du premier par mâles, fut
fait en 1484 marquis d'Arcos et duc de Cadix par les rois catholiques.
Faute de mâles, sa fille porta Arcos, etc., en mariage au petit-fils par
mâles de son grand-oncle paternel. Les rois catholiques lui retirèrent
Cadix, et en échange le firent en 1498 duc d'Arcos et lui donnèrent
d'autres terres. Celui que j'ai vu fort familièrement à Madrid était le
septième duc d'Arcos de mâle en mâle, fils de l'héritière d'Aveiro si
comptée en Espagne, dont il est parlé t. III, p.~95-196, et le même dont
il est parlé t. III, p.~224, à propos du voyage forcé qu'il fit en
France et en Flandre avec le comte de Baños, son frère. Ce duc d'Arcos
était un homme d'une belle et noble représentation, sa femme aussi, très
riches et très magnifiques, ayant un très beau et grand palais, des
meubles admirables, et fort aumôniers et gens de bien, fort considérés à
Madrid, fréquentant peu la cour et se plaisant en leurs haras et à la
plus superbe écurie d'Espagne, en nombre et en beauté de chevaux\,: tous
deux très polis, beaucoup d'esprit et de grandeur\,; et le duc d'Arcos
fort instruit et du goût pour les livres\,; tous deux parlant bien
français et de fort agréable conversation et même libre avec moi.

Aremberg, \emph{Ligne}. Était en Flandre attaché à la cour de Vienne.

Arion, \emph{Sotomayor y Zuniga}. Je parlerai de cette maison sous le
titre de Bejar. Ce duc d'Arion était oncle paternel du duc de Bejar,
quoique de peu plus âgé que lui. Il portait le nom de marquis de Valero,
et il était un des quatre majordomes du roi quand Philippe V arriva en
Espagne, qui prit pour lui un goût et une estime qui a toujours duré\,;
il était vice-roi du Mexique lorsque j'étais en Espagne, où il était en
vénération\,; c'est lui que le roi d'Espagne, bien qu'absent, fit
majordome-major de la princesse des Asturies, puis duc d'Arion et grand
en arrivant en Espagne peu après que j'en fus parti.

Atri, \emph{Acquaviva}. Napolitain, frère du cardinal Acquaviva et neveu
d'un autre cardinal Acquaviva\,; il était capitaine des gardes du corps
de la compagnie italienne, et en Italie lorsque j'étais en Espagne.

Atrisco, \emph{Sarmiento}.

Baños, \emph{Ponce de Léon}, frère du duc d'Arcos. Il s'était retiré et
établi en Portugal dans les biens d'Aveiro, de sa mère, lorsque j'étais
en Espagne.

Béjar, \emph{Sotomayor y Zuniga}. Les rois catholiques érigèrent cette
terre, qui est en Estramadure, en 1488, pour Alvar de Zuniga, second
comte de Placencia, et dès 1460 fait duc d'Arevalo par les rois
catholiques qui peu après mirent ce titre sur Placencia et enfin sur
Bejar, et réunirent à leur couronne Arevalo et Placencia. La nièce du
second duc de Bejar en hérita et porta Bejar en mariage en 1533 à Fr.~de
Sotomayor, cinquième comte de Belalcazar, dont le fils, qui joignit à
son nom celui de Zuniga, fut quatrième duc de Bejar. Cette maison de
Sotomayor, dans laquelle cette grandesse s'est depuis continuée de mâle
en mâle, descend masculinement de Gutiere de Sotomayor, grand maître de
l'ordre d'Alcantara, mort en 1456, dont le fils aîné Alphonse fut créé
comte de Belalcazar par Henri IV roi de Castille. Le douzième duc de
Bejar est celui que j'ai connu familièrement en Espagne. C'était un
homme d'esprit, sage, timide, qui désirait fort quelque utile
réformation dans le gouvernement, et qui m'en entretint particulièrement
plusieurs fois avec le comte de Priego en tiers, son ami intime, par qui
il m'avait fait demander ses conversations, et qui, me voyant si bien
avec Leurs Majestés catholiques et avec le marquis de Grimaldo,
désiraient ardemment que je m'y employasse, ce que je ne jugeai point du
tout à propos, quoique au fond je pensasse comme eux, ce que je ne leur
désavouai pas, ainsi que l'impossibilité radicale du remède. Ce duc de
Bejar était fort honnête homme, instruit et fort pieux\,; il avait eu
dès l'âge de six ans, chose unique, la Toison de son père, tué, en 1686,
volontaire au siège de Bude. L'empereur s'intéressa fort pour cette
grâce si singulière. Longtemps depuis mon retour il maria son fils aîné
à une fille de prince de Pons-Lorraine, qui fut dame du palais de la
reine, et quelques années après il fut majordome-major du prince des
Asturies, gendre du roi de Portugal.

Berwick, \emph{Fitzjames}. Bâtard de Jacques II, roi d'Angleterre, étant
duc d'York et de la soeur du fameux duc de Marlborough, duc et pair de
France et d'Angleterre, maréchal de France, général des armées de France
et d'Espagne, chevalier des ordres de la Jarretière, de la Toison d'or
et du Saint-Esprit, gouverneur de Limousin, tué devant Philippsbourg
dont il faisait le siège en {[}1734{]}. Je remets au titre de Liria à
parler de cette grandesse.

Bournonville, \emph{idem}. Cette maison est originaire du Boulonnais où
est la terre de Bournonville dont elle tire son nom, et connue dès
1070\,; longtemps Français, puis transplantés en Flandre. Il s'agit ici
de Michel-Joseph de Bournonville, qui a longtemps porté le nom de baron
de Capres. Son père, frère cadet du père de la première maréchale de
Noailles, mourut en 1718 gouverneur d'Oudenarde et lieutenant général
des armées de Philippe V, et sa mère était Noircarmes-Sainte-Aldegonde,
seconde femme de son mari. Le baron de Capres monta par les degrés en
Flandre au service d'Espagne\,; il fit si bien sa cour aux maîtresses de
l'électeur de Bavière qu'avec fort peu de réputation dans le monde et de
pas plus à la guerre, il devint lieutenant général et chevalier de la
Toison d'or, qu'il reçut en 1710 des mains de l'électeur à Compiègne.
N'ayant plus rien à gagner avec lui, il passa en Espagne, où il
s'attacha servilement à la princesse des Ursins\,; qui, comme on l'a vu
ailleurs, l'envoya de sa part à elle à Utrecht pour cette souveraineté
qu'elle voulait qu'on lui établît, et qui accrocha si étrangement la
paix d'Espagne. Bournonville ne put être admis à Utrecht, y fut méprisé
comme il le fut aussi en France et en Espagne de s'être chargé d'une si
vile commission. Mais avec un esprit médiocre, il l'avait très souple, à
qui les bassesses, quelles qu'elles fussent, ne coûtaient rien, et qui
l'avait tout tourné aux intrigues et à la fortune avec force langages et
beaucoup de désinvolte et de grand monde. Ce bel emploi lui dévoua
entièrement la princesse des Ursins, qui le mit si bien auprès du roi
d'Espagne que, même après sa chute à elle, il fut fait, en 1715, grand
d'Espagne et bientôt après capitaine des gardes du corps de la compagnie
wallonne\,; il prit le nom de duc de Bournonville et eut encore la clef
de gentilhomme de la chambre, mais pas un d'eux n'en avait aucune sorte
de fonction que le duc del Arco et le marquis de Santa Cruz.

J'en reçus à Madrid toutes les avances et toutes les caresses
imaginables. Il voulait aller ambassadeur en France, où résolument on
n'en voulait point, dont il se doutait bien. C'était donc pour lever cet
obstacle qu'il me courtisait. J'avais ordres de l'y barrer sous main,
même à découvert de la part du roi s'il était nécessaire. C'était un
éclat que je voulus éviter, qui me coûta un vrai tourment les derniers
mois que je passai en Espagne, parce qu'ils se passèrent en importunités
journalières là-dessus de sa part, et en efforts de la mienne, pour lui
en faire perdre la pensée, jusqu'à la veille de mon départ qu'il
m'obséda deux heures le soir dans la cour du Retiro, pour me persuader
de l'intérêt qu'on avait en France de l'y avoir ambassadeur, et me
conjurer de le persuader à M. le duc d'Orléans et au cardinal Dubois.
S'il ne réussit pas dans ce dessein, il obtint en 1726 l'ambassade de
Vienne, dont il n'eut pas lieu d'être content\,; mais, accoutumé à
savoir se reployer, il ne laissa pas d'être nommé, l'année suivante,
premier plénipotentiaire au congrès de Soissons, où il ne se fit que des
révérences et des repas, d'où il retourna en Espagne, peu content de
Paris et de notre cour, malgré la protection des Noailles auxquels il
était fort homogène, excepté à sa cousine la maréchale, à qui il ne
ressemblait point, car il était faux au dernier point, et le sentait
fort loin, et d'une avarice extrême.

Il avait un frère aîné sans fortune dont il prit le fils auprès de lui.
Il n'était point marié, et son dessein était de lui faire tomber sa
grandesse et sa charge. Il était fort parmi le monde pendant que j'étais
à Madrid, et en même temps peu désiré, peu estimé et peu compté.

Doria, \emph{idem}, à Gênes, dont il est d'une des quatre premières
maisons.

Estrées, \emph{idem}, Français, à Paris. On a vu en son lieu comment il
fut fait grand\footnote{Éteint. (\emph{Note de Saint-Simon}.)}.

Frias, \emph{Velasco}, en Castille, près de Burgos. Les rois catholiques
l'érigèrent en duché pour Bernardin Fernandez de Velasco, troisième
comte de Haro, et connétable de Castille, après son père, office
personnel jusqu'à ce second connétable, qui le rendit héréditaire,
tellement qu'ils ont été bien plus connus sous le seul nom de
connétables de Castille, que sous celui de ducs de Frias, grandesse qui,
pour être toute masculine, n'est jamais sortie de la maison de Velasco.
Cette illustre maison, qui a fait plusieurs branches, vient toute de J.
de Velasco, rico-hombre et seigneur de Bibriesca et de Pomar avant 1400.
Les offices de connétable et d'amirante avaient anciennement des rangs,
des droits et des fonctions dans les divers royaumes dont ils l'étaient,
qui composent celui d'Espagne\,; mais devenus depuis longtemps
héréditaires, par conséquent abusifs, tout ce qui y était attaché
s'était tellement perdu qu'il n'en Restait plus que le titre, qui
n'était que pour les oreilles, et ne donnait plus quoi que ce soit.
Cette inutilité, l'insolence et la perfidie de l'amirauté, et l'enfance
du connétable engagèrent Philippe V, il y a quelques années, à en
supprimer même les titres pour toujours par un diplôme exprès et sans
dédommagement, parce que ce n'était qu'un titre vain et vide de tout. Je
n'ai point vu le dernier de ces connétables, parce que son jeune âge
l'empêchait de paraître dans le monde. Il était fort riche et fort grand
seigneur, le dixième duc de Frias.

Gandie, \emph{Llançol} dit \emph{Borgia}, au royaume de Valence, près de
la mer. Alphonse Borgia, fait cardinal, 1445, par Eugène IV, succéda,
1455, à Nicolas V, prit le nom de Calixte III, et mourut 1458. Sa soeur
avait épousé Geoffroy Llançol, d'une ancienne maison du royaume de
Valence, aux enfants duquel le pape Calixte III fit prendre le nom et
les armes de Borgia, dont il ne restait plus de mâles. Geoffroy Llançol
eut de la soeur du pape deux fils et trois filles P. L. Borgia, préfet
de Rome, et Rodriguez Borgia, qui fut pape sous le nom d'Alexandre VI,
lequel, étant cardinal, avait eu de Venosa, femme de Dominique Arimano,
Romain, quatre fils et une fille, qui épousa successivement Jean Sforze,
seigneur de Pesaro, Alphonse d'Aragon, duc de Bisceglia, et Alphonse
d'Este, duc de Ferrare. Les fils furent Pierre-Louis Borgia, fait par
Ferdinand le Catholique duc de Gandie en 1485, qui mourut accordé avec
M. Enriquez, fille de l'amiral de Sicile\,; César Borgia, d'abord
cardinal qu'il ne demeura pas, et qui devint célèbre par ses crimes,
sous le nom de duc de Valentinois\,; Jean Borgia, qui succéda au duché
de Gandie de son frère aîné, et qui épousa M. Enriquez, qui lui avait
été destinée\,; enfin Godefroy Borgia, prince d'Esquillace\footnote{Nous
  avons suivi l'orthographe de Saint-Simon\,; mais le nom de cette ville
  du royaume de Naples est Squillace (Calabre ultérieure).}, marié à une
bâtarde d'Alphonse, roi d'Aragon, et dont la branche qui a duré
longtemps s'est éteinte. César Borgia fit tuer Jean Borgia dans Rome, et
jeter son corps dans le Tibre\,; mais il laissa un fils et une fille. Ce
fils fut Jean II Borgia, duc de Gandie, qui de J., fille d'Alphonse,
bâtard de Ferdinand, roi d'Aragon, laissa François Borgia, duc de
Gandie, qui, après avoir perdu sa femme, F. de Castro, se fit jésuite,
dont il fut bientôt après général\,: c'est le célèbre saint François de
Borgia, mort 1572, et canonisé cent ans après. Il laissa une grande
postérité qui se divisa en plusieurs branches, desquelles l'aînée a
toujours masculinement conservé le duché et le titre de duc de Candie.
C'est le treizième duc de Gandie, que j'ai vu en Espagne, jeune, sans
monde ni esprit, obscur et embarrassé de tout, que toutefois la
considération de son nom, du duc d'Hijar son beau-père, du cardinal
Borgia son oncle, fit l'un des deux gentilshommes de la chambre du
prince des Asturies à son mariage.

Giovenazzo \emph{del Giudice}, Génois transplanté à Naples. C'était le
prince de Cellamare, ambassadeur en France, qui ourdit avec le duc et la
duchesse du Maine la conspiration dont il a été parlé, et tant de lui à
cette occasion qu'il n'en reste rien à ajouter ici, non plus que sur le
cardinal del Giudice, son oncle paternel, dont il a été beaucoup parlé
ici, tant à l'occasion de son voyage à Paris qu'à celle de son expulsion
d'Espagne par le cardinal Albéroni. Son frère, le vieux duc de
Giovenazzo, qui avait encore plus d'esprit et d'intrigue que lui, et
bien plus de sens, alla s'établir en Espagne, où il parvint à une grande
considération. Charles II le fit grand, mais seulement pour trois races,
et enfin conseiller d'État. Son fils Cellamare, qui, étant encore
ambassadeur à Paris, prit à sa mort le nom de duc de Giovenazzo, avait
épousé à Rome une Borghèse, veuve du duc de La Mirandole, et mère du duc
de La Mirandole que je trouvai établi en Espagne. Cellamare en avait une
fille unique, demeurée à Rome dans un couvent, qui avait cette troisième
race de grandesse et de grands biens à porter au mari qui l'épouserait.
On la disait étrangement laide. Je ne sais ce qu'elle est devenue.

Longues années après mon retour, la cour d'Espagne lit un long voyage à
Cadix, Séville, Grenade, etc., et don Joseph Patino était lors premier
ministre et chef des finances en particulier. Cellamare, je l'appelle
toujours ainsi, y était comme grand écuyer de la reine, charge qu'il
avait dès le temps qu'il était à Paris. Patiño avait le défaut d'être
également infatigable en promesses réitérées et en inexécutions, même de
choses à faire sur-le-champ, surtout quand il s'agissait d'argent. Il y
avait longtemps qu'il menait Cellamare de la sorte sur le payement de
l'écurie de la reine, livrée, fourrages et réparations de voitures, dont
Cellamare était outré, n'osant trop pousser un premier ministre dans le
plus haut crédit et la puissance la plus vaste et la plus absolue. La
chose traîna ainsi jusqu'au départ de la cour pour revenir à Madrid,
toujours en promesses, et la plupart d'être payé sur-le-champ, sans
jamais d'exécution la plus légère. Le matin du départ, Cellamare fut
chez Patino lui représenter l'état de l'écurie de la reine, etc.\,; il
en eut peu de satisfaction, il se fâcha, en vint aux grosses paroles, et
entra dans une telle colère qu'il eut peine à regagner son logis, où il
se trouva si mal qu'il en mourut le jour même à près de quatre-vingts
ans.

Gravinades \emph{Ursins}, à Naples et à Rome. C'est à présent l'aîné de
cette grande maison, si tant est qu'il en reste d'autres branches.
M\textsuperscript{me} des Ursins fit donner la grandesse au duc de
Gravina, neveu du pape Benoît XIII.

Havré, \emph{Croï}, en Flandre. Philippe III l'érigea en duché pour
Charles-Alexandre de Croï, de la branche d'Arschot, qui, de gentilhomme
de la chambre de l'archiduc Albert, et conseiller au conseil de guerre à
Bruxelles, prit le nom de duc de Croï après la mort de Charles duc de
Croï, son cousin et son beau-frère. Philippe III le fit conseiller
d'État, surintendant des finances des Pays-Bas, chevalier de la Toison
d'or et grand d'Espagne. Il fut tué dans sa maison à Bruxelles, à
cinquante ans, en 1624, d'un coup de mousquet qui lui fut tiré par une
fenêtre. Il avait épousé, en 1599, Yolande, fille de Lamoral, prince de
Ligne, dont il n'eut qu'une fille unique, qui porta sa grandesse et tous
ses biens en mariage à Pierre-François, second fils de Philippe de Croï,
comte de Solre, qui prit par elle le nom de duc d'Havré. Il fut
chevalier de la Toison d'or, gouverneur de Luxembourg et du comté de
Chiny, et chef des finances des Pays-Bas, mort à Bruxelles en 1650. Son
fils unique, Ferdinand François-Joseph de Croï, duc d'Havré, fut
chevalier de la Toison d'or, et mourut à Bruxelles en 1694. Il avait
épousé, en 1668, l'héritière d'Halluyn dans le château de Wailly près
d'Amiens, dont il eut Charles-Joseph, duc d'Havré\,; tué sans alliance à
la bataille de Saragosse, 10 septembre 1710, lieutenant général et
colonel du régiment des gardes wallonnes, et Jean-Baptiste-Joseph, duc
d'Havré et colonel du régiment des gardes wallonnes après son frère. La
princesse des Ursins lui fit épouser la fille de sa soeur et d'Antoine
Lanti, dit della Rovere, seigneur romain, à qui sa belle-soeur procura
l'ordre du Saint-Esprit en 1669. La chute de la princesse des Ursins
attira des dégoûts au duc et à la duchesse d'Havré qui était dame du
palais de la reine. Le duc d'Havré quitta l'Espagne et se retira en
France avec sa femme, et mourut à Paris en 1627. Il laissa deux fils,
dont l'aîné, duc d'Havré, grand d'Espagne, s'est fixé au service de
France où il est lieutenant général, et a épousé une fille du maréchal
de Montmorency, dernier fils du maréchal duc de Luxembourg. Le cadet
s'est marié en Espagne à la fille héritière du frère de sa mère qui,
comme on le verra ci-après, le fera grand d'Espagne.

Hijar, \emph{Silva}, ancienne baronnie en Aragon, puis duché, a passé
d'héritière en héritière en différentes maisons, et enfin en celle de
Silva, où elle ne fut que sur une seule tête par son mariage, dont une
seule fille héritière, qui porta ses biens et cette grandesse à Rodrigue
de Silva y Sarmiento et Villandrado, comte de Salinas et Ribadaneo,
second marquis d'Alenquer, mort au château de Léon, prisonnier d'État,
ayant trempé dans la conjuration de Charles Padille contre Philippe IV.
Son fils aîné, duc d'Hijar, eut des fils qui n'eurent point d'enfants,
et laissèrent leur soeur héritière, qui porta ses biens et cette
grandesse en mariage, décembre 1688, à son cousin paternel Frédéric de
Silva y Portugal, marquis d'Orani, dont le petit-fils, par mâles, est le
huitième duc d'Hijar, que j'ai vu en Espagne, qui fréquentait peu la
cour et le monde, mais qui avait de la considération. Je l'ai fort peu
vu et point du tout fréquenté.

L'Infantado\footnote{Saint-Simon écrit \emph{l'Infantade} et
  \emph{l'infantado} indistinctement.}, \emph{Silva}. Cette maison,
cette grandesse et le duc del Infantado, du temps de mon ambassade en
Espagne, sont traités ci-devant, en sorte qu'il n'en reste rien à
expliquer ici, sinon que l'érection en est des rois catholiques en 1475,
sous le nom de l'Infantado, et d'héritage en héritage tomba enfin vers
1657 dans la maison de Silva, au cinquième duc de Pastrane.

Pastrane, terre en Castille, vendue avec d'autres, en 1572, par Gaspard
Gaston de La Cerda et Mendoza, à Ruy Gomez de Silva, prince d'Eboli,
qu'il fit peu après ériger en duché et grandesse pour lui par Philippe
II, qui l'avait fait grand d'Espagne et duc d'Estremera dès 1568\,; et
le nouveau duc de Pastrane en préféra le titre à celui de duc
d'Estremera qu'il quitta. Il eut plusieurs enfants d'Anne Mendoza y La
Cerda son épouse, favorite si déclarée de Philippe II, dont descendent,
outre les ducs de Pastrane, les comtes de Salinas, les ducs d'Hijar et
les marquis d'Orani d'Elisede et d'Aguilar. L'aîné, Roderic de Silva y
Mendoza, fut second duc de Pastrane et troisième prince d'Eboli, et
grand-père d'autre Roderic de Silva, cinquième duc de Pastrane, qui
devint duc de l'Infantado et de Lerma par sa femme, soeur et héritière
de Roderic Diaz de Vivar Hurtado de Mendoza et Sandoval, septième duc
del Infantado, mort sans enfants en janvier 1657, et de Diego Gomez de
Sandoval, mort aussi sans enfants, juillet 1668. Le duc del Infantado,
du temps que j'étais en Espagne, est petit-fils du duc de Pastrane,
devenu, comme il vient d'être expliqué, duc de l'Infantado, dont les
Silva, depuis cette époque, ont préféré le titre à celui de duc de
Pastrane.

Il résulte de ce détail que la date de la grandesse del Infantado doit
être prise de la première qu'il ait eue, qui est celle de 1568 de duc
d'Estremera qui, sous Charles-Quint, a passé de l'état de rico-hombre à
celui de grand d'Espagne.

Licera \emph{y Aragon}.

Linarès, \emph{Alencastro}. (Voir t. III, p.~97.) À quoi rien ici à
ajouter, sinon que, la grandesse étant tombée à l'évêque de Cuença, qui
en prit le titre et cessa de porter le nom d'évêque de Cuença, je le
laissai en partant d'Espagne sans avoir fait sa couverture, parce qu'il
voulait la faire avec son bonnet, et que les grands s'y opposaient et
voulaient qu'il se couvrit avec son chapeau. Cette contestation, qui
durait, depuis longtemps, retenait ce prélat à la cour, qui n'en était
pas fâché, et qui n'était pas sans ambition ni sans esprit. Il était,
comme on l'a vu au renvoi, de la maison d'Alencastro.

Liñarès, en Portugal, érigé en comté par le roi Emmanuel de Portugal
pour Antoine de Noroña, fils puîné de Pierre de Noroña y Menesez, issu
de la maison royale de Castille. Une fille héritière épousa un autre
Noroña, dont le fils fut fait duc de Liñarès par Jean IV, roi de
Portugal. Son fils fut fait grand d'Espagne par Charles II, et grand
écuyer de la reine sa seconde femme, et mourut à sa suite à Tolède en
1703. Ses deux fils moururent sans postérité, et sa \emph{fille} aînée
portale duché et grandesse de Liñarès en mariage au second duc
d'Abrantès.

Liria, fils unique du premier lit du duc de \emph{Berwick} ci-dessus,
qui, après avoir fait tout jeune ses premières armes en Hongrie,
retourna en Angleterre sur le point de la révolution, et passa en France
avec Jacques II, dont il était fils naturel. Il y servit d'abord
volontaire, et tôt après lieutenant général tout d'un coup\,; il eut
bientôt des commandements en chef. Il a tant été parlé de lui dans ces
Mémoires, et de l'occasion glorieuse qui lui acquit la grandesse et la
Toison à lui et à son fils, qu'il n'est besoin de s'arrêter que sur la
singularité de sa grandesse, sur quoi il faut reprendre les choses de
plus haut. Il avait été marié deux fois, et n'avait de son premier lit
qu'un fils unique et plusieurs du second. Il s'était si parfaitement
flatté d'obtenir son rétablissement en Angleterre que, lorsqu'il fut
fait duc et pair de France, il obtint une chose inouïe dans ses lettres,
qui fut l'exclusion de son fils aîné, parce qu'il le destinait à
succéder en Angleterre à ses dignités et à ses biens\,; mais lorsqu'il
fut fait grand d'Espagne, il s'était enfin désabusé de cette trop longue
espérance, et voulut établir tout à fait en Espagne ce fils aîné.
Philippe V, en le faisant grand d'Espagne, lui avait donné en même temps
les duchés de Liria et de Quirica, dans le royaume de Valence, qui
avaient été des apanages des infants d'Aragon. Le duc de Berwick obtint
de les pouvoir donner actuellement à son fils aîné, et qu'il jouit en
même temps de la grandesse conjointement avec lui, ce qui était
jusqu'alors sans exemple. Son fils aîné prit donc alors le nom de duc de
Liria, fit sa couverture, reçut l'ordre de la Toison d'or, et bientôt
après épousa la soeur unique du duc de Veragua qui, par l'événement,
devint héritière de très grands biens. C'était une femme très bien
faite, l'air fort noble et les manières, avec de l'esprit, du sens et de
la piété, et fort estimée et considérée. On a vu qu'elle fut dame du
palais de la princesse des Asturies à son mariage.

Le duc de Liria était lieutenant général, et fut gentilhomme de la
chambre du roi d'Espagne très peu avant que j'y arrivasse. On a vu toute
l'amitié et les services que j'en reçus. Il avait par deux fois couru
grand risque en Écosse et en Angleterre. Il avait de l'esprit, beaucoup
d'honneur et de valeur, et une grande mais sage ambition, était aimé,
estimé et compté en Espagne, et le fut partout où il alla. Sa
conversation était très agréable et gaie, instructive quand on le
mettait sur ce qu'il avait vu et très bien vu en pays divers et en
affaires, très bien avec tout ce qu'il y avait de meilleur en Espagne,
ami le plus intime de Grimaldo qu'il n'avait point abandonné dans sa
disgrâce du temps d'Albéroni, et Grimaldo ne l'avait jamais oublié\,;
quoiqu'il eût beaucoup de dignité, il ne laissait pas d'être souple avec
mesure et justesse, et fort propre à la cour qu'il connaissait
extrêmement bien. Il avait un talent si particulier pour les langues,
qu'il parlait latin, français, espagnol, italien, anglais, écossais,
irlandais, allemand et russien comme un naturel du pays, sans jamais la
moindre confusion de langues. Avec cela il aimait passionnément le
plaisir\,; et la vie compassée, uniforme, languissante, triste de
l'Espagne lui était insupportable. Il était fait pour la société libre,
variée, agréable, et c'était ce qu'on n'y trouvait pas.

Quelque temps après mon départ, il obtint l'ambassade de Russie, avec
une commission à exécuter à Vienne. Il réussit en l'une et en l'autre,
tellement que la tzarine, sans l'en avertir, lui jeta un jour le collier
de son ordre au cou. Il repassa à Paris, où il se dédommagea tant qu'il
put de l'ennui de l'Espagne, et où nous nous revîmes avec grand plaisir.
Il me voulut même bien donner quelques morceaux fort curieux qu'il avait
faits sur l'état de la cour et du gouvernement de Russie. Il demeura à
Paris tant qu'il put, et bien moins qu'il n'eût voulu, et pour éloigner
son retour en Espagne, il obtint permission d'aller voir le roi
d'Angleterre à Rome\,; de là il alla à Naples, où il fit si bien, qu'il
demeura si longtemps que, s'y abandonnant aux plaisirs de la société, et
peu à peu à l'amour d'une grande dame, il en mourut de phtisie, laissant
plusieurs enfants. C'est un homme que je regretterai toujours. Son fils
aîné a recueilli sa grandesse, est grandement établi, mais ne lui
ressemble pas.

Medina-Coeli, \emph{Figuerroa y La Cerda. La} grandeur de l'origine de
cette grandesse, et la singularité de sa première continuation,
m'engagent à m'y étendre. Alphonse X, roi de Castille, dit l'Astrologue,
de son goût pour l'étude, et en particulier pour les mathématiques et
l'astronomie, et des fameuses tables dites Alphonsines de son nom qu'il
fit dresser sous ses yeux, eut deux fils d'Yolande, infante d'Aragon,
son épouse\,: Ferdinand l'aîné fut gendre de saint Louis\,; et Sanche
dit le Brave. Ferdinand donna des preuves de son courage contre les
Mores, et mourut à vingt et un ans, en 1275, neuf ans avant son père, et
laissa deux fils, Alphonse et Ferdinand, qui, je n'ai pu savoir
pourquoi, prirent dans la suite le nom de La Corda. Sanche, fils cadet
de l'Astrologue, voyant les deux fils de son aîné si fort en bas âge, et
le roi son père si enterré dans ses études qu'il ne put jamais se
résoudre d'aller en Allemagne où il avait ôté élu unanimement empereur,
le méprisa, et conçut le dessein de régner. Les instances persévérantes
des princes d'Allemagne, ni les exhortations du pape, n'ayant pu
l'ébranler pendant plusieurs années, quoiqu'il eût accepté l'empire,
pris le nom d'empereur, souvent promis de passer en Allemagne, les
princes de l'Empire, rebutés de tant de remises, se tournèrent du côté
du roi d'Angleterre, qui eut plus de volonté, mais non plus de succès,
ce qui engagea les Allemands à renoncer à l'un et à l'autre, et à élire
Rodolphe, comte d'Hapsbourg, chef fameux de la maison d'Autriche.

Sanche, ravi du mépris, où l'attachement à l'étude et la privation de
l'Empire qui en fut l'effet avait précipité son père, profita de cette
passion d'étude pour lui persuader de se décharger sur lui de tous les
soins du gouvernement, qui le détournaient de ses occupations les plus
chères. Parvenu à régner sous son nom et {[}à{]} s'être acquis toute la
Castille par sa valeur et sa manière de gouverner', il songea à faire
déshériter ses neveux, et à se faire associer par son père, et couronner
roi de son vivant, car jusqu'à la réunion des divers royaumes qui
composent l'Espagne, c'est-à-dire jusqu'aux rois catholiques
inclusivement, tous ces différents rois se faisaient couronner. Le père
y consentit, et presque tout le royaume\,; au moins on n'osa y branler.
Ce ne fut pas tout, Sanche trouva que son père demeurait trop longtemps
avec lui sur le trône\,; il résolut de l'en précipiter, il en vint à
bout. Le malheureux père, réduit à ses livres, ne put s'en consoler avec
eux. Il implora l'assistance de toute l'Europe contre un fils si
dénaturé, qui ne lui en procura aucune. Alors réduit au désespoir, il
donna sa malédiction à son fils, le déshérita et sa race autant qu'il
fut en lui, rappela ses petits-fils aînés à leurs droits, et à défaut de
leur race, appela à sa couronne celle de saint Louis. Il mourut dans ce
désespoir, et Sanche sut bien empêcher l'effet des dernières volontés de
son père. Ce prince et Jacques Ier, roi d'Angleterre, montrent ce que
sont des cuistres couronnés. Des deux malheureux neveux, Alphonse de La
Cerda fit la branche dite de Lunel, et Ferdinand fit celle dite de Lara,
de la femme que chacun des deux épousa. Cette branche s'éteignit dans le
petit-fils de Ferdinand, qui n'eut qu'un fils mort au berceau, et des
filles mariées, qui furent emprisonnées et empoisonnées par l'ordre de
Pierre le Cruel, roi de Castille, en 1361. Ainsi je ne parlerai point de
cette branche.

Alphonse de La Cerda n'oublia rien pour recouvrer le royaume qui lui
appartenait, et dont il prit le nom de roi de Castille, que Sanche, son
oncle, avait usurpé. Ses efforts furent inutiles\,; il fut réduit à se
retirer en France, où Charles le Bel le fit son lieutenant général en
Languedoc. Il épousa Mahaud, dame de Lunel, dont il eut un seul fils
connu sous le nom de prince des Îles Fortunées, d'où sont sortis les
Medina-Coeli. Il se remaria à Isabeau, dame d'Antoing et d'Espinoy,
veuve d'Henri de Louvain, seigneur de Gaësbeck, qui épousa en troisièmes
noces J. Ier de Melun, vicomte de Gand. De son second mariage Alphonse
de La Cerda eut Charles, dit de Castille ou d'Espagne, connétable de
France, qui figura dignement et grandement, et qui fut empoisonné à
Laigle en Normandie, où il mourut, par ordre de Charles le Mauvais, roi
de Navarre. Ce connétable ne laissa point d'enfants de Marguerite de
Châtillon-Blois. Il eut deux frères sans établissements ni alliances,
dont un fut archidiacre de Paris, et une soeur mariée en Espagne, à Ruys
de Villalobos. Ainsi finit promptement cette branche du connétable.
Revenons maintenant à son frère aîné, Louis d'Espagne, prince des îles
Fortunées, duquel sont sortis les Medina-Coeli.

Ce Louis de La Cerda eut le don du pape des îles Fortunées, dont il fut
couronné roi dans Avignon, par le même pape Clément VI, vers 1344. Ces
îles sont les Canaries, qu'il se résolut d'aller chercher sur l'exemple
de ceux de Gênes et de Venise sur le bruit de leur découverte\,; mais ce
fut un dessein qu'il ne put exécuter. Il fut amiral de France, comte de
Clermont et de Talmont\,; il épousa vers 1370 Léonor de Guzman, dame du
port Sainte-Marie, près Cadix, dont il ne laissa qu'une seule fille
héritière, appelée Isabelle de La Cerda, dame de Medina-Coeli et du port
Sainte-Marie, qui fut veuve sans enfants de Roderic Alvarez d'Asturie.
Voyons maintenant à qui elle se remaria.

Gaston Phoebus, comte de Foix, vicomte de Béarn et de Bigorre, dit
Phoebus pour sa beauté, dont la magnificence et la cour, la puissance et
l'autorité chez tous les princes de son temps sont si vantés dans
Froissart, fut toujours brouillé avec Agnès, fille puînée de Philippe
III, roi de Navarre, à la cour duquel elle passa presque toute sa vie,
et que ce comte de Foix avait épousée en 1348. Il n'en avait qu'un fils
unique qu'il avait marié avec Béatrix, fille de Jean comte d'Armagnac,
lequel passait toute sa vie tant qu'il pouvait auprès de sa mère et du
roi de Navarre son oncle. Étant venu voir son père à Orthez, qui
haïssait sa femme, et ne l'aimait guère lui-même, et ne pouvait souffrir
le roi de Navarre, son beau-frère, il en fut assez bien reçu. Au bout de
quelques jours le comte de Foix, au retour de la chasse, se mit à table
pour souper\,; son fils lui présenta la serviette pour laver. Dans cet
instant le soupçon et la colère surprirent si à coup le comte de Foix
que, croyant que son fils lui allait porter le coup de la mort en lui
donnant la serviette, il tira un poignard de son sein, dont il l'abattit
mort à ses pieds, en 1380\,; et c'était un jeune homme de très grande
espérance, très bien né et bien éloigné d'avoir jamais eu une si
horrible pensée. Le père, revenu à lui-même, fut au désespoir, et ne put
s'en consoler, et en mourut enfin de douleur, qui lui causa l'apoplexie
qui l'étouffa dans l'instant qu'il se lavait les mains en se mettant à
table à Orthez pour souper, en 1391, à quatre-vingts ans, de même façon
qu'il avait tué son fils. Ce fils n'avoir point eu d'enfants, tellement
que Matthieu de Foix, vicomte de Castelbon, succéda à Gaston Phoebus au
comté de Foix. Plusieurs années auparavant, sa soeur unique, Isabelle,
avait épousé Archambaud de Grailly qui, par elle, succéda au comté de
Foix, etc., par la mort sans enfants de Matthieu comte de Foix, etc.,
frère de sa femme. Le duc de Foix fut fait duc par Louis XIV en 1663,
avec M\textsuperscript{me} de Senecey sa grand'mère, et la comtesse de
Fleix sa mère\footnote{Voy. t. Ier, le récit de la réception des ducs et
  pairs au parlement. Le duc de Foix faisait partie de cette promotion.}
, toutes deux dames d'honneur de la reine-mère, et mort il n'y a pas
fort longtemps sans enfants, a été le dernier de cette maison de Grailly
qui, par ce même héritage de Foix, eut celui de Navarre ensuite aussi
par héritage, en porta peu la couronne, qui tomba par une héritière dans
la maison d'Albret, et d'elle par la même voie dans la maison de
Bourbon, avec les comtés de Foix, Bigorre, Béarn, etc. Reprenons
présentement notre sujet.

César Phoebus, comte de Foix, n'avait d'enfants que le fils qu'il
poignarda\,; mais il laissa quatre bâtards dont les deux derniers n'ont
point paru dans le monde. Bernard, l'aîné des quatre, eut un bonheur
extrême, comme on le va voir. Yvain, le second des quatre, le favori du
père, brilla à la cour de Charles VI, fut de ce funeste bal où ce roi et
sa suite se masquèrent en sauvages, où le feu prit à leurs habits, dont
plusieurs moururent brûlés, dont Yvain fut un, sans avoir été marié. Ce
fut le 30 janvier 1392.

Bernard, bâtard de Gaston Phoebus, comte de Foix, et l'aîné des trois
autres bâtards, alla chercher fortune en Espagne dès 1367, y établit sa
demeure, s'y distingua par sa valeur au service du comte de Transtamare
contre Pierre le Cruel, roi de Castille, dont il était frère bâtard,
mais qu'il vainquit et tua, et fut roi de Castille en sa place sous le
nom d'Henri II. Bernard eut le bonheur de plaire, à Isabelle de La
Cerda, dame de Medina-Cœli et du port Sainte-Marie, fille et seule
héritière de Louis de La Cerda ou d'Espagne, prince des îles Fortunées,
etc., petit-fils de Ferdinand, fils aîné de Castille et de Blanche,
troisième fille de saint Louis, sur lesquels Sanche le Brave, après la
mort du même Ferdinand son frère aîné, avant le roi Alphonse
l'Astrologue, leur père, avait usurpé la couronne de Castille. Cet
heureux bâtard de Foix fit donc ce grand mariage si disproportionné de
lui, et fut fait comte de Medina-Coeli. Il prit en plein et en seul le
nom de La Cerda, et les armes au premier et quatrième partis\footnote{Le
  mot \emph{parti}, en terme de blason indiquait que l'écu était divisé
  de haut en bas en parties égales. Parmi ces compartiments, l'un
  contenait les armes de Castille, un autre les armes de Léon, et deux
  autres les armes de France.}, de Castille et de Léon, au second et
troisième de France, et tous ces quartiers sans brisure, ainsi qu'il
appartenait à ces malheureux princes déshérités, pères de cette royale
héritière. Les trois générations suivantes comtes de Medina-Coeli
figurèrent fort à la guerre et dans l'État et par leurs alliances. La
quatrième fut Louis II de La Cerda, servit si bien les rois catholiques
contre les Mores, qu'en 1491 ils le créèrent duc de Medina-Coeli\,; le
troisième duc fut fait marquis de Cogolludo\,; le sixième épousa
l'héritière du duché d'Alcala. Son fils, le septième, épousa l'héritière
des duchés de Ségorbe et de Cardonne, des marquisats de Comarès et de
Denia et du comté de Sainte-Gadea. Je ne marque sur chacun que les
grandesses qu'ils accumulèrent et point les autres terres. Le huitième
fils du septième finit la race de ces heureux bâtards de Foix. Ce fut
Louis-François, huitième duc de Medina-Coeli, général des côtes
d'Andalousie, puis des galères de Naples, ambassadeur à Rome, vice-roi
de Naples appelé à Madrid, fait gouverneur du prince des Asturies, et
premier ministre d'État 1709. La jalousie et les menées de la princesse
des Ursins le rendirent suspect. Il fut accusé d'une conspiration contre
l'État, et arrêté comme il allait au conseil, conduit à Pampelune, puis
à Fontarabie, où il mourut fort tôt après sans aucuns enfants de la
fille du duc d'Ossone qu'il avait épousée en 1678. Ses soeurs avaient
épousé, l'aînée le marquis de Priego\,; la seconde le marquis
d'Astorga\,; la troisième le dernier amirante de Castille\,; la
quatrième le duc d'Albuquerque\,; la cinquième le marquis de Solera\,;
la sixième le connétable Colonne\,; la septième le duc del Sesto\,; la
dernière le comte d'Oñate, tous grands d'Espagne. Ainsi la soeur aînée
du huitième duc de Medina-Coeli des bâtards de Foix hérita de toutes ses
grandesses qu'elle porta après son mariage à son mari le marquis de
Priego. Voyons maintenant qui était ce marquis de Priego, qui était
aussi duc de Feria, et doublement grand d'Espagne.

Laurent II, Suarez de Figuerroa, fut fait comte de Feria en Estramadure
par Henri IV, roi de Castille, en 1467. Il était petit-fils de Laurent
Ier Suarez de Figuerroa, maître de l'ordre de Saint-Jacques, qui acquit
cette terre, et il fut grand-père d'autre Laurent III Suarez de
Figuerroa\,; tout cela de mâle en mâle, qui épousa, en 1518, Catherine,
fille aînée et héritière de Pierre Fernandez de Cordoue, marquis de
Priego, par laquelle il unit les deux grandesses de Feria et de Priego,
et le nom de Fernandez de Cordoue, de sa femme, au sien de Suarez de
Figuerroa dans sa postérité. Pierre leur fils, mort après son père, mais
avant sa mère, fut quatrième comte de Feria, et ne laissa qu'une fille
unique laquelle fut bien marquise de Priego, mais non comtesse de Feria,
qui ne pouvait passer aux filles. Ainsi son oncle paternel devint
cinquième comte de Feria, et ce fut en sa faveur qu'en 1567 Philippe II
le fit duc de Feria, dont le fils, second duc de Feria, venu à Paris de
la part de Philippe II, servit si ardemment la Ligue. Sa race s'éteignit
dans le quatrième duc de Feria.

Alphonse Suarez Figuerroa était troisième fils de Laurent III, troisième
comte de Feria, et de Catherine, héritière de Pierre Fernandez de
Cordoue, marquis de Priego, et frère cadet du premier duc de Feria, dont
il épousa la fille, et fut par elle marquis de Priego. Sa postérité
masculine réunit Feria et Priego, par la succession du cinquième marquis
de Priego au quatrième duc de Feria. Le fils de celui-ci fut ainsi
sixième duc de Feria, et aussi sixième marquis de Priego, et c'est lui à
qui Philippe IV accorda les honneurs de grand de la première classe.

Il maria son fils à la soeur aînée du dernier duc de Medina-Coeli des
bâtards de Foix, laquelle en recueillit la succession depuis qu'elle fut
veuve et qu'elle transmit à son fils Emmanuel Figuerroa de Cordoue et La
Cerda, marquis de Priego, duc de Feria et Medina-Coeli, etc., père de
celui que j'ai vu en Espagne, et qui y était fort considéré. Il avait un
fils déjà grand, qui portait le nom de marquis de Cogolludo et qui,
depuis mon retour, acquit de nouvelles grandesses par son mariage avec
la fille unique héritière du marquis d'Ayétone. Le père, et le fils
étaient autant du grand monde et de la cour que des seigneurs espagnols
naturels en pouvaient être, fort polis\,: je les voyais fort
familièrement. Ce sont ceux de cette cour qui se sont souvenus le plus
longtemps de moi, par leurs lettres, bien des années depuis mon retour.
Le palais de Medina-Coeli, presque au bout de Madrid, vers Notre-Dame
d'Atocha, est peut-être le plus spacieux qu'il y ait dans la ville et
très somptueusement meublé. Le roi d'Espagne s'y retira à la mort de la
reine sa première femme, et y a demeuré jusque fort près de son second
mariage. J'y ai vu une comédie extrêmement magnifique, dans une salle
faite pour ce spectacle, où le duc de Medina-Coeli avait convié toute la
cour et le plus distingué de la ville, hommes et femmes, après le retour
de Lerma, où je vis le duc de Linarez, tout évêque qu'il était, et le
cardinal Borgia\,; tout y était plein, mais avec un grand ordre et
décence, et rien de plus magnifique que l'abondance des
rafraîchissements et de tout ce qui accompagna la fête.

Medina de Rioseco, \emph{Enriquez y Cabrera}, amirante héréditaire de
Castille. Cette maison depuis son origine, ses grandesses, le personnel
de l'amirante de Castille, lors de l'avènement de Philippe V à la
couronne d'Espagne, ont été traités avec un si grand détail, (t. III,
p.~122), sa conduite depuis sa fuite en Portugal, le triste personnage
qu'il y fit jusqu'à sa mort, (t. III, p, 435), qu'il ne s'en pourrait
faire ici que d'ennuyeuses redites.

2 Medina-Sidonia, \emph{Guzman}. C'est le premier duché des Castilles.
Les antérieurs à celui-là sont éteints. Il est en Andalousie, vers le
détroit de Gibraltar. Jean II, roi de Castille l'avait donné, sans
érection, à J. Guzman, maître de l'ordre de Calatrava. Cette terre tomba
à Henri Guzman, second comte de Niebla, dont le fils aîné, Jean-Alphonse
de Guzman, fut créé, en février 1445, par le même roi Jean II, duc de
Medina-Sidonia, mais seulement pour sa personne. Le roi Henri IV
l'étendit, en février 1460, non seulement à sa postérité légitime, mais
encore à son défaut à l'illégitime. Cela sent bien le mauresque et
l'Afrique. La maison de Guzman est une des plus anciennes, des plus
grandes et des plus illustres d'Espagne, et y figurait fort dès le Xe
siècle. Le duché de Medina-Sidonia est demeuré dans la postérité
masculine et légitime du premier duc. On a suffisamment parlé du duc de
Medina-Sidonia à l'occasion du testament de Charles II et de l'arrivée
de Philippe V en Espagne\footnote{Voy. t. III, p.~7. C'est un des
  passages supprimés dans les anciennes éditions.}, dont il fut grand
écuyer, puis chevalier du Saint-Esprit, et de son fils qui aima mieux
conserver sa golille et vivre obscur que de faire sa couverture après la
mort de son père. C'est ce fils qui était duc de Medina-Sidonia lorsque
j'étais en Espagne, et que je n'ai vu ni rencontré nulle part.

Saint-Michel, \emph{Gravina}, d'une des plus grandes maisons de Sicile,
où il avait très bien servi et s'était fort endetté à soutenir le parti
de Philippe V tant qu'il avait pu\,; en considération de quoi il avait
obtenu la grandesse. Il était venu à Madrid pour y faire sa
couverture\,; mais, comme je l'ai dit ailleurs, je l'y laissai encore
sans s'être couvert faute d'avoir pu payer la médiannate et les frais,
qui vont loin, sans avoir pu obtenir ni remise ni diminution, ce que
tout le monde trouvait fort injuste. Il était vieux, estimé et
accueilli\,; mais la tristesse de sa situation le rendait obscur. Comme
toute sa famille était en Sicile où il comptait retourner, je ne m'y
étendrai pas davantage.

LaMirandole, \emph{Pico}. Je ne m'arrête sur ce seigneur italien, fait
grand d'Espagne par Philippe V, qui le fit aussi son grand écuyer, que
parce qu'il s'est établi en Espagne après avoir perdu toute espérance de
rétablissement dans ses petits États d'Italie, où ses pères étaient
comme souverains, et dont l'empereur Léopold les a dépouillés sans
retour, parce qu'ils se sont trouvés à sa bienséance. Les Pic sont
connus dès 1300, par Fr.~Pico, seigneur de La Mirandole et vicaire de
l'Empire. J. Pic et Fr.~son frère, quatrième génération de ce premier
François, furent faits comtes de Concordia, 1414, par l'empereur
Sigismond. Le fameux Pic de La Mirandole, le phénix de son siècle par
son immense savoir, mort sans alliance en 1494, n'ayant pas encore
trente-deux ans, était frère cadet de Galeot Pic, seigneur de La
Mirandole, comte de Concordia, qui était la quatrième génération du
premier comte. Galeot Pic, second du nom, comte de Concordia et premier
comte de La Mirandole, mort 1551, était petit-fils du frère du savant
Pic de La Mirandole, et père de Silvie et de Fulvie, qui épousèrent le
comte de La Rochefoucauld et un autre La Rochefoucauld, comte de Randan,
du premier desquels viennent les ducs de La Rochefoucauld. Ce même père
de ces deux dames de La Rochefoucauld le fut aussi d'un comte de La
Mirandole et de Concordia, duquel le fils, nommé Alexandre, fut fait duc
de La Mirandole, en en 1619, par l'empereur Ferdinand II, duquel le duc
de La Mirandole, que j'ai vu en Espagne, est la quatrième génération.
Son frère a depuis été cardinal par Clément XI, dont il était maître de
chambre. Ce duc de La Mirandole s'était vu sur le point d'être rétabli
dans ses États et d'épouser la princesse de Parme, qui eut depuis
l'honneur d'être la seconde femme de Philippe V, et qui conserva
toujours de l'amitié et une grande distinction pour lui et pour la femme
qu'il épousa depuis, soeur du marquis de los Balbazès, que j'ai vue
aussi en Espagne et qui fut noyée dans sa maison de Madrid, réfugiée
dans son oratoire, par une subite inondation dont j'ai parlé ailleurs,
quoique arrivée depuis mon retour. Ce duc de La Mirandole était un fort
bon et honnête homme, fort pieux et considéré\,; sa mère était Borghèse,
fille du prince de Sulmone, remariée à Cellamare qui en était veuf, et
qui vivait avec lui dans une étroite amitié.

Monteillano, \emph{Solis}. Cette maison peut être comparée à quelques
françaises qui se sont élevées à une grande fortune. Celui-ci était
proprement de ce que nous appelons de robe. Il s'éleva par ses talents
jusqu'à être gouverneur du conseil de Castille, et il eut assez de
faveur pour être fait grand d'Espagne et duc de Monteillano par Charles
II, depuis quoi il n'a presque plus paru. Il avait épousé une soeur du
prince d'Isenghien, gendre du maréchal d'Humières, qui avait de
l'esprit, du monde, encore plus de sens. Ce fut elle que la princesse
des Ursins choisit pour lui garder la place de camarera-mayor de la
reine, lorsqu'elle frit chassée la première fois et qu'elle reprit à son
retour triomphant en Espagne. Cette grande place l'avait fait connaître,
aimer et considérer dans le peu de temps qu'elle l'occupa, et c'est ce
qui la fit choisir dans la suite pour remplir la même place auprès de la
princesse des Asturies, où on en fut fort content\,: dans l'entre-deux
elle avait perdu son mari. Elle avait un fils qui était jeune, dont on
disait du bien. Je l'ai vu, mais sans aucun commerce. Il avait, dit-on,
du goût pour la lecture et la retraite, et il paraissait peu à la cour
et dans le monde. Je ne répondrais pas que cette grandesse n'eût été
achetée dans les grands besoins où Charles II s'est trouvé plus d'une
fois\,; car il manqua toujours d'argent.

2 Monteléon, \emph{Pignatelli}. On connaît Jacques Pignatelli,
gouverneur de la Pouille dès 1326, et cette maison, qui est fort
étendue, pour une des grandes, des plus illustrées de titres et des plus
hautement alliées du royaume de Naples. Hector\footnote{Cet Hector avait
  épousé Jeanne, héritière de Tagliavia, dont le grand-père paternel fut
  fait en 1561 duc de Terranova, et en 1565 grand d'Espagne, chevalier
  de la Toison d'or, etc., par Philippe II, dont il fut ambassadeur en
  Allemagne, et après gouverneur du Milanais. C'est cette héritière,
  cinquième duchesse de Terranova qui étant veuve d'Hector Pignatelli,
  duc de Monteléon avec postérité, fut faite par Charles II
  camarera-mayor de sa première femme, fille de Monsieur, frère de Louis
  XIV, en 1619, à qui elle se rendit si insupportable par sa rigidité et
  ses insolences que la reine se la fit ôter, chose sans exemple en
  Espagne. Elle fut mise en cette même charge auprès de la reine, mère
  de Charles II, et y mourut, mai 1692, au Buen-Retiro, laissant
  héritière de ses biens et de sa grandesse de Terranova Jeanne
  Pignatelli, qui avait épousé, 1679, Nicolas Pignatelli, frère de son
  bisaïeul, père du duc de Monteléon, qui fait {[}cet{]} article.
  (\emph{Note de Saint-Simon}.)} Pigriatelli, quatrième duc de
Monteléon, vice-roi de Catalogne, fut fait grand d'Espagne en 1613, par
Philippe III. Nicolas Pignatelli, vice-roi de Sardaigne et chevalier de
la Toison d'or, fils dernier cadet de cet Hector, épousa la fille
héritière du septième duc de Monteléon, petit-fils de son frère, et
devint par elle huitième duc de Monteléon et de Terranova, dont la mère
de son père était héritière, et fut ainsi grand d'Espagne. Ce fut lui
qui, comme le plus ancien chevalier de l'ordre de la Toison d'or qui fût
lors en Espagne, y donna en cérémonie le collier à Philippe V à son
arrivée. On a parlé de lui, t. III, p.~128, en son lieu, à propos de la
saccade du vicaire. Il se retira bientôt après à Naples où étaient ses
duchés et tous ses biens, y fut très -partial de la maison d'Autriche,
et n'est pas revenu depuis en Espagne, ni aucun de sa famille.

Mortemart, \emph{Rochechouart}, François, duc et pair, à Paris. C'est la
grandesse du duc de Beauvilliers que Philippe V lui donna en arrivant en
Espagne, dont il avait été le gouverneur. Elle passa au duc de
Mortemart, qui avait épousé sa fille unique héritière, et par la mort
d'eux et de leurs enfants cette grandesse est éteinte depuis mon retour.

Nagera, \emph{Osorio y Moscoso}, frère cadet du comte d'Altamire, à
l'article duquel je remets à parler de leur maison. Najera ou Nagera,
car il s'écrit et se lit des deux façons. Cette terre, qui est en
Castille, fut érigée en duché par les rois catholiques, 1482, pour
Pierre Manrique de Lara, dit le Vaillant, second comte de Trevigno, et
dixième seigneur d'Amusco. Cette grandesse est tombée cinq fois en
différentes maisons par des filles héritières. Pendant que j'étais en
Espagne, don Joseph Osorio y Moscoso, frère cadet du comte d'Altamire,
eut cette grandesse par son mariage avec Anne de Guevara y Manrique, qui
en était l'héritière et file du défunt frère du dixième comte d'Oñate.

Nevers, \emph{Mancini}, son père, fils d'une soeur du cardinal Mazarin,
fut duc à brevet. Il ne put ou négligea d'obtenir l'enregistrement de
ses lettres, quoique la toute-puissante faveur de son oncle se soit
trouvée dans la suite presque la même pour lui par celle de
M\textsuperscript{me} de Montespan, dont il avait épousé la nièce, fille
de M\textsuperscript{me} de Thianges sa soeur, dont la faveur était
grande aussi auprès du roi, et a duré autant que sa vie qui a dépassé de
plusieurs années le renvoi de M\textsuperscript{me} de Montespan. M. de
Nevers, qui personnellement n'avait jamais rien mérité du roi, et son
fils beaucoup moins encore, fort fâché de ne pouvoir espérer que son
fils fût duc, chercha partout une grandesse à lui faire épouser. Il
trouva enfin M. A. Spinola, fille aînée et héritière de J. B. Spinola,
qui pour de l'argent s'était fait faire prince de l'Empire, en 1677, par
l'empereur Léopold, et depuis, par la même voie, grand d'Espagne par
Charles II, dans leurs pressants besoins de finances. Ce mariage ne se
fit pourtant célébrer qu'en 1709, deux ans après la mort du duc de
Nevers, et son fils qui jusqu'alors avait porté le nom de comte de
Donzy, prit celui de prince de Vergagne, mais sans rang ni honneurs qu'à
la mort de son beau-père en Flandre, où il était lieutenant général et
gouverneur d'Ath. La duchesse Sforza, soeur de sa mère, et dans la plus
grande et plus longue intimité de M\textsuperscript{me} la duchesse
d'Orléans, profita de la régence de M. le duc d'Orléans, et le fit faire
duc et pair sans avoir jamais vu ni cour ni guerre.

Noailles, \emph{idem}. Il y a eu tant et tant d'occasions ici de parler
et de s'étendre sur le duc de Noailles, qu'il suffit de dire qu'avec la
faveur de sa famille et celle de M\textsuperscript{me} de Maintenon,
dont il avait épousé l'unique nièce et héritière, fille de son frère, il
ne lui fut pas difficile d'obtenir en Espagne tout ce qu'il voulut.

Ossuna, \emph{Acuña y Tellez Giron}. La maison d'Acuña, fort nombreuse
en branches tant espagnoles que portugaises, et la maison de Silva,
prétendent sortir de la même origine aussi illustre qu'ancienne, et y
sont autorisées par les meilleurs auteurs, qui les font masculinement
descendre de Fruela, roi de Léon, des Asturies et de Galice, par le
ricohombre Pélage Pelaez, duquel sont masculinement sortis Gomez Paez de
Silva, dont toute la maison de Silva est descendue, et Ferdinand Paez
qui le premier prit le nom d'Acuña, du lieu d'Acuña-Alta, qu'Alphonse
Ier, roi de Portugal, lui avait donné, et duquel toute sa postérité
conserva le nom. La septième génération masculine de ce Ferdinand Paez,
seigneur d Acuña, fut Martin Vasquez de Acuña, qui fut comte de Valence,
épousa 1° Thérèse, fille et héritière d'Alphonse Tellez-Giron, dont il
eut un fils qui porta le nom de Tellez-Giron\,; {[}2°{]} il épousa
l'héritière de la maison de Pacheco, et en eut deux fils. Jean, l'aîné,
porta le nom de Pacheco de sa mère, et Pierre, le cadet, prit le nom de
Giron, de la mère de son père. L'aîné de ces deux frères est le chef de
la branche aînée de toute la maison d'Acuña-Pacheco, ducs d'Escalope. Le
cadet, mort, 1466, maître de l'ordre de Calatrava, est le chef de la
seconde branche d'Acuña-Tellez-Giron, ducs d'Ossone.

Son arrière-petit-fils de mâle en mâle fut Pierre d'Acuña-Giron,
cinquième comte d'Urenna, vice-roi de Naples, créé, 1562, duc d'Ossone
en Andalousie, entre Séville et Malaga, par Philippe II. C'est de mâles
en mâles aînés la cinquième génération que nous avons vue\,; savoir\,:
le sixième duc d'Ossone qu'on a vu en son lieu être venu à Paris lors de
l'avènement de Philippe V à la couronne d'Espagne pour y saluer son
nouveau roi, voir la cour de France et joindre le roi d'Espagne avant
son arrivée à Madrid\,; le même duc d'Ossone, premier plénipotentiaire
d'Espagne à Utrecht, et mort en Flandre peu après la signature de cette
paix\,; et son frère le comte de Pinto, duc d'Ossone, après la mort de
son frère, ambassadeur d'Espagne en France pour le mariage du prince des
Asturies avec la fille de M. le duc d'Orléans. On a suffisamment parlé
de l'aîné en son temps, et le cadet n'a rien eu qui mérite d'en rien
dire.

Saint-Pierre, \emph{Spinola}, Génois, de l'une des quatre grandes
maisons de Gênes, trop connue et trop nombreuse pour m'y étendre.
Quoique accoutumée aux honneurs, aux grandeurs, aux plus grands emplois
et fertile en grands hommes, il est pourtant constant en Espagne que
François-Marie Spinola, duc de Saint-Pierre et gendre de
Philippe-Antoine Spinola, quatrième marquis de Los Balbazès, grand
d'Espagne et général des armes du Milanais, acheta la grandesse de
Charles II en 1675\,; il acheta aussi la principauté de Piombino que
l'empereur s'appropria sans le rembourser. Il chercha protection dans ce
malheur pour y intéresser les cours de France et d'Espagne, et comme il
était veuf il épousa, 1704, à Paris, la seconde soeur du marquis de
Torcy, ministre d'État et secrétaire d'État des affaires étrangères, qui
était veuve avec des enfants du marquis de Resnel, Clermont-d'Amboise.
Lui aussi en avait de sa première femme qui ont figuré en Espagne avec
beaucoup de réputation à la guerre où l'aîné a commandé des armées et
est devenu capitaine général et grand d'Espagne après son père. Le duc
de Saint-Pierre, lassé à Paris de ne voir point avancer ses affaires sur
Piombino, emmena sa femme errer en Italie, quelque peu en Allemagne, la
ramena à Paris, puis en Espagne. Il fut peu de temps à Bayonne
majordome-major de la reine douairière d'Espagne, soeur de la mère de
l'empereur et de l'électeur palatin\,; mais voyant que son crédit à
Vienne ne lui servait de rien, il la quitta et s'en alla à Madrid où sa
femme fut dame du palais de la reine et fort bien avec elle. Je les
trouvai ainsi à Madrid où je les vis fort et en reçus toutes sortes de
prévenances et de civilités. Elle avait enfin apprivoisé la jalousie et
l'avarice de son mari, qui d'ailleurs était un homme d'esprit, fort
instruit et de bonne compagnie, avec des manières naturellement fort
nobles et fort polies. Les étrangers s'assemblaient chez eux, et des
Espagnols quelquefois aussi\,; on y jouait quand on voulait, et ils ne
laissaient pas de donner assez souvent à manger. Depuis mon départ, le
duc de Saint-Pierre fut gouverneur de don Carlos, et enfin chevalier du
Saint-Esprit. Il avait de la valeur, avait peu de temps commandé une
armée, et était capitaine général de Charles II. Il mourut à Madrid,
fort vieux, en 1727. C'était un grand homme blond, maigre, bien fait, de
bonne mine, et qui sentait fort son grand seigneur. Sa veuve demeura
longtemps à Madrid, où, ennuyée enfin de la vie peu gaie et peu libre
qu'on y mène, elle obtint permission de venir faire un tour en France.
Elle y a conservé tant qu'elle a pu sa place et ses appointements de
dame du palais de la reine d'Espagne qu'elle amusait de ses lettres, et
le cardinal Fleury des réponses qu'elle en recevait. Ce manége ne lui
valut pas la moindre chose en France, et lassa la reine d'Espagne, qui
la rappelait inutilement, et qui lui ôta enfin sa place et ses
appointements, tellement qu'elle est demeurée pour toujours à Paris avec
beaucoup de goutte, très peu de bien, et moins encore de considération,
quoique bien dans sa famille. Elle n'a point eu d'enfants du duc de
Saint-Pierre.

Popoli, \emph{Cantelmi}. Une des meilleures maisons du royaume de
Naples. Lors de l'avènement de Philippe V à la couronne d'Espagne, le
cardinal Cantelmi était archevêque de Naples, et son frère le duc de
Popoli grand maître de l'artillerie de Naples, de la conduite desquels
le roi, et le roi son petit-fils, furent extrêmement contents. Ce duc de
Popoli avait succédé à ce duché de son frère aîné et à presque tous ses
biens fort considérables dans le royaume de Naples, par son mariage avec
la fille de son frère aîné, qui n'en avait que deux, et point de
garçons. Ce dernier duc de Popoli était un grand homme brun, bien
fourni, avec un beau visage mâle, qui sentait son grand seigneur, et un
général d'armée avec toutes les manières, grandes, avantageuses, polies.
Il ne se pouvait rien ajouter à son extérieur. Il avait beaucoup
d'esprit et de conduite, encore plus de manége et d'intrigue, beau
parleur, et disant ou taisant ou accommodant tout ce qu'il voulait à ses
vues, avec beaucoup d'insinuation et de grâces, haut par nature, bas à
l'excès quand il croyait en avoir besoin, ambitieux, avare à l'excès,
encore plus poltron, faux, double, extrêmement dangereux, et ne se
souciant que de son argent et de sa fortune à laquelle il sacrifia
toutes choses.

Il passa à Paris allant en Espagne. Le roi, qui cherchait à attacher au
roi son petit-fils les grandes maisons et les grands seigneurs de ses
nouveaux royaumes, et fort content de tout ce que ces deux frères
avaient fait à Naples, le reçut avec distinction\,; lui en habile homme
tira sur le temps, fit valoir ce que pouvait à Naples le cardinal son
frère qui en était archevêque, leur grande parenté, leurs amis, et
demanda l'ordre que le roi lui promit, et dont il lui envoya les marques
longtemps même avant qu'il y eût reçu le collier du roi d'Espagne, qui
lui donna aussi celui de la Toison. Les révolutions qu'on a vues en leur
lieu ayant mis toute l'Espagne en armes, le duc de Popoli servit et eut
des commandements, qui avec la considération de sa personne, et à l'aide
de ses intrigues et de ses propos avantageux, le portèrent promptement
au dernier grade militaire d'Espagne, qui est capitaine général, dont il
s'acquitta fort mal à la tête de l'armée de Catalogne, qu'il remit au
duc de Berwick, et s'en retourna à Madrid comme on allait commencer le
siège de Barcelone. Lorsque Philippe V se donna des compagnies des
gardes du corps sur le modèle inconnu jusqu'alors en Espagne de celles
du roi son grand-père, le duc de Popoli, déjà grand maître de
l'artillerie, obtint la compagnie des gardes du corps italienne, et la
querelle du \emph{banquillo}\footnote{Voy. sur cette querelle t. III,
  p.~287.} étant survenue, qu'on a vue en son lieu, le roi d'Espagne fit
grands d'Espagne ceux des capitaines de ses gardes du corps qui ne
l'étoient pas, entre autres le duc de Popoli. Enfin il devint gouverneur
du prince des Asturies, puis son majordome-major à son mariage.

Je le trouvai dans cet éclat en Espagne, et toutefois le seigneur de la
cour le plus parfaitement décrié. Sa femme, à qui il devait tous ses
grands biens, et qu'on disait fort aimable de figure et de manières,
avait été faite dame du palais de la reine qui l'aimait fort, et sa
réputation sur la vertu était entière. Elle mourut un peu étrangement,
et il passait publiquement pour l'avoir empoisonnée par jalousie,
jusque-là que la reine le lui a souvent reproché. Il en avait un fils
unique qui portait le nom de prince de Peltorano, bon garçon, point du
tout méchant, et ayant même de la valeur\,; mais étourdi, fou, débauché
à l'excès. Son père, en ne lui donnant rien ou fort peu par avarice,
l'avait rendu escroc, et il le fut et grand dissipateur toute sa vie. Le
duc de Popoli voyant ses instructions, exhortations, répréhensions,
punitions inutiles, imagina un moyen de le contenir. Il était
compatriote et ami intime du vieux duc de Giovenazzo, père de
Cellamare\,; il lui demanda en grâce de tenir son fils à son côté, de le
mener avec lui faire ses visites, et de le veiller et tenir de près
comme il aurait pu faire lui-même. Il crut que, quel que fût son fils,
le respect et la présence de ce vieillard le retiendrait, lequel pour
son esprit, ses talents, les places qu'il avait remplies était dans une
grande considération et respecté de tout le monde. Ce bonhomme eut assez
d'amitié pour le duc de Popoli, pour lui accorder sa demande, en sorte
que le jeune Peltorano était chez lui et avec lui du matin au soir, et
l'accompagnait partout où il allait, et qu'il n'avait pas un instant de
libre. Voici de quoi il s'avisa\,:

Il sut par hasard qu'un seigneur, dont j'ai oublié le nom, ne serait pas
sûrement chez lui, et il proposa au duc de Giovenazzo de l'aller voir,
parce qu'il le visitait quelquefois, et qu'il y avait du temps qu'il n'y
avait été. Le bonhomme le loua de cette attention et de son désir
d'aller voir un homme auprès duquel il y avait toujours à apprendre, et
il lui dit qu'il l'y mènerait l'après-dînée. Peltorano, sûr de son fait,
prit ses précautions. Les maisons de Madrid, même les plus belles, n'ont
point de cours, au moins y sont-elles fort rares. Les carrosses arrêtent
dans la rue où on met pied à terre\,; on entre par la porte qui est
comme nos portes cochères dans un lieu large et long, qui ne reçoit de
jour que par la porte, et qui a des recoins très obscurs, et l'escalier
est au fond par lequel on monte dans les appartements. Arrêtés à la
porte de ce seigneur, on leur vint dire qu'il n'y était pas\,; tout
aussitôt Peltorano pria le vieux duc de lui permettre de descendre un
moment pour un besoin dont il était fort pressé, saute à bas et entre
dans ce porche couvert\,; le temps qu'il y fut parut un peu long au
bonhomme, et il était prêt d'envoyer voir s'il ne se trouvait point mal,
lorsque Peltorano revint et monta en carrosse tranquillement avec
beaucoup d'excuses. Comme le carrosse partait et se mettait au pas,
comme on va dans Madrid, une courtisane sort du porche, se jette au
carrosse, se prend par les mains à la portière, crie et injurie
Peltorano qu'il l'escroque, qu'il lui a donné ce rendez-vous, qu'il lui
a promis quatre pistoles, et qu'il s'en va sans la payer. Le vieux duc
tout effaré la veut chasser\,; elle crie plus fort, qu'elle sera payée,
qu'elle ne quittera point prise qu'elle ne le soit, et qu'elle criera à
tout le peuple qu'ils la veulent affronter\,; elle fit tant de bruit, et
avec une telle résolution, que le bonhomme, comblé de honte, de colère
et d'indignation, tira quatre pistoles de sa poche qu'il lui donna pour
se délivrer d'elle, tandis que le Peltorano, qui n'avait pas un sou sur
lui, s'était tapis dans le coin du carrosse, et riait sous cape du
désarroi du bon vieillard, par qui il s'était fait mener à son
rendez-vous, et à qui encore il le faisait payer. Le duc de Giovenazzo,
délivré pour son argent de cette effrontée, s'en alla droit chez le duc
de Popoli, à qui il conta son aventure, lui remit son fils pour ne plus
s'en jamais mêler, et lui déclara qu'il ne s'exposerait pas à un second
affront. Le Peltorano fut bien pouillé et chapitré, ne fit qu'en secouer
les oreilles, et n'en devint pas plus sage\,; il ne fit qu'en rire et
conter son joli exploit.

C'est ce garnement-là qui épousa la fille du maréchal de Boufflers,
comme on l'a vu en son lieu, et que je trouvai à Madrid dame du palais
de la reine, et fort bien avec elle, et avec tout le monde sur un pied
d'estime et de considération. Son beau-père en avait beaucoup pour elle,
et son mari aussi, qui la laissait vivre à la française, voir qui elle
voulait, et donner presque tous les jours à souper, où mes enfants et
ceux qui étaient venus avec moi soupaient souvent, et passaient leurs
soirées jusque fort tard, avec fort bonne compagnie d'étrangers dont le
mari profitait aussi, et ils y jouaient quelquefois. Le duc de Popoli,
qui ne logeait pas avec eux, mais au palais, le savait bien, et le
trouvait bon, la reine aussi, quoique là-dessus assez difficile\,; mais
ils connaissaient le mari qui avait fait plus d'une fois d'étranges
présents à sa femme, et ils lui voulaient adoucir les malheurs d'avoir
un tel mari. À la fin depuis mon départ ses maux mal guéris et repris
augmentèrent\,; elle se tourna entièrement à la dévotion jusque-là
qu'elle voulut quitter sa place et se retirer dans un couvent. La reine
qui l'aimait et la plaignait la retint tant qu'elle put\,; mais enfin,
vaincue par ses prières, elle y consentit, mais à condition qu'elle
irait dans les \emph{descalceales reales}\footnote{Abbaye de fondation
  royale de religieuses déchaussées. Saint-Simon a écrit
  \emph{descalceales} pour indiquer des \emph{religieuses
  déchaussées\,;} mais ce mot ne se trouve pas dans les lexiques
  espagnols. Les précédents éditeurs l'ont changé en \emph{descalcez},
  qui veut dire \emph{nudité des pieds}, et, par extension, ordre de
  moines déchaussés. La véritable expression pour indiquer des
  religieuses déchaussées serait \emph{descalzas}.}**, dans un
appartement qu'elle lui ferait accommoder, qu'elle viendrait voir la
reine, et que la reine l'irait voir par la communication du palais à ce
couvent, qu'elle garderait toujours sa place sans en faire de fonctions
pour les reprendre quand il lui plairait, et ajouta une pension aux
appointements de sa place. Elle fut généralement regrettée de tout le
monde. Sa retraite ne fut que de deux ou trois ans qu'elle y passa dans
la plus grande piété et beaucoup de souffrances, au bout desquels elle y
mourut, tandis que son mari, devenu très riche par la mort de son père,
dissipait les trésors qu'il avait amassés. Il eut dans la suite des
aventures fâcheuses qui le firent enfermer, et longtemps, plus d'une
fois en Espagne et en Italie.

À l'égard du père, dès qu'on l'avait vu deux ou trois fois, on
s'apercevait aisément de presque tout ce qu'il était avec ses
compliments outrés. Malgré sa figure imposante, on sentait le faux de
loin, et l'affronteur en tous ses propos, à tel point que je n'ai jamais
compris comment il a pu parvenir à une si grande fortune. Ses grands
emplois de capitaine des gardes du corps, et de gouverneur du prince des
Asturies, et son talent d'intrigue et de cabale le faisaient compter,
mais au fond tout le monde s'en défiait et le méprisait.

J'ai déjà dit qu'il fut le seul seigneur dont je ne reçus aucune
civilité, si on excepte les compliments à perte de vue dont il
m'accablait quand je le rencontrais, ce qui n'arrivait qu'au palais, et
encore rarement\,; aussi ne m'en contraignis-je pas en propos, et en ne
lui rendant aucune sorte de devoir. Il se fit écrire une seule fois et
fort tard à ma porte\,; j'avais été chez lui en allant la seconde fois
chez le prince des Asturies. En partant pour mon retour, je ne manquai à
aucune visite moi-même, quelque nombreuses qu'elles fussent, excepté la
sienne, et je pris mon temps de m'envoyer faire écrire chez lui que
j'étais au Mail à faire ma cour à Leurs Majestés Catholiques, et qu'il
ne pouvait l'ignorer. Pendant cette promenade où la reine, toujours à
côté du roi, faisait toujours la conversation avec le peu de gens
considérables qui l'accompagnaient, et une conversation fort agréable et
familière, je pris la liberté de lui demander où elle me croyait
alors\,; elle se mit à rire et me dit\,: «\,Mais ici où je vous vois.
--- Point du tout, madame, repris je, je suis actuellement chez le duc
de Popoli, où je prends congé de lui\,;» et de là en plaisanteries, car
elle ne l'aimait point tout Italien qu'il fût.

Il ne la fit pas longue après mon départ. Il mourut dans le mois de
janvier suivant, regretté de personne. On lui trouva un argent immense
que son avarice avait accumulé. Le duc de Bejar fut majordome-major du
prince des Asturies en sa place.

3 Sesse, c'est Sessa, \emph{Folch-Cardonne}. Ce duché dans le royaume de
Naples fut donné par Ferdinand le Catholique au grand capitaine Gonzalve
de Cordoue, qui n'eut point de mâles, et dont la fille héritière porta
ce duché en mariage à Fernandez de Cordoue, comte de Cobra, de sa même
maison. Elle en eut un fils que Philippe H fit en 1566 duc de Baëna, qui
est un lieu à huit lieues de Cordoue, et qui par sa mère fut aussi duc
de Sesse. Il ne laissa que deux filles Françoise, l'aînée, veuve sans
enfants d'Alphonse de Zuniga, marquis de Gibraleon, fit cession de ses
duchés à Antoine Folch de Cardonne, descendu du premier comte de
Cardonne, second duc de Somme au royaume de Naples, fils du premier duc
de Somme, et de Béatrix, soeur cadette de Françoise. C'était un seigneur
dont Philippe II estimait fort l'esprit et le sens. C'est de lui que
descend de mâle en mâle le duc de Sesse, que j'ai fort vu en Espagne,
qui ne ressemblait guère à celui dont on vient de parler. Celui-ci était
un grand garçon, fort bien fait, ayant la tête plus que verte, aimant
fort le vin, chose fort rare dans un Espagnol, et d'ailleurs étourdi et
débauché à merveilles, par conséquent méprisé, quoique assez dans le
monde, mais fort rarement au palais. Il n'était point marié.

Saint-Simon, \emph{idem}, et mon second fils conjointement avec moi pour
en jouir tous les deux ensemble et en même temps.

Solferino, \emph{Gonzague}, cadet d'une branche de Castiglione. Son
père, fort pauvre déjà, l'était devenu tout à fait par les guerres
d'Italie, de sorte qu'il envoya ce fils en France avec un petit collet,
dans l'espérance qu'il y attraperait quelque bénéfice pour vivre. Il
était noir, vilain, crasseux, et paraissait un pauvre boursier de
collège. Personne ne le recueillit, personne même ne lui parlait dans
les appartements de Versailles\,; il n'entrait que dans les maisons
ouvertes, où on ne lui disait mot, et encore n'allait-il que dans fort
peu. Il importuna tellement le roi de sa présence qu'il revint une fois
de Trianon, où tout le monde pouvait aller lui faire sa cour, quelques
jours plus tôt que ce qu'il avait fixé, et ne put s'empêcher de dire,
tout mesuré qu'il était toujours, qu'il n'avait pu tenir davantage à
voir à tous les coins dallées, et à toutes les portes de son passage, ce
petit abbé de Castillon et Fornare, dont on a parlé ailleurs. À Paris,
cet abbé n'était pas mieux venu. Sa ressource était chez le duc d'Albe,
ambassadeur d'Espagne. Il y fit si bien sa cour à la duchesse d'Albe
qu'après la mort de son mari, elle le remena avec elle en Espagne, où
tant fut procédé qu'elle l'épousa, et pour ne pas déchoir, le roi
d'Espagne eut pour elle la considération de le faire grand d'Espagne, et
peu après lui accorda une clef de gentilhomme de sa chambre\,; mais sans
exercice, comme ils étaient tous. Il perdit sa femme comme j'arrivais à
Madrid. La douleur lui persuada de se faire capucin, et quand je l'allai
voir, je trouvai sa chambre sans tapisserie ni meubles, avec un châlit
sans ciel ni rideaux, et trois ou quatre méchants sièges de paille, avec
un capucin avec lui. Cette grande douleur ne fut pas longue. Il épousa
avant mon départ une Caraccioli, fille du prince de Santo-Buono, qui
était peut-être la seule belle personne qui fût dans Madrid. L'esprit
lui était venu avec le pain assuré, et il était fort dans le grand
monde, estimé et bien reçu partout, et bien mieux peigné qu'il ne
l'était à Paris.

Tursis, \emph{Doria}, Génois, et à Gênes, de l'une des quatre grandes
maisons de Gênes, où ces ducs de Tursis se sont fait compter depuis
longtemps par une escadre de galères qu'ils ont depuis longtemps à eux,
et dont ils ont souvent fort bien servi les rois d'Espagne.

Veragua, \emph{Portugal y Colomb}. On a parlé et tâché d'expliquer (t.
III, p.~88 et suiv.), les branches royales de Portugal\footnote{Le
  passage, auquel renvoie Saint-Simon avait été supprimé par les anciens
  éditeurs.}, Oropesa, Lemos, Veragua, Cadaval, etc.\,; ainsi je n'en
ferai point de redites\,; j'ai assez touché le personnel de ce duc de
Veragua, depuis, pour n'avoir que peu à ajouter. On se souviendra
seulement que c'est de lui que j'ai reçu le plus de bonnes instructions
sur les grandesses, les maisons, et les personnages d'Espagne\,; qu'il
était frère de la duchesse de Liria, et qu'elle a hérité de ses grands
biens, parce qu'il était veuf sans enfants d'une soeur du duc de Sesse,
et qu'il ne se remaria point.

Ce duché et grandesse fut institué et donné en 1537, par Charles-Quint,
à Diego Colomb, second grand amiral des mers, et vice-roi des Indes ou
des terres découvertes par son père, le fameux Christophe Colomb, qui
était de Ligurie, et qui avait été le premier vice-roi et grand amiral
des Indes. Philippe II, en 1556, échangea Veragua contre la Vega, dans
l'île de la Jamaïque, avec Louis Colomb, fils aîné de Diego, et revêtit
La Vega des mêmes titres et honneurs accordés à Veragua par l'empereur
son père, nonobstant quoi Louis Colomb, ainsi que ses successeurs, ont
toujours pris les titres de ducs de Veragua et La Vega, et de seigneurs
de la Jamaïque, ce dernier on ne sait sur quoi fondé. Louis Colomb ne
laissa que deux filles. L'aînée se fit religieuse, l'autre porta tous
ses biens et ses titres en mariage à son cousin germain, fils du frère
cadet de son père, et n'eut point d'enfants. Les deux soeurs de Louis
Colomb, disputèrent ce grand héritage, Marie et Isabelle, qui fut enfin
adjugé au petit-fils d'Isabelle Nuñez de Portugal y Colomb, qui fut
ainsi quatrième duc de Veragua et père d'Alvare, cinquième duc de
Veragua, et celui-ci père de Pierre-Emmanuel, sixième duc de Veragua,
qui eut la Toison, et fut vice-roi de Galice, de Valence et de Sicile,
et enfin conseiller d'État, tout cela avec beaucoup d'esprit et de
talents, grande avarice, foi très douteuse entre la maison d'Autriche et
le nouveau roi d'Espagne, Philippe V, en tout un homme habile, adroit,
dangereux, et de fort mauvaise réputation.

C'est le père du duc de Veragua que j'ai vu en Espagne, et qui, avant la
mort de son père, portait le nom de marquis de la Jamaïque, et était
venu en France sous ce nom, avec la chimère de rattraper sur les Anglais
l'île de la Jamaïque\,; dont il se prétendait dépouillé par eux.
Longtemps après mon retour, il revint en France pour la même chimère,
qu'il poursuivit près de deux ans fort inutilement, quoi que le duc de
Berwick et moi lui pussions dire, et dépensa cependant fort gros avec
une fameuse chanteuse de l'Opéra. À la fin il tomba malade assez
considérablement\,; la peur du diable le prit, il eut peine néanmoins à
se séparer de cette fille, à qui il donna fort gros. Les vapeurs et les
scrupules l'enfermèrent à ne vouloir voir personne. Il fit de grandes
aumônes, et s'écriait souvent qu'il se repentait bien d'avoir fâché
Dieu\,: c'était son expression. Enfin il s'en retourna dans cet état en
Espagne à fort petites journées\,; il y vécut deux ans toujours enfermé
dans les mêmes vapeurs, ne voyant presque que sa soeur la duchesse de
Liria, qu'il laissa enfin par sa mort une des plus puissantes héritières
qu'il y eût en Espagne. Il avait été à la tête des finances et du
conseil des Indes avec capacité et probité. La jalousie d'Albéroni
l'avait tenu deux ans prisonnier dans le château de Malaga, où il
s'était si bien accoutumé qu'il n'en voulait point sortir. C'était un
homme de beaucoup d'esprit et de connaissances, d'une paresse de corps
incroyable qui diminuait son ambition, un peu avare, fort doux et bon,
sale et malpropre à l'excès, ce qu'on lui reprochait sans nul
ménagement, de fort bonne, agréable et instructive compagnie, et
charmant dans la société, quand il faisait tant que de s'y prêter. Il
était aimé et fort mêlé avec le meilleur monde, souvent malgré lui et sa
paresse, jusqu'à ce que ses vapeurs en eurent fait un reclus. En lui
finit cette branche de Portugal.

Villars, \emph{idem}. Le maréchal de Villars, sans avoir jamais servi le
roi d'Espagne, ni eu aucun rapport avez lui, fut fait grand d'Espagne au
commencement de la régence, au grand étonnement de tout le monde, et
sans qu'on ait jamais su pourquoi. Il le dut, je crois, à ses vanteries
et à ses rodomontades dont la cour d'Espagne fut la dupe, et crut faire
par là une acquisition importante qui ne lui servit jamais à rien. On a
vu ailleurs ses étranges frayeurs à la découverte de la conspiration de
Cellamare et du duc du Maine, dont il fut très réellement sur le point
de mourir. Il ne tint pas à lui d'être fait par l'empereur prince de
l'Empire. Richesses et grandeurs tout lui fut bon.

UZEDA, \emph{Acuña Pacheco Tellez-Giron}. Cette terre qui est en
Castille, fut érigée en duché par Philippe III pour Christophe de
Sandoval y Roxas, fils aîné du duc de Lerme, son premier ministre,
depuis cardinal. Christophe fut marié, mourut avant son père en 1624,
laissa un fils de la fille du huitième amirante de Castille, et ce fils,
qui fut second duc d'Uzeda, mourut en Flandre en 1635, et ne laissa que
deux filles. L'aînée porta le duché de Lerme et beaucoup d'autres biens
en mariage à Louis Ramon Folch, sixième duc de Cardonne et de Segorbe\,;
et la cadette, j'ignore par quelle exception, porta le duché d'Uzeda en
mariage, en 1645, à Gaspard d'Acuña Tellez-Giron, cinquième duc
d'Ossone, dont elle n'eut que des filles, desquelles l'aînée porta le
duché d'Uzeda en mariage, en 1677, à J. Fr.~d'Acuña Pacheco,
Tellez-Giron, troisième comte de Montalvan, qui descendait de mâle en
mâle du fils aîné du premier duc d'Escalope, marquis de Villena, et de
l'héritière de Tellez-Giron, par son troisième fils Alphonse, dont ce
troisième comte de Montalvan fut la septième génération masculine, et
par son mariage troisième duc d'Uzeda. C'est lui qui se trouva
ambassadeur d'Espagne à Rome, à la mort de Charles II et à l'avènement
de Philippe V à la couronne d'Espagne. On a vu en son lieu qu'il s'y
conduisit si bien d'abord qu'il fut compris dans les cinq premiers
chevaliers du Saint-Esprit espagnols que le roi fit à la prière du roi
son petit-fils, mais que, voyant les affaires mal bâter en Italie, il
quitta à Rome le caractère d'ambassadeur de Philippe V, renvoya le
collier du Saint-Esprit au feu roi, chose jamais arrivée jusqu'alors,
prit la Toison que l'archiduc lui envoya, erra longtemps en Italie sans
nulle considération dans le parti qu'il avait embrassé, se retira enfin
à Vienne où il vécut longtemps fort pauvre et fort méprisé, y mourut
dans cet état, et y laissa ses enfants.

Princes de

Bisignano, \emph{Saint-Séverin} à Naples, dont à tous égards c'est une
des premières et plus grandes maisons, qui y a dans tous les temps
puissamment figuré, et qui prétend avec fondement tenir le fief de
Saint-Séverin de Robert Guiscard, en récompense des services rendus à ce
conquérant. Louis de Saint-Séverin, septième comte de Saponara, et
sixième prince de Bisignano, né en 1588, fut fait grand d'Espagne, dont
sa postérité masculine jouit encore aujourd'hui.

Santo-Buono, \emph{Carraccioli}. On peut à peu près dire de cette maison
napolitaine ce qui a été dit de la précédente. Celle-ci prétend tirer
son origine de Grèce, et avoir grandement figuré sous les empereurs de
Constantinople grecs. Elle est divisée en deux par les armes\,: les
Carraccioli rouges qui portent d'or à trois bandes de gueules au chef
d'azur, et les Carraccioli au lion qui portent d'or au lion d'azur. Si
ces deux divisions ont la même origine, laquelle en ce cas est sortie de
l'autre, c'est ce que je laisserai à expliquer. Ces différents points
ont tous leurs conjectures. L'opinion la plus reçue est que c'est la
même maison, puisque de toute ancienneté ces deux divisions ont porté
jusqu'à présent le même nom de Carraccioli, et qu'il n'est pas rare que
les branches anciennes de la même maison, en conservant le même nom,
aient pris des armes différentes. Celle de Joyeuse en France,
c'est-à-dire Châteaurandon, qui est son vrai nom, en fournit un exemple
qui est encore sous nos yeux. Quoi qu'il en soit, le prince de
Santo-Buono que j'ai vu en Espagne, homme d'esprit, et qui savait
beaucoup, avouait, après s'être fort appliqué aux recherches de sa
maison, que les Carraccioli au lion, dont il était, étaient cadets des
Carraccioli rouges, mais masculinement et de la même maison. Ces deux
divisions se sont étendues en une infinité de branches presque toutes
illustres par les emplois, les titres, les alliances et les grandes
possessions.

Matthieu Carraccioli, quatrième prince de Santo-Buono, et second duc de
Castelsangro, mort en 1694, et marquis de Buchiniaco, et comte de
Nicastro, fut fait grand d'Espagne. Il était père de celui que j'ai vu
en Espagne, qui avait été ambassadeur à Venise, et vice-roi du Pérou.
C'était un fort honnête homme, très considéré, d'une conversation
charmante et instructive, et que j'ai beaucoup vu. Il était allé fort
goutteux au Pérou. Il y trouva une herbe qui, prise comme du thé,
guérissait de la goutte, sans aucun des inconvénients des remèdes de
l'Europe qui, en guérissant la goutte en apparence, ne font que déranger
le cours ordinaire de cette humeur qui se porte sur les parties
intérieures, et tue, peu de temps après l'apparente guérison des
membres. Le prince de Santo-Buono eut la curiosité de faire un voyage de
plus de cinquante lieues du côté des montagnes pour voir cette herbe en
son pays natal. Il la vit, il en usa, il se diminua beaucoup la
goutte\,; mais comme il y était sujet dès sa jeunesse, et qu'il en était
déjà estropié, il ne put que diminuer et rendre rares ses attaques de
goutte, et demeura estropié à peu près comme il l'était avant que d'en
avoir pris. Je lui reprochai de n'en, avoir point apporté avec lui pour
en faire des épreuves, et voir quel soulagement en tireraient les
goutteux ainsi séchée et après un si long voyage. La difficulté qu'avait
le prince de Santo-Buono à marcher et à se tenir debout, jointe à la
considération de sa personne, lui avait procuré la distinction d'aller
en chaise à porteur, quoiqu'il n'eût pas la qualité de conseiller
d'État, et qu'au palais on lui apportait un tabouret en attendant que le
roi parût. Il avait des enfants fort honnêtes gens, d'une Ruffo, fille
du quatrième duc de Bagnara au royaume de Naples, où je les crois
retournés depuis la mort de leur père, arrivée peu après mon retour. Les
étrangers s'accoutument difficilement à l'Espagne. Il faut de grands
liens pour les y fixer.

Butera, \emph{Branciforte}, à Naples.

Gariati, \emph{Spinelli}, à Naples.

Chalais, \emph{Talleyrand}, à Paris, Français. La princesse des Ursins
avait épousé en premières noces l'oncle paternel aîné de ce nouveau
prince de Chalais, qui fut de ce fameux duel des La Frette, dont il a
été parlé ailleurs, et qui fut obligé de sortir promptement du royaume.
Il mourut à Venise, allant trouver sa femme à Rome, qui y resta et qui y
épousa le duc de Bracciano, aîné de la maison des Ursins, dont
l'histoire a été racontée ici. Devenue arbitre de tout en Espagne et
ayant fort aimé son premier mari, et par conséquent voulant élever ce
qui lui était proche, elle fit venir en Espagne ce neveu de son premier
mari, dont on a vu en son lieu les voyages et les manoeuvres, et enfin
le fit faire grand d'Espagne sans la permission du roi, qui déclara
qu'il pouvait demeurer en Espagne et qu'il ne lui permettrait jamais de
jouir en France du rang ni des honneurs de grand d'Espagne. La chute de
M\textsuperscript{me} des Ursins lui fit perdre le peu de considération
qu'il s'était acquise.

Je le vis beaucoup en Espagne, et le désir qu'il avait de venir jouir de
sa grandesse dans sa patrie, et la part qu'il savait que j'avais dans
l'amitié et la confiance de M. le duc d'Orléans, et qui avait tant de
puissantes raisons de ne lui être pas favorable, l'engagea à ce que je
n'oserais dire, me faire beaucoup sa cour. Il n'en avait pas besoin.
L'inconcevable et toujours infructueuse débonnaireté de M. le duc
d'Orléans fit, sans ma participation, tout ce qu'il put désirer dès
qu'il sut ce qu'il désirait. Il fit, après mon retour, plusieurs voyages
en France où il voulait se stabilier\footnote{S'établir.}.

Il était pauvre et seulement exempt des gardes du corps en Espagne, dont
il tirait peu, et ne le voulait pas perdre, et n'avait jamais servi en
France et fort peu en Espagne. À la fin, lassé de passer si souvent et
si peu utilement les Pyrénées, il prit congé de l'Espagne pour toujours,
et il épousa la soeur du duc de Mortemart, veuve de Cani, fils unique de
Chamillart, et dont elle était ennuyée de porter le nom, quoiqu'elle en
eût des enfants, qu'elle et lui traitèrent toujours avec tendresse.
Ayant ce tabouret, elle devint dame du palais de la reine. Chalais
pourchassa longtemps l'ordre du Saint-Esprit sans avoir pu l'attraper. À
l'ivresse de la cour, dans tous les deux, succéda le dégoût\,; elle
donna sa place à sa fille qu'ils avaient mariée à son cousin germain,
neveu de Chalais, et {[}ils{]} se sont presque tout à fait retirés de la
cour et du grand monde.

Chimay, \emph{Hennin Liétard}, de Flandre. Lui et son troisième frère se
distinguèrent fort à la guerre et devinrent de bonne heure lieutenants
généraux au service de Philippe V. L'électeur de Bavière, étant
gouverneur général des Pays-Bas sous Charles II, l'avait pris en amitié
tout jeune, et tout jeune lui procura de ce roi l'ordre de la Toison
d'or, dont il reçut le collier des mains de l'électeur. Après
l'avènement de Philippe V à la couronne d'Espagne, et tandis que la
princesse des Ursins la gouvernait, il passa avec son troisième frère en
Espagne, où ils continuèrent à servir, tandis que le second frère,
archevêque de Malines, suivit la révolution des Pays-Bas soumis par
l'empereur, malgré lequel ensuite, comme on l'a vu en son lieu, il se
fit tout dévotement cardinal. Le prince de Chimay fit si bien sa cour à
la princesse des Ursins qu'elle {[}le{]} fit faire grand d'Espagne. Il
devint mon gendre\,: j'en parlerai ailleurs.

Castiglione, \emph{Aquino}, à Naples, que nous prononçons Aquin, maison
qui tire son origine de ces seigneurs lombards qui, à la chute de leur
royaume, se répandirent dans ce qui a fait depuis le royaume de Naples
et s'y emparèrent de plusieurs villes, en sorte que, dès l'an 1073,
Artenulphe était comte d'Aquin et duc de Gaëte, dont la postérité
masculine a possédé Aquin jusqu'à aujourd'hui, et par ses grandes
possessions, ses grands emplois, ses grandes alliances, passe avec
raison pour une des premières maisons d'Italie, et a donné saint Thomas
d'Aquin à l'Église. Thomas, prince de Castiglione, de Fercoletto et de
San Mango, duc de Néocastre, comte de Martorano, dernier cadet de la
maison d'Aquin, et gendre, en 1686, d'Alexandre Pie, duc de La Mirandole
et de Concordia, fut fait grand d'Espagne par Charles II, et a eu
postérité masculine. Charles II fit grand d'Espagne, 1699, Thomas
d'Aquin, sixième prince de Castiglione.

Colonne, \emph{idem}, à Rome, où cette grande et puissante maison figure
si hautement depuis près de sept cents ans, et dans toute l'Italie, par
ses diverses branches, ses grandes possessions, ses grands emplois, ses
illustres alliances sans nombre, plusieurs papes, une foule de cardinaux
et beaucoup de grands hommes et qui ont eu le plus de part aux guerres
et aux grands mouvements de l'Italie. Fabrice Colonne, duc de Paliano et
de Taliacolto, mort en 1520, fut le premier de sa maison connétable du
royaume de Naples, charge qui, jusqu'à aujourd'hui, est demeurée
héréditaire à sa postérité masculine. Laurent Onuphre fut le septième,
eut la Toison d'or et fut fait grand d'Espagne. Il mourut en 1641.

Doria, \emph{idem}, à Gènes, de l'une des quatre premières maisons de
cette république.

Ligne, \emph{idem}, en Flandre, dont la mère était Lorraine Chaligny,
nièce de la reine Louise, épouse du roi Henri III, et petit-fils du
premier prince de Ligne, créé 1601 par l'empereur Rodolphe
III\footnote{Il y a dans le manuscrit Rodolphe III\,; mais il faut lire
  Rodolphe II, empereur qui régna de 1576 à 1612.}. Il eut la Toison
d'or, ainsi que son père, son grand-père, son bisaïeul, et son frère
aîné, mort, en 1641, sans enfants. Il fut général de la cavalerie aux
Pays-Bas, ambassadeur d'Espagne en Angleterre, vice-roi de Sicile,
gouverneur général du Milanais, grand d'Espagne 1650, conseiller d'État,
mort à Madrid en décembre 1679\,; il épousa une
Nassau-Dilembourg-Siégen, veuve de son frère aîné, avec dispense. Cette
grandesse est demeurée en sa postérité masculine, qui a servi Philippe
V, et qui est retournée au service de l'empereur, lorsque les Pays-Bas
espagnols sont retournés sous sa domination.

Masserano, Ferrero, originaires du diocèse de Verceil, avec la chimère
de descendre de la grande et illustre maison Acciaïoli\,; mais la vérité
est qu'on ne les connaît guère avant l'an 1500 qu'ils eurent un
cardinal, un évêque de Verceil en 1506, et un autre cardinal en 1517\,;
ils en ont eu depuis trois autres et plusieurs évêques et abbés dans les
États des ducs de Savoie. Le neveu du premier de ces cardinaux fut
marquis de Masseran, situé dans le Piémont. Sa mère était Fiesque\,;
dont ils ont depuis mis les armes sur le tout des leurs qui sont
d'Acciaïoli, sans aucune preuve d'en être, au premier et quatrième\,; au
second et au troisième de l'Empire, par quelque concession\,; ainsi, à
proprement parler, ils n'ont point d'armes à eux. Dans la suite, ils se
sont trouvés si honorés de l'alliance de Fiesque qu'ils en ont ajouté le
nom au leur. Ce premier marquis de Passeran épousa une Sforze
Santa-Fiore, puis une Raconis, des bâtards de Savoie. Son fils épousa
une bâtarde du duc Charles-Emmanuel de Savoie, de laquelle vinrent ses
enfants, puis une Grillec-Saint-Trivier du même nom qu'était Brissac si
longtemps major des gardes du corps de Louis XIV. Ce second marquis de
Masseran fut fait prince de l'Empire et de Masseran par la protection du
même duc de Savoie dont il avait épousé la bâtarde. Son fils épousa une
Simiane Pianezze, dont il eut un fils unique qui épousa, en 1686, une
bâtarde du duc Charles-Emmanuel de Savoie\,; car il y en a eu trois de
ce nom.

Le mariage du roi d'Espagne Philippe V avec une fille de Savoie fit
espérer à ce troisième prince de Masseran quelque fortune pour son fils
en Espagne. Il l'y envoya jeune et fort bien fait. On l'appelait le
marquis de Crèvecoeur. Il avait de l'esprit, de la galanterie, savait
mêler la réserve avec la hardiesse, avait grande envie de faire fortune
et tous les talents de courtisan qui y conduisent. Il s'attacha à faire
sa cour à la princesse des Ursins et à la reine\,; sa faveur pointa et
s'augmenta tellement auprès de l'une et de l'autre que le monde en
parla. Il n'en fut que mieux avec elles, et il en profita pour ménager
habilement les ministres et les plus grands seigneurs. Son père
mourut\,; il prit le nom de prince de Masseran, et la même faveur le
fit, tôt après, grand d'Espagne. Il fut un des six seigneurs affidés à
la princesse des Ursins, qu'elle laissa seuls approcher du roi d'Espagne
après la mort de la reine, et il eut l'adresse et le bonheur que la
chute de M\textsuperscript{me} des Ursins ne lui nuisit point auprès du
roi ni même de la nouvelle reine, avec qui je l'ai vu fort familier. Il
était gendre du prince de Santo-Buono, et il perdit sa femme comme
j'arrivais à Madrid, qui était belle et dame du palais de la reine, dont
il avait des enfants tout petits. Il en fut fort affligé, et demeura
toujours extrêmement uni avec son beau-père. C'était un homme
extrêmement aimable et un de ceux avec qui j'ai le plus vécu et le plus
familièrement. Il était fort ami des ducs de Veragua et de Liria, lié
avec Grimaldo et avec tout ce qu'il y avait de grand ou de plus choisi.
On disait pourtant qu'il ne fallait pas trop s'y fier\,; mais je n'ai ni
vu ni rien ouï dire qui pût autoriser ce bruit. En un mot, il était
aimé, considéré, désiré, reçu avec plaisir partout, même des plus
gourmés et des plus vieux seigneurs espagnols. Il avait de la grâce et
de la prudence en tous ses discours et en toutes ses manières, quoique
gai et libre et de la meilleure compagnie du monde. Depuis mon retour,
il alla faire un voyage en Italie et vint faire un tour en France, où
nous fumes ravis de nous retrouver. Il y fut peu, et dans ce peu, hommes
et femmes de la cour le couraient, et tout le monde fut affligé de son
départ. À son retour en Espagne il eut les hallebardiers de la garde,
qui sont comme nos Cent-Suisses, par la mort du marquis de Montalègre,
et longtemps après la compagnie des gardes du corps italienne, qui était
sa grande ambition, lorsque le duc d'Atri la quitta pour être
majordome-major de la reine à la mort du marquis de Santa Cruz, et
mourut assez jeune quelques années après dans cette charge. En arrivant
en Espagne je le trouvai ayant déjà la Toison d'or et la clef de
gentilhomme de la chambre.

Le vieux marquis Ferrero qui avait l'Annonciade, et qui a été
ambassadeur de Savoie auprès de Louis XIV, il y a fort longtemps, était
d'une branche cadette de cette maison. C'était un homme de beaucoup
d'esprit, de capacité et de mérite. Sa bisaïeule était aussi Fiesque.
Ces Ferrero ont eu quelques grandes alliances.

Melphe, \emph{Doria}, Génois, d'une des quatre grandes et premières
maisons de la république, transplanté à Naples.

Palagonia, \emph{Gravina, en} Sicile, d'une des plus grandes maisons du
pays.

Robecque, \emph{Montmorency}, branche sortie de celle de Fosseux. Le
second prince de Robecque quitta le service d'Espagne en 1678 et se mit
en celui de France, où il eut un régiment. Il mourut de maladie à
Briançon en Dauphiné, en 1691. Il avait épousé la soeur du comte de
Solre, chevalier du Saint-Esprit en 1688 et lieutenant général dont la
mère était soeur du père du prince d'Isenghien, gendre du maréchal
d'Humières. Il laissa deux fils. L'aîné, prince de Robecque, servit avec
réputation jusqu'à être maréchal de camp, puis passa au service de
Philippe V, qui le fit lieutenant général, lui donna la Toison d'or et
le fit, en 1713, grand d'Espagne. Il était extrêmement bien avec la
princesse des Ursins, qui cherchait à s'attacher les seigneurs
étrangers. Il épousa à Madrid, en 1714, la fille du comte de Solre, sa
cousine germaine, qui fut aussitôt dame du palais de la reine. Il
continua à servir et eut le régiment des gardes wallonnes, lorsque
Albéroni força le duc d'Havré à le quitter et à se retirer en France\,;
mais le prince de Robecque mourut un mois après, en octobre 1716, sans
enfants.

Son frère cadet, qui portait le nom de comte d'Estaires, servit avec
réputation longtemps en France. Il prit le nom de prince de Robecque à
la mort de son frère. Il eut la Toison d'or et succéda à sa grandesse,
dans le diplôme de laquelle il était compris. Il fut lieutenant général,
et au retour en France de la fille ce feu M. le duc d'Orléans, veuve du
roi Louis, il en fut nommé majordome-major par Philippe V. Il épousa
tout à la fin de 1722 Catherine du Bellay, morte en 1727, et lui,
quelques années après, tout à fait établi en France, et y a Bissé un
fils marié à une fille du duc de Luxembourg.

Sermonetta, \emph{Gaetano}, que nous prononçons Cajetan. Cette maison,
féconde en titres et en emplois, et toujours en grandes alliances, n'est
connue qu'après l'an 1200, par Mathias Cajetan, général des troupes du
bâtard Mainfroy, en Sicile, qui prit son nom c e la ville de Gaëte, au
royaume de Naples, dont on ne voit aucune raison. Son petit-fils fut
l'étrange Boniface VIII, qui n'oublia pas l'établissement de sa maison.
Ces grands d'Espagne n'y sont jamais venus et sont toujours demeurés à
Naples.

Sulmone, \emph{Borghèse}, de Sienne, famille d'avocats et de
jurisconsultes. Antoine Borghèse, fatigué des troubles domestiques de sa
patrie, se retira à Rome, y fut avocat consistorial\footnote{C'est-à-dire
  attaché à un des consistoires ou assemblées de cardinaux, qui
  servaient à la fois de conseils du pape et de tribunaux.}, et s'y
enrichit assez pour acheter à son fils aîné une charge d'auditeur de la
chambre fort chèrement, qu'il perdit fort peu après avec ce fils.
Clément VIII en eut pitié et donna sa charge à Camille son frère, qui
devint cardinal en 1594, à quarante-quatre ans, et pape Paul V, en 1605,
à cinquante-trois ans, et mourut, en janvier 1621, à soixante-huit ans.
Ce fut un terrible pape, qui éleva sa famille tout d'un coup en terres,
en titres, en grandes alliances, en richesses. Il fit le fils de son
frère prince de Sulmone, obtint pour lui la grandesse d'Espagne, et lui
fit épouser la fille du duc de Bracciano, chef de la maison des Ursins.
Celui d'aujourd'hui est le quatrième grand d'Espagne, dont les alliances
et les possessions se sont toujours accrues. Ces Borghèse, depuis Paul
V, ont toujours demeuré à Rome.

Surmia, \emph{Orleschalchi}. Innocent XI était fils d'in riche banquier
de Côme, dans le Milanais, et servit jeune dans les troupes impériales.
Il embrassa depuis l'état ecclésiastique, et l'argent de sa famille
l'avança dans les prélatures. Il fit sa cour, comme les autres, à la
fameuse dona Olympia, belle-soeur d'innocent X, qui pouvait tout sur le
pape et qui le fit cardinal en 1645, et il fut pape en 1676. Avec un
génie austère, borné, opiniâtre et un coeur tout autrichien, il s'y
abandonna avec une partialité qui le rendit odieux à tout ce qui n'était
pas vendu à la maison d'Autriche et la dupe de l'usurpation de
l'Angleterre par le prince d'Orange, qu'il favorisa d'argent et de tout
ce qu'il put, croyant ne favoriser {[}que{]} la maison d'Autriche contre
la France. S'il ne se servit pas de ses parents dans les affaires, il
fit pis de les abandonner au cardinal Cibo. Son neveu Odeschalchi en
était incapable, dont il fit un des plus puissants champignons de
l'Italie en possessions et en dignités, qu'il était bien raisonnable que
la maison d'Autriche lui prodiguât\,; l'empereur le fit prince de
l'Empire et traiter d'Altesse par tous ses dépendants à Rome et en
Italie, et Charles II le fit grand d'Espagne. Cette grandesse subsiste
encore dans je ne sais qui de sa famille, dont pas un n'a été en
Espagne.

J'ai oublié Ottaïano, \emph{Médicis}, d'une branche cadette et fort
séparée de celle des grands ducs de Toscane, et cinq générations avant
que celle-ci parvint à la souveraineté, et c'est la seule qui reste de
toute la maison de Médicis. Elle est depuis très -longtemps établie dans
le royaume de Naples et a toujours été méprisée par les souverains de
Toscane et par tout ce qui est sorti d'eux, les reconnaissant pourtant
toujours pour être Médicis comme eux.

Bernard de Médicis, baron d'Ottaïano, dans le royaume de Naples, épousa
une bâtarde d'Alexandre, duc de Florence, veuve de Fr.~Cantelmi. Il
était frère d'Alexandre de Médicis, archevêque de Florence, 1574,
cardinal, décembre 1583, à quarante-huit ans, pape, Léon XI, en avril
1605, mort le 27 des mêmes mois et année à soixante et dix ans. Ce même
frère de ce pape eut un fils, aussi baron d'Ottaïano qui, d'une
Saint-Séverin, eut deux fils qui, l'un après l'autre, furent princes
d'Ottaïano, qui épousèrent chacune un Carraccioli. L'aîné n'eut point
d'enfants\,; le cadet eut Joseph de Médicis, troisième prince
d'Ottoïano, fait grand d'Espagne en 1700, par Charles II, dont la
postérité masculine subsiste à Naples, d'où elle n'est point sortie\,;
princes d'Ottaïano, ducs de Sarno et grands d'Espagne.

Marquis de

Arizza, \emph{Patafox}.

Ayetona, \emph{Moncade}, colonel du régiment des gardes espagnoles.
Cette maison est une des plus grandes et des plus illustres d'Espagne,
indépendamment de ce qui peut être chimérique. Moncade est la première
baronnie de Catalogne, et est depuis plus de quatre cents ans dans cette
maison de mâle en male. Elle prétend venir d'un Dapifer\footnote{Ce nom
  latin désignait le grand officier, qu'on a appelé depuis sénéchal. Il
  servait à la table du roi et commandait l'armée en son absence. Dans
  la suite, les fonctions du sénéchal furent partagées entre le
  connétable et le grand maître de la maison du roi.}, général de
l'armée française au secours du pays de Barcelone contre les Sarrasins,
vers 733, dont le fils, Arnaud, fut investi par l'empereur Louis le
Débonnaire de la terre de Moncade, ce qui a été cause que les
successeurs de cet Arnaud, c'est-à-dire sa postérité, ont pris
indifféremment le nom de Dapifer ou celui de Moncade. Cette maison a
aussi possédé le Béarn et le Bigorre. Guillaume Ramon de Moncade épousa
Constance, fille de Pierre II, roi d'Aragon. Il était sénéchal de
Catalogne et fut le premier seigneur d'Ayétone, qui est, comme on l'a
dit, la première baronnie de la Catalogne. Il eut deux fils\,: Pierre de
Moncade, seigneur d'Ayétone et sénéchal de Catalogne, dont est descendue
la branche de Moncade et celles qui en sont sorties, demeurées en
Espagne, et Ramon de Moncade qui a fait la branche sicilienne des ducs
de Montalte, princes de Paterno, etc., dont les ancêtres y ont suivi les
Aragonnais et se sont établis à Naples et en Sicile. Ayétone est
toujours demeuré dans la branche restée en Espagne masculinement.

Je n'ai pu trouver la date ni le règne en Espagne de l'érection de la
grandesse d'Ayétone. Les différentes et les plus apparentes conjectures
et leurs combinaisons laissent peu de lieu de douter qu'elle ne soit la
première de l'érection de Philippe Il, vers 1560, et c'est par cette
raison que je l'y ai rangée. Ce qui ne peut être douteux est que les
Moncade, premiers seigneurs d'Ayétone et sénéchaux d'Aragon, en étaient
ricos-hombres\,; qu'ils ne passèrent point en grandesse sous
Charles-Quint, qui par là les abrogea tacitement, et furent rétablis en
grandesse par Philippe II. Celui que j'ai fort vu et pratiqué en
Espagne, et qui, avec son frère, le comte de Baños, qui en savait encore
plus que lui, m'a instruit de bien des choses, était le sixième marquis
d'Ayétone, qui avait une grande réputation de probité, de
désintéressement et de valeur la plus distinguée et la plus brillante,
et en même temps la plus simple, à laquelle néanmoins on prétendait que
les talents ne répondaient pas assez. Il était de tout temps fort
attaché à Philippe V, qui l'avait fait capitaine général de ses armées.
C'était un homme fort aimable dans la société, avec les manières du
monde, simples, nobles et polies, et l'air d'un grand seigneur. Lui et
son frère, que nous verrons, parmi les comtes, être grand par sa femme,
et veufs tous deux, n'avaient point de garçons, et des biens assez
médiocres. Le marquis d'Ayétone, depuis mon départ, maria sa fille
unique au marquis de Cogolludo, fils aîné du duc de Medina-Coeli, lequel
m'écrivit pour m'en donner part avec beaucoup d'amitié, quoique je ne
lui en eusse point donné du mariage de mon fils fait auparavant. Quoique
le marquis d'Ayétone portât le nom de Moncade, et non celui de Dapifer,
il ne portait point les armes de Moncade, qui sont de gueule à huit
besans d'argent en pal\footnote{Le \emph{pal}, en terme de blason, était
  une des pièces honorables de l'écu, qui représentait un pal, ou pieu
  posé debout, depuis le chef de l'écu jusqu'à la pointe.}, quatre de
chaque côté, mais il porte les armes de Bavière seules et en plein.
Cette chimère vient du nom de Dapifer, qui signifie le grand sénéchal,
et depuis, le grand maître, qui lui a succédé dans l'autorité intérieure
du palais, et non dans celle que le grand sénéchal avait dans le
royaume\,; ces charges héréditaires sont éteintes partout, excepté dans
l'Empire, où l'électeur de Bavière la possède, et par elle est électeur.
Cette similitude, tout étrangère qu'elle est, aura donné lieu à cette
singularité du marquis d'Ayétone\,; au moins n'en ai-je pu découvrir
d'autre raison\,; et pour la date de sa grandesse, c'est ce que je me
gardai bien de lui demander.

LosBalbazès, \emph{Spinola}, Génois, de l'une des quatre grandes maisons
de Gènes. Philippe III érigea cette terre, en 1621, en marquisat et
grandesse pour le fameux capitaine Ambroise Spinola fils de Philippe
Spinola, marquis de Venafro, et d'une Grimaldi, fille du prince de
Salerne. Il avait épousé une Bassadonna, et mourut en septembre 1630. Il
laissa le cardinal Spinola, mort en février 1639, une fille mariée au
premier marquis de Leganez, et Philippe Spinola, second marquis de Los
Balbazès, qui eut la Toison d'or, et qui épousa une fille de Paul Doria,
duc del Sesto, grand d'Espagne, qui lui apporta cette nouvelle
grandesse, et lui fit joindre le nom de Doria à celui de Spinola. Il
mourut en 1659. Son fils, né en février 1632, Paul Spinola-Doria,
troisième marquis de Los Balbazès et duc del Sesto, est celui qui se
trouva au mariage de Louis XIV, qui accompagna la cour depuis la
frontière d'Espagne jusqu'à Paris en qualité d'ambassadeur d'Espagne,
qui parut avec tant de magnificence et de galanterie à l'entrée du roi
et de la reine à Paris, et qui y fit admirer l'une et l'autre pendant
tout le cours de son ambassade. Il fut après du conseil d'État et de
celui de guerre, et majordome-major de la seconde femme de Charles II.
Il était gendre du connétable Colone, et mourut à Madrid, en décembre
1699, n'ayant pas encore soixante ans. Son fils, quatrième marquis de
Los Balbazès, fut gentilhomme de la chambre de Charles II, et général de
ses armées en Milanais. Il était gendre du huitième et dernier duc de
Medina-Cœli, des bâtards de Foix, qui mourut prisonnier à Fontarabie. Je
ne sais s'il eut peur de la disgrâce de son beau-père et d'être impliqué
dans ce dont on l'accusait\,; mais tout à coup il se fit prêtre avec
dispense de recevoir tous les ordres à la fois, dont on fut fort surpris
à la cour d'Espagne. Quelques-uns ont prétendu qu'outre cette raison,
car les prêtres sont fort difficiles à arrêter et à juger en Espagne
pour causes laïques, il avait des vues de se faire cardinal. Quoi qu'il
en soit, il vécut, depuis, peu d'années, et laissa le cinquième marquis
de Los Balbazès, que j'ai fort vu en Espagne, et qui était gendre du duc
d'Albuquerque et frère des duchesses de Medina-Coeli, d'Arcos, de La
Mirandole et de la princesse Pio.

Il avait de l'esprit, du monde, de l'application et des lettres, qui
n'empêchaient point beaucoup d'ambition, les talents de courtisan et
d'être plus mêlé avec le grand monde, où il était aimé et estimé par ses
manières nobles et polies, que ne le sont d'ordinaire les seigneurs
espagnols, et passait pour un fort honnête homme. Je l'ai beaucoup
fréquenté. Il fut gentilhomme de la chambre du prince des Asturies, à
son mariage, et l'était déjà du roi, et à la mort du prince Pio noyé
dans l'inondation de l'hôtel de La Mirandole, il fut grand écuyer de la
princesse des Asturies.

Bedmar, \emph{Bertrand La Cueva}. Cette maison a été expliquée au titre
d'Albuquerque\,; le marquis de Bedmar est cadet de cette maison. Il
servit presque toute sa vie au dehors de l'Espagne, en Italie et aux
Pays-Bas. Il y était capitaine général et gouverneur des armes à
l'avènement de Philippe V à la couronne d'Espagne, où on fut extrêmement
content de sa conduite, tant alors que depuis. Il y fut commandant
général pendant l'absence de l'électeur de Bavière, gouverneur général,
qui alla dans ses États, et le marquis de Bedmar roulait d'égal avec nos
maréchaux de France, commandait des armées séparées et aux troupes
françaises, comme aux espagnoles et wallonnes, comme à celles-ci
réciproquement nos généraux français. Il se conduisait si bien et
d'ailleurs avec tant de correspondance avec nos généraux et nos troupes
qu'il gagna entièrement leur amitié et leur estime par sa valeur et son
désintéressement, et par la magnificence avec laquelle il vivait. Louis
XIV lui en sut tant de gré qu'il lui donna l'ordre du Saint-Esprit en
1704, et le collier en 1705, en passant pour aller de Flandre vice-roi
de Sicile. Il fut le seul Espagnol pour qui le roi demanda et obtint la
grandesse. Je le trouvai en Espagne conseiller d'État et président du
conseil de guerre et de celui des ordres, et dans une grande
considération. On a vu qu'il fut premier commissaire d'Espagne pour la
signature des articles du contrat de mariage de l'infante avec le roi,
et, par très grande distinction, on lui apportait un siège chez le roi
d'Espagne, en attendant que Sa Majesté Catholique parût.

C'était un homme fort poli, dont toutes les qualités et les manières
étoient aimables, nobles, et d'un grand seigneur, en même temps polies
et familières. Il était goutteux, ne sortait guère de chez lui que pour
des fonctions, ou pour aller au palais, et avait presque toujours
compagnie chez lui\,; il avait de l'esprit, du sens, et tant vu au
dehors que sa conversation était également agréable, gaie, et
instructive. Je l'ai extrêmement vu et pratiqué à Madrid, où Leurs
Majestés Catholiques, les ministres, et tout le monde en faisaient
beaucoup de cas. Il se piquait fort d'aimer et de caresser les Français,
et d'une grande reconnaissance pour la mémoire de Louis XIV. Il avait
très bonne mine, et l'air fort français. J'admirai avec quelle facilité
il s'était remis à vivre à l'espagnol, à son puchero\footnote{Pot-au-feu,
  marmite.}, à manger seul un morceau, après avoir été un si grand
nombre d'années hors d'Espagne, à vivre avec tout le monde comme nous
vivons ici, et avec une grande et bonne table bien remplie de mets et de
convives.

Il n'avait qu'une fille unique mariée au marquis de Moya, second fils du
marquis de Villena, auquel elle porta cette grandesse. Elle était dame
du palais de la reine, et cruellement laide. Longtemps depuis mon
retour, le marquis de Moya, qui, avec peu d'esprit, mais une valeur
distinguée et beaucoup d'honneur, était fort dans le monde, devint par
la mort de son beau-père marquis de Bedmar, dont il prit le nom, et par
la mort de son père, capitaine des gardes du corps de la compagnie
espagnole, que son frère aîné quitta pour monter à la charge de
majordome-major du roi, qu'avait le marquis de Villena, leur père, qui
était une faveur sans exemple.

Camaraça, \emph{Los Cobos}. Il ne laisse pas d'y avoir en Espagne, comme
en France, des grandesses de faveur, et dont les races ne remontent pas
haut. Fr.~de Los Cobos était secrétaire d'État, favori de Charles-Quint
qui le fit conseiller d'État, grand commandeur de Léon de l'ordre de
Saint-Jacques, grand trésorier de Castille, et lui fit épouser M.
Mendoza y Sarmiento. Leur fils épousa Fr.-L., fille de Fr.~de Luna,
rico-hombre de Sangro en Aragon, et seigneur de Camaraça, laquelle en
fut faite marquise. C'est d'eux que sortent masculinement les Los Cobos,
marquis de Camaraça. Diego de Los Cobos, troisième marquis de Camaraça
mort tout à la fin de 1645, fut fait grand d'Espagne, et ne laissa
qu'une fille religieuse. Emmanuel de Los Cobos, appelé à sa grandesse,
lui succéda. Il sortait de mâle en mâle du frère cadet de Los Cobos,
premier marquis de Camaraça, il fut bisaïeul de Balthasar de Los Cobos,
cinquième marquis de Camaraça, chevalier de la Toison d'or, gentilhomme
de la chambre de Charles Il, général des galères de Naples, puis de
celles d'Espagne, enfin vice-roi d'Aragon. Sa mère, Acuña Portocarrero,
fille du troisième comte de Montijo, mourut, en 1694, camarera-mayor de
la reine-mère de Charles II.

Castel dosRios, \emph{Semmenat}, Catalan. C'est celui qui était
ambassadeur d'Espagne en France à la mort de Charles II, duquel il a
suffisamment été parlé à cette occasion, qui lui valut la grandesse et
la vice-royauté du Pérou, comme on l'a vu au même endroit. Il y mourut
après quelques années. Son fils aîné, connu ici avec lui sous le nom de
marquis de Semmenat, qui l'avait accompagné au Pérou, y resta fort
longtemps après sa mort, et, n'en est revenu en Espagne que depuis mon
retour où il fit aussitôt après sa couverture.

Castel-Rodrigo, \emph{Homodeï}. C'est une cité en Portugal. L. de Moura,
d'une maison noble et ancienne de ce royaume-là, alcade ou gouverneur de
cette cité, eut un fils Christophe de Moura, que Philippe II en fit
comte pour les services qu'il en avait reçus lorsqu'il s'empara du
Portugal, à la mort du cardinal-roi Henri. Le même Christophe de Moura
fut fait par Philippe III marquis de Castel-Rodrigo et grand d'Espagne.
Il avait été le premier vice-roi de Portugal pour l'Espagne. Son fils et
le fils de son fils ont été gouverneurs généraux des Pays-Bas\,; le
dernier mourut à la fin de 1675, gendre du sixième duc de Montalte, et
ne laissa que deux filles. L'aînée, veuve sans enfants d'un Guzman, fils
puîné du duc de Medina de Las Torres, se remaria à Ch. Homodeï, et la
cadette à Gilbert Pio, prince de Saint-Grégoire en Lombardie, dont elle
eut des enfants. Après sa mort elle se remaria à L. Contarini, alors
ambassadeur de Venise à Rome.

Les Homodeï sont des jurisconsultes, des citadins et des gens de robe de
Milan, connus dès 1340, et sont demeurés tels sans illustration ni
alliances jusque vers 1600, que Ch. Homodeï, extrêmement riche, se fit
marquis de Piopera, et poussa si bien un de ses fils dans les charges de
la prélature de Rome qu'il fut cardinal en 1652, et mourut en 1685.
C'est l'aîné de ce cardinal qui fut père de Ch. Homodeï, connu sous le
nom de marquis d'Almonacid, qui épousa la fille aînée de Moura,
marquise, héritière de Castel-Rodrigo\,; et qui, après avoir essuyé de
longues chicanes avec peu de fondement pour le droit, mais causées par
la légèreté de sa naissance, se couvrit enfin en 1679, par la grandesse
que sa femme lui avait apportée. Il se trouva homme d'esprit, d'honneur
et de mérite, et parvint sous Charles II à être conseiller d'État\,; il
se conduisit si bien à l'avènement de Philippe V à la couronne
d'Espagne, qu'il fut choisi pour l'ambassade de Turin, y négocier le
mariage du roi d'Espagne, et faire la demande pour lui de la fille de
Savoie, soeur cadette de M\textsuperscript{me} la duchesse de Bourgogne,
et l'amener au roi d'Espagne en Catalogne où il était pour lors prêt à
passer à Naples, et commander les armées en Lombardie. Castel-Rodrigo
fût déclaré grand écuyer de la reine en arrivant avec elle, et fut
toujours fort compté et considéré. À la mort de cette princesse, il
renonça à la cour, et se retira dans sa maison de Madrid. Il perdit
bientôt après sa femme. Ce changement domestique et de fortune lui
affaiblit la tête, tellement que lorsque j'arrivai à Madrid, il n'était
plus en état de paraître ni de voir personne chez lui. Je ne laissai pas
d'y aller à mon retour de Lerma, à cause de ma grandesse, et d'y
retourner avec mon second fils, quelques jours avant sa couverture,
comme c'est l'usage établi à l'égard de tous les grands. Je ne le vis
point, comme je m'y étais bien attendu, et comme il n'était plus en état
de rien, je ne reçus même contre la coutume aucune civilité ni
compliment de sa part.

Par la mort de sa femme, sans enfants, la grandesse de Castel-Rodrigo
passa à l'autre soeur, mère du prince Pio, quoique le mari veuf en
conserve le rang et les honneurs toute sa vie. Ainsi, après sa mère, la
grandesse vint au prince Pio qui fit sa couverture. C'est ce même prince
Pio, capitaine général et gouverneur de Catalogne, quoique jeune, dont
on a vu qu'Albéroni se joua si longtemps et si cruellement sur le
commandement de l'armée qu'il faisait assembler en Catalogne pour passer
en Sardaigne, etc., et le même que j'ai vu à Madrid, et qui fut fait
grand écuyer de la princesse des Asturies. C'était un grand homme fort
bien fait, poli, glorieux, ambitieux au possible, qui avait très bonne
opinion de soi, plus de valeur que de talents et d'esprit, quoiqu'il ne
manquât pas de l'un ni des autres. Il fut entraîné par le torrent qui,
depuis mon départ, inonda tout à coup l'hôtel de La Mirandole, et son
corps fut trouvé à une lieue de Madrid, dans une espèce de cloaque. Il
laissa des enfants fort petits. Il ne laissait pas d'être assez compté,
et fort parmi le monde. Il dansa et fort bien aux bals, car en Espagne,
comme je l'ai déjà dit, hommes et femmes dansent à tout âge.

Castromonte, \emph{Baeza}. C'est une famille de robe, et sans alliances
d'autour de Valladolid, inconnue et dans l'obscurité jusqu'à J. Baeza,
second marquis de Castromonte, dont la mère était Lara, et le frère aîné
mort sans enfants premier marquis de Castromonte. Ce second marquis fut
fait grand d'Espagne par Charles II, en janvier 1698, sans service, sans
charge, sans faveur précédente, et l'acheta fort cher à ce qu'ils
prétendent tous en Espagne. Il n'a point eu d'enfants de deux femmes. Le
fils de son frère lui a succédé et a des enfants. C'est un homme qui
paraissait fort peu, et que je n'ai fait qu'apercevoir en Espagne.

Clarafuente, \emph{Grillo}, à Gènes, de la première noblesse de la
république\,:

Santa-Cruz, \emph{Benavidez y Bazan}, majordome-major de la reine
seconde femme de Philippe V. La maison de Benavidez est masculinement
issue d'Alphonse IX, roi de Léon et d'Adonce Martinez, son épouse, par
don Alonzo, seigneur de Aliquer, leur fils cadet, dont le fils, Pierre
Alonzo de Léon, épousa l'héritière de Benavidez, issue d'Alphonse VIII,
empereur des Espagnes\,; d'autres donnent une autre origine à cette
maison, et la font descendre d'Inniguez, seigneur de Biedma, dans le
royaume de Tolède. Ils donnent une origine illustre à ce nom d'Inniguez,
de la délivrance d'une reine d'Aragon des mains des Mores. Cet Inniguez
épousa une Castro\,; les alliances directes de Ponce de Léon, et de
Sotomayor, furent celles du second et du troisième degré. Le quatrième
degré fut Mendus Rodriguez de Biedma et Benavidez.

C'est à celui-ci qu'il faut s'arrêter un moment. Il épousa 1° une
Tolède\,; 2° une Martinez\,; 3° une Cordoue\,; 4° apparemment par amour
la bâtarde d'une Manrique de Lara, archevêque de Tolède. Ce Mendus
Rodriguez de Biedma fit son premier mariage en 1344. Jusqu'à lui nulle
terre, nulle fille dans sa maison qui portât le nom de Benavidez, lequel
depuis lui qui le prit sans qu'on en puisse deviner la raison, passa à
toute sa postérité, sans qu'il y ait été jamais plus de mémoire de leur
ancien nom de Biedma\,: or, toute la maison de Benavidez descend de ce
Mendus Rodriguez, qui le prit le premier, parce que ses frères n'eurent
point d'enfants mâles, et que les mâles sortis de ses oncles et
grands-oncles s'éteignirent de son temps. Mais revenant à l'autre
origine des rois de Léon, la raison de ce changement de nom se
découvre\,: on a vu ci-devant que Pierre Alonzo de Léon, fils de Roderic
Alonzo, seigneur de Aliquer, fils cadet d'Alphonse IX, roi de Léon,
avait épousé l'héritière de Benavidez, issue d'Alphonse VII, empereur
des Espagnes. Leur fils, leur petit-fils, et leurs deux
arrière-petit-fils de mâle en mâle, ne prirent plus que le nom seul de
Benavidez. L'aîné des arrière-petit-fils mourut sans enfants, son seul
frère cadet fit un majorasque\footnote{Voy. sur les majorasques t. III,
  p.~247.} de plusieurs terres avec celle de Benavidez, auquel il donna
ce nom, et, se voyant sans enfants, il le substitua à son cousin Mendus
Rodriguez, seigneur de Biedma, à condition que ledit Mendus Rodriguez et
toute sa postérité ne porteraient plus que le nom seul de Benavidez. Or,
comment ce Mendus Rodriguez, seigneur de, Biedma, substitué au
majorasque et au nom de Benavidez était-{[}il{]} le cousin de J. Alonzo
de Benavidez issu de mâle en mâle des rois de Léon, fondateurs du
majorasque qu'il lui substitua\,? Était-ce parenté proche ou éloignée,
masculine ou féminine\,? Quoi qu'il en soit, il entra en possession de
ce majorasque en 1364. Deux ans après Henri IV, roi de Castille, en
démembra trois terres qu'il donna à Gonzalve Bazan, son favori et son
sommelier de corps, et donna en échange à Mendus Rodriguez de Benavidez,
la terre d'Iznotarafe, qui, pour avoir été conquise sur les Mores le
jour de Saint-Étienne, premier martyr, fut changée de nom, et toujours
depuis appelée San-Estevan del Puerto, ce dernier nom pour la distinguer
des autres de même nom parce que celle-ci est à une ouverture ou passage
de montagnes, et ces passages s'appellent \emph{puerto} en espagnol,
d'où vient par exemple le nom de Saint-Jean-pied-de-Port, et non de porc
comme dit le vulgaire, parce que cette place est au pied et à l'entrée
des Pyrénées du côté de France, à qui elle appartient. Cette terre de
San-Estevan, que Mendus Rodriguez eut en échange de ce qu'Henri IV, roi
de Castille, lui avait pris, était beaucoup plus considérable que ce
qu'il avait laissé prendre à ce roi.

Son arrière-petit-fils fut fait, en 1473, comte de San-Estevan del
Puerto, et fut père d'autre Mendus Rodriguez de Benavidez, comte de
San-Estevan del Puerto, duquel de mâle en mâle sont sortis les comtes de
San-Estevan del Puerto, grands d'Espagne, qu'on verra ci-après, et les
marquis de Santa-Cruz, leurs cadets. Le cinquième comte de San-Estevan
del Puerto, épousa, en 1548, une Cueva, qui lui apporta la terre depuis
marquisat de Solera, ce qui lui fit ajouter le nom de La Cueva au sien
et à ses descendants, comtes de San-Estevan. Son arrière-petit-fils,
huitième comte de San-Estevan et premier marquis de Solera, eut un frère
cadet Henri de Benavidez, marquis de Bajona et comte de Chinchon,
capitaine général des galères d'Espagne, et conseiller d'État qui épousa
Mencia Pimentel, dont le frère unique mourut sans enfants, et qui devint
héritière des marquisats de Santa-Cruz, Bajona et Viso par sa mère,
héritière de la maison de Bazan, ce qui fit ajouter le nom de Bazan à
celui de Benavidez à leur postérité, quelquefois même le prendre seul à
cause de la grandesse attachée au marquisat de Santa-Cruz pour le
grand-père paternel de l'héritière de Bazan, épouse d'un Pimentel qui
n'avait eu que cette fille héritière, qui épousa cet H. de Benavidez,
lequel en fut grand d'Espagne et grand-père du marquis de Santa-Cruz que
j'ai vu en Espagne, auquel je reviendrai après cette courte parenthèse.

Le grand-père de l'héritière de Bazan qui épousa le Pimentel, dont la
fille héritière porta la grandesse de sa mère à Henri de Benavidez,
frère cadet du huitième comte de San-Estevan, ce grand-père, dis-je,
était Alvar de Bazan, marquis de Santa-Cruz, ou Sainte-Croix, comme nos
Français l'appelaient, capitaine général de la mer, sous Philippe II. Ce
fut lui qui se rendit maître de l'escadre qu'après la mort du cardinal
roi de Portugal, Catherine de Médicis fit équiper pour porter un grand
secours en Portugal à Antoine, prieur de Crato, bâtard du duc de Beja,
second fils du roi Emmanuel de Portugal et d'une juive, qui voulut
prouver le mariage de sa mère, et après la mort du cardinal roi, se fit
proclamer roi à Santarem et Lisbonne, et eut un grand parti. Ses
aventures ne sont pas de mon sujet. Catherine de Médicis, qui, pour
relever sa naissance, se mit aussi sur les rangs sans nulle apparence de
fondement de prétendre à la couronne de Portugal, avait intérêt
d'afficher cette prétention, et d'empêcher la ruine du prieur de Crato,
comptant avoir meilleur marché de ce bâtard que de Philippe II. Comme
cette vanité de la reine la touchait sensiblement, et qu'elle était
toute puissante en France, ce fut à qui s'embarquerait sur cette escadre
de toute la noblesse de la cour, et Strozzi même, parent proche de la
reine, et fort avant dans ses bonnes grâces. Le marquis de Sainte-Croix,
ayant battu cette escadre, 26 juillet 1582, fit mettre pied à terre à
tout ce qui la montait, fit égorger de sang-froid dans l'une des
Terceires Ph. Strozzi qui la commandait, toute cette jeune noblesse et
tous les officiers, et emmena les vaisseaux et les équipages en Espagne.
Une si monstrueuse inhumanité fut détestée dans toute l'Europe, mais
elle plut si fort à Philippe II, qu'il fit aussitôt le marquis de
Santa-Cruz grand d'Espagne. Revenons maintenant au Benavidez qui jouit
{[}de cette grandesse{]}, après avoir passé par une autre maison.

Le marquis de Santa-Cruz que j'ai vu en Espagne était pauvre et retiré
chez lui dans la Manche, sous Charles II, et à l'avènement de Philippe V
à la couronne. Il avait essuyé un étrange contraste. Sa femme l'avait
accusé d'impuissance. Il y eut sur cela un grand procès\,; il le perdit,
et peut-être qu'il n'en fut pas fâché. Son humeur peu accorte ne
convenait guère au mariage. Il fut même permis à sa femme de se
remarier. Assez peu après, il fut attaqué par une fille bourgeoise pour
qu'il eut à se charger d'un enfant qu'elle prétendit qu'il lui avait
fait. Nouveau procès, et il le perdit encore. On voit qu'il n'était pas
heureux en procès.

Il vivait donc solitairement chez lui pendant les premières années du
règne de Philippe V, sans aucun accès à la cour ni à Madrid, malgré sa
naissance et sa dignité, lorsque le duc de Berwick vint la première fois
en Espagne où le feu de la guerre était de tous côtés. Il sut que le
marquis de Santa-Cruz, avec ce qu'il avait pu rassembler de ses vassaux,
avait si fermement combattu une partie de l'armée ennemie, à un passage
important de ce pays si montueux, qu'il l'avait arrêtée, et qu'après une
défense opiniâtre, il l'avait obligée à se retirer et à chercher où
passer ailleurs, ce qui, dans les circonstances où on se trouvait alors,
fut un service très utile. Le duc de Berwick en parla au roi d'Espagne,
lui fit donner du commandement, le fit venir à la cour, et lui procura
tous les agréments qu'il put. Santa-Cruz, d'abord sauvage, s'y
apprivoisa peu à peu, continua à servir avec distinction, mais sans
grade, il était trop vieux pour en vouloir, et s'attacha enfin à la cour
où il devint avec le temps, je n'ai point su par quelle intrigue,
majordome-major de la reine seconde femme de Philippe V, et parfaitement
bien avec le roi et avec elle. Il fut gentilhomme de la chambre seul
toute l'année en exercice avec le duc del Arco, et tous deux amis
intimes, qui, par leurs charges, passaient leur vie ensemble ou dans
l'intérieur du roi et de la reine ou à leur suite, à leurs chasses et à
leurs voyages. Il était fort des amis de Grimaldo, et témoigna toujours
au duc de Liria qu'il n'oubliait point ce qu'il devait à son père, avec
tendresse, intérêt et grande familiarité.

C'était un fort grand homme et bien fourni, un visage brun et rouge, de
gros sourcils noirs et des yeux qui regardaient volontiers de côté,
l'air et le jeu sournais et moqueur, beaucoup de fierté\,; tout montrait
en lui de la hauteur et de la noblesse jusque dans ses fonctions auprès
de la reine. Il n'était pas ignorant, avait beaucoup d'esprit et de
finesse dans l'esprit et dans les manières, et quoique mesuré, se
contraignait peu par grandeur sur les gens et sur les choses. Il se
communiquait fort peu, se retranchait sur l'assiduité de ses
fonctions\,; mais au fond c'était son goût et le fruit de la longue
solitude où il avait passé tant d'années. On le craignait pour ses dits,
pour sa morgue dédaigneuse, pour la difficulté de son accès même aux
lieux publics, au palais, encore plus son silence et ses yeux qui
parlaient de compagnie. Il ne laissait pas de parler un peu et de rire
même assez volontiers\,; mais toujours son rire était malin et
expressif. Il n'aimait point du tout les Français ni les Italiens, sans
que sa faveur et sa familiarité avec le roi et la reine en souffrissent
la moindre atteinte. Il se mêlait difficilement de quelque chose par
paresse et par dédain. Avec cela il avait des amis et de l'estime, et il
ne manquait ni aux devoirs ni à la politesse\,; mais il ne la prodiguait
pas, et en savait mesurer les degrés. Tout François et ambassadeur de
France que j'étais, j'étais parvenu à l'apprivoiser avec moi par le duc
de Liria, et par toutes sortes d'attentions et de prévenances au palais,
et j'avoue qu'il me plaisait fort, et me divertissait assez souvent,
quoique avare de discours et même de paroles, et il me paraissait qu'il
ne se déplaisait point avec moi. J'aurai lieu de parler de lui à
l'occasion de l'échange des princesses dont il fut chargé. Sur ses
dernières années, il fut fait chevalier du Saint-Esprit et de la Toison
d'or.

Laconi, \emph{idem}. Il était depuis longtemps aux Indes espagnoles
lorsque j'étais en Espagne.

Lede, \emph{Bette}. J'ai fort parlé de lui à l'occasion de l'expédition
de la Sardaigne et de la Sicile, dont le cardinal Albéroni le chargea en
chef, et dont il s'acquitta en capitaine, au retour de laquelle, quoique
malheureuse par la supériorité extrême de l'armée navale des Anglais et
de leurs troupes de débarquement, il fut fait grand d'Espagne, puis
envoyé en Afrique faire la guerre aux Mores, dont il s'acquitta avec
beaucoup de capacité et de bonheur. Je le trouvai en Espagne avec la
Toison d'or, dans la première considération et dans une grande estime.
Il vivait même avec assez de splendeur, avait une bonne table, et y
rassemblait les Flamands, d'autres étrangers, les Espagnols qu'il
pouvait, peu ou point de François, qu'il haïssait.

C'était un Liégeois sans naissance, qui s'était élevé par son courage,
son assiduité, ses talents pour la guerre, d'autant plus rapidement que
l'Espagne manquait de généraux, et il le devint excellent. Je n'ai guère
vu un plus vilain petit homme, en plus malotru, plus tortu, un peu
bossu, fort rousseau, l'air très bas, mais les manières nobles, avec de
l'esprit beaucoup, de la vivacité, de la hauteur, et le visage allongé,
décharné, le plus désagréable du monde. J'avais pris à tache de
l'apprivoiser, et j'y étais parvenu. Nous causions souvent ensemble au
palais, et il était de ceux qui venaient manger familièrement chez moi
sans prier. Sa, conversation était simple et agréable, souvent mêlée de
traits fort justes et fort naturels, quelquefois plaisants, quoique
sérieux et réservé. Depuis mon retour, il fit un voyage en Flandre où il
eut l'honneur d'épouser une Croï, qui n'avait rien, qu'il remena en
Espagne, lui sans s'arrêter à Paris, où elle fut danse du palais de la
reine, dont il a eu postérité.

Mancera\footnote{Voy t. III, p.~17.}.

Mondejar, \emph{Ivannez}. Cette terre, qui est en Castille, fut érigée
en marquisat et en grandesse d'Espagne, vers 1612, pour Innigo Lopez de
Mendoza, et tomba depuis en plusieurs maisons par des filles héritières.
Enfin celle de Cordoue et Mendoza l'apporta en mariage à Gaspard
Ivannez, comte de Tendilla, d'une naissance pourtant fort commune et peu
connue, qui prit le nom de marquis de Mondejar, et fit sa couverture en
1678\,; son fils épousa pourtant une soeur du connétable de Castille,
dont le fils était le marquis de Mondejar, du temps que j'étais en
Espagne, mais fort obscur et retiré.

Montalègre, \emph{Guzman}. C'est celui que j'ai vu en Espagne. Il
portait autrefois, du vivant de son père, le nom de marquis de Quintana,
et était majordome de semaine de Charles II, qui le prit en amitié et le
fit fort tôt gentilhomme de sa chambre. Sa faveur augmenta, en sorte
qu'il fut regardé comme un favori, et fut capitaine des hallebardiers de
la garde, enfin grand d'Espagne à la fin de 1697. Il conserva ces deux
charges à l'avènement de Philippe V à la couronne d'Espagne, où je le
trouvai sommelier du corps, mais sans nul exercice comme je
l'expliquerai en son lieu, et comme étaient presque toutes les charges
du palais. Il se trouvait, quand elle vaqua, le plus ancien de tous les
gentilshommes de la chambre. Cette raison, sa naissance, sa dignité, un
reste de teinte de ce qu'il avait été auprès de Charles II, l'élevèrent
à cette grande charge. C'était un bon et très honnête homme, fort
paresseux, fort retiré, par dégoût de n'avoir que le titre vain d'une si
belle charge, un esprit médiocre, peu à son aise, incapable de se mêler
de rien, doux et modeste, toutefois compté et considéré par estime, et
aussi par l'habitude de respecter fort les sommeliers, quoique celui-ci
n'en eût que la plus légère écorce. Il m'avait pris assez en amitié.
J'aurai lieu de parler de lui encore sur la fin de mon séjour en
Espagne. Son fils était gentilhomme de la chambre du roi.

Pescaire, \emph{Avalos}. Maison espagnole qui se prétend originaire de
Navarre, puis transplantée en Andalousie, où Loup Ferdinand d'Avalos fit
des prodiges de valeur contre les Mores grenadins, sous les rois de
Castille Ferdinand IV et Alphonse XI, qui l'en récompensèrent en biens
et en dignités qu'il transmit à ses descendants. Cette descente
masculine leur est contestée par des auteurs qui prétendent que cette
descendance finit en une fille héritière, appelée Mencia d'Avalos, qui
porta ses biens en mariage à Ruïs de Baeza y Haro, dont le fils s'appela
Roderic Lopez d'Avalos, et laissa le nom de son père pour prendre seul
celui de sa mère, comme lit après lui toute sa postérité.

Ce Roderic Lopez d'Avalos fut un homme illustre qu'Henri III, roi de
Castille, en fit connétable, en 1396, qui, entre autres enfants qui
firent des branches demeurées en Espagne, eut un fils cadet qui chercha
fortune auprès des rois d'Aragon, qui fut grand trésorier du royaume de
Naples, et qui épousa Ant. d'Aquino, soeur et héritière du marquis de
Pescaire. Ses enfants firent comme lui d'illustres alliances, qui se
soutinrent ou devinrent encore plus grandes dans sa longue postérité.
Alphonse d'Avalos, marquis de Pescaire et del Vasto après son frère
aîné, mort sans enfants, grand trésorier de Naples et général des armées
de Charles-Quint, Alphonse, dis-je, fut vice-roi de Naples et grand
d'Espagne\,; il mourut en 1546. Il laissa son fils aîné grand trésorier
de Naples, et vice-roi de Sicile, sixième aïeul du marquis de Pescaire à
Naples, du temps que j'étais en Espagne, d'où cette branche n'est point
sortie depuis son premier établissement dans ce royaume-là, et des
cadets dont l'un fut chancelier de Naples, cardinal en 1561, et mourut
en 1600, et l'autre fit la branche des princes de Montesarchio et de
Troja.

Richebourg, \emph{Melun}. Fr.-Ph. de Melun, fils puîné du second prince
d'Espinoy, et frère du troisième grand-père du dernier, mort sans
enfants, fait duc et pair de Joyeuse, et gendre du duc de Bouillon\,; ce
marquis de Richebourg, dis-je, eut la Toison d'or et le gouvernement et
grand-bailliage de Mons et de Hainaut, et mourut en 1690. Son fils porta
après lui le nom de marquis de Richebourg, passa en Espagne, y reçut la
Toison d'or, et fut fait grand d'Espagne par Philippe V, capitaine
général de ses armées, puis de Galice, après de Catalogne, enfin colonel
du régiment des gardes wallonnes. Il était dans ses gouvernements
lorsque j'étais en Espagne. Il n'a laissé que deux filles demeurées en
Flandre, qui ne se sont point mariées, et la grandesse s'éteint
nécessairement.

Ruffec, \emph{Saint-Simon}. Mon second fils, conjointement avec moi, et
pour en jouir ensemble l'un et l'autre, dont c'est le premier exemple en
Espagne.

Torrecusa, \emph{Carraccioli}. Voir p.~411-413, ce qui a été dit de
cette maison sur l'article des princes de Santo-Buono.

Philippe Carraccioli, des Carraccioli rouges, était troisième fils de
l'amiral Jean Carraccioli, frère de la mère du pape Boniface IX
Tomacelli. Ce même Philippe était frère d'H. comte de Gierace, grand
trésorier de Naples en 1348, de Gualterius, gouverneur de la Pouille, de
Louis, maréchal de l'Église romaine, et de Nicolas, général de l'ordre
de Saint-Dominique, cardinal 1376, mort 1389. Ce même Philippe épousa
Marcella Brancaccia, c'est-à-dire Marcelle de Brancas. D'eux est sortie
la branche des marquis de Vico et de Torrecusa, des comtes de Biecavi et
des ducs de Airola et de S.-Vito.

La septième génération de ce Philippe Carraccioli fut Lelius
Carraccioli, marquis de Torrecusa, dont le fils Charles-André, second
marquis de Torrecusa, mort en 1646, fut fait grand d'Espagne, bisaïeul
de celui que j'ai vu fort peu à Madrid, obscur, et qui passait pour un
fort pauvre homme, mais qui avait une femme d'esprit et de mérite, damé
du palais, aimée de la reine et fort considérée.

Villena, \emph{ducs d'Escalone, Acuña y Pacheco}. On peut voir plus
haut, au titre d'Ossone, ce qui est dit de cette grande, illustre et
nombreuse maison d'Acuña, et que les marquis de Villena, ducs
d'Escalone, en sont les aînés. Les titres de marquis de Villena et, de
duc d'Escalona ont toujours été dans cette maison sur la même tête. On a
fait remarquer plus d'une fois que les titres de duc, de prince, de
marquis et de comte sont entièrement indifférents en Espagne, et que
celui seul de grand y est tout. C'est ce qui a fait que ces aînés de la
maison d'Acuña, marquis de Villena et ducs aussi d'Escalope, grands
d'Espagne par l'un et par l'autre, ont préféré porter le nom de marquis
de Villena, parce que, étant le premier marquisat de Castille, cette
primauté, quoique sans rang et sans effet comme primauté, les a flattés,
et comme on l'a remarqué ailleurs, leur a donné occasion d'usurper la
singularité de signer \emph{El Marquez} tout court, sans y rien ajouter.
Ne pouvant donc traiter séparément deux titres qui ont toujours été
assemblés sur les mêmes têtes de ces aînés de la maison d'Acuña, j'ai
préféré de le faire sous celui qu'ils portent préférablement, quoiqu'ils
soient souvent désignés aussi par l'autre.

On a vu article d'Ossuna quels étaient les deux frères Jean et Pierre
d'Acuña, et d'où sortis\,; que Jean, aîné de la maison entière, fit la
branche de Villena, et Pierre celle d'Ossone, et les raisons qui
engagèrent ces deux frères et leur postérité à joindre au nom d'Acuña,
l'aîné celui de Pacheco, le cadet celui de Giron. Ce J. d'Acuña y
Pacheco, maître de l'ordre de Saint-Jacques, fut favori d'Henri IV, roi
de Castille, qui lui donna la terre de Villena qu'il érigea pour lui en
marquisat, et peu après, en 1469, érigea en sa faveur Escalone en duché,
à huit lieues de Tolède. En 1480 les rois catholiques, mécontents de ce
que son fils, second marquis de Villena, et second duc d'Escalone, avait
penché pour le roi de Portugal et Jeanne de Castille, pour la succession
à cette couronne, lui ôtèrent Villena, le réunirent à leur couronne où
il est toujours depuis demeuré réuni. Néanmoins les ducs d'Escalope, ses
descendants, n'y ont jamais renoncé, et pour marque de leur prétention
affectent, et on le souffre, de porter un titre dont ils n'ont plus la
terre {[}joint{]} à celui dont ils l'ont.

Le marquis de Villena, duc d'Escalone, que j'ai vu en Espagne, était
majordome-major du roi, et le seigneur d'Espagne le plus considéré, le
plus respecté et le plus digne de l'être. Il avait alors
soixante-quatorze ans, et une fort bonne santé. Il avoir été vice-roi et
capitaine général de Catalogne, de Navarre, d'Aragon, de Sicile, enfin
de Naples, où il reçut Philippe V, {[}étant{]} le huitième marquis de
Villena, duc d'Escalone, et le cinquième ayant la Toison d'or. J'ai
parlé de lui sur la bataille du Ter, où il fut battu, et sur la belle
défense qu'il fit dans le royaume de Naples, où à bout de moyens, il
soutint le siège de Gaëte si longtemps, et y fut pris enfin barricadé
dans les rues, les armes à la main, indignement traité et mis aux fers
par les Impériaux, irrités des obstacles et des retardements qu'il avait
mis à leur conquête, parmi la révolte et le manquement de troupes et de
toutes choses, et longtemps enfermé par eux à Pizzighitone, en sorte
qu'il avait les jambes tout arquées de ses fers, et marchait assez mal.
J'ai parlé de sa délivrance par la belle action de son fils aîné, qui la
procura devant Brighuela, à l'occasion de la prise de cette place, et de
la bataille de même nom, que les Espagnols gagnèrent\,; ainsi je n'en
répéterai rien. Enfin j'en ai parlé à l'occasion des coups de bâton
qu'il donna en présence de la reine et du roi, fort malade dans son lit,
au cardinal Albéroni, en sorte qu'il n'y a rien à en répéter ici. Je me
suis fait conter le dernier par lui, tel que je l'ai écrit, et il m'en
instruisit fort en détail avec modestie, mais avec complaisance. Avec
beaucoup de dignité, de gravité, les manières hautes, nobles, civiles,
mais avec poids, mesure et discernement\,; l'air simple, mais toutefois
très imposant\,; la taille médiocre, maigre, un visage majestueux\,:
tout sentait et montrait en lui un très grand seigneur, malgré sa
modestie et sa simplicité, et un seigneur devant lequel on voyait tous
les plus grands se ranger, lui faire place, lui céder sans qu'on en fût
surpris, même sans le connaître\,; tout cela avec un médiocre esprit,
aucun crédit et beaucoup des fonctions de sa charge retranchées. Il
n'était pas riche, avait une médiocre maison, mais une belle
bibliothèque. Il savait beaucoup, et il était de toute sa vie en
commerce avec la plupart de tous les savants des divers pays de
l'Europe. Il avait établi une académie pour la langue espagnole sur le
modèle de notre Académie française, dont il était le chef, qui
s'assemblait toutes les semaines, et qui dans les occasions
complimentait le roi comme les autres corps, comme fait la nôtre.
C'était un homme bon, doux, honnête, sensé, je le répète encore, simple
et modeste en tout, pieux solidement et sans superstition en homme bien
instruit, enfin l'honneur, la probité, la valeur, la vertu même. Son
père avait été vice-roi des Indes et de Navarre, et son grand-père
vice-roi de Sicile.

Ces marquis de Villena, ducs d'Escalona, avaient toujours fait les plus
grandes alliances. Celui-ci avait épousé la soeur du comte de
San-Estevan del Puerto, dont on parlera bientôt. Il avait marié son fils
aîné, comte de San-Estevan de Gormaz, à la soeur du comte d'Altamire,
dont la mère héritière de la marquise Folch, des ducs de Cardonne, était
camarera-mayor de la reine, et le marquis de Moya, son fils, à la fille
héritière du marquis de Bedmar. Le marquis de Villena était non
seulement le maître absolu dans sa famille, mais le patriarche de celles
où ses enfants s'étaient mariés. L'union entre toutes les trois était
intime, et il en était l'oracle et le dictateur. Le comte de San-Estevan
de Gormaz était un peu épais, peu d'esprit, courtisan, timide, capitaine
de la compagnie des gardes du corps espagnoles, et, à ce titre, fait
grand d'Espagne, du vivant de son père, lors de l'affaire du banquillo,
et majordome-major du roi, à la mort de son père, chose sans exemple en
Espagne. Il eut aussi sa Toison d'or et sa présidence académique.
C'était un honnête homme, et fort courageux, capitaine général, mais
sans talents pour les sciences et pour l'académie. Le marquis de Moya,
avec peu d'esprit, et force babil, était fort dans le monde. Il avait
défendu le palais de Madrid longuement, et avec un grand courage contre
les troupes de l'archiduc. Ces deux frères, quoique aimés tendrement de
leur père, chez qui ils demeuraient, étaient devant lui comme de petits
garçons, à qui il taillait les morceaux à mesure qu'ils en avaient
besoin.

Je m'étais attaché à mériter l'amitié du marquis de Villena, et j'y
étais parvenu. Je le voyais souvent, et j'y apprenais toujours quelque
chose de bon. Il fut presque le seul qui osât me venir voir à mon
quartier d'Almanzo\footnote{Ce village est appelé plus haut
  Villahalmanzo.}, après ma petite vérole, avant que j'eusse été à
Lerma, tant le roi la craignait. Il envoyait plus que le reste de la
cour savoir de mes nouvelles. Tant que j'ai été en Espagne, j'en ai reçu
toutes sortes d'amitiés, ainsi que de ses deux fils.

Visconti, \emph{idem}, à Milan. La grandesse est de 1679, pour César
Visconti, chevalier de la Toison d'or.

COMTES DE

Aguilar, \emph{Manrique de Lara}. Terre en Castille, donnée par le roi
Jean Ier de Castille, en 1385, à J. Ramirez d'Arellano, dit le Noble,
seigneur de Los Cameros, rico-hombre de Castille. Alphonse, de mâle en
mâle, arrière-petit-fils de J. Ramirez d'Arellano, en fut fait comte et
grand d'Espagne en 1475 par les rois catholiques. On a vu dans ce qui a
été expliqué sur la dignité de grands d'Espagne, qu'elle n'est connue
que depuis Charles-Quint, qui la substitua adroitement aux anciens
ricos-hombres, qui en avaient le rang et les honneurs, quels ils
étaient, et comment ils s'étaient multipliés à l'excès, enfin ce qu'ils
perdirent pour faire leur cour à Philippe le Beau, père de
Charles-Quint. Il faut donc entendre les grandesses avant Charles-Quint
des ricos-hombres, qui en avaient le rang et plus que les avantages, et
qu'on n'appelle ici grands et grandesses érigés avant Charles-Quint que
pour se conformer au langage d'aujourd'hui. On a vu encore dans cette
espèce de court traité de la grandesse, fait ici à l'occasion de
l'avènement de Philippe V à la couronne d'Espagne, que Charles-Quint, en
substituant la dignité de grand d'Espagne qu'il inventa à l'ancienne
dignité de ricohombre qu'il abolit, comprit les plus puissants des
ricos-hombres dans ces nouveaux grands d'Espagne, et n'y comprit point
ceux qu'il crut pouvoir ne pas ménager, qui de fait demeurèrent
dégradés. Apparemment que les comtes d'Aguilar furent de ce nombre, puis
dès le fils de celui qui avait été fait comte d'Aguilar, et grand
d'Espagne, pour continuer à s'exprimer dans le langage connu, ce fils et
sa postérité cessèrent de jouir du rang et des honneurs de grand
d'Espagne jusqu'au 6 janvier 1640, que Philippe IV les rendit à J.
Ramirez d'Arellano, huitième comte d'Aguilar. Cette maison d'Arellano
était pourtant bien grande et bien illustre, puisqu'elle descendait
masculinement de Sanche Ramirez, seigneur de Peña Cerrada, frère de
Garcias, dit le Restaurateur, roi de Navarre, mort en 1151. C'était
peut-être pour cela même que Charles-Quint, la voulut abaisser et
confondre. Leurs armes mêmes étaient très singulières, et ne pouvaient
avoir été prises sans quelque cause curieuse que je n'ai pu découvrir.
Elle n'écartelait point, et portait l'écu parti de gueules d'or à trois
fleurs de lis de l'un en l'autre, deux et une, et celle-ci mi-partie de
l'un en l'autre, ces fleurs de lis faites comme celles que nos rois
portent aujourd'hui.

Ce J. Ramirez d'Arellano, huitième comte d'Aguilar, rétabli grand
d'Espagne par Philippe IV en janvier 1640, épousa la fille unique,
héritière de J. de Mendoza, premier marquis de Saint-Germain et de
Hinoyosa, dont il eut le neuvième comte d'Aguilar, qui mourut en 1668,
et d'une fille du huitième comte d'Oñate, qui était Guévara, ne laissa
qu'une fille qui porta sa grandesse avec Aguilar Hinoyosa Los Cameros,
etc., en mariage, en 1670, à Roderic Emmanuel Manrique de Lara, comte de
Frigilliane, duquel j'ai amplement parlé en traitant des conseillers
d'État et seigneurs distingués d'Espagne, à l'occasion du testament de
Charles II et de l'avènement de Philippe à la couronne d'Espagne. J'ai
aussi parlé à la même occasion du comte d'Aguilar, son fils, en celle du
premier siège de Barcelone, qu'il vint proposer au feu roi, et qui eut
de si fâcheuses suites, à l'occasion de Flotte, et de Renaud qu'il fit
arrêter dans l'armée que commandait le maréchal de Besons en Espagne, à
qui il ne le vint dire qu'après l'exécution faite à son insu\,; enfin à
l'occasion de la disgrâce commune du duc de Noailles et de lui,
lorsqu'ils voulurent donner une maîtresse au roi d'Espagne pour faire
tomber le crédit de la reine et celui de la princesse des Ursins, qui
gouvernait, et par la maîtresse régner eux-mêmes. Son caractère exposé
en ces différentes occasions me dispensera de le retoucher ici. Je me
contenterai de dire seulement que c'était l'homme de toutes les Espagnes
qui avait le plus d'inquiétude d'esprit, et d'ambition, à qui les moyens
coûtaient le moins, et qui était le plus dangereux\,; aussi le duc de
Noailles et lui se sentirent d'abord l'un l'autre dès qu'ils se virent,
et lièrent une amitié la plus intime qui a duré autant que leur vie. Il
ne me reste donc plus qu'à dire ce qui est arrivé à ce comte d'Aguilar
depuis cette disgrâce commune avec le duc de Noailles en 1710. Ce comte
d'Aguilar avait été successivement et rapidement à la tête des finances,
des affaires de la guerre, commandé en chef, et capitaine général des
armées, colonel du régiment des gardes espagnoles, enfin capitaine de la
compagnie espagnole des gardes du corps qu'il perdit par cette disgrâce,
et qui fut donnée au comte de San-Estevan de Gormaz, fils aîné du
marquis de Villena. Exilé dans une riche commanderie de l'ordre de
Saint-Jacques, dont il était grand chancelier, et avait pour cela quitté
la Toison d'or par une avarice qui lui fit grand tort dans le monde, il
intrigua tant qu'il obtint de servir la campagne suivante, à condition
de n'approcher point de Madrid ni de la cour. L'Altesse donnée à la
princesse des Ursins et au duc de Vendôme qui indigna toute l'Espagne,
et qui en outra tous les grands, fut plus sensible au comte d'Aguilar
qu'à pas un, parce que, servant dans son armée, il ne pouvait éviter de
lui donner cet étrange traitement qui jamais n'a appartenu qu'aux
infants et au bâtard don Juan d'Autriche, qui l'usurpa dans les troubles
qu'il excita pendant la minorité de Charles II et le parti qui l'éleva
jusqu'à arracher le gouvernement d'entre les mains de la reine-mère
régente. Pendant cette campagne de 1711, le duc de Vendôme mourut fort
brusquement et fort solitairement à Vignaroz, au bord de la mer, comme
on l'a vu en son lieu, et cru empoisonné sans aucun doute. Aguilar eut
le malheur d'en être fort publiquement accusé, et fut renvoyé dans sa
commanderie pour n'en plus sortir. Quoique la mort du duc de Vendôme eût
été reçue avec une joie marquée par tout ce qui était distingué en
Espagne en dignité ou en naissance, par l'extrême dépit de ce traitement
d'Altesse, Aguilar, craint et haï de grands et de petits, ne trouva
point de protecteurs, de sorte qu'il passa bien des années sans sortir
de sa commanderie. Vers 1720, il obtint permission de venir faire un
tour court à Madrid, sous prétexte d'affaires et de santé, à condition
de ne se présenter pas devant Leurs Majestés Catholiques. Dans le peu
qu'il y séjourna il se jeta à la tête du parti italien, dont je parlerai
bientôt, et il lui fut permis après de venir à Madrid, pendant l'absence
de la cour, qui était à Lerma, puis d'y faire quelque petit séjour, mais
en s'y montrant sobrement, et à la fin de se présenter une fois devant
Leurs Majestés Catholiques au palais.

C'était un très méchant homme sur qui personne ne pouvait compter, mais
si plein d'esprit, de nerf, d'ambition et de ressources qu'il n'était
pas à mépriser. Ainsi par ces raisons, je fus conseillé d'envoyer lui
faire compliment par un gentilhomme comme à un seigneur que j'avais vu à
notre cour autrefois. Dès le lendemain, il m'en envoya un me remercier
et s'excuser sur son indisposition de n'être pas encore venu me rendre
ses devoirs, dont il s'acquitterait incessamment. En effet, il me vint
voir deux jours après, et me trouva. Je la lui rendis promptement, et le
trouvai seul. Tout se passa en compliments et en discours de philosophe
de sa part, de retraite, etc. Je n'en voulais pas davantage\,; il s'en
retourna tôt après à sa commanderie sans avoir réitéré nos visites. Je
découvris sans peine un homme piqué, frétillant, désolé de son exil,
abattu de santé, et cachant ce qui s'en montrait, malgré lui, sous des
propos de la satisfaction qui se trouve dans le repos et dans la
jouissance de soi-même. Son exil s'est adouci depuis, mais la disgrâce a
duré jusqu'à sa mort, qui n'est arrivée que plusieurs années depuis mon
retour.

Le duc de Noailles et lui ont toujours été en commerce de lettres, et le
roi et la reine d'Espagne le savaient et le trouvaient très mauvais, et
toutefois les laissaient faire avec une sorte de mépris pour tous les
deux. Le comte d'Aguilar était gendre du septième duc de Monteléon
Pignatelli, qui, peu après l'arrivée de Philippe V en Espagne, s'était
retiré à Naples, où il avait pris le parti de la maison d'Autriche, à
laquelle il était demeuré attaché le reste de sa vie.

La maison de Manrique de Lara ne cède à aucune autre en Espagne en
ancienneté et en grandeur d'origine, en alliances, possessions, en
dignités et en emplois\,; elle descend de mâle en mâle des comtes
souverains de Castille, qui sortaient de même des rois des Asturies et
de Galice. Ils ont donné des reines à la Navarre, à Léon et à la
Castille, et ils en ont épousé des filles. Ils ont été vicomtes de
Narbonne, de la branche desquels est sortie celle de ces derniers comtes
d'Aguilar\,; enfin ils sont immédiatement alliés de tout temps aux plus
grands et aux plus puissants de tous les ricos-hombres du Portugal et de
tous les royaumes particuliers qui composent aujourd'hui celui des
Espagnes, dont le détail ferait un volume.

Altamira, \emph{Ossorio y Moscoso}. Roderic de Moscoso, seigneur
d'Altamire, perdit son fils unique tout jeune, et eut deux filles.
Agnès, l'aînée, épousa Vasco Lopez d'Olloa, dont un fils créé par Jean
II, roi de Castille, comte d'Altamire, qui eut un fils mort jeune, à qui
succéda la soeur cadette de sa mère Urraque de Moscoso, femme de Pierre
Alvarez Ossorio, fils puîné du premier comte de Transtamare, et frère du
premier marquis d'Astorga. C'est de ce mariage que descend de mâle en
mâle le comte d'Altamire que j'ai vu en Espagne\,; il en est le neuvième
comte, et cette grandesse, érigée pour son trisaïeul paternel de mâle en
mâle, est vers 1610. Son père mourut en 1698 à Rome, ambassadeur de
Charles II, après avoir été vice-roi de Naples\,; et sa mère fille du
sixième duc de Segorbe et de Cardonne, de la maison Folch, était de mon
temps, et longuement depuis, camarera-mayor de la reine avec une très
grande considération.

Ce comte d'Altamire son fils était fort jeune, et néanmoins fort
considéré, lorsque j'étais en Espagne. Il était bien fait, appliqué, peu
répandu, de l'esprit, de la conduite, fort grave, fort dévot, fort
mesuré, fort espagnol, et regrettant toutes les étiquettes, fort homme
d'honneur, l'air d'un grand seigneur, mais un air un peu embarrassé et
très réservé, et une politesse qui semblait vouloir bien faire à travers
la crainte d'en trop faire. Il fut sommelier de corps du roi Louis,
après l'abdication de Philippe V, son favori dans ce court règne, au
point qu'il aurait tout gouverné. Il avait déjà rétabli toutes les
étiquettes espagnoles et aboli tout ce qui n'était pas des manières et
des coutumes antiques. On pouvait dire de lui que c'était un jeune
seigneur qui n'avait point vieilli depuis le temps de Philippe II. Il
fut nommé chevalier du Saint-Esprit avant l'âge, et mourut bientôt après
sans l'avoir encore reçu et sans avoir été marié. On commençait déjà de
mon temps à le compter beaucoup\,; il savait et s'appliquait fort à la
lecture, et je ne sais qui aurait pu l'apprivoiser.

Aranda, \emph{Roccafull}. Cette terre en Aragon a été possédée
premièrement en comté par lope Ximenez de Urrea, et passa par sa fille
dans la maison d'Heredia, dont le cinquième comte d'Aranda fut fait
grand d'Espagne vers 1590. Cette grandesse est enfin tombée par des
héritières en 1696 à l'héritière Henriette-Françoise d'Heredia et Urrea
qui la porta en mariage à Guillaume de Roccafull, et Rocaberti, comte
d'Albaterre. MM. de Roquefeuille qui sont François, et en France, et ont
eu un grand maître de Malte, prétendent être de même maison que les
Roccafull d'Espagne.

LosArcos, \emph{Figuerroa y Laso de La Vega}. Philippe III l'érigea en
comté pour Pierre, quatrième fils de Gomez Suarez de Figuerroa et
d'Elvire Laso de La Vega, lequel Pierre avait épousé Blanche de
Sotomayor, dame de Los Arcos\,: c'est le troisième comte d'Arcos, sorti
de mâle en mâle du premier qui fut fait grand d'Espagne, en 1697, par
Charles II, et c'est son fils que j'ai vu, mais assez peu en Espagne.

Atarez, \emph{Villalpando}, de Philippe V.

Banos, \emph{Moncade}. Gonzalve, marquis de Landrada, second fils de J.,
cinquième duc de Medina-Cœli, et frère du sixième des bâtards de Foix,
eut un fils aîné marié à M. A. I., héritière de Leyva et de Baños. Il en
devint veuf, fut vice-roi du Mexique, et se fit carme en 1676. Son fils
aîné, comte de Baños et marquis de Landrada, grand écuyer de Charles II,
fut fait par lui grand d'Espagne en novembre 1692. Il ne laissa qu'une
fille qui apporta cette grandesse en mariage à Emmanuel de Moncade,
comte de Baños par elle, frère du marquis d'Ayétone, duquel j'ai parlé
au titre d'Ayétone. Il avait servi avec distinction, et avait perdu une
jambe, mais par accident. Il n'avait qu'une fille non plus que son
frère.

Benavente, \emph{Pimentel}. Cette maison est des plus grandes et des
plus illustres de Portugal. J. Alphonse Pimentel avait épousé J. Tellez
de Menesez qui lui avait apporté la ville et terre de Bergança, laquelle
était fille du comte de Barcellos, et soeur d'Éléonore, femme de
Ferdinand, roi de Portugal. Ce Pimentel passa de Portugal en Castille
avec l'infante Béatrix, femme de Jean, premier roi de Castille. Henri
III, roi de Castille, lui échangea Bergança pour Benavente en Léon, et
l'érigea en comté en récompense de ses services, entre autres d'avoir
défendu Bergança jusqu'à la dernière extrémité contre le roi Jean de
Portugal. Cet échange et érection est de 1398, et c'est le titre de la
grandesse qui est toujours depuis demeurée dans sa postérité masculine.

J'ai fort parlé du douzième comte de Benavente à l'occasion des
seigneurs principaux qui étoient lors du testament de Charles II et de
l'avènement de Philippe V à la couronne d'Espagne\footnote{T. III, p.~3
  et suiv. Ce passage avait été supprimé dans les anciennes éditions,
  qui ne laissent pas de s'y référer.}. Celui-ci, qui était sommelier du
corps de Charles II, et qui le demeura de Philippe V, fut de la junte de
la régence par le testament, et dans la suite fut un des cinq premiers
Espagnols à qui Louis XIV envoya le collier du Saint-Esprit. Il était
gendre du comte d'Ouate Guevara, et mourut fort vieux et fort considéré,
et dans sa charge. Je n'ai point vu son fils qui avait épousé une soeur
du duc de Gandie-Borgia. Il passait sa vie reclus dans ses terres dans
une extrême dévotion, affolé des jésuites dont cinq ou six l'y
assiégeaient toujours. Il y tenait sa femme et ses enfants auxquels il
ne donnait rien, ne voulait voir personne, et désolait sa famille et
toute sa parenté, qui, avec tous leurs efforts, n'avaient pu le tirer de
cette obscurité ni le persuader de marier pas un de ses enfants, quoique
fort riche. Ce qui est étrange, c'est qu'ils disaient tous qu'il avait
de l'esprit et du savoir, et pestaient tous contre les jésuites qu'ils
prétendaient l'avoir ensorcelé\,; ses soeurs étaient les duchesses de
Medina-Sidonia et d'Hijar.

Castrillo, \emph{Crespi}.

Egmont, \emph{Pignatelli}. Egmont est en Hollande, d'où une des plus
grandes et des plus illustres maisons des Pays-Bas a tiré son origine et
son nom de cette seigneurie. La souveraineté de Gueldre et de quelques
autres pays a été un assez court espace de temps dans une branche de
cette maison qui s'éteignit après l'avoir perdue. Ses autres branches
s'attachèrent à la maison d'Autriche, et eurent de grands emplois, de
grands honneurs, de grands biens, mais des honneurs par les dignités. Je
n'ai pu démêler si leur grandesse est de Charles-Quint, comme il est
assez apparent, ou de Philippe II. La dernière branche de cette maison
s'éteignit en la personne du dernier comte d'Egmont, en 1707, qui, à
l'avènement de Philippe V à la couronne d'Espagne, suivit le sort des
Pays-Bas, qui se soumirent à ce nouveau monarque. Il servit en France et
en Espagne avec beaucoup de valeur et de distinction, était lieutenant
général et chevalier de la Toison d'or. Il avait épousé en 1697, à
Paris, M\textsuperscript{lle} de Cosnac, nièce paternelle du célèbre
archevêque d'Aix, commandeur du Saint-Esprit et parente fort proche de
la duchesse de Bracciano, si connue depuis sous le nom de princesse des
Ursins, qui fit ce mariage, et qui logeait M\textsuperscript{lle} de
Cosnac chez elle, à Paris, où elle était alors. Le père de ce dernier
comte d'Egmont mourut à Cagliari en 1682, vice-roi de Sardaigne, était
arrière-petit-fils du comte d'Egmont à qui le duc d'Albe fit couper la
tête, et au comte d'Horn, à Bruxelles, 1568. Par la mort du dernier
comte d'Egmont sans enfants, de M. Ang. de Cosnac, à Fraga, en
Catalogne, 15 septembre 1707, dans l'armée d'Espagne, sa succession et
sa grandesse vint à l'aînée de ses soeurs mariée à Nicolas Pignatelli,
duc de Bisaccia au royaume de Naples et à leur postérité. Ce dernier
comte d'Egmont mourut à trente-huit ans, et sa veuve à quarante-trois, à
Paris, en 1717, et cette grande maison d'Egmont fut éteinte.

Nicolas Pignatelli, quatrième duc de Bisaccia, épousa en 1695 la soeur
aînée du dernier comte d'Egmont, qui en devint en 1707 l'héritière. Lui
et le prince de Cellamare, dont il a été tant parlé ici, étaient amis
intimes et enfants du frère et de la soeur, et le père de ce duc de
Bisaccia et le pape Innocent XII étaient enfants des issus de germain.
Nicolas, duc de Bisaccia, mari de l'héritière d'Egmont, s'attacha au
service de Philippe V, et s'y distingua fort. Il fut pris dans Gaëte,
combattant aux côtés du marquis de Villena, et conduit avec lui dans les
prisons de Pizzighetone. Il perdit sa femme en 1714, et vint s'établir à
Paris, où il maria son fils unique à la seconde fille du feu duc de
Duras, fils et frère aîné des deux maréchaux ducs de Duras, qui a pris
le nom et les armes de sa mère, avec ses biens et sa grandesse. Sa soeur
a épousé le duc d'Aremberg, grand bailli et gouverneur de Mons et du
Hainaut pour l'empereur. Ce comte d'Egmont, après la mort à Paris du duc
de Bisaccia, son père, fit un voyage à Naples, où il mourut, laissant
deux fils, dont l'aîné, comte d'Egmont, et grand d'Espagne, a épousé la
fille unique du duc de Villars, fils unique du maréchal duc de Villars,
dont il n'a point d'enfants\,; il a un frère\,; tous deux dans le
service du roi. Leur branche est la cadette de toute la maison
Pignatelli.

San-Estevan de Gormaz, \emph{Acuña y Pacheco}, fils aîné du marquis de
Villena, dans l'article duquel on trouve tout ce qui regarde ce fils,
fort distingué par sa valeur et ses actions, et par sa probité, peu par
ses talents, d'esprit assez court et courtisan timide. Je l'ai fort vu
et pratiqué en Espagne.

San-Estevan delPuerto, \emph{Benavidez}. On a vu ci-devant, à l'article
de Santa-Cruz, quelle est la maison de Benavidez, et de quelle de ses
branches sont issus les comtes de San-Estevan del Puerto enfin l'origine
du nom de San-Estevan del Puerto. Je me contenterai donc de dire que le
neuvième comte de San-Estevan del Puerto, frère de l'épouse du marquis
de Villena, duc d'Escalope, fut un homme de beaucoup d'esprit, de traits
plaisants et en même temps de capacité. Il fut capitaine général du
royaume de Grenade en 1672, et en 1678 vice-roi de Sicile, dont il
éteignit et punit à Messine les restes de la révolte passée\,; vice-roi
de Naples, en 1687, qu'il quitta au duc de Medina-Coeli, en 1696, et en
arrivant à Madrid il fut fait grand d'Espagne par Charles II, conseiller
d'État et grand écuyer de la reine palatine. Il se conduisit si bien à
la mort de Charles II, et à l'arrivée de Philippe V en Espagne, qu'il
fut majordome-major de la reine sa première femme. Il mourut fort vieux
et fort considéré, sans enfants. Son frère, appelé à sa grandesse,
quitta force bénéfices, lui succéda, se maria, et eut un fils qui est le
comte de San-Estevan del Puerto, qu'on a vu premier ambassadeur
plénipotentiaire d'Espagne au congrès de Cambrai, gouverneur et premier
ministre de l'infant don Carlos en Toscane, enfin chevalier du
Saint-Esprit, et grand écuyer du prince des Asturies. Je n'ai point vu
son père ni lui en Espagne.

Fuensalida, \emph{Velasco}, terre en Castille. Henri IV, roi de
Castille, la fit comté pour Pierre Lopez d'Ajala. Bernardin de Velasco y
Roïas et Cardenas, fils de la soeur et héritière du sixième comte de
Fuensalida Ajala, mort sans enfants, lui succéda et quitta le nom de
Folmenar qu'il portait pour prendre celui de comte de Fuensalida. Son
fils fut successivement vice-roi de Navarre, de Sardaigne, de Galice, et
gouverneur général de Milan. Il ne faut pas omettre qu'il avait un frère
aîné, mort sans enfants, à qui il succéda. Charles II le fit grand
d'Espagne vers 1670\,; c'est son petit-fils de mâle en mâle que j'ai vu
à Madrid, mais peu, et j'en ai encore ouï moins parler. C'était un grand
garçon, assez bien fait, de vingt-six ou vingt-sept ans. J'ai parlé de
la maison de Velasco au titre des ducs de Frias, connétables de
Castille.

Lamonclava, \emph{Bocanegra y Portocarrero}. Louis Bocanegra y
Portocarrero, fait comte de Palma, en 1507, épousa 1° une Tellez-Giron,
fille du comte d'Urena, en 1499\,; et en secondes noces Éléonore Laso de
La Vega, fille du seigneur de Los Arcos. Du premier lit, il eut un fils
qui continua les comtes de Palma, et une fille religieuse\,; du second
lit il eut Antoine, seigneur de Lamonclava, duquel est sortie cette
branche. Son petit-fils fut fait comte de Lamonclava, et eut Melchior,
second comte de Lamonclava, que Charles II fit grand d'Espagne vers
1693, et l'envoya gouverneur de la Nouvelle-Espagne. Il eut des fils
d'une Urena, fille du seigneur de Berbedel, qu'il avait épousée, qu'il
emmena avec lui en Amérique, où il mourut, et qui y sont restés,
tellement que, lorsque j'étais en Espagne, ils étaient encore aux Indes
Occidentales\,; je ne sais si le comte de Lamonclava en est revenu
depuis. Je remets à parler des maisons Bocanegra et Portocarrero à
l'article de Palma.

Lemos, \emph{Portugal y Castro}. On a tâché d'expliquer, t. III, p.~88
et suiv., les branches royales de Portugal, Oropesa, Lemos, Veragua,
Cadaval, etc.\footnote{Ces passages ont été omis dans les anciennes
  éditions, et cependant les éditeurs n'ont pas manqué d'y renvoyer le
  lecteur.}, ainsi on n'en répétera rien. Lemos en Galice a passé dans
plusieurs maisons par des héritières, et tomba par cette voie à Pierre
Alvarez Ossorio, seigneur de Cabrera et Ribera, qui en fut fait comte en
1457, par Henri IV, roi de Castille. Son fils mourut avant lui, qui ne
laissa qu'un bâtard, lequel fut héritier de son grand-père. Ce bâtard,
second comte de Lemos, ne laissa que deux filles\,; l'aînée hérita de
Lemos et des biens de son père, et Denis de Portugal, fils puîné du
troisième duc de Bragance, n'eut pas honte à la maurisque de l'épouser.
Aussi était-il lui-même de race bâtarde, quoique couronnée. C'est de lui
que sont masculinement venus les comtes de Lemos, grands d'Espagne,
jusqu'à présent. J'ignore la date de cette grandesse, qu'on peut
vraisemblablement attribuer à Charles-Quint.

C'est le onzième comte de Lemos que j'ai vu en Espagne\,; il avait été
vice-roi de Sardaigne, et capitaine général des galères de Naples, sous
Charles II, qui lui avait donné aussi la Toison d'or. On peut voir dans
l'article de l'Infantado ce qui est dit de sa conduite, et de celle de
la duchesse sa femme, soeur du duc de l'Infantado, à l'égard de Philippe
V. Ce comte de Lemos avait de l'esprit, et se faisait craindre par la
liberté de ses traits. D'ailleurs son extrême paresse et sa parfaite
incurie l'empêchait de faire usage de son esprit, et le tenait renfermé
à fumer sans cesse, chose fort extraordinaire pour un Espagnol\,: aussi
n'était-il compté pour rien. Sa femme l'était et fort considérée\,; sa
figure était agréable, et sentait extrêmement ce qu'elle était. Elle
avait de l'esprit, du sens, de la politesse, de l'intrigue, aimait la
conversation et le monde, et en voyait chez elle plus que les autres
dames espagnoles. Je l'ai fort vue\,; souvent elle m'envoyait ce qu'on
appelle un \emph{recao}, qui n'est qu'un compliment par un gentilhomme,
et savoir de mes nouvelles, et la coutume est d'y répondre par une
visite. Elle avait un beau palais à une extrémité de Madrid, qui donnait
sur la campagne, magnifiquement meublé. Son mari se tenait dans son
appartement. On ne le voyait jamais dans celui de sa femme, qui s'en
passait très bien, quoique en grande et juste réputation de vertu. On
fut surpris avec raison qu'elle eût accepté d'être camarera-mayor de
M\textsuperscript{lle} de Beaujolais, destinée alors à l'infant don
Carlos. On n'en pouvait choisir une plus agréable par elle-même ni plus
capable de former une princesse. Aussi réussit-elle très bien, et s'en
fit fort aimer.

Maceda, \emph{Lanços}. C'est une maison de Galice, ancienne, mais qui
n'a rien d'illustre. Le comte de Maceda que j'ai vu à Madrid était un
très bon et très honnête homme, fort simple, fort modeste, peu répandu
et d'un esprit médiocre. Il n'était jamais sorti de chez lui lorsque la
guerre mit en feu toutes les provinces d'Espagne. Sa fidélité pour
Philippe V se distingua dans la sienne par les efforts de sa bourse,
quoique peu riche, de son crédit et de ses soins. Il se présenta à tout
avec valeur et jugement, secondé du comte de Taboada son fils, qui avait
tout l'esprit, la valeur, le sens et l'activité possible. La guerre
finie, Philippe V, qui avait beaucoup ouï parler de leurs services, s'en
souvint\,; il fut surpris de ne les point voir à Madrid\,; il leur fit
dire d'y venir, et fort peu après, il fit le comte de Maceda grand
d'Espagne, et tout le monde y applaudit. Dans la suite, il fit la
comtesse de Taboada, dame du palais, qui avait aussi de l'esprit et du
mérite, et ils étaient aimés et considérés à Madrid où il se fixèrent,
et l'étaient fort en Galice. Le comte de Taboada était borgne
d'accident\,; il en plaisantait le premier\,; il était fort dans le
monde, et désiré et estimé partout. Il était fort des amis des ducs de
Veragua et de Liria, du prince de Masseran et de beaucoup d'autres.
C'est un de ceux qui venait le plus familièrement manger ou causer chez
moi. Je n'ai point vu d'homme plus gai ni qui eût la repartie plus vive,
plus fine, plus à la main. Ces trois amis que je viens de nommer
l'attaquaient sans cesse. C'était entre eux des escarmouches
ravissantes. Il était déjà lieutenant général, quoique jeune, et a
toujours depuis continué à servir. Il a perdu son père depuis mon
retour, et est devenu capitaine général avec beaucoup de réputation, de
valeur et de talent pour la guerre, et d'homme d'honneur et de probité.
Il a pris le nom de comte de Maceda, et a fait sa couverture depuis la
mort de son père.

Miranda, \emph{Chaves}. Cette terre, qui est sur le Duero, fut érigée en
comté par Henri II, roi de Castille, pour Pierre de Zuniga, second fils
du premier comte de Ledesma. Après avoir passé en diverses maisons par
des filles héritières, la dernière fut Anne, fille unique de Ferdinand
de Zuniga, comte de Miranda et duc de Peñeranda, qui porta l'un et
l'autre avec beaucoup de grands biens en mariage à J. de Chaves, comte
de La Calçada et de Casarubios, fils de Melchior de Chaves, frère et
héritier de Balthasar de Chaves, comte de La Calçada et
d'Isabelle-Joséphine Chacon y Mendoza, comtesse de Casarubios, et mourut
en 1696, et laissa des fils et des filles. Cette maison de Chaves est
ancienne et grandement alliée. Je ne vois point la date de la grandesse
de Miranda, mais la date de celle du duché de Peñeranda me persuade que
l'autre est de même date\,; car Miranda est certainement grandesse, et
le Chaves que j'ai vu à Madrid, qui les possédait toutes deux,
s'appelait comte de Miranda, ce qu'il n'eût pas fait étant duc de
Peñeranda, qui est grandesse, si Miranda ne l'était pas. Disons donc un
mot de Peñeranda, son érection en duché par Philippe III, pour Jean de
Zuniga y Avellaneda y Cardenas, vice-roi de Catalogne, puis de Naples,
enfin président des conseils d'État et de guerre. Il était fils puîné de
Fr.~de Zuniga, quatrième comte de Miranda, et il avait épousé la fille
de son frère aîné, héritière de la maison de Miranda. Leur fils Diègue
lui succéda, et fut père de Fr., troisième duc de Peñeranda, auquel
Philippe IV accorda la grandesse de première classe en 1629\,; car ce
n'est que depuis très peu d'années que tous les duchés sont peu à peu
devenus grandesses, avant quoi ils ne donnaient qu'une dénomination
distinguée, mais sans rang et sans honneurs. L'année suivante il devint
comte de Miranda par la mort de sa grand'mère susdite. Sa postérité
masculine défaillit, et ses biens et ses deux grandesses furent portés
dans la maison de Chaves, comme il a été expliqué au commencement de cet
article.

Montijo, \emph{Acuña y Portocarrero}. On peut voir au titre d'Ossone ce
qu'il est dit de la maison d'Acuña, et que les marquis de Villena, ducs
d'Escalone, en sont les aînés. Pierre d'Acuña, second fils du premier
duc d'Escalone et marquis de Villena, et de Marie, héritière de
Portocarrero, en ajouta le nom au sien, et fit cette branche qui souvent
porta le nom seul de Portocarrero. Son fils fut seigneur de Montijo, et
le fils de celui-là en fut fait comte par Charles II, en 1697. C'est le
cinquième comte de Montijo que j'ai vu en Espagne. Il était fort jeune
et fort bien fait, et avait déjà la Toison d'or. Son père avait été fait
grand d'Espagne par Charles II, et avait laissé son fils enfant qui fut
marié de fort bonne heure, servit dès qu'il le put dans la fin de la
guerre, s'incommoda, et eut le bon sens de se retirer avec sa femme dans
ses terres pour raccommoder ses affaires. Il y avait déjà longtemps
qu'il vivait dans cette retraite, qui n'était pas fort loin de Lerma,
lorsqu'il y parut au mariage du prince des Asturies. Il y fut très bien
reçu du roi, et de la reine qui avait pris de la bonté pour lui. Cette
retraite lui avait fait honneur\,; et il avait montré de la valeur à la
guerre. Toute la cour marqua de la joie et de l'empressement de le voir.
Il retourna chez lui de Lerma, et ne vint à Madrid que peu avant mon
départ où il fut très bien reçu de tout le monde, et où je le vis assez.
Il me parut de l'esprit, instruit, sage et beaucoup de politesse et
d'envie de faire. C'est lui qui longtemps depuis fut ambassadeur en
Angleterre et à Francfort, pour l'élection de l'empereur, électeur de
Bavière. Il se plaignit fort de mon absence à la Ferté dans ses courts
passages à Paris. Il fut grand écuyer de la reine après Cellamare, et
son majordome-major après Santa-Cruz, ce qui enfin lui a procuré l'ordre
du Saint-Esprit.

Oñate, \emph{Velez de Guevara}. Terre en Biscaye, est possédée depuis
plusieurs siècles par l'ancienne maison Velez de Guevara, illustre par
ses possessions, ses alliances et ses emplois. Henri IV, roi de
Castille, fit en 1449, Inigo Velez de Guevara comte d'Oñate, dans la
postérité masculine duquel elle s'est toujours conservée de père en fils
ou deux seules fois par des héritières qui ont épousé de leurs parents
du même nom, armes et maison qu'elles. Le huitième comte d'Oñate, dont
la grand'mère était Tassis ou Taxis, succéda à l'utile charge
héréditaire de grand maître des postes d'Espagne et au comté de
Villamediana à Jean de Tassis, second comte de Villamediana, neveu de sa
grand'mère, qui fut tué d'un coup de pistolet, 21 août 1622, à Madrid
étant dans son carrosse avec don Louis de Haro\,; et on prétendit alors
que Philippe IV l'avait soupçonné d'être amoureux de la reine son
épouse, Élisabeth de France, et avait fait faire le, coup. Ce comte de
Villamediana n'avait point d'enfants, et ce huitième comte d'Oñate
transmit ses biens et sa charge à sa postérité, laquelle, je crois, a eu
le même sort que les charges héréditaires de connétable et d'amirante de
Castille, supprimées par Philippe V, et que celle de grand maître des
postes, dont le profit était grand, et les fonctions importantes et peu
convenables à une succession héréditaire, aura changé de forme\,; mais
c'est de quoi je ne me suis pas avisé de m'informer. C'est le onzième
comte d'Oñate que j'ai vu fort peu à Madrid, où il vivait fort retiré,
où peut-être l'avait jeté la disgrâce de son puissant beau-père, le
huitième duc de Medina-Coeli, mort en prison en 1711, à Fontarabie,
comme on le peut voir à l'article de Medina-Cœli.

Quant à la date de la grandesse, il paraît qu'elle est la même que
l'érection en comté, c'est-à-dire que Inigo Velez de Guevara, premier
comte d'Ouate en 1469, devint en même temps rico-hombre, et que de cette
dignité les comtes d'Ouate passèrent sous Charles-Quint à celle de
grands d'Espagne, ayant toujours été grands d'Espagne depuis.

Oropesa, \emph{Portugal y Toledo}. J'ai expliqué, ce me semble (t. III,
p.~88), les branches royales de Portugal, Oropesa, Lemos, Veragua,
Cadaval, etc., en sorte que je n'ai plus rien à y ajouter ici. J'ai de
même exposé, lors de l'avènement de Philippe V à la couronne d'Espagne,
ce qui regardait le personnel du comte d'Oropesa d'alors\footnote{Même
  remarque que plus haut. Le passage du t. III, auquel renvoie
  Saint-Simon, est un de ceux que les précédents éditeurs ont
  retranchés.}, président du conseil de Castille, sous Charles II, exilé
par lui, rappelé tout à la fin de la vie de ce roi, exilé de nouveau peu
après l'arrivée de Philippe V en Espagne, et mort dans cet exil. Depuis
mon retour son fils revint à Madrid, y épousa une fille du comte de
San-Estevan de Gormaz, et fut après chevalier de la Toison, en même
promotion avec son beau-père.

Palma, \emph{Bocanegra y Portocarrero}. Alphonse XI, roi de Castille,
donna cette terre en 1342 à Gilles Bocanegra, qui s'était mis à son
service, et était pour lui général de la mer. Son frère était duc de la
république de Gênes. Gilles avait épousé Marie de Fiesque. Leur
troisième petit-fils, quatrième seigneur de Palma, épousa
Fr.~Portocarrero, et ses descendants s'honorèrent tellement de cette
alliance qu'ils quittèrent leur nom de Bocanegra, et ne prirent plus que
le nom de Portocarrero. Louis, arrière-petit-fils du Bocanegra et de la
Portocarrero, et huitième seigneur de Palma, en fut fait comte par la
reine Jeanne, mère de Charles-Quint, en 1507\,; et son petit-fils,
troisième comte de Palma, fut fait marquis d'Almenara en 1623, par
Philippe IV. Le fils de ce troisième comte de Palma, et premier marquis
d'Almenara, mourut avant son père, et laissa deux fils dont le cadet fut
le fameux cardinal Portocarrero, promu par Clément IX, en 1669, depuis
archevêque de Tolède, dont il a été tant parlé ici, à l'occasion du
testament de Charles II\footnote{Voy. t. III, p.~3 et 5. C'est encore un
  des passages supprimés par les anciens éditeurs.}, de l'arrivée de
Philippe V en Espagne, et plusieurs fois depuis. Son frère aîné L. Ant.
Th. Bocanegra y Portocarrero, cinquième comte de Palma, fut rétabli, en
1679, par Charles II, dans le rang et honneurs de grand d'Espagne, dont
ses pères, ricos-hombres avant Charles-Quint, avaient été laissés par
cet empereur et roi d'Espagne dans l'état commun de ceux qu'il avait
comme dégradés, en abolissant cette dignité pour établir en sa place
celle de grands d'Espagne, où il n'avait point admis le comte de Palma
ni ses successeurs jusqu'à Charles II. Ce premier grand d'Espagne, comte
de Palma, eut un fils qui fut persécuté par la princesse des Ursins,
sous Philippe V, par haine pour sa femme, qui avait beaucoup d'esprit,
qui voyait beaucoup de monde à Madrid, qui était extrêmement considérée,
et y tenait une manière de tribunal où tout était apprécié, et où on ne
pardonnait pas à la princesse des Ursins sa conduite fort étrange à
l'égard du cardinal Portocarrero, dont on a parlé ici plus d'une fois. À
la fin même le comte et la comtesse de Palma furent exilés\,; c'est leur
fils qui leur avait succédé du temps que j'étais en Espagne, mais que je
n'y ai point vu, qui vivait mécontent et fort retiré, qui venait fort
rarement à Madrid, et qui ne se présentait point au palais.

Parcen, \emph{Sarcenio}.

Parédes, dit \emph{Tolede y La Cercla}, en Castille, appartenant à
Roderic Manrique qu'Henri IV en fit comte et grand de Castille en 1452.
De cette maison de Manrique de Lara elle passa en plusieurs autres par
des filles héritières, puis à un cadet de la maison de Gonzague, dont
l'héritière épousa Th., des bâtards de Foix, marquis de La Laguna en
1675. Il était frère du huitième duc de Medina-Coeli, et oncle du
dernier duc de Medina-Coeli, mort prisonnier à Fontarabie, dernier duc
de Medina-Coeli des bâtards de Foix. Le marquis de La Laguna, devenu
ainsi par sa femme comte de Parédes, fut capitaine général de la mer,
vice-roi de la Nouvelle-Espagne, enfin majordome-major de la palatine,
seconde femme de Charles II, qui en même temps le fit grand de la
troisième classe, et seulement pour sa personne en 1689\,; il mourut en
1692. Fort peu après, Charles II accorda la grandesse à sa veuve pour
elle et pour ses héritiers à toujours, en considération de ce que les
comtes de Parédes avaient été grands de Castille jusqu'à Charles-Quint,
c'est-à-dire ricos-hombres, et n'avaient pas été compris parmi ceux qui
de ce rang passèrent, sous Charles-Quint, à celui de grands d'Espagne,
et demeurèrent dégradés. Cette même dame fut, en 1694, camarera-mayor de
la reine, mère de Charles II, jusqu'à la mort de cette princesse, qui
arriva en 1696. Elle laissa un fils né à Mexique en 1683, que j'ai vu à
Madrid.

Peneranda, \emph{Velasco}. Terre qu'il ne faut pas confondre avec une
autre du même nom qui est duché, dont il a été parlé en l'article de
Miranda. Celle-ci fut érigée en comté par Philippe III pour Alph. de
Bracamonte, gouverneur de l'infant Charles son fils. Balthasar Emmanuel,
fils aîné d'Alph. de Bracamonte, second comte de Peñeranda n'eut que
deux filles. L'aînée porta le comté de Peñeranda en mariage à Gaspard de
Bracamonte, frère de son père, qui fut conseiller d'État, président des
conseils des ordres, des Indes et d'Italie, vice-roi de Naples, ensuite
ambassadeur plénipotentiaire d'Espagne à la paix de Munster\,; enfin, à
la mort de Philippe IV, un des gouverneurs de la monarchie. Il mourut à
Madrid en 1676. Son fils mourut tout à la fin de 1689 sans enfants.

Son héritière fut Ant. de Bracamonte, seconde fille de
Balthasar-Emmanuel, second comte de Peñeranda, dont la soeur aînée
l'avait porté en mariage au frère de leur père. Cette Ant. avait épousé
Pierre Fernandez de Velasco, second marquis del Fresno, qui fut
ambassadeur d'Espagne en Angleterre et conseiller d'État. Son père, né
sourd et muet, avait appris à se faire entendre, à lire, à écrire, etc.,
avec le prince de Carignan, en 1638, à Madrid, par l'industrie d'un
Espagnol, nommé Emmanuel Ramirez de Carion. Ce second marquis del
Fresno, devenu comte de Peñeranda par sa femme, obtint de Charles II la
grandesse à vie de troisième classe, puis de l'étendre à la vie de son
fils, qui l'a enfin obtenue perpétuelle de Philippe V. On a parlé sous
le titre de Frias de la maison de Velasco. Ce comte de Peñeranda était à
Madrid de mon temps.

Peralada, \emph{Rocaberti}.

Priego, \emph{Cordoue}. J'ai fort connu et pratiqué à Madrid le comte de
Priego, qui était ami intime du duc de Bejar, avec lesquels j'ai eu en
tiers, plusieurs bonnes et sages conversations et quelquefois assez
instructives. Le comte de Priego était un petit homme noir, rougeaud,
ventru, des yeux pétillants d'esprit et de feu, et qui ne trompaient
pas\,; aussi avant dans le grand monde qu'un seigneur espagnol y pouvait
être, et qui se fit faire grand d'Espagne fort plaisamment.

Il avisa que la princesse des Ursins avait fait venir d'Italie à Madrid
le fils de sa défunte soeur de Lanti, qu'elle avait fort aimée, qu'il
était pauvre et qu'elle cherchait à le marier richement\,; lui fit
accroire que sa fille unique serait un fort grand parti. Il sut si bien
conduire que tous les examens qu'elle en fit faire l'en persuadèrent si
bien qu'elle pensa tout de bon au mariage, et le lui fit proposer.
Priego, en habile homme, se fit prier et si bien qu'il déclara qu'il
voulait une condition sans laquelle il ne le ferait point et avec
laquelle il conclurait de tout son coeur\,; que cette condition était au
pouvoir de la princesse des Ursins et à l'avantage de son neveu\,; qu'en
un mot il voulait être grand d'Espagne. M\textsuperscript{me} des
Ursins, surprise de la sécheresse avec laquelle cette proposition se
faisait, fit la froide, se montra étonnée que quelqu'un prétendit lui
faire la loi. Priego n'en fut pas la dupe et laissa tomber la chose.
M\textsuperscript{me} des Ursins le voyant si résolu ne voulut pourtant
pas manquer une si bonne affaire, lui fit reparler et proposer de faire
donner la grandesse à son neveu en épousant sa fille. Priego répondit
qu'on se moquait de lui, qu'il savait bien que M\textsuperscript{me} des
Ursins ne manquerait pas tôt ou tard de procurer la grandesse à son
neveu\,; que peu lui importait à lui qui, avec ses grands biens, ne
serait pas embarrassé de trouver un grand d'Espagne ou un fils aîné de
grand pour sa fille, mais que, la voulant bien donner à un homme aussi
peu riche qu'était Lanti, parce qu'il était neveu de la princesse des
Ursins qui le désirait, et par respect et par attachement pour elle,
c'était bien le moins qu'il en profitât et qu'il eût la grandesse qui,
après lui qui était déjà vieux, et il le paraissait bien plus qu'il ne
l'était, passerait à sa fille et à son gendre. M\textsuperscript{me} des
Ursins, qui vit bien qu'il n'en démordrait pas, essaya de le résoudre à
faire le mariage en lui promettant qu'elle prendrait après son temps
pour lui faire obtenir ce qu'il désirait. Mais Priego sentit bien que,
s'il mariait sa fille sur ces belles promesses, on se moquerait de lui
après\,; que M\textsuperscript{me} des Ursins ferait faire Lanti grand
d'Espagne, et s'excuserait sur ce qu'elle n'avait pu obtenir qu'il le
fût. Il renvoya donc la proposition bien loin, fit dire net à
M\textsuperscript{me} des Ursins que, pouvant tout ce qu'elle voulait,
il ne comprenait point tant de difficultés\,; qu'en un mot, l'affaire
était à prendre ou à laisser, et qu'elle pouvait compter que le mariage
ne se ferait point qu'il ne fût grand d'Espagne, qu'il n'en eût toutes
les expéditions, et que de plus il n'eût fait sa couverture. Il y tint
ferme, fut fait grand d'Espagne, eut toutes ses expéditions, fit sa
couverture, après quoi le mariage suivit immédiatement. Il logea chez
lui son gendre, et sa fille fut dame du palais. Mais
M\textsuperscript{me} des Ursins et son neveu ne furent pas longtemps
sans s'apercevoir que ce grand parti était et serait en effet des plus
médiocres, et que Priego les avait joués pour être fait grand d'Espagne.
Ils furent enragés de la duperie, mais ils firent en gens sages\,:
l'affaire était faite\,; le gendre, qui était doux et honnête homme,
n'en vécut pas moins bien avec sa femme et son beau-père\,; pour
M\textsuperscript{me} des Ursins, elle eut toujours une dent contre lui,
elle la cachait, mais on s'apercevait aisément qu'elle ne pouvait lui
pardonner de l'avoir attrapée on ne convenait pas trop en Espagne que ce
comte de Priego fût de la maison de Cordoue.

Tous les matins en se levant, en toute saison, on lui versait doucement
une aiguière d'eau à la glace sur la tête, dont il ne tombait pas une
goutte à terre. Sa tête la consumait toute à mesure. Il prétendait que
cela lui faisait le plus grand bien du monde. L'abbé Testu, l'ami de
M\textsuperscript{me} de Maintenon et de tant de gens considérables de
la cour et de la ville, avec qui il a passé sa longue vie, et dont il a
été parlé ici plus d'une fois, avait la même pratique, et il n'en
tombait pas non plus une goutte à terre, mais c'était de l'eau
naturelle, ni chauffée, ni à la glace, en aucune saison. Depuis mon
départ, Lanti perdit sa femme, longtemps avant son beau-père, et n'en
avait qu'une fille, en sorte qu'il ne pouvait plus être grand, parce que
la grandesse passait par-dessus lui du grand-père à la petite fille
immédiatement. Il fut du temps en cet état\,; à la fin il obtint de
Philippe V une grandesse personnelle de troisième classe, et prit alors
le nom de duc de Santo-Gemini. Il perdit depuis son beau-père et maria
sa fille au second fils de la duchesse d'Havré, sa soeur, qui par là fut
grand d'Espagne et comte de Priego, qui alla s'y établir.

Salvatierra, \emph{Sarmiento y Sotomayor}.

Tessé, \emph{Froulay\,;} Français, à Paris. Le maréchal de Tessé,
premier écuyer de M\textsuperscript{me} la duchesse de Bourgogne, qui se
piqua de l'aimer pour avoir fait la paix de Turin et traité son mariage.
Elle lui procura la grandesse à bon marché, en 1704, lorsqu'il maria son
fils si richement à la fille unique de Bouchu, conseiller d'État\,; il
fit accroire au roi que contre tout usage, le roi d'Espagne lui avait
permis de suivre l'usage de France et de se démettre, comme font les
ducs, depuis le dernier connétable de Montmorency qui se démit le
premier, et au roi d'Espagne que le roi l'avait voulu ainsi. La
tromperie fut découverte, mais la belle-fille avait eu le tabouret et le
garda.

Visconti, \emph{idem}, Génois. Ainsi, il y a deux Visconti grands
d'Espagne, l'un avec le titre de marquis, l'autre de comte.

On verra par la liste {[}suivante{]} tous les grands d'Espagne et de
quelle maison ils sont, existant aujourd'hui, d'un seul coup d'oeil, en
même ordre qu'en détail ci-devant.

DUCS

Abrantès, \emph{Alencastro}.

Berwick, \emph{Fitzjames}.

Albe, \emph{Tolède}.

Bournonville, \emph{idem}.

Albuquerque, \emph{Bertrand y La Cueva del Arco Manrique de Lara}.

Doria, \emph{idem}.

Arcos, \emph{Ponce de Léon}.

Estrées, \emph{idem}. Éteint.

Aremberg, \emph{Ligne}.

Frias, \emph{Velasco}, connétable de Castille.

Arion, \emph{Sotomayor y Zuniga}.

Gandie, \emph{Llançol y Borgia}.

Atri, \emph{Acquaviva}.

Giovenazzo, \emph{Giudice}.

Atrisco, \emph{Sarmiento}.

Gravina, \emph{des Ursins}.

Baños, Ponce de \emph{Léon}.

Havré, \emph{Croï}.

Bejar, \emph{Sotomayor y Zuniga}.

Hijar, \emph{Silva}.

Del Infantado, \emph{Silva}.

\emph{Noailles, idem}.

Licera, \emph{Aragon}.

Osuna, \emph{Acuña y Tellez-Giron}.

Liñarès, \emph{Alencastro}.

Saint-Pierre, \emph{Spinola}.

Liria, \emph{Fitzjames}.

Popoli, \emph{Cantelmi}.

Medina-Coeli, \emph{Figuerroa y La Cerda}.

Sessa, \emph{Folch y Cardonna}.

Medina de Riosecco, \emph{Enriquez y Cabrera}.

Saint-Simon, \emph{idem}.

Medina-Sidonia, \emph{Guzman}, amirante de Castille.

Solferino, \emph{Gonzague}.

Saint-Michel, \emph{Gravina}.

Tursis, \emph{Doria}.

La Mirandole, \emph{Pico}.

Veragua, \emph{Portugal y Colomb}.

Monteillano, \emph{Solis}.

Villars, \emph{idem}.

Monteléon, \emph{Pignatelli}.

Uzeda, \emph{Acuña y Pacheco y Tellez-Giron}.

Mortemart, \emph{Rochechouart}. Éteint.

46, dont deux sont les mêmes que leurs fils, conjointement, et deux
éteints, ainsi 44 depuis.

Nagera, \emph{Ossorio y Moscoso}.

Nevers, \emph{Mancini}.

PRINCES DE

Bisignano, \emph{Saint-Séverin}.

Ligne, \emph{idem}.

Santo-Buono, \emph{Caraccioli}.

Masseran, \emph{Ferrero}.

Buttera, \emph{Branciforte}.

Melphe, \emph{Doria}.

Cariati, \emph{Spinelli}.

Ottaïano, \emph{Médicis}.

Chalais, \emph{Talleyrand}.

Palagonia, \emph{Gravina}.

Chimay, \emph{Hennin-Liétard}.

Robecque, \emph{Montmorency}.

Castiglione, \emph{Aquino}.

Sermoneta, \emph{Gaetano}.

Colonne, \emph{idem}, connétable de Naples.

Sulmone, \emph{Borghèse}.

Doria, \emph{idem}.

Surmia, \emph{Odeschalchi}.

\begin{enumerate}
\def\labelenumi{\arabic{enumi}.}
\setcounter{enumi}{17}
\tightlist
\item
\end{enumerate}

MARQUIS

Arizza, \emph{Palafox}.

Mancera.

Ayetona, \emph{Moncade}.

Mondejar, \emph{Ivannez}.

Los Balbazès, \emph{Spinola}.

Montalègre, \emph{Guzman}.

Bedmar, \emph{Bertrand y La Cueva}.

Pescaire, \emph{Avalos}.

Camaraça, \emph{Los Cobos}.

Richebourg, \emph{Melun}. Éteint.

Castel dos Rios, \emph{Sernmenat}.

Ruffec, \emph{Saint-Simon}.

Castel-Rodrigo, \emph{Hoinodeï\,; Pio}.

Torrecusa, \emph{Caraccioli}.

Castromonte, \emph{Breza}.

Tavara, \emph{Tolède}.

Clarafuente, \emph{Grillo}.

Villena, \emph{Acuña y Pacheco}.

Santa-Cruz, \emph{Benavidez y Bazan}.

Villafranca, \emph{Tolède}.

Laconi, \emph{idem}.

Visconti, \emph{idem}.

Lede, \emph{Bette}.

\begin{enumerate}
\def\labelenumi{\arabic{enumi}.}
\setcounter{enumi}{22}
\tightlist
\item
\end{enumerate}

COMTES

Aguilar, \emph{Manrique de Lara}.

Altarez, \emph{Villalpando}.

Altamira, \emph{Ossorio y Moscoso}.

Baños, \emph{Mancade y La Cerda}.

Aranda, \emph{Roccafull}.

Benavente, \emph{Pimentel}.

Los Arcos, \emph{Guzman}.

Castrillo, \emph{Crespi}.

Egmont, \emph{Pignatelli}.

Palma, \emph{Bocanegra y Portocarrero}.

San-Estevan de Gormaz, \emph{Acuña y Pacheco}.

Parcen, \emph{Sarcenio ou Sarterio}.

San-Estevan del Puerto, \emph{Benavidez}.

Parédes, dit \emph{Tolède y La Cerda}, mais de \emph{Medina-Coeli} des
bâtards de Foix.

Fuensalida, \emph{Velasco}.

Peneranda, \emph{Velasco}.

Lamonclava, \emph{Bocanegra}.

Peralada, \emph{Rocaberti}.

Lemos, \emph{Portugal y Castro}.

Priego, \emph{Cordoue}.

Maceda, \emph{Lanços}.

Salvatierra, \emph{Sarmiento}.

Miranda, \emph{Chaves}.

Tessé, \emph{Froulay}.

Montijo, \emph{Acuña y Portocarrero}.

Visconti, \emph{idem}.

Oropesa, \emph{Portugal y Tolède}.

\begin{enumerate}
\def\labelenumi{\arabic{enumi}.}
\setcounter{enumi}{26}
\tightlist
\item
\end{enumerate}

Ainsi 112 grands\footnote{On a reproduit exactement les chiffres donnés
  par Saint-Simon, quoiqu'ils ne soient pas toujours d'accord avec les
  listes.}.

On y compte les trois éteints depuis.

Mais Philippe V en a fait beaucoup depuis.

On n'y compte que pour deux les deux pères qui le sont conjointement
avec leur fils.

Ducs en Espagne

32

Marquis en Espagne

18

en France

5

en France

1

en Flandre

1

en Flandre

»

en Italie

6

en Italie

3

44

Princes en Espagne

2

Comtes en Espagne

25

en France

3

en France

2

en Flandre

1

en Flandre

»

en Italie

12

en Italie

1

18

28

Espagnols.

Français.

Flamands.

Italiens.

Anglais.

Duc

25

5

3

10

1

Prince

0

1

3

14

»

Marquis

14

1

2

5

»

Comte

23

1

0

4

»

62

8

8

33

1

Grands en tous pays, 112.

Outre ses grands, il y en a par charges ou état, qui sont\,:

Le majordome-major du roi.

Le général de la Mercy.

Le grand prieur de Castille de Malte.

Le général des dominicains.

L'abbé de Cîteaux.

Le général des cordeliers.

L'abbé de Clairvaux.

Le général des capucins.

Mais ces grands sont imperceptibles. Rien de si rare qu'un
majordome-major du roi d'Espagne ne soit pas pris d'entre les grands, et
plus rare encore, s'il se peut, qu'il ne soit pas fait grand, s'il ne
l'est pas, fort tôt après être fait majordome-major. Le grand prieuré de
Castille de l'ordre de Malte, qui vaut cent mille écus de rente, est
donné à un des infants, et tant qu'il y aura de ces princes, il y a
toute apparence que ce riche morceau demeurera entre leurs mains. À
l'égard des moines, ce n'est que très improprement qu'on les dit être
grands d'Espagne. Ils n'ont jamais eu nulle part hors de l'Espagne
aucune des distinctions, rangs ni honneurs des grands d'Espagne\,; ils
en reçoivent à titre de généraux d'ordre, et quoique ce puisse être à
titre de grandesse, et jusqu'à présent les choses ont toujours été ainsi
en Espagne\,; même quand ils y vont pour la visite de leurs couvents ou
les affaires de leurs ordres, ils n'y sont pas autrement traités qu'à
titre de généraux d'ordre. Tout ce qu'ils ont de particulier en Espagne,
et qu'ils n'ont nulle part ailleurs, c'est que la première fois
seulement qu'ils vont saluer le roi d'Espagne, il les fait couvrir, et
ils se couvrent en effet, et c'est de là qu'ils sont dits grands
d'Espagne. Mais après cette première fois, s'ils reparaissent devant le
roi d'Espagne, ils ne se couvrent point et n'y ont aucune distinction
différente de celles qu'y ont les autres généraux d'ordre qui ne sont
point grands, c'est-à-dire qui ne se couvrent jamais devant le roi
d'Espagne. Il en est de même en Espagne à leur égard partout, comme à
l'égard de ces derniers, d'avec lesquels ils n'ont aucune différence.
Depuis mon retour, le général des jésuites a été associé au même
honneur, aussi imperceptible pour lui que pour les six autres.

Il faut maintenant réparer l'oubli que j'ai fait des marquis de Tavara
et de Villafranca. Je veux me flatter qu'il n'y en a point d'autre dans
ce qu'il y avait de grands d'Espagne existant en avril 1722, que je suis
parti de la cour d'Espagne pour revenir en celle de France. Je n'oserais
toutefois m'en répondre, quelque soin que j'y aie pris dans le peu de
temps que j'ai pu y donner en Espagne, et en matière si étendue en tant
de pays, et si diverse par tant de transmissions d'héritières. Cet oubli
n'est pas dans la table des grands précédente.

Tavara, \emph{Tolède}, Emmanuel, par sa mère A. M. de Cordoue y
Pimentel, dont la mère était A. M. Pimentel, sixième marquise de Tavara.
Tavara m'a été donné pour grandesse par le duc de Veragua, et j'ai de sa
main une liste de grands d'Espagne, à laquelle j'ai conformé celle que
j'ai mise ici, dans laquelle Tavara est compris entre les marquis grands
d'Espagne. Mais je n'ai pas eu ou le temps de m'instruire de toutes les
grandesses, ou d'en garder toutes les instructions en note, ou de
retenir tout ce que je n'ai pas eu par écrit. Il s'en faut donc beaucoup
que je puisse rendre compte de toutes ces grandesses\,; et celle de
Tavara est de ce nombre.

VILLAFRANCA, \emph{Tolède}. Ce marquis et le précédent étaient enfants
des deux frères. Cette terre, dans le royaume de Léon, fut érigée par
les rois catholiques en marquisat, vers 1490, en faveur de Louis
Pimentel, mort, en 1497, avant son père, quatrième comte de Benavente.
Sa fille unique porta sa grandesse et ses biens en mariage à Pierre
Alvarez de Tolède, second fils du second duc d'Albe, dans la maison
duquel cette grandesse est demeurée jusqu'à celui que j'ai vu en
Espagne, qui était petit-fils du marquis de Villafranca, duquel il a été
tant parlé à l'occasion du testament de Charles II\footnote{Voy. t. III,
  p.~9.}\,; de l'arrivée de Philippe V en Espagne, dont il fut
majordome-major, et qui fut un des cinq premiers seigneurs espagnols à
qui le feu roi envoya l'ordre du Saint-Esprit. Son même petit-fils fut
par sa mère, héritière de Moncade y Aragon, duc de Montalte et de
Vibonne, et par sa femme, marquis de Los Velez, de sorte que je le
laissai avec quatre grandesses. Il était jeune, et ne faisait pas encore
souvenir de son grand-père. Ces trois dernières sont en Aragon, en
Sicile, et au royaume de Naples, toutes trois de Ferdinand le
Catholique, les quatre rico-hombreries alors sont devenues grandesses
sous Charles-Quint, et n'ont fait que passer d'un état et d'un nom à un
autre.

On a vu, lorsque j'ai traité, t. III, p.~224 et suiv., des grands et de
leur dignité, le soin qu'ils apportent de tout temps à faire un mystère
de leur ancienneté et de leurs classes. Tous conspirent à vouloir cacher
leurs différentes classes, qui, en effet, ne sont sensibles que dans
leur diplôme d'érection dans leur couverture, et dans le style de
chancellerie à leur égard\,; et quant à l'ancienneté à laisser croire,
en l'étouffant parmi eux, qu'ils viennent tous de ces anciens
ricos-hombres abolis par Charles-Quint, et transformés en grands
d'Espagne, dont il imagina la dignité destituée de la puissance de
celles des ricos-hombres qu'il abolit peu à peu en leur substituant la
grandesse. J'ai tâché de pénétrer autant qu'il m'a été possible le
secret de l'ancienneté. Il est vrai qu'il m'en est échappé une vingtaine
sur cent douze grands, existant en 1722 que j'ai quitté l'Espagne, et
qu'il y en a plusieurs autres, dont je n'ai pu fixer l'érection qu'avec
incertitude, en disant vers telle année. Dans ces cas, je me suis réglé
aux générations ou aux emplois le plus vraisemblablement qu'il m'a été
possible, sans reculer ni avancer trop celui qui le premier a eu la
qualité de grand d'Espagne, et dont les pères ne l'avaient pas. Et comme
ces grandesses, dont les héritières femelles sont presque toutes
capables, tombent quelquefois par elles à des grands postérieurs aux
grandesses qu'elles leur apportent, j'ai eu soin de les marquer quand
cela est arrivé, ce qui s'est trouvé rare.

Quand aux classes, je n'ai pu rien y démêler, sinon que Philippe II,
comme je l'ai remarqué (t. III, p.~233), en traitant de la grandesse,
n'a fait de grands que de la seconde classe. On voit assez au long, dans
la première liste alphabétique des grandesses, ce qui regarde ceux qui
les ont possédées. Je me contenterai, dans l'abrégé suivant, rangé, non
plus par ordre alphabétique ni de titres, mais par l'ordre d'ancienneté
que j'ai pu découvrir, {[}d'indiquer{]} pour qui érigées, et à qui
tombées, sans m'y étendre davantage, ni rien répéter de ce qui se trouve
dans la première liste alphabétique, sinon quelques légers suppléments.

ÉTAT DES GRANDS D'ESPAGNE, EXISTANT EN AVRIL 1722, SUIVANT CE QU'ON A PU
DÉCOUVRIR DE DATES DE LEUR ANCIENNETÉ RESPECTIVE.

Ricos-Hombres, dont l'ancienne dignité trop multipliée, abrogée par
Charles-Quint, et transmuée en celle de grand d'Espagne qu'il inventa, a
passé sous ce prince en grandesse, sans nouvelle érection, les autres
qui n'y passèrent pas, étant demeurées abolies, et les grands d'Espagne
de Charles-Quint, et depuis.

HENRI II.

C'est le fameux comte de Transtamare, frère bâtard du roi Pierre le
Cruel qui le vainquit, le tua, et fut élu roi de Castille en sa place,
dont la couronne passa à sa postérité.

Medina-Coeli, comté 1368, duché 1491, par les rois catholiques. Il y a
lieu de croire que cette érection en duché ne fut que pour une
dénomination plus distinguée, parce qu'on ne peut pas douter que ce
bâtard de Foix, qui eut l'honneur d'épouser l'héritière de Medina-Coeli,
laquelle était vraiment La Cerda, et qui en fut fait comte, ne fût pas
dès lors ricohombre. De cette race des bâtards de Foix, ce duché passa
par l'héritière dans la maison de Figuerroa, en épousant le marquis de
Priego, duc de Feria, deux fois grand d'Espagne, père du duc de
Medina-Coeli que j'ai vu en Espagne, dont elle fut mère, et apporta les
grandesses de Medina-Coeli, duché\,; Ségorbe, duché\,; Cardonne,
duché\,; Alcala, duché\,; Denia, marquisat\,; Comarès, marquisat\,;
Cogolludo, marquisat\,; San-Gadée, comté. Ces Figuerroa Medina-Coeli en
ont encore accumulé plusieurs autres depuis.

HENRI III.

Benavente, comté 1398, pour J. Alph. Pimentel, d'où il n'est point
sorti.

Amirante de Castille, charge héréditairement donnée par le même roi,
vers 1400, à Alph. Enriquez, fils puîné de Frédéric, maître de l'ordre
de Saint-Jacques, et frère jumeau du roi Henri II, fils bâtards tous
deux du roi Alphonse XI, et de sa maîtresse Éléonore de Guzman. On ne
peut, ce me semble, contester la qualité de rico-hombre à ce premier
amirante. Jean II le fit comte de Melgar, vers 1438. Ces dignités ne
sont point sorties de cette maison, non plus que celle de duc de Medina
di Riosecco, ajoutée par Charles-Quint, 1520.

JEAN II.

Arcos, comte 1440, pour Pierre-Ponce de Léon, marquis, 1484, par les
rois catholiques, duc par les mêmes, 1498, sans qu'Arcos soit jamais
sorti de cette maison.

HENRI IV.

Lemos, comté 1457, pour Pierre Alvarez Ossorio, dont le fils eut un
bâtard, la fille duquel le porta en mariage, mais un peu à la morisque,
à Denis de Portugal, second fils du troisième duc de Bragance dans la
postérité masculine {[}duquel{]} il est demeuré.

Medina-Sidonia, duché février 1460, pour J. Alph. de Guzman. Jean II
l'en avait fait duc en 1445, mais pour sa vie seulement. Henri IV
l'étendit à toute sa postérité légitime, et même à son défaut à
l'illégitime, suivant les moeurs morisques. Il est demeuré dans sa
postérité masculine et légitime.

Miranda, comté vers 1460, pour Diego Lopez de Zuniga. L'héritière de
Zuniga le porta vers 1670 à J. de Chaves y Chacon avec Peñeranda, duché
érigé en 1621, par Philippe III, pour J. de Zuniga, devenu comte de
Miranda, par son mariage avec sa nièce, héritière de Miranda. Ainsi, par
soi et par elle, il fut deux fois grand d'Espagne. Mais ces doubles
grands, soit de la maison de Zuniga, soit de celle de Chaves, ont
toujours porté le nom et le titre de comte de Miranda plus ancien
préférablement à celui de duc de Peñeranda, qui tous deux sont demeurés
dans la maison de Chaves.

Albuquerque, duché 1464, pour Bertrand de La Cueva. Sa postérité
masculine défaillit bientôt après, et l'héritière le porta en mariage à
un François appelé Hugues Bertrand, qui prit le nom seul et les armes de
La Cueva, et de ce mariage est issue toute la maison de La Cueva, d'où
ce duché n'est point sorti.

Villena, marquisat 1468, pour J. d'Acuña y Pacheco, qu'il fit encore
l'année suivante, 1469, duc d'Escalona. Henri IV était impuissant\,;
Isabelle, sa soeur, voulut le faire déclarer tel, et lui succéder. Cela
causa de grands troubles et des partis. Celui d'Isabelle déposa Henri IV
en 1465. Il se soutint tant qu'il put, et continua à faire des actes
valides de royauté. Isabelle, pour s'appuyer sur le trône de Castille,
épousa en 1469, Ferdinand, roi d'Aragon, son cousin issu de germain par
mâles sortis du roi Henri II. C'est eux qui se sont appelés les rois
catholiques, du titre de roi catholique que Ferdinand obtint à Rome\,;
et comme chacun d'eux gouvernait son propre royaume avec indépendance
l'un de l'autre, on prit l'habitude en Espagne, en parlant d'eux, de
dire \emph{les rois}. Cette façon de parler s'y est tellement établie
qu'on y dit encore \emph{les rois}, quand on y parle du roi et de la
reine ensemble, quoique depuis fort peu de règne de Jeanne, fille
d'Isabelle, et mère de Charles-Quint, les reines d'Espagne
n'ont\footnote{On a conservé le texte de Saint-Simon\,; les précédents
  éditeurs avaient changé \emph{n'ont} en \emph{n'aient}.} rien gouverné
que quand elles ont été veuves et régentes. Ce peu d'historique eût été
mieux en sa place dans la précédente liste détaillée\,; j'ai mieux aimé
en réparer ici l'oubli.

Henri IV étant mort en 1474, il y eut des prétentions du Portugal sur la
Castille, et des troubles qui ne sont pas de mon sujet. J. d'Acuña y
Pacheco qui avait été favori d'Henri IV, et par conséquent peu attaché à
Isabelle, sa soeur, qui de son vivant en voulait à sa couronne, favorisa
le Portugal, dont les efforts furent impuissants. La reine Isabelle l'en
punit en lui ôtant le marquisat de Villena qui est en Castille, et
l'unit à sa couronne, où il est toujours demeuré réuni, sans que la
postérité masculine de ce J. d'Acuña y Pacheco en aient quitté la
prétention, et le titre qu'ils ont toujours porté de préférence à celui
de duc d'Escalone. On en voit encore d'autres à l'article de Villena
dans la précédente liste détaillée. Cette même postérité masculine est
encore en possession du duché d'Escalone, et du titre de Villena, sans
le marquisat.

Albe, duché, 1469, pour Garcias Alvarez de Tolède, et il est demeuré
depuis dans cette maison. Jean II l'avait donné en titre de comté dès
1430, à Guttiere-Gomez de Tolède, qui était évêque, comme on le voit en
la précédente liste détaillée\,; le légua à son neveu, père de celui qui
fut fait duc. La distance en est si courte que je n'ai pas cru m'y
devoir arrêter, d'autant que cela a commencé par un évêque qui n'était
pas dans le cas des ricos-hombres, ni par conséquent d'en communiquer la
dignité aux siens. Ainsi, je me suis fixé à l'érection d'Albe en duché.

Oñate, comté, 1469, pour Inigo Velez de Guevara. Il est sorti, puis
rentré par des filles héritières, et demeuré enfin dans cette maison.

ROIS CATHOLIQUES.

Infantado, duché 1475, pour Diego Hurtado de Mendoza. Il passa enfin
d'héritière en héritière par mariage, vers 1680, à Roderic de Silva,
quatrième duc de Pastrane, prince d'Eboli, et est demeuré à leurs
descendants masculins, qui ont tous porté le titre de duc del Infantado
préférablement à celui de duc de Pastrane, comme plus ancien. On a vu,
pages 381 et suiv., ce qui regarde Pastrane, omis ailleurs, parce que
cette grandesse est sur la même tête que celle de l'Infantado.

Oropesa, comté 1475, pour Ferdinand de Tolède. Sa postérité masculine
défaillit au cinquième comte d'Oropesa, dont la fille aînée porta ce
comté avec d'autres biens en mariage à Édouard de Portugal, frère puîné
de Théodose II de Portugal, père du duc de Bragance, ou du roi Jean IV
de Portugal, en 1640, par la révolution de Portugal en sa faveur, qui en
chassa les Espagnols. Ce comte d'Oropesa, par sa femme, s'alla établir
en Espagne, où sa postérité masculine est demeurée avec le comté
d'Oropesa. Il fallait que cette rico-hombrerie, devenue tout de suite
grandesse sous Charles-Quint, n'eût pas été mise dans la première classe
lorsque les classes furent inventées depuis et établies, puisqu'elle n'y
fut mise que par Charles II, en août 1690, pour ce comte d'Oropesa qu'il
exila depuis, qui, après être revenu à Madrid, à l'arrivée de Philippe
V, en fut bientôt après exilé, qui se déclara pour l'archiduc, en 1706,
qui mourut un an après à Barcelone, dont il a été parlé ici en plusieurs
occasions, et dont le fils, comte d'Oropesa, est revenu depuis mon
retour, et a épousé à Madrid une fille du comte de San-Estevan de
Gormaz.

Najera, duché 1482, pour Pierre Manrique de Lara. D'héritières en
héritières, A. de Guevara le porta en mariage à Jos. Ossorio y Moscoso,
frère cadet du comte d'Altamire, pendant que j'étais en Espagne.

Gandie, duché 1485, pour Pierre-Louis Llançol, dit Borgia, second fils
bâtard du pape Alexandre VI, et père de saint François de Borgia. Ce
duché s'est masculinement conservé dans cette maison.

Sessa, duché vers 1486, pour le grand capitaine Alphonse de Cordoue.
Fr.~de Cordoue, héritière, fit cession de ce duché et de ses autres
biens, n'ayant point d'enfants, au fils de sa soeur cadette et unique,
Ant. Folch de Cardonne, qui par là fut aussi duc de Baëna, et qui par
son père était aussi duc de Somme. Ces grandesses se sont masculinement
conservées dans cette maison.

Bejar, duché 1488, pour Alvare de Zuniga. Thérèse de Zuniga, héritière,
porta ses biens et ce duché en mariage à Fr.~de Sotomayor, cinquième
comte de Belalcazar, en la postérité masculine duquel il est demeuré.

Frias, duché vers 1488, pour Bernardin-Fernandez de Velasco, second
connétable de Castille, de sa maison, qui y rendit cette charge
héréditaire. Son père avait eu cette charge le premier de sa maison, en
1473, après six autres connétables\,; ainsi, n'ayant été qu'à vie
jusqu'à son fils, j'ai cru ne devoir fixer son ancienneté qu'à
l'érection du duché de Frias, qui est depuis masculinement demeuré à sa
postérité.

Villafranca, marquisat 1490, pour Louis Pimentel. L'héritière de
Pimentel porta ce marquisat et ses autres biens en mariage à Pierre
Alvarez de Tolède, second fils du second duc d'Albe, dans la postérité
masculine {[}duquel{]} il est demeuré, laquelle a depuis acquis par des
héritières trois autres grandesses, qui sont les duchés de Montalte et
de Vibonne, et le marquisat de Los Velez. Il était aussi duc de
Ferrandine, mais Villafranca étant plus ancien que ces autres titres, il
leur a préféré, ainsi que ses pères, de porter le nom de marquis de
Villafranca.

CHARLES-QUINT.

Egmontest sûrement de ce prince\,: je n'ai pu en découvrir la date. Il y
a tout lieu de croire que ce roi des Espagnes n'oublia pas un aussi
grand seigneur de ses sujets des Pays-Bas, lorsque, à l'occasion de son
voyage d'Espagne en Allemagne pour y recevoir la couronne impériale, il
prit son temps d'abolir l'ancienne dignité des ricos-hombres, d'imaginer
et d'établir celles des grands d'Espagne qu'il y substitua, d'en faire
en même temps des anciens ricos-hombres par une simple conservation et
transmission d'une dignité à l'autre, en dégradant tacitement ceux
d'entre eux qu'il ne conservait pas par cette transition, et de leur
associer en même temps des plus grands seigneurs à la nouvelle dignité
de grands d'Espagne, qui n'avaient point été ricos-hombres, des uns et
des autres, desquels, devenus grands d'Espagne, il se fit accompagner à
son couronnement impérial, où il leur procura des distinctions-, des
rangs, et l'honneur de se couvrir en sa présence et au couronnement.

Cette grandesse est demeurée jusqu'à nos jours dans la maison d'Egmont
qui s'est entièrement éteinte. La soeur du dernier comte d'Egmont, et
dernier mole, mort sans enfants, hérita de ses biens et de sa grandesse.
Elle avait épousé le duc de Bisaccia de la maison Pignatelli, dont il a
été parlé plus d'une fois ici, et dont le fils prit le nom et les armes
d'Egmont, et s'est établi en France par son mariage avec la seconde
fille du duc de Duras, fils aîné et frère des maréchaux ducs de Duras.

Veragua, duché 1557, pour Diegue Colomb, fils du fameux Christophe. Ce
duché passa par Isabelle Colomb, héritière, à son petit-fils Nuisez de
Portugal, dans les descendants masculins duquel il est demeuré.

Pescaire, marquisat 1537, pour Alphonse d'Avalos, dans la postérité
duquel cette grandesse est demeurée.

PHILIPPE II.

Ayétone, marquisat vers 1560, pour J. de Moncade, dans la postérité
masculine duquel cette grandesse est toujours demeurée.

Ossuna, duché 1562, pour Pierre d'Acuña y Giron, dans la postérité
masculine duquel cette grandesse est depuis demeurée.

Terranova, duché 1565, pour Charles Tagliavia. J. Tagliavia, héritière,
porta ses biens et cette grandesse en mariage à Hector Pignatelli en
1679. Leur fils aîné épousa la fille héritière du septième duc de
Monteléon, Pignatelli aussi, dont la grandesse était de Philippe III, en
1613. Ces deux grandesses sont demeurées dans leur postérité
masculine\,; et depuis, ces grands ont préféré de porter le nom de duc
de Monteléon, comme venant de leur maison, à celui de duc de Terranova,
plus ancien, mais leur venant par femme.

Santa-Cruz, marquisat 1582, pour Alvare Bazan, général de la mer,
aussitôt après sa victoire navale et l'horrible massacre de sang-froid
qu'il fit de tous les prisonniers français dans l'île de Saint-Michel,
juillet 1582. Cette grandesse, d'héritière en héritière, tomba enfin à
Français Diaz de Benavidez, mort en 1680, père de celui que j'ai vu en
Espagne.

Aranda, comté 1590, pour Antoine Ximénès d'Urrea. Cette grandesse passa
par la maison d'Heredia, dont l'héritière la porta en mariage à
Guillaume de Roccafull, dans la postérité masculine duquel elle est
demeurée.

PHILIPPE III.

UZEDA, duché vers 1610, pour Fr.~Gomez de Sandoval, fils aîné du duc de
Lerme, premier ministre, et mort avant lui, dont l'héritière, après
diverses générations, quoique cadette, et je n'ai pu découvrir la cause
de ce partage, porta la grandesse d'Uzeda en mariage à Gaspard d'Acuña y
Tellez-Giron, qu'on a vu ici ambassadeur d'Espagne à Rome, à la mort de
Charles II, qui fit très bien à l'avènement de Philippe V, et qui, étant
encore son ambassadeur à Rome, se jeta dans le parti de l'archiduc, où
il est mort et, a laissé un fils.

Peñeranda, comté vers 1511, pour Alphonse de Braccamonte, qui par
l'héritière de Braccamonte, a été porté en mariage à Pierre-Fernandez de
Velasco, deuxième marquis del Fresno. J'ignore par quelle difficulté, en
la transmission de cette grandesse, ce même Pierre-Fernandez de Velasco
a été fait grand d'Espagne par Charles II, d'abord à vie, puis pour
celle aussi de son fils. C'est une difficulté dont je n'ai pas été
éclairci, car les Braccamonte, comtes de Peñeranda ont été certainement
grands d'Espagne à ce titre, et de la date ci-dessus de Philippe III.

Mondejar, marquisat vers 1612, pour Inigo Lopez de Mendoza. Cette
grandesse passa en plusieurs maisons par des filles héritières. Celle de
Cordoue la porta enfin en mariage à Gaspard Ivannez, comte de Tendilla,
qui en prit le nom, fit sa couverture en 1678, et a laissé un fils,
marquis de Mondejar, que j'ai vu à Madrid.

Hijar, duché 1614, pour J. Chr. L. Fernandez d'Hijar, arrière-petit-fils
de mâle en mâle de J. Fernandez, seigneur d'Hijar, en faveur duquel ce
duché avait été érigé en 1483, et n'avait point passé en grandesse sous
Charles-Quint. De filles en filles héritières il tomba dans la maison de
Silva, dont l'héritière le porta en mariage, à la fin de 1688, à
Frédéric de Silva, marquis d'Orani, son cousin, de même maison, dans la
postérité masculine duquel il est demeuré.

Havré, duché vers 1616, pour Ch. Alex. de Croï, de la branche d'Arschot.
Sa fille unique porta ses biens et sa grandesse à P. Fr.~de Croï, second
fils de Ph. de Croï, comte de Solre, qui prit le nom de duc d'Havré, et
cette grandesse est demeurée en sa postérité masculine.

Sulmone, principauté vers 1521, pour un Borghèse, fils du frère du pape
Paul V, à qui cette grandesse ne put être refusée, et qui est demeurée
dans cette postérité masculine.

LosBalbazès, marquisat 1621, pour le fameux Ambroise Spinola, dans la
postérité masculine duquel cette grandesse s'est conservée, avec celle
du duc del Sesto, par le mariage de la fille héritière de Paul Doria,
duc del Sesto\,; mais ils ont toujours préféré de porter le titre propre
de leur maison à celui de duc del Sesto.

PHILIPPE IV.

Altamire, comté 1621, pour Gaspard Ossorio y Moscoso, dans la postérité
masculine duquel cette grandesse s'est conservée. Gaspard était pourtant
le septième comte d'Altamire lorsqu'il obtint de Philippe IV la
grandesse, dont ses pères étaient déchus, qui l'avaient eue par
l'héritière d'Ulloa y Moscoso. Cette rico-hombrerie, érigée pour Lopez
d'Ulloa y Moscoso, dans les fins du règne de Jean II, vers 1452, n'était
pas passée en grandesse sous Charles-Quint, et était ainsi demeurée
dégradée.

Abrantès, duché vers 1625, pour Alphonse d'Alencastro, issu par mâles de
Georges, bâtard de Jean II, roi de Portugal, dans la postérité masculine
duquel cette grandesse est demeurée avec celle de Liñarès, par le
mariage du deuxième duc d'Abrantès avec l'héritière de Noroña y Silva,
fille de Ferdinand duc de Liñarès.

Bisignano, principauté 1626, pour Louis de San-Severino, dans la
postérité masculine duquel cette grandesse est demeurée.

Castel-Rodrigo, marquisat vers 1629, pour Chr. de Moura, qui avait été
premier vice-roi de Portugal, et c'est ce qui me fait craindre de m'être
trompé, et qu'encore qu'il fût fort vieux quand il fut fait grand
d'Espagne, il ne le soit de Philippe III. Quoi qu'il en soit, son fils
et son petit-fils lui succédèrent et furent l'un après l'autre
gouverneurs généraux des Pays-Bas. La fille héritière du dernier épousa,
à la fin de 1678, Charles Homodeï, marquis d'Almonacid, qui devint
marquis de Castel-Rodrigo, et en prit le nom, mais qui ne put faire sa
couverture qu'un an après, sur les difficultés qu'il essuya\,; je n'ai
point su sur quoi fondées. Il n'eut point d'enfants, et perdit sa femme
dont hérita sa soeur cadette, qui avait épousé Gilbert Pio, mère du
prince Pio, que j'ai vu en Espagne, qui recueillit la grandesse après
elle, sans préjudice du rang et des honneurs restés personnellement au
marquis d'Almonacid avec, sa vie durant, le nom et le titre de marquis
de Castel-Rodrigo.

Torrecusa, marquisat vers 1630, pour Ch. André Carraccioli, dont la
grandesse est masculinement demeurée à sa postérité.

Colonne, principauté, connétable héréditaire du royaume de Naples vers
1632, pour Laurent-Onuphre, septième connétable Colonne. Cette grandesse
est demeurée dans sa postérité masculine.

Camaraça, marquisat vers 1635, pour Diego de Los Cobos, dans la maison
duquel cette grandesse s'est conservée.

Aguilar, comté janvier 1640, pour Jean-Ramirez d'Arellano. Il épousa
Anne-Marie, fille unique de J. de Mendoza, premier marquis d'Hinoyosa
qu'elle lui apporta, et fut ainsi doublement grand d'Espagne, comme
comte d'Aguilar et marquis d'Hinoyosa. Lui et les siens ont préféré au
titre d'Hinoyosa celui d'Aguilar, dont il était huitième comte. J.
Ramirez d'Arellano eut Aguilar du roi Jean ter\,; en 1681. Il était
rico-hombre de Castille. Son petit-fils, Alph. Ram. d'Arellano en fut
fait comte en 1475, par les rois catholiques, et jouit des honneurs de
la grandesse ou rico-hombrerie d'alors. Mais n'ayant point passé en
grandesse sous Charles-Quint, elle demeura abrogée jusqu'au
rétablissement qui vient d'être expliqué. Celui qui fut rétabli ne
laissa qu'une fille qui épousa, en 1670, Emmanuel Manrique de Lara,
deuxième marquis de Frigilliana à qui elle apporta ces deux grandesses,
et qui a laissé un fils, comte d'Aguilar, que j'ai vu à Paris, et depuis
en Espagne. C'est de ce père et de ce fils qu'il est parlé ici à
plusieurs reprises.

Aremberg, duché vers 1650, pour Ph. Fr.~de Ligne, fils aîné de Ph. Ch.
de Ligne, de la branche de Barbançon, prince d'Aremberg, chevalier de la
Toison d'or, mort à Madrid en 1640, et de sa deuxième femme Is. de
Barlaymont. Ph. Fr., premier duc d'Aremberg, et fait grand d'Espagne,
fut chevalier de la Toison, général des mers des Pays-Bas espagnols,
gouverneur du Hainaut et de Valenciennes, et capitaine des archers de la
garde bourguignonne de Philippe IV, et de Charles II, en Flandre, où il
mourut sans postérité en 1674. Ses biens et sa grandesse passèrent à Ch.
Eugène, son frère, dans la postérité masculine duquel elle est demeurée,
mais passée et retournée au service de la maison d'Autriche, depuis que
les Pays-Bas espagnols sont rentrés sous son obéissance. J'ai voulu
suppléer ici à la négligence de cet article dans le précédent état
détaillé.

Ligne, principauté 1660, pour Cl. Lamoral de Ligne, grand-père de celui
qui existait lorsque j'étais en Espagne, qui a postérité masculine, et
est à Bruxelles au service de l'empereur. Il est de Philippe IV.

CHARLES II.

Fuensalida, comté 1670, pour Bernardin de Velasco y Rojas et Cardonne.
Cette grandesse s'est conservée dans sa postérité masculine.

Saint-Pierre, duché 1675, pour Fr.~M. Spinola. Cette grandesse est
demeurée dans sa postérité masculine.

Palma, comté juillet 1679, pour L. Ant. Th. Bocanegra y Portocarrero.
Louis Bocanegra y Portocarrero avait été fait comte de Palma par la
reine Jeanne, 1507, mais cette rico-hombrerie n'ayant point passé en
grandesse sous Charles-Quint, fils de cette reine, demeura abrogée.
Depuis le rétablissement de cette grandesse, elle est demeurée dans la
postérité masculine de celui qui l'a obtenue.

Nevers, 1680, pour J. B. Spinola, dont la fille aînée l'a porté en
mariage, en 1709, à L. J. Fr.~Mancini, dit Mazzarini, fait depuis duc et
pair de Nevers.

Santo-Buono, principauté 1684, pour Matthieu Carraccioli. Cette
grandesse est demeurée dans sa postérité masculine.

Surmia, principauté vers 1686, pour Odeschalchi, neveu du pape Innocent
XI. Cette grandesse est encore dans les mâles de cette famille.

Giovenazzo, duché 1690, pour del Giudice, mais pour trois vies ou
générations seulement. Cette troisième génération est la fille unique du
prince de Cellamare plus connu, et dont il a été tant parlé ici sous ce
nom. Elle était dans un couvent à Rome. Je ne sais qui elle a épousé.

Liñarés, duché 1692, pour de Noroña, dont la fille unique l'a porté au
deuxième duc d'Abrantès qui, par un moyen ou grâce, à moi inconnu, a
divisé ces deux grandesses entre ses fils ou petits-fils.

Baños, comté 1692, pour Pierre, dit de La Cerda y Leyva, mais branche
cadette des ducs de Medina-Coeli, bâtards de Foix, dont la fille
héritière épousa, en 1693, Emmanuel de Moncade, frère du marquis
d'Ayétone, dont ce comte de Baños n'a eu qu'une fille point mariée,
lorsque j'étais en Espagne. Je n'ai point appris depuis à qui elle aura
porté sa grandesse.

Parédes, comté 1692, pour Th., marquis de La Laguna, frère du huitième
duc de Medina-Cœli. En 1689, il avait été fait grand à vie\,; ce ne fut
que trois ans après qu'il le fut fait à toujours, et cette grandesse est
demeurée dans sa postérité masculine. C'est une rico-hombrerie érigée
par Henri IV, 1452, pour Roderic Manrique, qui, n'ayant point passé en
grandesse sous Charles-Quint, demeura abrogée, et dont la terre passa
par des héritières de maison en maison jusqu'à l'épouse de ce marquis de
La Laguna, qui obtint la grandesse, et prit le nom de comte de Parédes.

Lamonclava, comté vers 1693, pour Melchior Bocanegra y Portocarrero,
dont la grandesse est demeurée dans sa postérité masculine.

San-Estevan del Puerto, comté 1696, pour Fr.~Benavidez, dont la
grandesse est demeurée dans sa postérité masculine.

Montalègre, marquisat octobre 1697, pour Martin-Dominique de Guzman, qui
a des fils.

LosArcos, comté octobre 1697, pour Joachim Figuerroa y Laso de La Vega,
qui a des fils.

Montijo, comté 1697, pour d'Acuña y Portocarrero. On a parlé ailleurs de
son fils que j'ai vu en Espagne, et qui a postérité masculine.

Baños, duché 1698, pour Ponce de Léon, frère du duc d'Arcos, établi
depuis en Portugal dans ses biens maternels.

Castromonte, marquisat 1698, pour Jean Baeza, a postérité masculine.

Castiglione, principauté 1699, pour Thomas d'Aquino que nous prononçons
d'Aquin.

OTTAÏANO, principauté 1700, pour Joseph de Médicis qui a postérité
masculine.

PHILIPPE V.

Castel dosRios, marquisat 1700, avant partir de Versailles, pour de
Semmenat, ambassadeur d'Espagne en France, à la mort de Charles II.
C'est le premier qui reconnut et baisa la main de Philippe V, qui, par
le conseil du roi son grand-père le fit grand de la première classe à
Versailles, et l'y fit couvrir comme grand d'Espagne la première fois
devant lui, pour lui tenir lieu d'avoir fait sa couverture. Sa grandesse
subsiste dans sa postérité masculine.

Mortemart, duché 1701. En arrivant à Madrid, une des premières choses
que fit Philippe V, fut de faire grand d'Espagne de la première classe
le duc de Beauvilliers, son gouverneur. Cette grandesse passa au duc de
Mortemart\,; par le mariage de sa fille unique, et s'est éteinte depuis
mon retour par la mort de la duchesse de Mortemart et de toute sa
postérité.

Estrées, comté 1702, pour le comte d'Estrées qui passa le roi d'Espagne
de Barcelone à Naples, étant vice-amiral de France. Longtemps depuis mon
retour, il est mort duc, pair et maréchal de France, sans postérité, et
sa grandesse est demeurée éteinte.

Liria, duché 1704, pour Fitzjames, duc de Berwick, à qui peu après son
fils fut adjoint en la même grandesse, pour en jouir avec les mêmes
rang, honneurs, etc., que lui. Il prit alors le nom de duc de Liria.
Cette grandesse est dans sa postérité masculine établie en Espagne.

Gravina, duché 1704, pour le chef de la maison des Ursins. Cette
grandesse est demeurée dans sa postérité masculine.

Bedmar, marquisat 1704, à la prière du roi pour Bertrand La Cueva,
commandant général des Pays-Bas espagnols. Cette grandesse, faute de
mêles, passe à son gendre, second fils du marquis de Villena, qui
s'appelle le marquis de Moya, et qui prendra le nom de marquis de
Bedmar.

Tessé, comté 1704, pour de Froulay, comte de Tessé, maréchal de France.
Cette grandesse est demeurée dans sa postérité masculine.

LaMirandole, duché 1705, pour Pico. Cette grandesse est demeurée dans sa
postérité masculine.

Atri, duché 1706, pour Acquaviva, frère du cardinal Acquaviva, chargé
des affaires d'Espagne à Rome. Son fils l'était du temps que j'étais en
Espagne. Il était lors en Italie et a postérité masculine.

Chimay, principauté 1706, pour Hennin-Liétard, chevalier de la Toison
d'or de Charles Il. Il a été mon gendre, est mort sans enfants. Sa
grandesse a passé à son frère, mort aussi depuis, et au fils qu'il a
laissé, et qui s'établit en France.

Monteillano, duché 1707, pour de Solis. Cette grandesse est demeurée
dans sa postérité masculine.

Priego, comté 1707, pour de Cordoue. Sa fille unique a épousé Lanti, dit
de La Rovère\,; elle est morte devant son père et n'a laissé qu'une
fille. Le père déchu par là de cette grandesse que sa femme n'a point
eue, a été fait grand à vie, sous le nom de duc de S.-Gemini, et a marié
sa fille, avec la grandesse, au second fils de la duchesse d'Havré, sa
soeur, Croï, qui s'établit en Espagne et prend le nom de comte de
Priego, -tout cela longtemps depuis mon retour d'Espagne.

Noailles, comté 1711, pour le duc de Noailles qui, longtemps depuis, a
obtenu de faire passer sa grandesse à son second fils qui en jouit et a
postérité masculine\,:

Popoli, duché 1711, pour Cantelmi. Cette grandesse est demeurée dans sa
postérité masculine.

Masseran, principauté 1712, pour Ferreiro. Cette grandesse est demeurée
dans sa postérité masculine.

Richebourg, marquisat 1712, pour de Melun. Éteinte, n'ayant laissé que
deux filles non mariées, et qui n'ont point voulu l'être, et hors d'âge
d'avoir postérité.

Chalais, principauté 1713, pour de Talleyrand. Sa fille unique a épousé
un fils de son frère.

Robecque, principauté 1713, pour de Montmorency. Son frère, faute de
postérité, et appelé, a recueilli cette grandesse, et a laissé un fils,
qui en jouit et a des garçons.

Maceda, comté 1714, pour Lanços. Cette grandesse est demeurée dans sa
postérité masculine.

Solfarino, duché 1714, pour Gonzague. Cette grandesse est demeurée dans
sa postérité masculine.

San-Estevan de Gormaz, comté 1715, pour Acuña y Pacheco, fils aîné du
marquis de Villena, duc d'Escalona, qui a postérité masculine.

Bournonville, duché 1715, pour \emph{idem} non marié, a fait longtemps
depuis passer sa grandesse et sa charge de capitaine de la compagnie des
gardes du corps wallons, au fils d'un de ses frères.

Villars, duché 1716, pour le maréchal duc de Villars. Son fils unique
n'a qu'une fille unique, mariée au comte d'Egmont.

Lede, marquisat 1717, pour Bette. Cette grandesse est demeurée dans sa
postérité masculine.

Saint-Michel, duché 1718, pour Gravina. Il a des fils, et s'est fait
depuis cardinal.

DelArco, duché 1718, sans enfants. Je ne sais à qui cette grandesse est
allée.

Saint-Simon, comté janvier 1722, pour le duc de Saint-Simon et le
marquis de Ruffec, son second fils conjointement.

Arlon, duché 1722, pour Sotomayor y Zuniga. Je ne sais à qui cette
grandesse est allée, car il n'a point été marié.

Il faut maintenant donner une liste toute simple des grands d'Espagne,
dont la date est ou nettement ou suffisamment reconnue, en marquant les
anciennes rico-hombreries que Charles-Quint fit passer tout de suite en
grandesses, sans érection, et celles qui, ayant été abrogées par le même
prince d'une manière tacite, mais très réelle, en ne les faisant point
passer en grandesses, ce qui de fait les dépouilla pour toujours de
leurs rangs, honneurs et distinctions, sont redevenues grandesses, mais
par des érections faites par les rois successeurs de Charles-Quint, ce
qui fixe leur ancienneté parmi les grands, sans la remonter à celle des
rico-hombreries abrogées, mais les réduisant à la date de l'érection de
leurs grandesses. Si on veut voir leurs dates et de quels rois, si on
veut voir leurs maisons et si les possesseurs actuels sont héritiers de
mâle en mâle, ou par des filles héritières, ou eux-mêmes impétrants de
ces grandesses, c'est ce qui se trouve exactement et différemment
détaillé dans les deux précédents états des grands d'Espagne. On fera
suivre la liste qu'on va donner des grands, suivant leur ancienneté
connue ou justement présumée, d'une autre liste toute nue, par titres et
par ordre alphabétique, des grands dont on n'a pu connaître ni présumer
les dates d'érection, non plus que de la plupart de ceux-là aucune autre
chose, desquels le grand nombre est d'Italiens jamais sortis d'Italie.

Si, au lieu de cent douze grands d'Espagne, il s'en trouve cent treize
dans ces deux listes jointes ensemble, c'est que le marquis de Mancera
avait été oublié. Je l'ai dans la liste des marquis grands d'Espagne de
la main du duc de Veragua. J'avouerai de plus que j'ai oublié quel il
est. Le duc de Veragua a écrit Portocarrero à côté de son nom, mais je
n'en suis pas plus avancé, parce que c'est peut-être le nom de
l'héritière qui a apporté cette grandesse. Le marquis de Mancera, qui
s'appelait Antoine-Sébastien de Tolède, deuxième marquis de Mancera, fut
fait grand d'Espagne en mai 1692, par Charles II. Il fut ambassadeur à
Venise et en Allemagne, vice-roi de la Nouvelle-Espagne, majordome-major
de la reine mère de Charles Il, enfin conseiller d'État. C'est lui dont
il {[}a{]} été parlé plus d'une fois par la fidélité et l'attachement
qu'il signala pour Philippe V d'une façon si éclatante, et dont la
singularité de ne manger jamais de pain, ni rien qui en tint lieu, a été
aussi expliquée. Il mourut en 1711, à l'âge de cent sept ans, ayant
jusqu'alors conservé sa tête entière et toute sa santé. Charles II
l'avait fait grand seulement à vie, Philippe V le fit pour toujours, et
je n'en sais pas la date. Il ne pouvait moins faire pour lui. Il ne
laissa qu'une fille, peut-être grand'mère lorsqu'il mourut. J'ai donc
ignoré ou oublié le mariage de cette fille, et ce qui s'en est suivi. Je
n'ai point vu de marquis de Mancera tant que j'ai été en Espagne,
tellement que je réserve ce titre pour la liste des grands dont la date
et souvent les personnes me sont demeurées inconnues.

LISTE SIMPLE DES GRANDS D'ESPAGNE, SUIVANT LEUR ANCIENNETÉ, NETTEMENT OU
SUFFISAMMENT RECONNUE, EN MARQUANT CEUX QUI D'ABORD OU DEPUIS SONT ISSUS
DES ANCIENS RICOS-SOMBRES, ABROGÉS PAR CHARLESQUINT, QUI SUBROGEA À
CETTE ANCIENNE DIGNITÉ LA NOUVELLE DES GRANDS ET CEUX QUI ONT PLUSIEURS
GRANDESSES\footnote{Les chiffres indiquent le nombre des grandesses.}.

12 Le duc de Medina-Coeli.

Le comte d'Egmont.

Le comte de Benavente.

Le duc de Veragua.

2 L'amirante de Castille, comte de Melgar, duc de Medina di Riosecco.

2 Le marquis de Pescaire. R.-H.

6 Le duc d'Arcos.

3 Le marquis d'Ayétone. R.-H.

Le comte de Lemos.

Le duc d'Ossone.

2 Le duc de Medina-Sidonia.

2 Le duc de Monteléon et de Terranova.

2 Le comte de Miranda.

Le marquis de Santa-Cruz.

Le duc d'Albuquerque.

2 Le comte d'Aranda.

3 Le marquis de Villena, duc d'Escalona.

Le duc d'Uzeda.

9 Le duc d'Albe.

Le comte de Peñeranda.

Le comte d'Oñate.

Le marquis de Mondejar.

5 Le duc del Infantado.

2 Le duc d'Hijar. R.-H.

Le comte d'Oropesa.

Le duc d'Havré.

Le duc de Najara.

Le prince de Sulmone.

Le duc de Gandie.

3 Le marquis de Los Balbazès.

3 Le duc de Sessa.

5 Le comte d'Altamire. R.-H.

Le duc de Bejar.

Le duc d'Abrantès.

3 Le duc de Frias, connétable de Castille.

Le prince de Bisignano.

On voit ci-devant à leurs titres pourquoi l'amirante et le connétable de
Castille sont ici différemment qualifiés.

Le marquis de Castel-Rodrigo.

Le marquis de Torrecusa.

4 Le marquis de Villafranca.

Le connétable Colonne.

Tous ces grands ont passé sous Charles-Quint directement de la dignité
de ricos-hombres à celle de grand d'Espagne sans érection.

Ceux dont la dignité de ricos-hombres est demeurée abrogée par le fait
lors de ce changement de Charles-Quint, et qui depuis ont été faits
grands d'Espagne, seront marqués à côté de leurs noms par ces deux
lettres R-H.

Le marquis de Camaraça.

3 Le comte d'Aguilar. R.-H.

2 Le duc d'Aremberg.

Le prince de Ligne.

Le comte de Fuensalida.

Le duc de Saint-Pierre.

Le comte de Palma. R.-H.

Le duc de Nevers.

Le comte de Los Arcos.

Le prince de Santo-Buono.

Le comte de Montijo.

Le prince de Surmia.

Le duc de Baños.

Le duc de Giovenazzo.

Le marquis de Castromonte.

Le duc de Liñarez.

Le prince de Castiglione.

Le comte de Baños.

Le prince d'Ottaïano.

2 Le comte de Parédes. R.-H.

Le marquis de Castel dos Rios.

Le comte de Lamonclava.

Le duc de Mortemart, éteint.

Le comte de San-Estevan del Puerto. R.-H.

Le maréchal d'Estrées, éteint.

Le marquis de Montalègre.

Le duc de Liria.

Le duc de Noailles.

Le duc de Gravina.

Le duc de Popoli.

Le marquis de Bedmar.

Le prince de Masseran.

Le maréchal de Tessé.

Le marquis de Richebourg, éteint.

Le duc de La Mirandole.

Le prince de Chalais.

Le duc d'Atri.

Le prince de Robecque.

Le prince de Chimay.

Le comte de Maceda.

Le duc de Monteillano.

Le duc de Solferino.

Le comte de Priego.

Le comte de San-Estevan de Gormaz.

Le duc de Saint-Michel.

Le duc de Bournonville.

Le duc del Arco.

Le maréchal duc de Villars.

Le marquis de Ruffec.

Le marquis de Lede.

Le duc d'Arion.

LISTE SIMPLE DES GRANDS D'ESPAGNE DONT J'IGNORE LES DATES D'ÉRECTION ET
BEAUCOUP D'AUTRES CONNAISSANCES, PAR ORDRE ALPHABÉTIQUE ET PAR TITRES.

Les ducs d'Atrisco.

Le marquis de Clarafuente.

Doria.

Laconi.

2 Licera.

Mancera.

Tursis.

Tavara.

Les princes de Butera.

Visconti.

Cariati.

Les comtes d'Atarès.

Doria.

Castrillo.

Melphe.

Parcen.

Palagonia

Peralada.

Sermonetta.

Salvatierra.

Les marquis d'Arizza.

Visconti.

\end{document}
